\documentclass{mai_book}

\defaultfontfeatures{Mapping=tex-text}
\setdefaultlanguage{russian}
\setcounter{page}{637} % ВОТ ТУТ ЗАДАТЬ СТРАНИЦУ
% \setcounter{thesection}{6} % ТАК ЗАДАВАТЬ ГЛАВЫ, ПАРАГРАФЫ И ПРОЧЕЕ.
% Эти счетчики достаточно задать один раз, обновляются дальше сами


\begin{document}
\subsection{Тензорный ранг произведения двух многочленов}
Установим теперь некоторые результаты относительно сложности (в только что уточненном смысле) умножения двух многочленов. Начнем с наименьшего числа умножений, необходимых для вычисления произведения многочленов. Если работать в теле, имеющем достаточно элементов, можно воспроизвести многочлен $n$-й степени по его вычислениям в $n + 1$ точках. Это выражается следующим результатом:
\begin{predl}
Для тела мощности $\geqslant$ $n + m$ тензорный ранг произведения многочлена степени $n$ и степени $m$ равен $n + m + 1$.
  \end{predl}
\begin{myproof}
Изложим идею доказательства, оставив детали до упражнения 26. Пусть $X(T) = x_nT^n + \ldots + x_1T + x_0$ --- многочлен степени $n$, $Y(T) = y_mT^m + \ldots + y_1T + y_0$ --- многочлен степени $m$ и $Z(T) = z_{n+m} + \ldots z_1T + z_0$ --- их произведение. Выразим двумя существенно различными способами \{$z_0$,$z_1$,\ldots,$z_{n+m}$\} при помощи линейных комбинаций $n + m + 1$ элементарных произведений.
\begin{itemize}
\item{Первый способ предполагает, что мощность тела $\geqslant n + m + 1$. Выберем $n + m + 1$ различных точек $t_0$,$t_1$,\ldots,$t_{n+m}$ и вычислим значения $X(T)$ и $Y(T)$ в $t_0$,$t_1$,\ldots,$t_{n+m}$. Это дает $n + m + 1$ пар линейных форм, и вычислим $z_k$ с помощью интерполяции посредством матрицы Вандермонда порядка ($n + m + 1$). Следует заметить, что процесс интерполяции не требует порождающих умножений. Все необходимые порождающие умножения содержатся среди произведений значений многочленов.}
\item{Второй метод предполагает лишь, что мощность тела $\geqslant n + m$. Он использует многочлен $W(T) = X(T)Y(T) - x_ny_m(T - t_1)(T - t_2)\ldots(T - t_{n+m}) = w_{n+m-1}T^{n+m-1} + \ldots +\linebreak+w_0$, где $t_1$,\ldots,$t_{n+m}$ --- $n + m$ различных точек. Этот многочлен степени, строго меньшей $n + m$, может, очевидно, интерполироваться по точкам $t_i$: билинейные формы $w_k$, являющиеся коэффициентами $W(T)$, получаются с помощью линейных комбинаций $n + m$ элементарных билинейных форм $X(t_k) \otimes Y(t_k)$. Так как при $0 \leqslant k < n+m, z_k = w_k + c_k * x_ny_m, c_k$ --- константа кольца, и так как $z_{n+m} = x_ny_m$, то получим $n + m + 1$ билинейных форм $z_k$ в виде линейных комбинаций $n + m + 1$ элементарных билинейных форм $\{X(t_k) \otimes Y(t_k) \mid 1 \leqslant k \leqslant n+m\} \cup \{x_ny_m\}$.}
\end{itemize}
\end{myproof}
Ш. Виноград в \textit{<<Some bilinear forms whose multiplicutive complexity\linebreak depends on the field of constants>>} показал, что оба этих метода, детализированные в предыдущем доказательстве, являются единственными методами, позволяющими получить $z_k$ в виде линейных комбинаций $n+m+1$ элементарных билинейных форм.
\begin{beznomera}
  \textbf{Непосредственные применения}
  \end{beznomera}
Проиллюстрируем только что обсуждавшееся доказательство на знаменитом примере: произведение $z_2T^2+z_1T+z_0$ двух многочленов степени 1,($x_1T+x_0$) и ($y_1T+y_0$):
\begin{itemize}
\item{первый метод доказательства дает при использовании точек (0,1,$-1$) для вычисления значений: $z_0=x_0y_0$,
  \begin{center}
    $
    \begin{array}{l}
  z_1=\frac{1}{2}(x_1+x_0)(y_1+y_0)-\frac{1}{2}(x_1-x_0)(y_1-y_0), \\
  z_2=\frac{1}{2}(x_1+x_0)(y_1+y_0)+\frac{1}{2}(x_1-x_0)(y_1-y_0)-z_0,
\end{array}
  $
  \end{center}
  }
\item{второй метод дает при использовании точек (0,1):
  \begin{center}
    $
    \begin{array}{l}
    z_0=x_0y_0,\ z_1=(x_0+x_1)(y_0+y_1)-x_0y_0-x_1y_1,\ z_2=x_1y_1,
    \end{array}
    $
\end{center}}
\item{тот же метод при использовании точек (0,$-1$):
  \begin{center}
    $
    \begin{array}{l}
      z_0=x_0y_0,\ z_1=x_0y_0+x_1y_1-(x_0-x_1)(y_0-y_1),\ z_2=x_1y_1.
    \end{array}
    $
\end{center}}
  \end{itemize}
  \par
Оставшуюся часть раздела посвятим исследованию тензорного ранга алгебры $K[T]/P$, где $P$ --- унитарный многочлен.
\begin{property}
  Пусть $P$ --- унитарный многочлен степени $n$ с коэффициентами в теле $K$ мощности $\geqslant 2(n-1)$ (например, в теле характеристики 0 или просто в бесконечном теле). Тогда тензорный ранг алгебры $K[T]/P$, т.~е. тензорный ранг семейства $n$ билинейных форм, определяющих произведения алгебры, не превосходит $2n-1$.
  \end{property}
\begin{myproof}
  Элемент из $K[T]/P$ представляется в виде многочлена из $K[T]$ степени $\leqslant n-1$ (лучше: семейство \{$T^{n-1},T^{n-2},\ldots,T,1$\} образует $K$-базис алгебры $K[T]/P$). Умножение многочлена степени $n-1$ на многочлен степени $n-1$ требует $(n-1)+(n-1)+1=2n-1$ <<общих>> умножений; остается лишь взять результат по модулю $P$, что является линейной операцией.
  \end{myproof}
\begin{beznomera}
  \textbf{Пример}
  \end{beznomera}
Проиллюстрируем доказательство на примере $P(T)=T^2+T+1.$ Пусть $X(T)=x_1T+x_0$ и $Y(T)=y_1T+y_0$ --- два элемента из $K[T]/P$. Вычисляя их произведение в $K[T]$, используем три умножения. Если $W(T)=X(T)Y(T)=w_2T^2+$ $+w_1T+w_0$, то можно использовать, например, $w_0=x_0y_0, w_1=x_0y_0+x_1y_1-(x_1-x_0)$ $(y_1-y_0) и w_2=x_1y_1.$ Для вычисления $z_1T+z_0=X(T)Y(T)\ ($mod$T^2+T+1)$ используется соотношение $T^2=-T-1$, что дает окончательно $z_0=x_0y_0-x_1y_1 и z_1=$ $=x_0y_0-(x_1-x_0)(y_1-y_0).$\par
Можно улучшить этот результат, разлагая $P$ в произведение простых сомножителей и применяя китайскую теорему об остатках. Для начала установим следующий элементарный результат:
\begin{predl}
  Пусть $E$ и $F$ --- две свободные $K$-алгебры конечного типа над $K$. Тогда справедливо следующее неравенство для тензорного ранга:
  \begin{center}
    \textup{rang}$_{\otimes}(E \oplus F) \leqslant $ \textup{rang}$_{\otimes}(E)+ $\textup{rang}$_{\otimes}(F).$
    \end{center}
\end{predl}
\begin{myproof}
  Положим $n\ =\ \dim E$, $m\ =\ \dim F$ и обозначим \{$f_1,f_2,\ldots,f_n$\} и \{$g_1,g_2,\ldots,g_m$\} билинейные формы, определяющие произведения на $E$ и $F$ (аксиома алгебры, касающаяся произведения в алгебре, утверждает, что это билинейные отображения). Пусть $a_1,b_1,\ldots,a_r,b_r$ и $c_1,d_1,\ldots,c_s,d_s$ --- линейные формы, выражающие тензорные ранги $E$ и $F$:
  \begin{center}
$\{f_1,\ldots,f_n\} \subset$ Vect$(a_1 \otimes b_1,\ldots,a_r \otimes b_r)$\par
    $\{g_1,\ldots,g_m\} \subset$ Vect$(c_1 \otimes d_1,\ldots,c_s \otimes d_s).$
\end{center}
    \noindent
    Билинейные формы, связанные с произведением алгебры $E \oplus F$, являются расширениями на эту последнюю алгебру $f'_i$ и $g'_j$ форм $f_i$ и $g_j$ ($f'_i(x,y) = f_i(x,y)$, если $x$ и $y$ принадлежит $E$, в противном случае $f'_i(x,y) = 0$, то же самое для $g'_j$). Аналогичные расширения линейных форм $a_i$ и $b_i$, $a'_i$ и $b'_i$ позволяют утверждать, что
    \begin{center}
      $\{f'_i,g'_j \mid 0 \leqslant i \leqslant n, 0 \leqslant j \leqslant m\} \subset $ Vect$\{a'_i \otimes b'_i, c'_j \otimes d'_j \mid 0 \leqslant i \leqslant r, 0 \leqslant j \leqslant s\},$
   \end{center}
        \noindent
что и доказывает, что $\text{rang}_{\otimes}(E \oplus F) \leqslant r + s$.
\end{myproof}
\begin{sled}
  Пусть $P = P_1P_2 \ldots P_q$ --- разложение многочлена степени $n$ в произведение \textbf{\textup{взаимно простых}} многочленов. Если основное кольцо $K$ имеет мощность не меньше $2(n - q)$, то тензорный ранг алгебры $K[T]/P$ не превосходит $2n - q$.
  \end{sled}
\begin{myproof}
Китайская теорема об остатках дает изоморфизм алгебр:\par
$$K[T]/P \simeq K[T]/P_1 \times K[T]/P_2 \times \ldots \times K[T]/P_q,$$\noindent
и, применяя предыдущее предложение при $ \deg (P_i) \leqslant n - (q - 1)$, получим требуемое.
\end{myproof}
\begin{sled}
  Пусть $\theta (n)$ --- число делителей $n$ (включая 1 и $n$). Если $K$ --- тело характеристики $p$, не делящей $n$, имеющее более $2(n - \theta (n))$ элементов, то тензорный ранг алгебры циклической свертки на $n$ точках, $K[T]/(T^n - 1)$, не превосходит $2n - \theta (n)$.
    \end{sled}
\begin{myproof}
  Рассматриваемая алгебра есть $K[\mathbb{Z}_n] \simeq K[T]/(T^n - 1)$. Многочлен $T^n - 1$ представляется в виде $T^n - 1 = \prod_{d \mid n} \Phi_d(T)$, где $\Phi_d(T)$ есть $d$-й цикломатический многочлен. Если $d$ и $d'$ --- два различных делителя $n$, то многочлены $\Phi_d(T)$ и $\Phi_{d'}(T)$ взаимно просты. Действительно, $\Phi_d(T)\Phi_{d'}(T)$ делит $T^n - 1$, который не имеет сомножителей, являющихся полными квадратами, --- у него нет общих делителей с его производной, так как $n \not\equiv 0\ ($mod $p$).
  \end{myproof}
В действительности Виноград вычислил точно значение тензорного ранга алгебры $K[T]/P$. Его доказательство можно найти в статье \textit{<<Some bilinear forms whose multiplicative complexity depends on the field of constants>>} или в книге Кнута \textit{<<Получисленные алгоритмы>>} [99], либо у Ауслендера и Толимиери [12].\par
\begin{thm}[Виноград]
  Пусть $P$ --- многочлен степени $n$, представимый в виде произведения неприводимых многочленов: $P = P^{\alpha_1}_1P^{\alpha_2}_2 \ldots P^{\alpha_q}_q$. Если основное тело бесконечно, то тензорный ранг алгебры $K[T]/P$ равен в точности $2n - q$.
  \end{thm}
Как было сказано в начале этого раздела, алгоритмы, оптимальные с точки зрения тензорного ранга, обладают, однако, определенной сложностью, ввиду того, что некоторые операции игнорировались, в частности, умножения на скаляр и сложения-вычитания (см. упражнения в конце главы и следующий раздел). Но Виноград имеет еще кое-что в запасе! Симметризировав некоторые ситуации, он получает из теоретически эффективных алгоритмов алгоритмы, эффективные с практической точки зрения. Позднее мы еще вернемся к этому.
\subsection{Пример: циклическая свертка порядка 4}
Согласно предыдущим результатам, умножение по модулю $T^n - 1$ может быть реализовано с помощью $2n - \theta (n)$ \textbf{порождающих} умножений, где $\theta (n)$ --- число делителей $n$. Проиллюстрируем этот факт для случая $n = 4$. Рассмотрим два многочлена $X(T) = x_0 + x_1T + x_2T^2 + x_3T^3$ и $Y(T) = y_0 + y_1T + y_2T^2 +\linebreak+y_3T^3$. Нужно вычислить, используя только $2 \times 4 - \theta (4) = 5$ умножений, билинейные формы $z_k$, определенные соотношением $z_0 + z_1T + z_2T^2 + z_3T^3 = X(T) \times\linebreak \times Y(T)$ (mod $T^4 - 1$). В матричном виде (в котором обычно описывается D{\footnotesize FT}) это тождество можно записать:\par
$$
\begin{pmatrix} z_0 \\ z_1 \\ z_2 \\ z_3 \end{pmatrix} = \begin{pmatrix} x_0y_0 + x_1y_3 + x_2y_2 + x_3y_1 \\ x_0y_1 + x_1y_0 + x_2y_3 + x_3y_2 \\ x_0y_2 + x_1y_1 + x_2y_0 + x_3y_3 \\ x_0y_3 + x_1y_2 + x_2y_1 + x_3y_0 \end{pmatrix} = \begin{pmatrix} x_0 & x_1 & x_2 & x_3 \\ x_1 & x_2 & x_3 & x_0 \\ x_2 & x_3 & x_0 & x_1 \\ x_3 & x_0 & x_1 & x_2 \end{pmatrix} \begin{pmatrix} y_0 \\ y_3 \\ y_2 \\ y_1 \end{pmatrix}
$$
\noindent
(Заметим, что прямое вычисление, использующее эти формулы, требует 16 порождающих умножений и 12 сложений). Предположим, что элемент 2 обратим в базовом кольце $K$. Применим китайскую теорему об остатках к многочлену $T^4 - 1 = (T - 1)(T + 1)(T^2 + 1)$, что дает нам изоморфизм колец $K[T]/(T - 1) \times K[T]/(T + 1) \times K[T]/(T^2 + 1)$ на $K[T]/(T^4 - 1)$.\par 
Вычислим сначала $X(T)Y(T)$ по модулю $T - 1$. Вычисление значений в точке $1$ вводит элементарную билинейную форму $m_1 = (x_0 + x_1 + x_2 + x_3)(y_0 + $ $+y_1 + y_2 + y_3)$. Затем вычисляем $X(T)Y(T)$ по модулю $T + 1$. Вычисление значения в точке $-1$ дает элементарную билинейную форму $m_2 = (x_0 - x_1 + x_2 - x_3)$ $(y_0 - y_1 + y_2 - y_3)$. Имея эти два результата, применим китайскую теорему об остатках, которая требует соотношения Безу для двух взаимно простых многочленов $P$ и $Q$: $UP + VQ = 1$, и дает
\begin{center}
  $Z \equiv UP \times (Z$ mod $Q) + VQ \times (Z$ mod $P)\ ($mod $PQ).$
  \end{center}\noindent
В интересующем нас случае ($P = T - 1$ и $Q = T + 1$), используя $\frac{1}{2}(T + 1) + $ $+\frac{-1}{2}(T - 1) = 1$, получим:\par
$$
\begin{array}{rl}
 \\
X(T)Y(T)\ \textup{mod} \ (T^2 - 1) & = \frac{T + 1}{2}m_1 - \frac{T - 1}{2}m_2 =
\\
& = \frac{m_1 + m_2 + T(m_1 - m_2)}{2}.
\\
\end{array}
$$
\noindent
Осталось вычислить значение $X(T)Y(T)$ по модулю $T^2 + 1$. Используем равенство $T^2 = -1$, что дает $X(T) = x_0 - x_2 + (x_1 - x_3)T$ (mod $T^2 + 1$) и $Y(T) = y_0 - y_2 + $ $+(y_1 - y_3)T$ (mod $T^2 + 1$). Осталось перемножить эти два многочлена при помощи трех произведений: $m_3 = (x_0 - x_2)(y_0 - y_2)$, $m_4 = (x_1 - x_3)(y_1 - y_3)$ и $m_5 = (x_0 - x_1 - x_2 + x_3)(y_0 - y_1 - y_2 + y_3)$, что дает нам (с учетом того, что $T^2 = -1$): $m_3 - m_4 + (m_3 + m_4 - m_5)T$. Китайская теорема об остатках, примененная к $T^4 - 1 = (T^2 + 1)(T^2 - 1)$, и соотношение Безу $\frac{1}{2}(T^2+1)+ \frac{-1}{2}(T^2-1)=1$ дают:
\begin{center}
\begin{multline*}
  X(T)Y(T) \mod (T^4-1)= \frac{T^2+1}{2}\frac{m_1+m_2+T(m_1-m_2)}{2}-\\ \frac{T^2-1}{2}(m_3-m_4+(m_3+m_4-m_5)T),
\end{multline*}
\end{center}
т.~е.
\begin{center}
\begin{multline*}
  \frac{m_1+m_2+2m_3-2m_4}{4}+ \frac{m_1-m_2+2m_3+2m_4-2m_5}{4}T+ \\ \frac{m_1+m_2-2m_3+2m_4}{4}T^2+ \frac{m_1-m_2-2m_3-2m_4+2m_5}{4}T^3.
\end{multline*}
  \end{center}\par
Короче говоря, найдены 5 явно выраженных элементарных произведений $m_i$, т.~е. 5 пар ($a_i, b_i$), где $a_i,b_i$ --- линейные формы с $m_i=a_i \otimes b_i$, с помощью которых можно выразить 4 билинейные формы $z_0,z_1,z_2,z_3$. Это можно сделать с помощью трех матриц:
\begin{center}
  \begin{equation} \label{n9}
    \begin{split}
      A=\begin{pmatrix} 1 & 1 & 1 & 1 \\ 1 & -1 & 1 & -1 \\ 1 & 0 & -1 & 0 \\ 0 & 1 & 0 & -1 \\ 1& -1 & -1 & 1 \end{pmatrix}&,\ \ \ B=\begin{pmatrix} 1 & 1 & 1 & 1 \\ 1 & -1 & 1 & -1 \\ 1 & 0 & -1 & 0 \\ 0 & 1 & 0 & -1 \\ 1& -1 & -1 & 1 \end{pmatrix}, \\ C=\frac{1}{4} &\begin{pmatrix} 1 & 1 & 1 & 1 \\ 1 & -1 & 1 & -1 \\ 2 & 2 & -2 & -2 \\ -2 & 2 & 2 & -2 \\ 0 & -2 & 0 & 2 \end{pmatrix}
    \end{split}
\end{equation}
\end{center}
\noindent
при $(z_i(x,y))= ^t C \cdot (a_i(x)b_i(y))$.\par
\subsection{Семейство билинейных форм и трилинейные формы}
В этом разделе мы рассмотрим связи между семейством билинейных форм и специальной трилинейной формой. Пусть $E$, $F$ и $G$ --- свободные модули размерностей $n$, $m$ и $s$ соответственно. Задание трилинейной формы $h$ на $E \times F \times G$ в базовом кольце $K$ эквивалентно, при фиксированных базисах, заданию кубической матрицы $(h_{ijk})$ размера $n \times m \times s$, причем соответствие выражается соотношением $h(x,y,z)= \sum h_{ijk}x_iy_jz_k$.
\begin{predl}
  Пусть $K$ — кольцо. Существует взаимно однозначное соответствие, ставящее в соответствие произвольному семейству $s$ билинейных форм, определенных на одном и том же свободном модуле $E \times F$, трилинейную форму, определенную на $E \times F \times K^s$. Если \{$h_1,h_2,\ldots,h_s$\} --- указанное семейство билинейных форм и $h$ --- ассоциативная трилинейная форма, то соответствие задается соотношением
  \begin{center}
    $h(x,y,z)= \sum_{1 \leqslant k \leqslant s}h_k(x,y,z_k),$ одной стороны,\par
    и\par
    $h_k(x,y)=h(x,y,e_k),$ с другой стороны,
  \end{center}
  \noindent
  где ($e_1,е_2,\ldots,e_s$) --- каноническая базис в $K^s$.
\end{predl}
Это утверждение, кажущееся достаточно банальным, позволяет симметризовать ситуацию: в трилинейной форме $h(x,y,z)$ переменные $x,y,z$ имеют один и тот же статус, и, благодаря этому соответствию, здесь индекс $k$ форм $h_k$ может <<проектироваться>> в $K^s$. Например, семейство $s$ билинейных форм на $K^n \times K^m$ позволяет с помощью трилинейной формы из предложения 46 определить семейство $m$ билинейных форм на $K^s \times K^n$, $(z,x) \mapsto h(x,e_j,z)$, или же $n$ билинейных форм на $K^m \times K^s,\ (y,z) \mapsto h(e_i,y,z)$.
  \begin{determ}
($i$) Аналогично тому, как был определен тензорный ранг билинейной формы, можно определить его для трилинейной формы. Если $h$ --- трилинейная форма, то ее \textbf{тензорный ранг} есть наименьшее целое $r$, такое, что $h$ записывается в виде суммы элементарных трилинейных форм, т.~е. $h = a_1 \otimes b_1 \otimes c_1 + \cdots + a_r \otimes b_r \otimes c_r$, где $a_i,\ b_i\ и\ c_i$ --- линейные формы. Разумеется, такое равенство всегда возможно, так как элементарные трилинейные формы являются системой образующих  пространства трилинейных форм.\par
    ($ii$) Говорят, что линейные отображения $a$, $b$ и $c$, определенные линейными формами $a_i$, $b_i$ и $c_i$, $a = (a_i,\dots,a_r)$, $b = (b_i,\dots,b_r)$ и $c= (c_i,\dots,c_r)$, или же их матрицами $A$, $B$ и $C$ (строчки которых являются матрицами соответствующих линейных форм), образуют \textbf{реализацию} трилинейной формы $h$.\par
Тот факт, что тройка линейных отображений $a$, $b$ и $c$ является реализацией трилинейной формы $h$, можно записать в виде\par
  $$h(x,y,z)= \langle a(x), b(y), c(z) \rangle ,$$ где $\langle$.,.,.$\rangle$ --- стандартное трилинейное произведение на $K^r \times K^r \times K^r$.
  \end{determ}
  \begin{predl}
Пусть $h$ --- трилинейная форма типа $n \times m \times s$ и \{$h_1,h_2,\dots,h_s$\} --- семейство билинейных форм типа $n \times m$, ассоциированных с $h$. Тогда эквивалентны следующие утверждения:\par
($i$) $h = a_1 \otimes b_1 \otimes c_1 + \cdots + a_r \otimes b_r \otimes c_r$, что выражает тот факт, что тензорный ранг $h$ не превосходит $r$,\par 
($ii$) $(h_1,\dots,h_s) = (a_1 \otimes b_1,\dots,a_r \otimes b_r)C$ --- выражение, в которое входят лишь строки (а не столбцы, как обычно), что означает, что в реализации матрица $C$ действует в транспонированном виде.\par 
Из этих двух свойств следует, что тензорный ранг $h$ равен тензорному рангу семейства \{$h_1,h_2,\dots,h_s$\}, с аналогичным результатом при замене ($n, m, s$) на ($s, n, m$) или ($m, s, n$).
\end{predl}
\begin{myproof}
  Если $h = a_1 \otimes b_1 \otimes c_1 + \cdots + a_r \otimes b_r \otimes c_r$, то при $1 \leqslant k \leqslant s$ имеем $h_k = c_{1k} \cdot a_1 \otimes b_1 + c_{2k} \cdot a_2 \otimes b_2 + \cdots + c_{rk} \cdot a_r \otimes b_r$, где $c_{lk}$ --- элементы матрицы $C$, что доказывает ($ii$). Для вывода обратной импликации достаточно переписать предыдущее доказательство в обратном порядке.
  \end{myproof}
Возвращаясь к исходной проблеме, т.~е. к представлению \{$h_1,\dots,h_s$\} в виде линейной комбинации элементарных билинейных форм $a_i \otimes b_i$, имеем равенство $(h_k(x,y)) =^tC(a_i(x)b_i(y))$, в которое матрица $C$, соответствующая линейному отображению $c$, входит в \textit{транспонированном} виде. Последнее равенство образует эффективную реализацию включения $s$ линейных форм $h_k$ в векторное пространство, порожденное $r$ \textit{элементарными} билинейными формами $\{h_1,\dots,h_s\} \subset$ Vect $(a_1 \otimes b_1,\dots,a_r \otimes b_r).$ Формы $c_i$ дают коэффициенты комбинаций.
\subsection{Пример: циклическая свертка порядка 4 (продолжение)}
В разделе 6.3 показано, что циклическая свертка $x \star _4 y$ порядка 4 может быть реализована с помощью 5 порождающих умножений. Каждое из этих 5 умножений получается как тензорный квадрат одной из 5 линейных форм, приведенных ниже в таблице справа (матрицы $A$ и $B$ формулы (\ref{n9}) равны). Тогда можно определить 4 билинейные формы, ассоциированные с 3-тензором $h$, используя матрицу $C$ из формулы (\ref{n9}).

\begin{wraptable}{i}{0.5\textwidth}
  \begin{tabular}{ccccccccc}
    $a_1(x)$ & = & $x_0$ & + & $x_1$ & + & $x_2$ & + & $x_3,$ \\
    $a_2(x)$ & = & $x_0$ & $-$ & $x_1$ & + & $x_2$ & $-$ & $x_3,$ \\
   $a_3(x)$ & = & $x_0$ &  &  & $-$ & $x_2,$ &  & \\
    $a_4(x)$ & = &  &  & $x_1$ & & & $-$ & $x_3,$ \\
    $a_5(x)$ & = & $x_0$ & $-$ & $x_1$ & $-$ & $x_2$ & + & $x_3,$ \\
    \end{tabular}
  \end{wraptable}
Это дает нам следующее выражение, в которое входят только 5 порождающих умножений (при этом не учитываем ни умножения, ни деления на скаляры). Но зато, число сложений существенно возрастает, и метод поначалу может оказаться не очень продуктивным.
\begin{center}
  \begin{equation} \label{n10}
    \begin{split}
&4h_0=a_1 \otimes b_1 + a_2 \otimes b_2 + 2a_3 \otimes b_3 - 2a_4 \otimes b_4, \\
&4h_1 = a_1 \otimes b_1 - a_2 \otimes b_2 + 2a_3 \otimes b_3 + 2a_4 \otimes b_4 - 2a_5 \otimes b_5, \\
&4h_2 = a_1 \otimes b_1 + a_2 \otimes b_2 - 2a_3 \otimes b_3 + 2a_4 \otimes b_4, \\
&4h_3 = a_1 \otimes b_1 - a_2 \otimes b_2 - 2a_3 \otimes b_3 - 2a_4 \otimes b_4 + 2a_5 \otimes b_5. \\
    \end{split}
  \end{equation}
  \end{center}Можно также заметить, что в эту формулу матрица $C$ входит в транспонированом виде, как и было уточнено ранее: строчки $C$ читаются сверху вниз в базисе системы ($a_i \otimes b_i$).\par
Обозначим через $h$ трилинейную форму, ассоциированную с семейством четырех билинейных форм \{$h_0,h_1,h_2,h_3$\}. Как было показано в 46, 3-линейная форма $h$ может выражаться равенством $h(x,y,z) = \sum h_k(x,y)z_k$. Здесь можно переписать $h(x,y,z) = (a_1 \otimes b_1 + a_2 \otimes b_2 + \cdots)z_0 + (a_1 \otimes b_1 — a_2 \otimes b_2 + \cdots)z_1 + \cdots$, что приводит нас к введению линейных форм:\par
$$c_1(z)= \frac{z_0+z_1+z_2+z_3}{4},\ \ \ c_2(z)= \frac{z_0-z_1+z_2-z_3}{4},$$
$$c_3(z)= \frac{2z_0 + 2z_1-2z_2-2z_3}{4},\ \ \ c_4(z)= \frac{-2z_0+2z_1+2z_2-2z_3}{4},$$
$$c_5(z)= \frac{-2z_1 + 2z_3}{4}.$$ Теперь соотношение (\ref{n10}) переписывается в виде: $h=a_1 \otimes b_1 \otimes c_1 + \cdots + a_5 \otimes b_5 \otimes c_5$.\par
Далее предполагаем, что многочлен $X = х_0 + x_1T + x_2T^2 + x_3T^3$ фиксирован, и постараемся собрать все операции, отличные от 5 порождающих умножений, необходимых для вычисления С{\footnotesize С}$_4$.\par 
3-линейная форма $h$, ассоциированная с циклической сверткой порядка 4, такова, что $h_{i,j,k}=1$, если $i + j \equiv k$ (mod 4), и 0 в противном случае. Следовательно, она обладает свойством симметрии $h_{i,j,k}=h_{k,4-j,i}$, где все индексы берутся по модулю 4: $h_{i,0,k}=h_{k,0,i}$, $h_{i,1,k}=h_{k,3,i}$, $h_{i,2,k}=h_{k,2,i}$, $h_{i,3,k}=h_{k,1,i}$. Значит, матрицы $A' = C$, $B' = B$, получающиеся перестановкой второго и четвертого столбцов\footnote{Внимание: так как нумерация начинается с 0, первый столбец матрицы (в обычном смысле) имеет номер 0, второй --- номер 1 и т.~д.}, $C' = A$ образуют реализацию 3-линейной формы $h$.
\begin{center}
  \begin{equation} \label{n11}
    \begin{split}
  A'= \frac{1}{4} \begin{pmatrix} 1 & 1 & 1 & 1 \\ 1 & -1 & 1 & -1 \\ 2 & 2 & -2 & -2 \\ -2 & 2 & 2 & -2 \\ 0 & -2 & 0 & 2 \end{pmatrix}&,\ \ \ B'=\begin{pmatrix} 1 & 1 & 1 & 1 \\ 1 & -1 & 1 & -1 \\ 1 & 0 & -1 & 0 \\ 0 & -1 & 0 & 1 \\ 1& 1 & -1 & -1 \end{pmatrix},\\
  C'= &\begin{pmatrix} 1 & 1 & 1 & 1 \\ 1 & -1 & 1 & -1 \\ 1 & 0 & -1 & 0 \\ 0 & 1 & 0 & -1 \\ 1 & -1 & -1 & 1 \end{pmatrix}
  \end{split}
  \end{equation}
  \end{center}\par
Здесь наиболее сложные вычисления (для которых коэффициенты отличны от 1 или $—1$) приходятся на первую матрицу $A'$, ассоциированную с линейными формами $x_i$, что позволяет заранее вычислить $A'(x)$ (так как $x_i$ фиксировано). Тогда циклическая свертка вычисляется с помощью нахождения значений 5 линейных форм относительно $y_i$ (с помощью сложений и вычитаний, как показывают строки матрицы $B'$),\begin{wrapfigure}{i}{0.45\textwidth}
$\begin{pmatrix} s_3 \\ s_4 \\ s_5 \\ s_6 \\ s_7 \end{pmatrix} = \begin{pmatrix} 1 & 1 & 1 & 1 \\ 1 & -1 & 1 & -1 \\ 1 & 0 & -1 & 0 \\ 0 & -1 & 0 & 1 \\ 1 & 1 & -1 & -1 \end{pmatrix} \begin{pmatrix} y_0 \\ y_1 \\ y_2 \\ y_3 \end{pmatrix}$
\end{wrapfigure} таким образом, чтобы сократить число операций насколько
возможно. В рассматриваемом примере, можно оптимизировать произведение, стоящее справа, вычислив сперва $y_0 + y_2$ и $y_1+y_3$, что позволит нам вычислить первое матричное произведение, используя лишь 7 сложений. Если же, выбрав другой путь <<оптимизации>>, рассматривать суммы $y_0+y_1$ и $y_2+y_3$, встречающиеся по два раза, то было бы невозможно осуществить это матричное произведение менее, чем за 8 сложений!\par
Затем осуществляем пять умножений этих форм на заранее вычисленные константы $\alpha_i$, полученные из строк $A'$:\par 
$$\begin{pmatrix} m_1 \\ m_2 \\ m_3 \\ m_4 \\ m_5 \end{pmatrix} \longleftarrow \begin{pmatrix} \alpha_1 * s_3 \\ \alpha_2 * s_4 \\ \alpha_3 * s_5 \\ \alpha_4 * s_6 \\ \alpha_5 * s_7 \end{pmatrix}, \begin{pmatrix} \alpha_1 \\ \alpha_2 \\ \alpha_3 \\ \alpha_4 \\ \alpha_5 \end{pmatrix} = \frac{1}{4} \begin{pmatrix} 1 & 1 & 1 & 1 \\ 1 & -1 & 1 & -1 \\ 2 & 2 & -2 & -2 \\ -2 & 2 & 2 & -2 \\ 0 & -2 & 0 & 2 \end{pmatrix} \begin{pmatrix} x_0 \\ x_1 \\ x_2 \\ x_3 \end{pmatrix}.$$ Наконец, произведем на этих результатах сложения (или вычитания), указанные \textit{столбцами} матрицы $C'$, минимизируя число операций.\par
\begin{table}[h!]
\begin{center}
  \begin{tabular}{|llll|}
  \hline
  $s_1 \leftarrow y_0 + y_2,$ & $s_2 \leftarrow y_1 + y_3,$ & $s_3 \leftarrow s_1 + s_2,$ & $s_4 \leftarrow s_1 - s_2,$
  \\
  $s_5 \leftarrow y_0 - y_2,$ & $s_6 \leftarrow y_3 - y_1,$ & $s_7 \leftarrow s_5-s_6,$ &
  \\
  \multicolumn{4}{|l|}{
    $m_1 \leftarrow \frac{x_0+x_1+x_2+x_3}{4} * s_3,\ \ \ m_2 \leftarrow \frac{x_0-x_1+x_2-x_3}{4} * s_4,\ \ \ m_3 \leftarrow \frac{x_0+x_1-x_2-x_3}{2} * s_5,$}
  \\
    \multicolumn{4}{|l|}{
$m_4 \leftarrow \frac{-x_0+x_1+x_2-x_3}{2} * s_6,\ \ \ m_5 \leftarrow \frac{-x_1+x_3}{2} * s_7,$}
  \\
  $s_8 \leftarrow m_1 + m_2,$ & $s_9 \leftarrow m_3 + m_5,$ & $s_{10} \leftarrow m_1 - m_2,$ & $s_{11} \leftarrow m_4 - m_5,$
  \\
  $s_{12} \leftarrow s_8 + s_9,$ & $s_{13} \leftarrow s_{10} + s_{11},$ & $s_{14} \leftarrow s_8-s_9,$ & $s_{15} \leftarrow s_{10} - s_{11},$
  \\
  & & & 
  \\
  $z_0=s_{12},$ & $z_1=s_{13},$ & $z_2=s_{14}$ & $z_3=s_{15}$
  \\
  \hline
  \end{tabular}
  \caption{Циклическая свертка $z=x \star_4 y$ на 4 точках}
\end{center}
\end{table}
\par
Схема вычисления C{\footnotesize C}$_4$ в таблице 2 использует 5 умножений и 15 сложений/вычитаний (вместо 16 умножений и 12 сложений). В этой схеме сложения нумеруются $s_1,\ s_2,\dots$, а умножения $m_1,\ m_2,\dots$
\section{Малые схемы для дискретного преобразования Фурье}
Методы Кули -- Тьюки и Гуда существенным образом используют тот факт, что показатель $n$, для которого вычисляется D{\footnotesize FT}$_n$, является \textit{составным}. Напротив метод Рейдера, опубликованный в его статье: \textit{<<Discrete Fourier transform when the number of data samples is prime>>}, применяется к преобразованию Фурье простого порядка $p$, сводя его к вычислению свертки порядка $p - 1$. Реализуя эти циклические свертки, получаем, благодаря методу Винограда, эффективные схемы для D{\footnotesize FT}$_p$, которые можно потом внедрить в методы Кули -- Тьюки или Гуда. Тогда для данных порядков это дает схемы вычисления дискретных преобразований Фурье с удивительно малым количеством операций.
\subsection{DFT порядка $p$ и СС порядка $p - 1$ (метод Рейдера)}
Прежде, чем изложить метод Винограда, мы покажем, как можно свести вычисление дискретного преобразования Фурье, простого порядка $p$ к циклической свертке порядка $p - 1$.\par
\begin{wrapfigure}{i}{0.35\textwidth}
$\begin{pmatrix} \hat{a}_0 \\ \hat{a}_1 \\ \hat{a}_2 \end{pmatrix} = \begin{pmatrix} 1 & 1 & 1 \\ 1 & w & w^2 \\ 1 & w^2 & w \end{pmatrix} \begin{pmatrix} a_0 \\ a_1 \\ a_2 \end{pmatrix}$
  \end{wrapfigure}
Используя исследование билинейных форм, проведенное в предыдущем разделе, мы сможем эффективно вычислить эту циклическую свертку, и, значит, эффективно вычислить преобразование Фурье. Но сначала рассмотрим пример: каковы следствия из C{\footnotesize C}$_2$ для D{\footnotesize FT}$_3$? В преобразовании Фурье D{\footnotesize FT}$_{3,w}$, расположенном слева (где $w$ --- кубический корень из единицы), выделим $2 \times 2$-подматрицу-циркулянт в правом нижнем углу матрицы $V_w$. Эта подматрица-циркулянт является матрицей циклической свертки на 2 точках.\par 
Для вычисления (см. упражнение 22) циклической свертки $z = x \star_2y$ на 2 точках для фиксированного $x$ с помощью 2 умножений и 4 сложений используем схему 3, приведенную ниже.\par
\begin{table}[h!]
  \begin{center}
\begin{tabular}{|llll|}
  \hline
  $s_1 \leftarrow y_0+y_1,$ & $s_2 \leftarrow y_0-y_1,$ & $m_1 \leftarrow \frac{x_0+x_1}{2} * s_1,$ & $m_2 \leftarrow \frac{x_0-x_1}{2} * s_2$
  \\
  $s_3 \leftarrow m_1 + m_2,$ & $s_4 \leftarrow m_1-m_2,$ & $z_0=s_3,$ & $z_1=s_4$
  \\
  \hline
\end{tabular}
\caption{Циклическая свертка на 2 точках}
  \end{center}
  \end{table}
\par
Тогда получим (упражнение 24) схему вычисления (4) для $\hat{a} = $D{\footnotesize FT}$_{3,w}(a)$, использующую 2 умножения и 6 сложений (вместо 4 умножений и 6 сложений в обычном методе).\par 
Существуют другие схемы, позволяющие вычислять не только D{\footnotesize FT}$_{3,w}$, нo также и $\lambda$D{\footnotesize FT}$_3(w)$ для фиксированного $\lambda$. Такой тип схем изучался Виноградом, который получил метод более эффективный, чем метод Гуда. Теперь перейдем к теореме Рейдера, которая точно определяет, в каком смысле дискретное преобразование Фурье простого порядка может вычисляться через циклическую свертку меньшего порядка.\par
\begin{table}[h!]
  \begin{center}
\begin{tabular}{|lll|}
  \hline
  $s_1 \leftarrow a_1+a_2,$ & $s_2 \leftarrow a_1-a_2,$ & $s_3 \leftarrow a_0+s_1,$
  \\
  $m_1 \leftarrow \frac{w+w^2}{2} * s_1,$ & $m_2 \leftarrow \frac{w-w^2}{2} * s_2,$ & 
  \\
  $s_4 \leftarrow a_0+m_1,$ & $s_5 \leftarrow s_4+m_2,$ & $s_6 \leftarrow s_4-m_2,$
  \\
  $\hat{a}_0=s_3,$ & $\hat{a}_1=s_5,$ & $\hat{a}_2=s_6$
  \\
  \hline
\end{tabular}
\caption{Дискретное преобразование Фурье на 3 точках}
  \end{center}
\end{table}
\par
В дальнейшем $p$ --- простое число, $V$ --- матрица Вандермонда порядка $p$, ассоциированная с корнем $w$ $p$-й степени из единицы: $V = (w^{ij})$. Как и в определении 30 раздела 5.2, если $\sigma$ --- перестановка интервала [0,$p$[, то обозначение $P_{\sigma}$ используется для перестановочной матрицы, определенной своим действием на каноническом базисе: $P_{\sigma}(e_j)=e_{\sigma (j)}$ и $a_{\sigma}$ --- вектор: $a_{\sigma}=(a_{\sigma (0)},\dots,a_{\sigma (p-1)})=P_{\sigma^{-1}}(a)$.
\begin{thm}[Рейдер, 1968]
Любой образующий элемент $\gamma$ группы обратимых элементов по модулю $p$ определяет перестановку $\sigma$ интервала \textup{[}0,$p$\textup{[} по правилу: $\sigma (0)=0$ и $\sigma (i)= \gamma^{i-1}$ \textup{mod} $p$. При этих условиях\par
    (i) матрица $P_{\sigma^{-1}}VP_{\sigma}$ имеет подматрицу порядка $p-1$, содержащуюся в правом нижнем блоке, $V^{\sigma}$, которая является циркулянтом,\par
    ($ii$) вычисление D{\footnotesize FT}$_p$ сводится к вычислению C{\footnotesize C}$_{p-1}$ в следующем смысле: если числа $b_j$ определены матричным равенством $(b_{\sigma (j)})_{0 < j < p}=\linebreak=V^{\sigma}(a_{\sigma (i)})_{0 < i < p}$, то $\hat{a}=$D{\footnotesize FT}$_p(a)$ задается равенствами:\par
\begin{center}
  $\hat{a_0}=a_0 + \cdots + a_{p-1}$ и для $0 < i < p : \hat{a_i}=a_0 + b_i$.
\end{center}
\end{thm}
\begin{mynotice}
Перестановка $\sigma$, которую можно назвать перестановкой Рейдера, ассоциированная с $\gamma$, фиксирует 0 и 1.\par
Матрица-циркулянт --- это матрица, в которой элемент с индексами $i$ и $j$ зависит только от $i + j$ mod $n$ (имеется другое определение, где вместо $i + j$ mod $n$ рассматривается $i — j$ mod $n$).
\end{mynotice}
\par 
\begin{myproof}
По результатам о перестановочных матрицах, установленным при исследовании метода Гуда (см. раздел 5.2), ясно, что элемент с индексом ($i,j$) матрицы $VP_{\sigma}$ есть $w^{i \sigma (j)}$ и что у матрицы $P_{\sigma^{-1}}VP_{\sigma}$ этот элемент есть $w^{\sigma (i) \sigma (j)}$. По определению $\sigma$ имеем для строго положительных $i$ и $j$ (т.е. рассматривая только подматрицу $V^{\sigma}$):\par 
$$(P_{\sigma^{-1}}VP_{\sigma})_{ij}=w^{g^{i+j-2}},$$ и по предыдущему замечанию получаем, что $V^{\sigma}$ --- циркулянт.
\end{myproof}
\begin{beznomera}
  \textbf{Пример}
  \end{beznomera}
Например, в $\mathbb{F}_7$ элемент 3 является образующим группы обратимых элементов, что дает перестановку $\sigma$ = (0,1,3,2,6,4,5) и приводит к следующей записи, в которой ясно видна свертка порядка 6:\par
$$ \begin{pmatrix}
  \hat{a}_0 \\ \hat{a}_1 \\ \hat{a}_2 \\ \hat{a}_3 \\ \hat{a}_4 \\ \hat{a}_5 \end{pmatrix} = \begin{pmatrix} 1 & 1 & 1 & 1 & 1 & 1 & 1 \\ 1 & w^1 & w^3 & w^2 & w^6 & w^4 & w^5 \\ 1 & w^3 & w^2 & w^6 & w^4 & w^5 & w^1 \\ 1 & w^2 & w^6 & w^4 & w^5 & w^1 & w^3 \\ 1 & w^6 & w^4 & w^5 & w^1 & w^3 & w^2 \\ 1 & w^4 & w^5 & w^1 & w^3 & w^2 & w^6 \\ 1 & w^5 & w^1 & w^3 & w^2 & w^6 & w^4 \end{pmatrix} \begin{pmatrix} a_0 \\ a_1 \\ a_2 \\ a_3 \\ a_4 \\ a_5 \end{pmatrix}$$
\subsection{Комбинация схем Рейдера и Гуда}
Виноград в его замечательной статье: \textit{<<On computing the Discrette\linebreak Fourier Transform>>} удивил <<общество FFT>>, анонсировав метод, использующий 20\% от числа умножений по методу Кули -- Тьюки и близкое к тому число сложений. Его метод основывается на внимательном анализе разложения Гуда и на суровой экономии при вычислении дискретных преобразований Фурье на небольшом числе точек (работа, которая теперь стала возможной благодаря исследованию билинейных форм и циклических сверток). Мы не будем обсуждать здесь принцип, разработанный Виноградом во всей его общности. Наоборот проиллюстрируем его на нашем излюбленном примере: вычислении D{\footnotesize FT}$_{15}$. Векторизация метода Гуда по Винограду исследуется в упражнении 37. 
\subsubsection{Подробное изучение D{\footnotesize FT}$_3$}
Рассмотрим сначала $\xi$ --- корень кубический из единицы. Мы используем $\xi$ обычным образом для вычисления преобразования Фурье порядка 3.
\begin{wrapfigure}{i}{0.34\textwidth}
$\begin{pmatrix} \hat{X}_0 \\ \hat{X}_1 \\ \hat{X}_2 \end{pmatrix} = \begin{pmatrix} X_0+X_1+X_2 \\ X_0+X_1 \xi +X_2 \xi^2 \\ X_0+X_1 \xi^2 +X_2 \xi \end{pmatrix}$
  \end{wrapfigure}
Но вполне возможно обобщить проделанное вычисление, чтобы получить вычисление, которое приведено справа, в котором $X_0$, $X_1$ и $X_2$ --- элементы произвольного векторного $K$-пространства $F$. Обозначим это вычисление D{\footnotesize FT}$_{F, \xi}$. Для получения метода вычисления $\hat{X}_i$ достаточно возвратиться к схеме 4, помещенной перед теоремой Рейдера.\par
  \begin{center}
\begin{tabular}{|lll|}
  \hline
  $S_1 \leftarrow X_1+X_2,$ & $S_2 \leftarrow X_1-X_2,$ & $\hat{X}_0 \leftarrow X_0+S_1,$
  \\
  $M_1 \leftarrow \frac{w+w^2}{2} * S_1,$ & $M_2 \leftarrow \frac{w-w^2}{2} * S_2,$ & 
  \\
  $S_4 \leftarrow X_0+M_1,$ & $\hat{X}_1 \leftarrow S_4+M_2,$ & $\hat{X}_2 \leftarrow S_4-M_2,$
  \\
  \hline
\end{tabular}
  \end{center}
  \par
Она показывает, что можно вычислить векторную схему D{\footnotesize FT}$_{F, \xi}$, используя 2 умножения на скаляр и 6 сложений векторов, что приводит к операциям в базовом теле, так что D{\footnotesize FT}$_{F, \xi}$ вычисляется с использованием 6$q$ сложений и 2$q$ умножений в базовом теле, где $q$ --- размерность пространства $F$.\par
Теперь слегка усложним условия и поставим вопрос, как можно эффективно вычислить D{\footnotesize FT}$_{F, \xi}(BX_0,\ BX_1,\ BX_2)$, если $B$ --- линейное преобразование пространства $F$. В только что приведенной схеме нужно применить $B$ к 3 выражениям, либо к значениям $X_i$ в начале или же к значениям $\hat{X}_i$ в конце, что приведет к появлению некоторого числа дополнительных скалярных операций. Тогда уместен вопрос: можно ли воспользоваться этими матричными умножениями, чтобы исключить 2$q$ умножений, присутствующих в первом вычислении?\par
Имеется несколько возможных ответов: невозможно сэкономить умножения в схеме вычисления, рассматривая ее так, как она была записана, но зато, если после того, как мы получили формулы, дающие D{\footnotesize FT}, ее слегка модифицировать, можно добиться некоторой экономии. Преобразование, которым мы будем оперировать со схемой вычисления, будет оправдано.\par
В нижеследующей схеме значения $\hat{X}_1$ и $\hat{X}_2$ выражены через $\hat{X}_0$ посредством лишь модификации выражения для $M_1$.\par
\begin{center}
\begin{tabular}{|lll|}
  \hline
  $S_1 \leftarrow X_1+X_2,$ & $S_2 \leftarrow X_1-X_2,$ & $\hat{X}_0 \leftarrow X_0+S_1,$
  \\
  $M_1 \leftarrow ( \frac{w+w^2}{2} - 1) * S_1,$ & $M_2 \leftarrow \frac{w-w^2}{2} * S_2,$ & 
  \\
  $S_4 \leftarrow \hat{X}_0+M_1,$ & $\hat{X}_1 \leftarrow S_4+M_2,$ & $\hat{X}_2 \leftarrow S_4-M_2,$
  \\
  \hline
\end{tabular}
\end{center}
\par
Далее увидим, что эта модификация вычисления основана на матричном соотношении между D{\footnotesize FT}$_p$ и C{\footnotesize C}$_{p-1}$, отличающемся от того, который мы использовали до сих пор. Теперь можно перегруппировать операции преобразования $B$ и умножения на константы. Мы заменим $\hat{X}_0 \leftarrow X_0 + S_1$ на $\hat{X}_0 \leftarrow B(X_0 + S_1)$, $M_1 \leftarrow ( \frac{w+w^2}{2} - 1) * S_1$ на $M_1 \leftarrow ( \frac{w+w^2}{2} - 1) \cdot B(S_1)$ и $M_2 \leftarrow \frac{w-w^2}{2} * S_2$ на $M_2 \leftarrow \frac{w-w^2}{2} \cdot B(S_2)$. Таким образом, мы эффективно вычисляем\linebreak D{\footnotesize FT}$_{F, \xi}(BX_0,\ BX_1,\ BX_2)$ и, вычисляя заранее константные преобразования $( \frac{w+w^2}{2} - 1) \cdot B$ и $\frac{w-w^2}{2} \cdot B$, экономим 2$q$ умножений.\par
Сложность, которую получают для этого вычисления, определяется как функция сложности вычисления $B$ (произведения на матрицу), что можно обозначить через $\mathcal{M} (B)$ для умножений и $\mathcal{A} (B)$ для сложений. Тогда получаем, что для вычисления D{\footnotesize FT}$_{F, \xi}(BX_0,\ BX_1,\ BX_2)$ нужно:\par
\begin{center}
$\mathcal{M} (B) + 2 \mathcal{M} (\lambda B)$ умножений и $\mathcal{A} (B) + 2 \mathcal{A} (\lambda B) + 6q$ сложений.
\end{center}
\subsubsection{Снова D{\footnotesize FT}$_5$}
Теорема Гуда утверждает, что для вычисления преобразования Фурье на 15 точках достаточно вычислить несколько преобразований Фурье на 5 точках, а затем применить к полученному результату преобразование Фурье порядка 3. Тогда получится искомый результат с точностью до нескольких перестановок компонент. Итак, дискретное преобразование Фурье является линейным преобразованием. Значит можно отвести преобразованию Фурье порядка 5 роль преобразования $B$, введенного в предыдущем разделе так, что с точностью до перестановок вычисление\linebreak D{\footnotesize FT}$_{F, \xi}(BX_0,\ BX_1,\ BX_2)$, описываемое как операция над элементами тела (а больше уже не векторного пространства), будет преобразованием Фурье на 15 точках. Для успешного окончания исследования осталось определить схему, позволяющую вычислять преобразование Фурье на 5 точках, а также произведение преобразования Фурье на 5 точках на константу ($\lambda B$).\par 
Преобразование Фурье порядка 5 будет вычисляться, согласно теореме Рейдера, с помощью циклической свертки порядка 4. Следовательно,\begin{wrapfigure}{i}{0.47\textwidth}
$\begin{pmatrix} \hat{a}_0 \\ \hat{a}_1 \\ \hat{a}_2 \\ \hat{a}_4 \\ \hat{a}_3 \end{pmatrix} = \begin{pmatrix} 1 & 1 & 1 & 1 & 1 \\ 1 & w^1 & w^2 & w^4 & w^3 \\ 1 & w^2 & w^4 & w^3 & w^1 \\ 1 & w^4 & w^3 & w^1 & w^2 \\ 1 & w^3 & w^1 & w^2 & w^4 \end{pmatrix} \begin{pmatrix} a_0 \\ a_1 \\ a_2 \\ a_4 \\ a_3 \end{pmatrix}$
\end{wrapfigure} нам нужно сначала определить перестановку Рейдера интервала [0,4]. Для этого выберем  образующий группы обратимых элементов по модулю 5. Число 2 является таким образующим, который, в силу равенства $(g^0,g^1,g^2,g^3)=(1,2,3,4)$ индуцирует перестановку $\sigma = (0\ 1\ 2\ 4\ 3)$ и приводит к равенству слева.\par
Продолжение и окончание этого исследования посвящены установлению явных формул для вычисления $\lambda$D{\footnotesize FT}$_5$, с \textbf{максимальной} экономией умножений и сложений. Как можно будет убедиться, это исследование требует иногда таких усилий, которые кажутся несоизмеримыми с полученным выигрышем, но это так и есть, ведь это дискретное преобразование Фурье\dots \par
Разложение, позволяющее обычно переходить от D{\footnotesize FT}$_5$ к C{\footnotesize C}$_4$, существенно использует следующее матричное соотношение:\par
$$\begin{pmatrix} 1 & 1 & 1 & 1 & 1 \\ 1 & w^1 & w^2 & w^4 & w^3 \\ 1 & w^2 & w^4 & w^3 & w^1 \\ 1 & w^4 & w^3 & w^1 & w^2 \\ 1 & w^3 & w^1 & w^2 & w^4 \end{pmatrix} = \begin{pmatrix} 1 & 1 & 1 & 1 & 1 \\ 1 & 0 & 0 & 0 & 0 \\ 1 & 0 & 0 & 0 & 0 \\ 1 & 0 & 0 & 0 & 0 \\ 1 & 0 &0 & 0 & 0 \end{pmatrix} + \begin{pmatrix} 0 & 0 & 0 & 0 & 0 \\ 0 & w^1 & w^2 & w^4 & w^3 \\ 0 & w^2 & w^4 & w^3 & w^1 \\ 0 & w^4 & w^3 & w^1 & w^2 \\ 0 & w^3 & w^1 & w^2 & w^4 \end{pmatrix}.$$\noindent
В этом выражении умножение первой матрицы на $\lambda$ индуцирует умножение на $\lambda$ двух следующих матриц. Вторая матрица, фигурирующая в сумме, умноженная на константу, остается все еще матрицей циклической свертки, которую можно вычислить обычными методами. Что касается первой матрицы, если она умножается на константу, то в процессе своего применения она вводит два новых умножения, и это число умножений не может быть уменьшено. Но зато, если записать выражение:\par
\begin{multline*}
  \begin{pmatrix} 1 & 1 & 1 & 1 & 1 \\ 1 & w^1 & w^2 & w^4 & w^3 \\ 1 & w^2 & w^4 & w^3 & w^1 \\ 1 & w^4 & w^3 & w^1 & w^2 \\ 1 & w^3 & w^1 & w^2 & w^4 \end{pmatrix} = \begin{pmatrix} 1 & 1 & 1 & 1 & 1 \\ 1 & 1 & 1 & 1 & 1 \\ 1 & 1 & 1 & 1 & 1 \\ 1 & 1 & 1 & 1 & 1 \\ 1 & 1 & 1 & 1 & 1 \end{pmatrix} + \\
  + \begin{pmatrix} 0 & 0 & 0 & 0 & 0 \\ 0 & w^1-1 & w^2-1 & w^4-1 & w^3-1 \\ 0 & w^2-1 & w^4-1 & w^3-1 & w^1-1 \\ 0 & w^4-1 & w^3-1 & w^1-1 & w^2-1 \\ 0 & w^3-1 & w^1-1 & w^2-1 & w^4-1 \end{pmatrix}
  \end{multline*}\noindent
в котором последняя матрица суммы есть матрица циклической свертки, то умножение на $\lambda$ перестановочной матрицы преобразования с D{\footnotesize FT}$_5$, приводит к появлению лишь одного элементарного умножения на первой матрице суммы, все строки которой совпадают. Это тоже преобразование, адаптированное к матрицам порядка 3, которое нам позволило сразу же сэкономить 2$q$ умножений в вычислении D{\footnotesize FT}$_3$.\par 
\begin{wrapfigure}{i}{0.2\textwidth}
$\begin{pmatrix} \hat{a}_0 \\ \hat{a}_0 \\ \hat{a}_0 \\ \hat{a}_0 \\ \hat{a}_0 \end{pmatrix} + \begin{pmatrix} 0 \\ \hat{a}_1 - \hat{a}_0 \\ \hat{a}_2 - \hat{a}_0 \\ \hat{a}_4 - \hat{a}_0 \\ \hat{a}_3 - \hat{a}_0 \end{pmatrix}$
\end{wrapfigure}
Продолжим. Предыдущее матричное соотношение при вычислении D{\footnotesize FT}$_{15}$ приводит к вычислению суммы, приведенной справа. Первое слагаемое этой суммы соответствует применению матрицы из единиц к вектору ($a_0,a_1,a_2,a_4,a_3$), а второе слагаемое --- операции матрицы циклической свертки на том же векторе. Изучим более детально вычисление циклической свертки, чтобы понять, как можно экономно скомбинировать ее с этим разложением. Для этого рассмотрим реализацию ($A',B',C'$), заданную формулой (\ref{n11}) на стр.~646. Первая матрица, $A'$, позволяет
\end{document}

