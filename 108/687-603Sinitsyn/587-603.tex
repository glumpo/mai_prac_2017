\documentclass{mai_book}

\defaultfontfeatures{Mapping=tex-text}
\setdefaultlanguage{russian}
\begin{document}
\section{ Примитивные корни из единицы}
\textbf{(2) Определение.}

\textit{Элемент $\omega$ кольца $A$ называется \textbf{примитивным}\footnote{В русской учебной литературе (по алгебре) употребляется термин 
первообразный корень из единицы. Однако все руководства прикладного плана (теория 
кодирования, криптография и т.д.) используют понятие примитивного корня, что 
несколько короче и, пожалуй, удачнее. — Прим. перев.} \textbf{корнем n-й}
степени из единицы, если $\omega^n = 1$ и если для $i - 1,2,\ldots ,n - 1$ элемент $1 - \omega^i$ обратим. В частности, $\omega$ и имеет порядок $n$ в группе $U(A)$ обратимых элементов $А$.}

Установим теперь некоторые простые характеристические свойства
примитивных корней из единицы, возможные в различных частных 
случаях. Раздел заканчивается установлением критерия, позволяющего 
находить кольца, в которых существуют такие корни.

\textbf{(3) Лемма.}

\textit{Пусть $\omega$ - корень $n$-й степени из единицы, такой, что $1 - \omega^i$ не являются делителями нуля для $i = 1,2,\ldots п - 1$. 
Тогда $V_\omega \ast V_{\omega^{-1}} = n\cdot Id$;
где обозначение $Id$ используется для единичной матрицы размеров $n \times n $.}

\textbf{Доказательство.}
\newlength{\MYwidth} % новый параметр длины
\def\MYvrule#1\par{
\par\noindent
\MYwidth=\textwidth\addtolength{\MYwidth}{-7pt}
\hbox{\vrule width 1pt\hspace{5pt}\parbox[t]{\MYwidth}{#1}}
}
\MYvrule Обозначим через $s_{ij}$ элементы матрицы $V_\omega \ast V_{\omega^{-1}}.$ 
 
 По определению: $s_{ij}=\sum\limits_{l=0}^{n-1} \omega^{il} \omega^{-lj}=\sum\limits_{l=0}^{n-1}( {\omega^{i-j}})^l.$

 Классическое тождество $(1 - X)(\sum\limits_{l=0}^{n-1}X^l) = 1 - X^n$ дает при

 $X = \omega^{i-j}$ следующее соотношение $(1 - \omega^{i-j}) \ast s_{ij} = 1 - ({\omega^{i-j}})^n = 0$.

 Если $i\neq j$, то $1-\omega^{i-j}$ не является делителем нуля и, следовательно,

 $s_{ij} = 0$. В противном случае, при $i=j$ непосредственно имеем

 $s_{ij} = n$, что доказывает лемму.
\bigskip

\textbf{(4) Предложение.}

Пусть $\omega$ — корень из единицы порядка n в коммутативном кольце A.

(i) Следующие два свойства эквивалентны:

a) $\omega$ является примитивным корнем n-й степени из единицы;

b) для i = 1,2,..., n — 1 ни один из элементов $1 — {\omega}^i$ не является

делителем нуля в А, и n — обратим в кольце А.
\

(ii) Если А — область целостности, то $\omega$ — примитивный корень

n-й степени из единицы тогда и только тогда, когда n обратим в А.

(iii) Если А — поле, то $\omega$ — примитивный корень (следовательно, n

не делится на характеристику поля).\bigskip

\textbf{Доказательство.}

Пусть $\omega$ — корень n-й степени из единицы. Предположим, что $1 - {\omega}^i$

для $1 \leq i < n$ не является делителем нуля. Тогда по предыдущей

лемме $V_{\omega} \ast V_{\omega^{-1}} = n \cdot {Id}$, откуда получаем:\bigskip

$det(V_{\omega})\cdot det(V_{\omega^{-1}}) = n^n$ $\Rightarrow$ \bigskip

$\prod\limits_{i < j}({\omega}^i - {\omega}^j) \prod\limits_{i < j}({\omega}^{-i} - {\omega}^{-j}) = 
\pm \prod\limits_{i \neq j}({\omega}^i - {\omega}^j) = \pm n^n.$ \bigskip

Это доказывает, что n обратим тогда и только тогда, когда $1 - {\omega}^i$

обратим.

Два других пункта (ii), (iii) — непосредственные следствия (i).

Другое (прямое) доказательство пункта (iii) состоит в замечании,

что в поле А характеристики р ${x^{mp}} - 1 = {{(x^{m}} - 1)}^{р}$, и потому

$x^{mp}=1$ $\Rightarrow$ $x^{m}=1$. Следовательно, элементы поля 

характеристики p не могут иметь порядок mp.\bigskip

Существует характеризация примитивных корней через циклотомические многочлены. Напомним вкратце определение и некоторые 
основные свойства этих многочленов (которые понадобятся в дальнейших
исследованиях).\bigskip

\textbf{(5) Определение и элементарные свойства.}

(i) Для $n \in \mathbb{N}^*$ обозначим через $U_n \subset \mathbb{C}^* $ мультипликативную 
группу,
состоящую из корней n-й степени из единицы, а через $P_n \subset U_n $
подмножество в $U_n$, состоящее из корней из единицы порядка n.

(ii) \textbf{n-й циклотомический многочлен} — это многочлен, 
определяемый формулой: $\Phi_n(X) = \Pi_{\zeta \in P_n}(X - \zeta). $

(iii) Рекуррентное соотношение \ 

\vspace{0pt}\hspace{100pt} $X^n - 1 = \prod\limits_{d|n}\Phi_d(X) = \Phi_n(X) \prod\limits_{d|n, d \neq n} \Phi_d(X),$  \hspace{50pt}(2)
 
 с одной стороны, позволяет вычислять многочлен $\Phi_n(X)$, а с другой,
показывает, что $\Phi_n(X)$ является \textbf{унитарным многочленом с целыми коэффициентами}.\\
В частности, если \textbf{р — простое число}, то
$\Phi_p(X) = X^{p-1} + X^{p-2}+...+X^2 +X +1.$ 

Упражнение 10 в конце главы позволяет найти таблицу из 12 первых
циклотомических многочленов:
\begin{table}[h]
\begin{center}
\begin{tabular}{l l}
$\Phi_1(X)  = X-1$  & $\Phi_7(X)  = X^6+X^5+...+X+1$ \\
$\Phi_2(X)  = X+1$  &$\Phi_8(X)  = X^4+1$ \\
$\Phi_3(X)  = X^2+X+1$  &$\Phi_9(X)  = X^6+X^3+1$ \\
$\Phi_4(X)  = X^2+1$  &$\Phi_{10}(X)  = X^4-X^3+X^2-X+1$ \\
$\Phi_5(X)  = X^4+X^3+X^2+X+1$  &$\Phi_{11}(X)  = X^10+X^9+...+X+1$ \\
$\Phi_6(X)  = X^2-X+1$  &$\Phi_{12}(X)  = X^4-X^2+1$ 
\end{tabular}
\end{center}
\end{table}\\
\textbf{(6) Теорема.}\\
\textit{Для элемента $\omega$ коммутативного кольца А следующие условия 
эквивалентны:\\
(i) $\omega$ — примитивный корень n-й степени из единицы;\\
(ii) $\Phi_n(\omega) = 0$, и n обратим в кольце А.}\\
\textbf{Доказательство.}\\
• Докажем сперва импликацию (i) $\Rightarrow$ (ii). Для любого целого d 
многочлен $\Phi_d(X)$ - делитель в $\mathbb {Z}[X]$ для $X^d - 1$. $\Phi_d(\omega)$, следовательно,
делит $\omega^d - 1$, и, если d - собственный делитель $n$, то, поскольку
$\omega$ примитивен, $\omega^d - 1$ обратим в A, и тоже самое верно для $\Phi_d(\omega)$.
Формула (2), примененная к $X = \omega$, позволяет тогда заключить,
что $\Phi_n(\omega) = 0$. Тем самым доказано, что $n$ обратим.\\
• Докажем обратное. Равенство многочленов $X^n - 1 =
\Phi_n(X)Q(X)$ показывает, что $\omega^n = 1$. Дифференцируя его, находим
$n {X}^{n-1} = \Phi_{n}^{'}(X)Q(X) + \Phi_n(X)Q^{'}(X)$. Применяя это равенство к
$X = \omega$, получаем $n{\omega}^{n-1} = \Phi_{n}^{'}(\omega)Q(\omega)$, что доказывает 
обратимость $Q(\omega)$. Следовательно, так как $Q(X) = \Pi\Phi_{d}(X)$, где $d$ —
собственный делитель $n$, то $\Phi_{d}(\omega)$ — обратим. Кроме того, для
такого $d$ равенство $X^d - 1 =  \Pi_{e|d}\Phi_{e}(X)$ показывает, что $\omega^d - 1$
обратим. В завершение доказательства пусть $i$ удовлетворяет 
условию $1 \leq i < n$. Если $d = \text{HOД}(i,n)$, то существуют такие целые
$u$ и $v$ ,что $d = ui+vn$ причем $u \geq 0$ (достаточно заменить $u$ на
$u+kn$ и $v$ на $v -ki$). Многочлен $X^i -1$ является делителем (в ${\mathbb{Z}}[X]$)
многочлена $X^{ui} - 1$ и, следовательно, $\omega^i - 1$ делит $\omega^{ui} -1 = \omega^d - 1$,
что доказывает обратимость $\omega^i - 1$.\\
\textbf{\textit{Приложение : кольца, содержащие примитивные корни
фиксированного порядка}}\\
Пусть $q$ — целое четное число. По упражнению 10 $\Phi_{4q}(X) = \Phi_{q}(X^4)$
(так как $q$ четно) и, в частности, $\Phi_{4q}(1+i) = \Phi_q ((1+i)^4) = \Phi_q ((2i)^2) =
\Phi_q(-4)$. Следовательно, если $m$ — делитель $\Phi_q(-4)$, взаимно простой с
$q$, то $1+i$ — примитивный корень порядка $4q$ в $\mathbb{Z} [i]/(m)$. Действительно,
в этом кольце $\Phi_{4q} (1+i)=0$ и $4q$ обратим по модулю $m$ ($m$ — взаимно
простое с $q$ и с 4, так как $m | \Phi_q (-4) | (-4)^q - 1)$.\\
\textbf{1}. Для $q = 18$ имеем $\Phi_{18} (X) = X^6 - X^3 + 1$  и  $\Phi_{18} (-4) = 4161 =
3 \times 19 \times 73$. Следовательно, $1 + i$ — примитивный корень порядка 72 в
кольце $\mathbb{Z} [i]/(19)$ (которое, кстати говоря, является полем, так как $19 \equiv 3
(mod 4))$.\\
\textbf{2}. Для $q = 24$ имеем $\Phi_{24}(X) = X^8 - X^4 + 1 $ и $ \Phi_{24} (-4) = 65 281 =
97 \times 673$. $1 + i$ является примитивным корнем порядка 96 по модулю 97.
Имеется также примитивный корень порядка 96 в $\mathbb{Z} / 97 \mathbb{Z}$, равный 5.\\
Первый из двух примеров показывает, что в $\mathbb{Z} [i]$ иногда можно 
найти примитивный корень по модулю $m$, большего порядка, чем $m - 1$
(поле $\mathbb{Z}/(19)$ дает только корни из единицы, порядок которых делит 18).
Второй пример показывает, что можно легко найти примитивные 
корни, даже если рассматриваемое кольцо, не является полем ($97 \equiv 1
(mod 4)$, и справедлива факторизация $97 = 4^2 + 9^2 = (4 + 9i)(4 - 9i))$.
\section{ Дискретное преобразование Фурье
и циклическая свертка}
Фундаментальное свойство примитивных корней, которое нас здесь
интересует, состоит в следующем: если $\omega$ — примитивный корень
$n$-й степени из единицы, то $V_\omega \ast V_{\omega^{-1}} = n\cdot Id$, что означает, что
$ V_{\omega^{-1}} = n^{-1}\cdot V_\omega  $ (так как по теореме 6 $n$ обратим в $А$). Это свойство
весьма существенно. В нем, по сути, утверждается, что интерполяция,
являющаяся обратной к вычислению значений, в действительности 
сводится к вычислению многочлена. Точнее: \bigskip \\
\textbf{(7) Предложение.} \bigskip \\
\textit{Пусть $\omega$ - примитивный корень $n$-й степени из единицы в кольце $А$.
Для интерполяции многочлена $P = \sum a_i X^i$ по значениям
$\hat{a}_j = P(\omega^j)$ при $j = 0,1,...,n - 1$ достаточно найти значения 
многочлена $\hat{P}$ определенного соотношением $\hat{P} = \sum a_j X^j$ в точках $\omega^{-i}$, где
$i = 0,1,...,n - 1$. Тогда коэффициенты $a_i$ многочлена $Р$ определяются
по формулам $a_i = n^{-1} \cdot \hat{P}(\omega^{-i})$ при $i = 0,1,..., n - 1$.}\bigskip \\
\textbf{Пример} \\
Проиллюстрируем это последнее утверждение с помощью корня $6$-й
степени из единицы. Циклотомический многочлен $\Phi_6(X) = X^2 - X + 1$,
и, например, $\Phi_6(X) = 10^2 - 10 + 1 = 91$. Поэтому $6$ обратимо по 
модулю $91$, $\omega = 10$ — примитивный корень 6-й степени из единицы в $\mathbb{Z} / 91 \mathbb{Z}$.
Обратный к 6 по модулю 91 элемент есть — 15. Поэтому для вектора
с коэффициентами из $\mathbb{Z}_{91}$ имеем: \\
\begin{center}
$\begin{pmatrix}
b_0\\
b_1\\
b_2\\
b_3\\
b_4\\
b_5\\
\end{pmatrix}$ =
$\begin{pmatrix}
1 & 1 & 1 & 1 & 1 & 1\\
1 & 10 & 9 & -1 & -10 & -9\\
1 & 9 & -10 & 1 & 9 & -10\\
1 & -1 & 1 & -1 &1 & -1\\
1 & -10 & 9 & 1 & -10 & 9\\
1 & -9 & -10 & -1 & 9 & 10
\end{pmatrix}$
$\begin{pmatrix}
a_0\\
a_1\\
a_2\\
a_3\\
a_4\\
a_5\\
\end{pmatrix}$
\end{center}
тогда и только тогда, когда\\
\begin{center}
$\begin{pmatrix}
a_0\\
a_1\\
a_2\\
a_3\\
a_4\\
a_5\\
\end{pmatrix}$ = --15
$\begin{pmatrix}
1 & 1 & 1 & 1 & 1 & 1\\
1 & -9 & -10 & -1 & 9 & 10\\
1 & -10 & 9 & 1 & -10 & 9\\
1 & -1 & 1 & -1 &1 & -1\\
1 & 9 & -10 & 1 & 9 & -10\\
1 & 10 & 9 & -1 & -10 & -9
\end{pmatrix}$
$\begin{pmatrix}
b_0\\
b_1\\
b_2\\
b_3\\
b_4\\
b_5\\
\end{pmatrix}$
\end{center}\bigskip 
\textbf{(8) Определение.} \bigskip \\
\textit{Пусть $\omega$ — примитивный корень $n$-й степени из единицы. 
Дискретное преобразование Фурье $\mathcal{F} $, ассоциированное с $\omega$, есть по 
определению преобразование из пространства $A_n [X]$ в пространство $A^n$,
определенное по правилу:\\
\vspace{0pt}\hspace{100pt}$\mathcal{F}_{\omega} (P) = (P({\omega}^0),P({\omega}^1),P({\omega}^2),...,P({\omega}^{n-1})), $
\hspace{50pt}(3)\smallskip\\
где в рассматриваемом равенстве для $ {\mathcal{F}}_{\omega} : A^n \ni a \longrightarrow \text{â}  \in A^n $ ,
$\text{â}_j = \Sigma a_i \omega^{ij}$ . Для краткости используется аббревиатура $DFT$ на $n$ 
точках (от английской фразы \textbf{discrete Fourier transform},), или $DFT_n$,
либо $DFT_{n,\omega}$.}\bigskip \\
В дальнейшем будем также рассматривать преобразование Фурье
как автоморфизм на пространства функций из $[0, n[$ в $A$. С этой 
\textit{функциональной} точки зрения результат преобразования Фурье $F = \mathcal{F}_{\omega} (f)$
некоторой функции $f:[0,n[ \longrightarrow A$ есть также функция $F:[0,n[ \longrightarrow A$
заданная соотношением $F(j) = \Sigma f(i) \omega^{ij}$ для $0 \leq j < n$ и обратным
преобразованию Фурье $\mathcal{F}_{\omega}$ является, с точностью до 
мультипликативной константы $n$, преобразование Фурье $\mathcal{F}_{\omega^{-1}}$.\bigskip \\
\textbf{Отступление.} \newline
Обозначим через $\mathbb{U} \subset \mathbb{C}^{*}$ подгруппу, состоящую из всех 
комплексных чисел, имеющих модуль 1, а через $\mathbb{U}_{n}$ подгруппу всех
корней $n$-й степени из единицы. Любая локально компактная
группа $G$ обладает положительной мерой, инвариантной 
относительно трансляций и единственной с точностью до 
мультипликативного множителя: это так называемая \textbf{мера Хаара}.\\
Обозначим через $ \hat{G}$ — группу характеров $G$, т.е. множество
непрерывных морфизмов $\chi$ из $G$ в $\mathbb{U}$ . $ \hat{G}$ наделена структурой
мультипликативной группы, индуцированной той же операцией,
что и $\mathbb{U}$.\\
Если $f \in L^1 (G)$, т.е. $f : G \rightarrow \mathbb{C}$ интегрируема на $G$, то можно
определить функцию $f$ на $ \hat{G}$ следующим образом: \\
\begin{center}
$\hat{f}(\chi) = \int_{G}^{} f(x) \chi(x) dx  ,\;\;\;\;\; \chi \in \hat{G},   $
\end{center}
и отображение $L^1 (G) \ni f \longmapsto \hat{f}$ со значениями во множестве
функций из $\hat{G}$ в $\mathbb{C}$ является \textbf{преобразованием Фурье}.
В частности, пусть $G$ — аддитивная группа $\mathbb{Z}_n$ целых чисел
по модулю $n$ (в этом случае мера Хаара является мерой, со 
значением 1 в каждой точке). Если $\alpha \in \mathbb{U}$ есть корень $n$-й степени из
единицы, то отображение $\chi_{\alpha}$, определенное для $j \in \mathbb{Z}_n$ с помощью
$ \chi_{\alpha}(j) = {\alpha}^j$, задает элемент из $\hat{\mathbb{Z}}_n$, т.е. характер на $\mathbb{Z}_n$, и 
нетрудно убедиться, что $\mathbb{U}_n \ni \alpha \longmapsto \chi_{\alpha} \in \hat{\mathbb{Z}}_n$ является каноническим
изоморфизмом групп. Если $f: \mathbb{Z}_n \rightarrow \mathbb{C}$ является функцией, то
преобразование Фурье функции $f$ можно отождествить с 
функцией $\hat{f}$  из $\mathbb{U}_n$ в $\mathbb{C}$:\\
\begin{center}
$\hat{f}(\alpha) = \sum\limits_{0 \leq i < n} \alpha^{i} f(i), \;\;\;\;\; \alpha \in \mathbb{U}_n.$
\end{center}
Если, кроме того, $\omega$ — \textbf{фиксированный примитивный корень}
$n$-й степени из единицы, то любой элемент $\alpha$ записывается 
единственным способом в виде $\omega^j$, где $j$ — класс вычетов по модулю
$n$ что дает изоморфизм (не естественный, так как он зависит от
выбора $\omega$) групп $\mathbb{U}_n$ и $\mathbb{Z}_n$. Преобразование Фурье функции $f$ дает
тогда $\hat{f}(j) = \sum \omega^{ij} f(i)$ для $j \in \mathbb{Z}_n$, что объясняет используемый
термин «преобразования Фурье».\\
Причиной рассмотрения преобразования Фурье в этой книге 
является его применение к произведению многочленов. Если $f$ и $g$ — две
функции из $[0, n[$ в $A$, представляющие два многочлена, произведение
$f \ast g$ которых имеет степень, меньшую $n$, то\\
\begin{center}
$\mathcal{F}_{\omega} (f \ast g) = \mathcal{F}_{\omega} (f) \times \mathcal{F}_{\omega} (g), $
\end{center}
где слева стоит произведение многочленов, а справа — обычное 
произведение функций. Но если условие $deg(f \ast g) < n$ не выполняется, то
и в этом случае существует одна и только одна функция $h : [0, n[ \rightarrow A$
такая, что $\mathcal{F}_{\omega} (h) = \mathcal{F}_{\omega} (f) \times \mathcal{F}_{\omega} (g)$. Как выразить $h$ через $f$ и $g$? Что же
такое $h$ по отношению к $f$ и $g$?\smallskip\\
\textbf{(9) Определение и свойство.}\smallskip\\
\textit{Для любых двух функций $f$ и $g$ из $[0, n[$ в $A$ \textbf{циклической сверткой}
$h = f \star g$ называется функция той же природы, определенная по правилу\\
\begin{center}
$h(k)=\sum\limits_{i+j \equiv k (mod\; n)} f(i)g(i),\;\; \text{удовлетворяющая}$\smallskip\\
\end{center}
\vspace{0pt}\hspace{100pt} соотношению $\mathcal{F}_{\omega} (f \star g) = \mathcal{F}_{\omega} (f) \times \mathcal{F}_{\omega} (g).$ \hspace{50pt}(4)\smallskip\\
Другое определение состоит в том, что функция $f \star g$ представляет
произведение многочленов $\sum f(i) X^i$ и $\sum g(j) X^j$ по модулю многочлена
$X^n - 1$. В этом случае говорят о \textbf{циклической свертке на $n$ точках},
которую обозначают $Cc_n$.}\smallskip\\
\textbf{Доказательство}  формулы (4).\\
Формула $\mathcal{F}_{\omega} (f \star g) = \mathcal{F}_{\omega} (f) \times \mathcal{F}_{\omega} (g).$ получается с помощью 
элементарных вычислений, использующих соотношение $\omega^n = 1$ (а не то,
что $\omega$ — примитивный корень $n$-й степени из единицы).\\
В действительности хороший способ понять преобразование 
Фурье состоит в том, чтобы определить его на любом $A[X]$ с 
помощью $\mathcal{F}_{\omega} (P) = (P(\omega^0),P(\omega^1),...,\\P({\omega}^{n-1}))$, учитывая, что свойство
$\mathcal{F}_{\omega} (P \ast Q) = \mathcal{F}_{\omega} (P) \times \mathcal{F}_{\omega} (Q)$, очевидно, и перейти к фактору по
модулю $X^n - 1$ (так как $\mathcal{F} (X^n - 1) = 0)$. При этом можно сразу
определять $\mathcal{F}_{\omega}$ на факторкольце $A[X]/(X^n - 1)$.\smallskip\\
\textbf{Пример}\smallskip\\
Циклическая свертка $f \star g$ на матричном языке может быть 
также получена различными способами как произведение \textit{циркулянтной}
матрицы, на вектор; например, при $ n = 4$:\\
\begin{center}
$f \star g$=
$\begin{pmatrix}
f_0 & f_3 & f_2 & f_1 \\
f_1 & f_0 & f_3 & f_2 \\
f_2 & f_1 & f_0 & f_3 \\
f_3 & f_2 & f_1 & f_0 \\
\end{pmatrix}$
$\begin{pmatrix}
g_0 \\
g_1 \\
g_2 \\
g_3 \\
\end{pmatrix}$ = 
$\begin{pmatrix}
f_1 & f_2 & f_3 & f_0 \\
f_2 & f_3 & f_0 & f_1 \\
f_3 & f_0 & f_1 & f_2 \\
f_0 & f_1 & f_2 & f_3 \\
\end{pmatrix}$
$\begin{pmatrix}
g_3 \\
g_2 \\
g_1 \\
g_0 \\
\end{pmatrix}$=\smallskip\\
= $\begin{pmatrix}
f_0 & f_1 & f_2 & f_3 \\
f_1 & f_2 & f_3 & f_0 \\
f_2 & f_3 & f_0 & f_1 \\
f_3 & f_0 & f_1 & f_2 \\
\end{pmatrix}$
$\begin{pmatrix}
g_0 \\
g_3 \\
g_2 \\
g_1 \\
\end{pmatrix}.$
\end{center}
\section{Обзор различных понятий}
Три объекта: произведение многочленов, циклическая свертка и 
преобразование Фурье — тесно связаны уже по своим определениям. Но
они также связаны и алгоритмически. В дальнейшем мы уточним, как
алгоритм, позволяющий вычислить один из этих объектов, может быть
использован для вычисления другого. Ясно, что быстрое вычисление
дискретного преобразования Фурье, как следствие, позволяет быстро
вычислить свертку, произведение многочленов и произведение целых
чисел. Обратно, способ быстрого вычисления свертки $P \star Q$ \textbf{
для 
фиксированного} $Р$ в некоторых случаях может быть применен к 
дискретному преобразованию Фурье. Вот достаточно простой пример этого.\\
Формула
\begin{flushleft}
$\begin{pmatrix}
x_0 & x_1 \\
x_1 & x_0 
\end{pmatrix}$
$\begin{pmatrix}
y_0 \\
y_1
\end{pmatrix}$ = 
$\begin{pmatrix}
\alpha(y_0 + y_1) + \beta(y_0 - y_1) \\
\alpha(y_0 + y_1) - \beta(y_0 + y_1)
\end{pmatrix}$,
где $\alpha=\frac{x_0 + x_1}{2}$ и $\beta=\frac{x_0 - x_1}{2},$
\end{flushleft}
позволяет осуществить для \textit{данных фиксированных} $x_0$ и $x_1$ свертку на
двух точках с помощью 4 сложений и 2 умножений (вместо 2 сложений
и 4 умножений).
\begin{flushleft}
$\begin{pmatrix}
1 & 1 & 1\\
1 & \omega & \omega^2 \\
1 &  \omega^2 & \omega
\end{pmatrix}$
$\begin{pmatrix}
a_0\\
a_1\\
a_2
\end{pmatrix}$
\end{flushleft}
Можно применить этот результат к $2 \times 2$-подматрице,
участвующей в дискретном преобразовании
Фурье ($D_{FT}$) на трех точках, представленном
слева. Это позволяет реализовать $D_{FT_3}$ с помо-
щью 6 сложений и двух умножений (вместо 6
сложений и 4 умножений). Данный метод 
рассматривается в упражнениях 22 и 24 в конце главы. В этом смысле 
реализация $Cc_2$ — основная часть в вычислении $D_{FT_3}$. В действительности
это является частным случаем результата Рейдера, показавшего, что:
\textit{для простого числа $р$ реализация} $Cc_{p-1}$ — \textit{основная часть для 
вычисления} $F_{FT_p}$.
Читатель может поставить законный вопрос: «каков интерес в 
выигрыше двух умножений на $D_{FT_3}$, тогда как изначальная цель — 
быстрое вычисление $D_{FT_n}$ для \textit{больших} числа $n$?» В следующем разделе
будет показано, как можно свести одно $D_{FT}$ на большом числе точек к
нескольким $D_{FT}$ на небольшом числе точек.
\chapter{ Быстрое преобразование\\ Фурье}
В предыдущем разделе показана важность корней $n$-й степени из 
единицы для исследования задачи интерполяции многочлена, как частного
случая вычисления значений. Но не только этим они интересны. 
Предметом данного раздела являются описание и оценка сложности одного
метода вычисления значений, известного под названием \textit{метода Кули
— Тъюки}, который позволяет осуществить вычисление значений 
многочлена в случае, когда число $n$ «сильно составное». Эффективная 
реализация этого метода будет получена в следующих разделах. С 
исторической и методической точек зрения удобно рассмотреть сначала случай,
когда $n$ есть степень двойки, затем случай, когда $n$ — произвольная 
степень какого-либо числа и, наконец, когда $n$ просто составное. Читатель
может узнать подробности в статьях Иейтса \textit{«The Design and Analysis
of Factorial Experiments*} и Кули и Тьюки \textit{«An algorithm for the Machine
Calculation of Complex Fourier Series*}. Напомним, что метод, который
будет далее излагаться, предназначен для вычисления и интерполяции
многочленов степени, строго меньшей $n$, в $n$ точках, являющихся 
степенями корня $n$-й степени из единицы.
\section{ Случай, когда порядок есть степень двойки}
Покажем теперь, что для вычисления значений многочлена степени\\ < $n$
в точках $\omega^{0},\omega^{1},\omega^{2},...,\omega^{n-1},$ где $\omega$ — корень $n$-й степени из единицы,
достаточно $\mathcal{O}(n \log n)$ операций кольца (умножений, сложений), если $n$
— степень двойки. Пусть $P = \sum a_i X^i$ — многочлен степени < $n$, 
который запишем, разделив на две части слагаемые, содержащие $X$ в
четных и нечетных степенях:\smallskip\\
$P(X)=a_0X^0 + a_2X^2 +...+ a_{n-2}X + X(a_1X^0 + a_3X^2+...+a_{n-1}X^{n-2}),$\smallskip\\
или же:
\begin{center}
$P(X)=P_0(X^2)+XP_1(X^2)$, где $P_0(X) = \sum\limits_{0 \leq i < n/2} a_{2i} X^i$\smallskip\\
и $P_1(X) = \sum\limits_{0 \leq i < n/2} a_{2i+1} X^i.$
\end{center}\smallskip
Сразу заметим, что многочлены $P_0$ и $P_i$ имеют степени $<n/2$. Теперь
важно убедиться, что для нахождения значений $P$ в $n$ точках $x_i=\omega^i,
i = 0,1,2,...,n- 1$, где $\omega$ — корень $n$-й степени из единицы, необходимо
вычислить значения $P_0$ и $P_i$, \textit{но только в $n/2$ точках}. Это следствие
равенства $P(x_i)=P_0(x_{i}^2) + x_iP_1(x_i^2)$ и того, что, ввиду $\omega^n = 1$, для
$n/2 \leq i <n$, имеем $x_i^2 = x_{\frac{n}{2}-i}^2$.\smallskip\\
\textbf{(10) Теорема} (Кули и Тьюки, 1965).\\
Пусть $\omega$ — корень $n$-й степени из единицы в кольце $A$ степени 
которого  $\omega^0,\omega^1,\\\omega^2,...,\omega^{n-1}$ затабулированы. Если $n$ — степень двойки,
то для вычисления многочлена $P$ степени $<n$ во всех этих точках 
достаточно $\mathcal{O}(n \log n)$ операций.\smallskip\\
\textbf{Доказательство.}\\
Пусть $T_n$ — число операций, необходимых для вычисления 
многочлена степени $<n$ в точках $\omega^0,\omega^1,\omega^2,...,\omega^{n-1}$. Равенство
$P(x_i)=P_0(x_i^2)+x_iP_1(x_i^2)$ для $0 \leq i < n$, в котором имеется 
одно сложение и одно умножение, может использоваться рекурсивно,
так как, с одной стороны, полиномы $P_0$ и $P_i$ имеют степени $<n/2$,
а с другой стороны, новые точки, в которых необходимо 
производить вычисления,$x_i^2=\omega^{2i}$, являются степенями $\omega^2$, корня степени
$(n/2)$ из единицы, чьи значения уже затабулированы. Это равенство
приводит к рекуррентному соотношению $T_n = 2T_{n/2} +2n$, где 
слагаемое $2T_{n/2}$ получается из вычислений значений многочленов $P_0$,
$P_i$ степеней $<n/2$, а слагаемое $2n$ — дополнительные $n$ сложений
и $n$ умножений. Так как $T_1 = 0$, то по индукции легко доказать,
что $T_{2^k}=2^{k+1}k=2^{k+1}\log_2n$, что дает возможность выразить как
функцию от $n = 2k: T_n = 2n\log_2n$.\\
Числа умножений можно еще уменьшить наполовину. Сначала 
нужно заметить, что $\omega^{n/2}$ — корень квадратный из единицы, и значит,
в частности, если $\omega$ — примитивный корень $n$-й степени из 
единицы, $\omega^{n/2}=-1$. В этом случае $x_{i+n/2}=\omega^{i+n/2}=-\omega^i=-x_i$, и
удобно совместить вычисления $P(x_i)$ и $P(x_{i+n/2})$:\smallskip\\
\begin{center}
$P(x_i)=P_0(x_i^2)+x_iP_1(x_i^2),\;\;\;\;\;\;\;\;P(x_{i+n/2})=P_0(x_i^2)-x_iP_1(x_i^2).$
\end{center}\smallskip
Если $T_n^{add}$ (соответственно, $T_n^{mult}$) обозначает число сложений 
(соответственно, умножений), необходимых для вычисления значений
многочлена степени $<n$ в точках $\omega^0,\omega^1,\omega^2,...,\omega^{n-1}$, то 
получаем следующие два рекуррентных соотношения: $T_n^{add}=2T_{n/2}^{add}+n$ и
$T_n^{mult}=2T_{n/2}^{mult}+n/2$, что дает $T_n^{add}= n \log_2 n$ и $T_n^{mult}= \frac{1}{2} n \log_2 n$.
Следует заметить, что это незначительное улучшение не меняет
порядка величины, который составляет $\mathcal{O}(n \log n)$.\smallskip\\
\textbf{(11) Определение} («быстрое преобразование Фурье»).\\
\textit{Быстрым преобразованием Фурье (коротко $F_{FT}$) называется 
принцип вычисления дискретного преобразования Фурье, описанный выше.
Чтобы выделить целое число $n$, говорят о быстром преобразовании 
Фурье на $n$ точках или, сокращенно, $F_{FT_n}$.}\smallskip\\
\textbf{Замечание.} Существуют другие эффективные методы 
вычисления (Гуда, Винограда), но лишь метод Кули — Тыоки 
называют \textit{быстрым преобразованием Фурье}, вероятно, по историческим
соображениям. Впрочем, увидим в дальнейшем, что обозначение
$F_{FT_n}$ применяется и в более общем случае, когда $n$ не 
обязательно степень двойки.\smallskip\\
\section{ Случай, когда порядок — произвольная степень}
Предположим теперь, что $n$ является степенью целого числа $q \geq 2$:
$n=q^k$. Тогда в многочлене $P(X)$ группировка одночленов от одного и
того же показателя по модулю $q$ приводит к записи:\smallskip
\begin{center}
$P(X)=P_0(X^q)+XP_1(X^q)+X^2P_2(X^q)+...+X^{q-1}P_{q-1}(X^q),$
$ \text{где} \deg{P_i}<q^{k-1},$ 
\end{center}
и позволяет установить, аналогичным способом, результат, 
доказательство которого содержится в упражнении 16.\smallskip\\
\textbf{(12) Предложение.}\smallskip\\
\textit{В кольце $A$ пусть и является корнем $n$-й степени из единицы, степени
которого $\omega^0,\omega^1,\omega^2,...,\omega^{n-1}$ затабулированы. Если $n$ — степень $q$,
то вычисление значений многочлена $P$ степени $< n$ во всех точках,
являющихся степенями $\omega$, требует не более $2(q - 1)n \log_q n$ операций.}\\
\textbf{Замечание.} Если положить $E_q = 2(q - 1)n \log_q n$, что 
является числом необходимых операций для предыдущего метода при
$n$, равном степени $q$, то можно увидеть, что $\frac{E_q}{E_2}=\frac{q-1}{\log_2q}$ и это
выражение принимает наименьшее значение при $q = 2$.
\section{ Пример итеративного алгоритма}
В этом разделе мы изучим более подробно рекурсивные соотношения из
предыдущего раздела для получения \textit{рекуррентных соотношений}, 
позволяющих реализовать преобразование Фурье в \textit{итеративной форме}.
Действительно, рекурсивная реализация требует больше памяти и 
времени, что выражается формально в увеличении константы, входящей в
обозначение $\mathcal{O}(n \log n)$. Будем двигаться в двух направлениях:\\
• проиллюстрируем дерекурсивизацию метода Кули — Тьюки на
примере $n = 2^3$, следуя доказательству теоремы из предыдущего
раздела; это позволит нам выявить важность записи индексов в
двоичной системе счисления (по основанию 2);\\
• наметим другой подход с помощью более общего метода дере-
курсивизации: это метод Йейтса, который легко преобразуется в
итеративную схему и, таким образом, более близок к реализации,
которую мы хотим осуществить.
\subsection{ От рекурсии к итерации}
Мы должны вычислить значения $d_j = P(\omega^j)$ многочлена $P=\alpha_0+\alpha_1X+\alpha_2X^2+\alpha_3X^3+\alpha_4X^4+\alpha_5X^5+\alpha_6X^6+\alpha_7X^7$ в 
последовательных степенях $\omega$. Следование доказательству теоремы 10 приводит к
представлению многочлена $P(X)$ в виде $P_0(X^2)+XP_1(X^2)$. Значит,
вычисление значений $P$ в точках, являющихся степенями $\omega$, приводит
к последовательному вычислению значений $P_0$ и $P_i$ в точках $1,\omega^2,\omega^4$
и $\omega^6$. Обозначим через $c_j$ числа, получаемые на этом этапе (от $c_0$ до $c_4$
для $P_0$ и от $c_4$ до $c_7$ для $P_1$), и запишем для удобства читателя:\\
\begin{center}
$d_j=P(\omega_j)$ для $j=0,1,...,7,\;\;c_j=P_0(\omega^{2j})$\\
и $\;\;\;c_{j+4} = P_1(\omega^{2j})$ для $\;\;j=0,1,2,3.$  
\end{center}
Применяя основную идею теоремы 10, разложим многочлены $P_0$ и $P_i$
на \textit{четную} и \textit{нечетную} части:
\begin{center}
$P_0(X)=P_{00}(X^2)+XP_{01}(X^2)\;\;\;$ и $\;\;\;P_1(X)=P_{10}(X^2)+XP_{11}(X^2)$
\end{center}
Для вычисления $c_j$ достаточно вычислить значения $b_j$ полученных 
четырех многочленов в точках 1 и $\omega^4$:\\
\begin{table}[h]
\begin{center}
\begin{tabular}{|c|c|}
\hline
$P_{00}=\alpha_0 + \alpha_4X$ & $P_{01}=\alpha_2 + \alpha_6X$\\
\hline
$b_0\;\;\;\;\;\;\;\; b_1$ & $b_2\;\;\;\;\;\;\;\; b_3$\\
\hline
$P_{10}=\alpha_1 + \alpha_5X$ & $P_{11}=\alpha_3 + \alpha_7X$ \\
\hline
$b_4\;\;\;\;\;\;\;\; b_5$ & $b_6\;\;\;\;\;\;\;\; b_7$\\
\hline
\end{tabular}
\end{center}
\end{table}
\\
Значения $b_j$ очень просто зависят от $\alpha_j$, что дает возможность 
реписать цепочку вычислений до $d_j$ следующим образом:\\
\begin{table}[h]
\begin{center}
\begin{tabular}{|c|c|c|c|}
\hline
$b_0=\alpha_0 + \alpha_4$&$b_1=\alpha_0+\alpha_4 \cdot \omega^4$&$b_2=\alpha_2+\alpha_6$&$b_3=\alpha_2+\alpha_6 \cdot \omega^4$\\
\hline
$b_4=\alpha_1 + \alpha_5$&$b_5=\alpha_1+\alpha_5 \cdot \omega^4$&$b_6=\alpha_3+\alpha_7$&$b_7=\alpha_3+\alpha_7 \cdot \omega^4$\\
\hline
$c_0=b_0+b_2$&$c_1+b_1+b_3 \cdot \omega^2$&$c_2=b_0+b_2 \cdot \omega^4$&$c_3=b_1+b_3 \cdot \omega^6$\\
\hline
$c_4=b_4+b_6$&$c_5+b_5+b_7 \cdot \omega^2$&$c_6=b_4+b_6 \cdot \omega^4$&$c_7=b_3+b_7 \cdot \omega^6$\\
\hline
$d_0=c_0+c_4$&$d_1=c_1+c_5 \cdot \omega^1$&$d_2=c_2+c_6 \cdot \omega^2$&$d_3=c_3+c_7 \cdot \omega^3$\\
\hline
$d_4=c_0+c_4 \cdot \omega^4$&$d_5=c_1+c_5 \cdot \omega^5$&$d_6=c_2+c_6 \cdot \omega^6$&$d_7=c_3+c_7 \cdot \omega^7$\\
\hline
\end{tabular}
\end{center}
\end{table}
\\
Заметим, что число сложений $3 \times 8 = 24$, что лучше, чем $7 \times 8 = 56$
операций, необходимых в наивном методе (аналогичный результат для
умножений). Можно также отметить, что имеется важное свойство,
присущее алгоритмам типа $F_{FT}$, помогающее в их реализации, — это
зависимость между вычисляемыми элементами: пара элементов $d$ 
вычисляется в качестве функции, определяемой однозначно парой 
элементов с с теми же индексами (например, $(d_0, d_4)$ вычисляется при помощи
$(c_0, c_4))$, причем это свойство еще сохраняется, когда происходит 
переход от $c$ к $b$. Однако, оно не выполняется, при вычислении $b$, что 
заставляет нас ввести элементы $a_j$, полученные простой перенумерацией
элементов $a_j$:\newline
\begin{center}
$a_0=\alpha_0,\;\;a_1=\alpha_4,\;\;a_2=\alpha_2,\;\;a_3=\alpha_6,$\smallskip\newline
$\;\;a_4=\alpha_1,\;\;a_5=\alpha_5,\;\;a_6=\alpha_3,\;\;a_7=\alpha_7,\;$\\
\end{center}
откуда выводятся формулы перехода от $a_j$ к $b_j$, которые будут 
подчиняться уже подмеченной общей закономерности:\newline
\begin{table}[h]
\begin{center}
\begin{tabular}{|c|c|c|c|}
\hline
$b_0=a_0+a_1$&$b_1=a_0+a_1 \cdot \omega^4$&$b_2=a_2+a_3$&$b_3=a_2+a_3 \cdot \omega^4$\\
\hline
$b_4=a_4+a_5$&$b_5=a_4+a_5 \cdot \omega^4$&$b_6=a_6+a_7$&$b_7=a_6+a_7 \cdot \omega^4$\\
\hline
\end{tabular}
\end{center}
\end{table}\\
Наконец, чтобы сделать вычисления явными, осталось записать 
правила, которые управляют образованием пар на каждом этапе:\\
• для перехода от $a$ к $b$ рассматриваются пары элементов с 
последовательными индексами и степени величины $\omega^4$. Получаем:
$b_{2i}=a_{2i}+a_{2i+1} ({\omega^4})^{2i}$ и $b_{2i+1}=a_{2i}+a_{2i+1} ({\omega^4})^{2i+1}$,\\
• для перехода от $b$ к с образуютя пары элементов с индексами $i$ и
$j = i+2$ и используются степени величины $\omega^2:c_i=b_i+b_j \cdot {(\omega^2)}^i$
и $c_j=b_i+b_j \cdot {(\omega^2)}^j$,\\
• наконец, для перехода от $c$ к $d$ необходимо построить пары 
элементов с индексами $i$ и $j = i+4$ и использовать степени величины $\omega$:
$d_i = c_i + c_j \cdot \omega^i$ и $d_j = c_i + c_j \cdot \omega^j$.\\
В заключение отметим, что решены не все проблемы реализации.
Не определен еще полностью метод, который необходимо использовать,
чтобы заставить индексы $r$, появляющиеся в приведенных выше 
формулах, пробегать множество нужных значений, когда хотят представить
каждую последовательность $a, b, c$ и $d$ единым массивом.
\subsection{ Суммы Йейтса}
Цель этого метода — выразить наиболее простым образом объект
некоторого этапа как функцию объекта предыдущего этапа: вместе с
подходом «вычисление многочленов в точке» рекурсивность 
применяется к одному многочлену данной степени, превращая его в два 
многочлена меньшей степени. По этой причине более удобны 
функциональные обозначения: на данном этапе объект является функцией, которая
просто выражается через объект \textit{в точности той же природы} и 
вычисленного заранее.\\
Проиллюстрируем этот метод с помощью того же примера $n=2^3$.
Возьмем функцию $f : [0,2^3[ \longrightarrow A$ и вычислим преобразование 
Фурье $\hat{f}: [0,2^3[ \longrightarrow A$, определенное, как обычно, с помощью $\hat{f}(j) =
\sum_{0 \leq i < 2^3} f(i) \cdot \omega^{ij}$ Для $0 \leq j < 2^3$. Обозначим $F_3 = \hat{f}$ преобразование
Фурье $f$, вводя обозначения таким образом, что $F_3$ выражается с 
помощью другой функции $F_2$, которая в свою очередь выражается через...
и, наконец, функция $F_0$ выражается просто через $f$, причем, все эти
функции являются функциями из $[0,2^3[$ в $A$. Теперь опишем различные
этапы, позволяющие переходить от $F_0$ к $F_1$, от $F_1$ к $F_2$ и от $F_2$ к $F_3$,
где $F_0$ используется в качестве начального; очевидно, что указанные
объекты соответствуют объектам $a, b, c, d$ предыдущего раздела.\\
Начнем с того, что представим индексы $i$ и $j$ в системе счисления
по основанию 2:\\
\begin{center}
$i = <i_2i_1i_0>=i_22^2 + i_12+i_0$\\
и $j=<j_2j_1j_0>=j_22^2+j_12+j_0$ для $0 \leq i, j < 2^3.$
\end{center}
Так как $\omega^{2^3}=1$, то имеем равенство $\omega^{ij} = \omega^{i_0<j_2j_1j_0>} \ast \omega^{2i_1<j_1j_0>} \ast \omega^{2^2i_2<j_0>},$
которое позволяет переписать сумму $\sum_i f(i) \omega^{ij}$ в виде\\
\begin{center}
$\sum\limits_{i_0}\omega^{i_0<j_2j_1j_0>} \sum\limits_{i_1}\omega^{2i_1<j_1j_0>} \sum\limits_{i_2}\omega^{2^2i_2<j_0>}.$
\end{center}
Можно определить частичные суммы $F_1,F_2,F_3$ следующим образом:
\begin{flushleft}
$F_1<i_0i_1j_0>=\sum_{i_2} \omega^{{2^2}i_2<j>} f<i_2i_1i_0>,$\\
$F_2<i_0j_1j_2>=\sum_{i_1} \sum_{i_2} ... = \sum_{i_1} \omega^{{2i_1}<j_1j_0>} F_1<i_0i_1j_0>,$\\
$F_3<j_2j_1j_0>=\sum_{i_0}\sum_{i_1}\sum_{i_2} ...=\sum_{i_0} \omega^{i_0<j_2j_1j_0>} F_2<i_0j_1j_0>.$
\end{flushleft}
Чтобы сделать формулу, определяющую $F_1$, похожей на формулы, 
определяющие $F_2,F_3$, естественно ввести функцию $F_0$, удовлетворяющую
соотношению $F_1<i_0i_1j_0>=\sum_{i_2} \omega^{{2^2}i_2<j>} F_0<i_0i_1i_2>,$ для чего 
предполагается\\ $F_0<i_0i_1i_2>=f<i_2i_1i_0>$ и, таким образом, как бы заново получаем
результаты предыдущего абзаца, но в более общей форме.
\chapter{ Точное вычисление FFT :\\ произведение
многочленов}
В дальнейшем мы докажем формулы, позволяющие вычислять $F_{FT}$ в
общем случае. А пока обратимся к формулам, позволяющим вычислять
дискретное преобразование Фурье, когда число точек интерполяции
(точек, в которых вычисляются значения) является степенью двойки.
Это позволяет реализовать $F_{FT}$ в указанном частном случае и 
применить метод вычисления произведения многочленов, рассмотренный в
начале этой главы.
\section{ Реализация FFT, когда число точек
есть\\ степень 2}
Итак, приведем явные формулы для вычисления $F_{FT}$, полученные при
условии $n=2^k$ из следствия 28, доказанного в разделе 4.2.2.\smallskip\\
\textbf{(13) Лемма.}\smallskip\\
\textit{Пусть $n=2^k$. Обозначим через $<i_mi_{m-1}...i_1i_0>=i_m2^m+i_{m-1}2^{m-1}+...+i_12+i_0$ представления целого числа $i$ в системе счисления с 
основанием 2. Пусть $\sigma$ — инволюция на интервале $[0, n[$, инвертирующая
двоичную запись числа:\\ $\sigma<i_{k-1}...i_1i_0>=<i_0i_1...i_{k-1}>$. Пусть $f$ — 
последовательность элементов из $A$, индексированная числами из 
интервала $[0,n[$.\\
Определим последовательность из $k +1$ \textbf{массива} $F_0, F_1,..., F_k $ с 
индексами на интервале $[0, n[$ посредством следующего рекуррентного 
соотношения, применимого для $0 \leq m <k$:\\
$F_{m+1}<\;t_0\;...\;t_{k-m-2}\;t_{k-m-1}\;t_{k-m}\;...\;t_{k-1}>=$\\
$=F_m<\;t_0\;...\;t_{k-m-2}\;0\;t_{k-m}\;...\;t_{k-1}>+$\\
$=F_m<\;t_0\;t_1\;...\;t_{k-m-2}\;1\;t_{k-m}\;...\;t_{k-1}> \omega^{2^{k-m-1}<t_{k-m-1}...t_{k-1}>},$\\
где массив $F_0$ определен по формуле $F_0(i)=f(\sigma(i))$. Последний массив
$F_k$ дает преобразование Фурье $\hat{f}$ для $f$, $F_k(j) = \hat{f}(j)$ при $0 \leq j < n$.}\\
Эти несколько сложные формулы могут выражаться различным
образом (и, возможно, даже проще), если исключить ссылки на 
бинарную нумерацию, прямое обращение к которой может представлять
трудность для языков программирования высокого уровня.\\
\begin{lstlisting}[mathescape=true]
T* FastFourierTransform($f \in A^{2^{k}}$) {
for(s=0; s <= $2^k$; s++) {
$F(s) \longleftarrow f(\sigma (s))$;
}
for(i=0; i <= $2^{k-m-1}-1$) {
$i_0 \longleftarrow 2 * i * 2^m$; $i_1 = \longleftarrow i_0 + 2^m$;
for(j = 0; j <= $2^m - 1$) {
$i$_$0$_$j$ $\longleftarrow i_0 + j$; $i$_$1$_$j$ $\longleftarrow i_1 + j$;
$G_0 \longleftarrow$ $F(i$_$0$_$j$); $G_1 \longleftarrow$ $F(i$_$1$_$j) * {\omega}^{j*2^{k-m-1}}$;
$F(i$_$0$_$j) \longleftarrow G_0 + G_1$; $F(i$_$1$_$j) \longleftarrow G_0 - G_1$;
}
}
return F;
}
\end{lstlisting}
\begin{center}
\textbf{Алгоритм 1.} Быстрое преобразование Фурье в кольце $A$
\end{center}






\textbf{(14) Следствие}\\
\textit{Пусть $\omega$ - \textbf{примитивный} корень $2^k$-степени из единицы. Используя предыдущие обозначения, для $m<k$, массив $F_{m+1}$ иожет быть вычислен, исходя из $F_m$ по следующим формулам
\begin{equation} 
\begin{split}
F_{m+1}(2^{m+1} i+j) &= F_m (2^{m+1} i+j) + \\
&+F_{m+1}(2^{m+1} i + 2^m +j) \times \omega^{j2^{k-m-1}},\\
F_{m+1}(2^{m+1} i+2^m+j) &= F_m (2^{m+1} i+j) - \\
&+F_{m+1}(2^{m+1} i + 2^m +j) \times \omega^{j2^{k-m-1}},
\end{split}
\end{equation}
при ограничениях $0 \leq i < 2^{k-m-1}$  и $0 \leq j < 2^m.$}

\textbf{Доказательство.}\\
Положим
$ i = < t_0 \ldots t_{k-m-2}>, j = <t_{k-m} \ldots t_{k-1}> $ и $r = t_{k-m-1}$.
Эти числа удовлетворяют неравенствам $0 \leq  i < 2^{k~m~1}$ и $0 \leq j < 2^m$.
Естественно, можно записать

\newpage

\begin{equation}
\begin{split}
&<t_0 \ldots t_{k-m-2} r t_{k-m} \ldots t_{k-1} > = 2^{m+1} i + 2^m r + j \\
\mbox{ и } &2^{k-m-1} <r t_{k-m} \ldots t_{k-1}> = 2^{k-m-1} (2^m r+j) = 2^{k-1} r + 2^{k-m-1}.
\end{split}
\end{equation}

В этих формулах можно заметить, что все эти последовательные массивы могут вычисляться сразу: действительно, в формуле 5 появляются только элементы с индексами $2^{m+1}i+j$ и $2^{m+1}i + 2^m +j$ массивов $F_m$ и $F_{m+i}$ образуют такие пары, и больше они не встречаются нигде
при переходе от $F_m$ к $F_{m+i}$. Теперь можно сформулировать полный алгоритм 1 вычисления быстрого преобразования Фурье, когда число точек есть степень двойки (аналогичный алгоритм для числа точек $q^k$ с произвольным $q$, менее предпочтительный, можно увидеть в
упражнениях 16 и 17).

\section{Как реализовать FFT?}
Какие средства необходимы для реализации $F_{FT}$? Из предыдущего видно, что нужно работать в некотором кольце $A$, содержащем примитивные корни степени $n$ из единицы, где $n$ — степень $2$, которое будет порядком используемого преобразования Фурье.

\textbf{(15) Лемма}\smallskip\\
\textit{Пусть $p$ — простое число. Алгоритм $F_{FT}$ позволяет вычислить дискретное преобразование Фурье на $n$ точках в $\mathbb{Z}/p\mathbb{Z}$
за $\mathcal{O} (n \log n)$ модулярных операций, если выполнены следующие условия: $n$ — степень
числа $2$ и $n$ делит $p-1$.
}

\textbf{Доказательство.}

Группа $U(\mathbb{Z}/p\mathbb{Z})$
обратимых элементов в $\mathbb{Z}/p\mathbb{Z}$ является
циклической порядка $p — 1$. Если $\xi$ — образующий элемент этой группы,то $\omega=\xi^{\frac{p-1}{n}}$ — элемент порядка $n$, являющийся корнем $n$-й степени из единицы. То, что сложность ранее полученного метода есть
$\mathcal{O}( n \log n)$, следует из теоремы 10 и свойств, которые будут
доказаны в ходе общего исследования метода Кули — Тьюки (см.раздел 4.2.1).

Второй вопрос: какие средства должен предоставлять пакет вычисления $F_{FT?}$ Для ответа на этот вопрос надо вспомнить, что в интересующем нас случае произведения многочленов, наряду с преобразованием Фурье необходимо использовать его обратное для интерполяции произведения многочленов. Поэтому нужно больше, чем можно было

\end{document}