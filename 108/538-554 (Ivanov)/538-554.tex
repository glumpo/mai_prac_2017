\documentclass{mai_book}

\defaultfontfeatures{Mapping=tex-text}
\setmainfont{DejaVuSans}
\setdefaultlanguage{russian}

\newlength{\MYwidth} % новый параметр длины 
\def\MYvrule#1\par{ 
\par\noindent 
\MYwidth=\textwidth\addtolength{\MYwidth}{-7pt} 
\hbox{\vrule width 1pt\hspace{5pt}\parbox[t]{\MYwidth}{#1}} 
}
%\clearpage
\setcounter{page}{538} % ВОТ ТУТ ЗАДАТЬ СТРАНИЦУ
%\setcounter{thesection}{5} % ТАК ЗАДАВАТЬ ГЛАВЫ, ПАРАГРАФЫ И ПРОЧЕЕ.
% Эти счетчики достаточно задать один раз, обновляются дальше сами
\usepackage{amsmath}
\usepackage{multicol}
%%%%%
\usepackage{floatflt}
\usepackage{wrapfig}
%%%%
\newtop{IV {\textit{Некоторые методы алгебраической алгоритмики}}}

\begin{document}


\textbf{a.} Предположим, что $n=p^{\alpha}$, где $p$ --- нечетное простое число.
Доказать, что $b\in B_n$ тогда и только тогда, когда $n$ --- псевдопростое по основанию $b$.
Какую долю в $U$($\mathbb {Z}_n$) составляют такие основания?

\smallskip
\textbf{b.} Доказать, что если $n$ сильно псевдопростое по основанию $b$, то оно сильно псевдопростое по основанию --$b$ mod $n$ и п основаниям $b^m$ mod $n$.

\smallskip
\textbf{c.} Пусть $n-1=2^kq$, где $q$ нечетно, и предположим, что $n$ делится на простое число $p\equiv 3 (mod\;4)$.
Доказать, что $b\in B_n$ тогда и только тогда, когда $b^q=\pm 1$.

\smallskip
\textbf{d.} Доказать, что  $B_n$ --- подгруппа $U$($\mathbb {Z}_n$) тогда и только тогда, когда $n$ --- степень простого числа или $n$ делится на простое число $\equiv 3 (mod\;4)$.

\smallskip
\textbf{e.} Привести пример целого числа $n$ и оснований $b$ и $b'$ таких, что $n$ --- сильно псевдопростое по основаниям $b$ и $b'$, но не по основанию $bb'$.

\bigskip
\noindent\textbf{48. Количество элементов множества $\{b\in U(\mathbb {Z}_n)\;|\;b^{\frac{n-1}2}=\pm1 \}$}

\medskip
Если $H$ --- подгруппа группы $G$, то через $[G : H]$ обозначим индекс $H$ в $G$, т.е. отношение $|G|/|H|$.
В данном упражнении  $n = p^{\alpha_1}_1 \dots p^{\alpha_r}_r$ --- нечетное целое число $\ge 3$.
Определим следующие подмножества $U$($\mathbb {Z}_n$):


\begin{center}
$H_n=\{b$ | $b^{n-1}=1\}$, \qquad $K_n=\{b$ | $b^\frac{n-1}2=\pm1\}$, \qquad  $L_n=\{b$ | $b^\frac{n-1}2=1\}$.
\end{center}



\textbf{a.} Проверить, что $L_n \subset K_n \subset H_n$ и все эти подмножества являются подгруппами группы $U$($\mathbb {Z}_n$); найти порядки $H_n$ и $L_n$.
Показать, что $[H_n : L_n] = 2^t$, где $t$ --- число таких индексов $i$, что $v_2(n-1) \le v_2(p_i-1)$.
Привести примеры, для которых $t=0, t=r$.

\smallskip
\textbf{b.} Доказать, что $[K_n : L_n]=2$ при $t = r$ и $[K_n : L_n]=1$ в противном случае.

\smallskip
\textbf{c.} Отсюда $[H_n : K_n]=2^{r-1}$ при $t = r$ и $[H_n : K_n]=2^t$ в противном cлучае.
Доказать, что $H_n=K_n$ тогда и только тогда, когда $n$ — сте­пень простого числа или $t=0$ (т.е. $v_2(p_i-1) < v_2(n-1)$ для любого $p_i$).
В последнем случае $H_n=K_n=L_n$.

\bigskip
\noindent\textbf{49. Оченка количества чисел $b\in U(\mathbb {Z}_n)$ таких, что $b^{n-1}=1$}

\medskip
Обозначим через $H_n$ подгруппу $\{b\in U(\mathbb {Z}_n)\;|\;b^{n-1}=1\}$ (см. упраж­нение 48).
Доказать, что если $n$ имеет квадратный множитель, то $|H_n|\le \frac{\varphi (n)}3$.
Привести примеры, для которых достигается эта граница.

\newpage

\newtop{\textit{Упражнения}}

\noindent\textbf{50. Псевдопростые числа Эйлера по основанию $b$}

\medskip
Нечетное число $n$ называется псевдопростым числом Эйлера по основанию $b\in U(\mathbb {Z}_n)$, если в $U(\mathbb {Z}_n)\;b^{\frac{n-1}2}=(\frac{b}n)$, где $(\frac{b}n)$ обозначает сим­вол Якоби, введенный в упражнении 31.
Пусть $n=p_1 \dots p_r$, где $p_i$ --- нечетные, не обязательно различные простые числа.

\textbf{a.} Доказать строгие включения:

$$
\begin{aligned}
\{\;\text{простые числа}\;\} &\subsetneq \{\;\text{псевдопростые числа Эйлера по основанию}\;b\;\} \\
                                        &\subsetneq \{\;\text{псевдопростые числа по основанию}\;b\;\}
\end{aligned}
$$

\medskip
\textbf{b.} Положим $n-1=2^k_q$, где $q$ нечетно, и допустим, что $n$ --- сильно псевдопростое по основанию $b\in U(\mathbb {Z}_n)$. 
Если $j$ --- наименьшее целое число с $b^{2^jq}=1$, то доказать, что $j\leq v_2(p_i-1)$ и $b^\frac{n-1}2=(-1)^t$,где $t$ --- число таких индексов $i$,что $v_2(p_i-1)=j$ (напомним: $v_2(m)$ обозначает показатель 2 в разложении на простые множители числа m). 
Вывести отсюда, что $n$ сильно псевдопростое по основанию $b$. 
Доказать, что обратное утверждение не верно.

\textbf{с.} Пусть $n\equiv3\;(mod\;4)$. 
Доказать, что понятия "псевдопростое число Эйлера по основанию $b$" и "сильно псевдопростое число" совпадают.

\bigskip
\noindent\textbf{51. Псевдопростые числа по основанию 2, не больше $10^4$}

\medskip
Написать программу нахождения псевдопростых чисел по основанию 2, не больших $10^4$.
Какие из них являются псевдопростыми числами Эйлера по основанию 2? Сильно псевдопростыми по основанию 2?

\bigskip
\noindent\textbf{52. Тест на простоту Соловея--Штрассена}

\medskip
\textbf{a.} В этом упражнении $n$ --- нечетное целое число. 
Проверить, что множество $E_n$ элементов $b$, обратимых по модулю $n$ и таких, что $n$ --- псевдопростое число Эйлера по основанию $b$, является подгруппой груп­пы $U(\mathbb {Z}_n)$.

\smallskip
\textbf{b.} Доказать, что $E_n=U(\mathbb {Z}_n)$ тогда и только тогда, когда $n$ --- простое. 
Доказать, что если $n$ непростое, то $|E_n|\leq|U(\mathbb {Z}_n)|/2$.
По-другому: если число $n$ --- псевдопростое число Эйлера по большей части оснований (из $U(\mathbb {Z}_n)$), то оно простое. 
Построить вероятный тест на простоту, аналогичный тесту Рабина--Миллера.

\newpage

\newtop{IV {\textit{Некоторые методы алгебраической алгоритмики}}}

\noindent\textbf{53. Числа, достигающие границы $\frac{1}2$ в тесте Соловея--Штрассена}

\medskip
Пусть $n=p^{\alpha_1}_1 \dots p^{\alpha_r}_r$ --- нечетное целое число. 
Сохраним обозначения для подгрупп $E_n, L_n \subset K_n \subset H_n$ группы $U(\mathbb {Z}_n)$, введенных в предыдущих упражнениях. 
Известно, что $|E_n|\le \frac{\varphi (n)}2$ или, по-другому, число, являющееся псевдопростым число Эйлера по большей части оснований, простое. 
Изучим порядок $E_n$ для того, чтобы получить общий вид чисел с $|E_n|\le \frac{\varphi (n)}2$. 
Пусть $n-1=2^kq$, $p_j-1=2^{k_j}q_j$, где $q$, $q_1,\dots,q_r$ --- нечетные, а $k=v_2(n-1)$ и $k_j=v_2(p_j-1)$. 
Напомним, что $k\geq inf\;k_j$.

\smallskip
\textbf{a.} Предположим, что $k=inf\;k_j$. 
Доказать, что $\sum_{i,k_i=k}\alpha_i$ нечетно и что $L_n$ --- подгруппа индекса 2 в $E_n$. 
Вывести отсюда, что $|E_n|\le \frac{\varphi (n)}4$ и привести примеры, для которых эта грань достигается.

\smallskip
\textbf{b.} Предположим, что $k > inf\;k_j$. 
Доказать, что $E_n \subset L_n$ и $[L_n\;:\;E_n]=2$ тогда и только тогда, когда существует такое $j$, что $k_j < k$ и $\alpha_j$ нечетно. 
Вывести отсюда, что $|E_n|\le \frac{\varphi (n)}2$ и равенство до­стигается тогда и только тогда, когда $n$ --- число Кармайкла с $k > k_j$ для любого $j$. 
Привести примеры.

\bigskip
\noindent\textbf{54. Построение $n,\;b$ с $b^{\frac{n-1}2}=\pm1$, но $b^{\frac{n-1}2}\neq (\frac{b}n)$}

\medskip
Пусть $p$ и $q$ --- два различных нечетных простых числа с $v_2(p-1)=v_2(q-1)$. 
Если $n=pq$, то показать, как построить такое $b\in U(\mathbb {Z}_n)$, что $b^{\frac{n-1}2}=1$ и $(\frac{b}n)=-1$. Привести примеры.

\bigskip
\noindent\textbf{55. Факторизация "$\grave{a}$ la Ферма"}

\medskip
Выбор простых чисел является основой метода RSA. 
Цель этого упражнения --- установить алгоритм факторизации чисел $n=pq$, если $p$ и $q$ близки (и, очевидно, нечетны).

\smallskip
\textbf{a.} Допустим, мы располагаем алгоритмом вычисления квадрат­ных корней из целых чисел. 
Написать алгоритм факторизации $n$.
Оце­нить его сложность. 
Указание: записать $n$ как функцию от $(p+q)/2$ и $(p-q)/2$ (предполагается, что $p > q$).

\textbf{b.} В действительности, при факторизации чисел Ферма можно не вычислять квадратные корни на каждой итерации; тогда две послед­ние десятичные цифры квадрата не произвольны; это позволяет устра­нять числа, заведомо не являющиеся квадратами. 
Вместо использова­ния двух последних десятичных цифр можно рассматривать квадраты

\newpage

\newtop{\textit{Упражнения}}

\noindent по модулям некоторых простых чисел и, не вычисляя квадратных кор­ней, эффективно отсеивать не квадраты. 
Написать алгоритм, реализу­ющий этот метод.

\bigskip
\noindent\textbf{56. Матрица Смита, формула Чезаро}

\medskip
Обозначим через $\varphi$ функцию Эйлера, а через $\mu$ --- функцию Мёбиуса; напомним, что $\sum_{d|n}\mu(d)$ равна $0$, если $n > 1$, и $1$, если $n=1$. 
Это означает, что в кольце арифметических функций функция $\mu$ и константная
функция $1$ взаимно обратны (единичная функция это функция Дирака $\delta_1$). 
Определим три матрицы А, М и S размеров $n\times n$:


$$
\begin{aligned}
a_{ij}=\text{НОД}(i,j),\qquad m_{ij}=\mu(i/j),\; &\text{если}\; j\;|\;i,\; \text{и}\; 0\; \text{в противном случае};\\
s_{ij}=1,\; \text{если}\; j\;|\;i, \text{и}\; &0\; \text{в противном случае}.
\end{aligned}
$$


\medskip
\noindent Трансформантной функции $h\; :\; [1,n] \rightarrow \mathbb{Z}$ называется функция $\hat{h}\; :\; [1,n] \rightarrow \mathbb{Z}$, определенная по правилу $\hat{h}(m)=\sum_{d|m}h(d)$.

\smallskip
\textbf{a.} Доказать формулу обращения $MS=SM=Id_n$. 
Что означает эта формула? 
Если $\hat{H}$ --- матрица $\hat{h}(\text{НОД}(i,j))$, то проверить, что матрица $S\hat{H}^tS$ диагональна и равна $diag(h(1),h(2),\dots,h(n))$.

\smallskip
\textbf{b.} Доказать, что $A=SD^tS$, где $D$ --- диагональная матрица $diag(\varphi(1),\dots,\varphi(n))$. 
Доказать, что $det\:A=\varphi(1)\dots \varphi(n)$ (формула Смита) и $A^{-1}= ^t\!\!MD^{-1}M$ (формула Чезаро).

\bigskip
\noindent\textbf{57. Определение степеней множителей многочлена над $\mathbb{F}_q$}

\medskip
Обозначим через $K$ конечное поле порядка $q$, а через $\tau$ --- гомоморфизм Фробениуса $x \rightarrow x^q$ какой-либо $K$-алгебры. Если $P\in K[X]$ и $i\in \mathbb{N}^*$, то определим под-$K$-алгебру Берлекэмпа алгебры $K[X]/P:$


$$
\begin{aligned}
B_{P,i}=Ker(\tau^i-Id)&=\{x\in K[X]/P\;|\;\tau^i(x)=x\}=\\
                      &=\{x\in K[X]/P\;|\;x^{q^i}=x\}.
\end{aligned}
$$


\textbf{a.} Привести примеры, для которых $\tau$ не инъективно. 
Проверить, что $B_{P,i}$ действительно под-$K$-алгебра K[X]/P. 
Предположим, что $P$ — неприводимый многочлен степени $n$; что представляет из себя алгебра $B_{P,n}?$ алгебра
$B_{P,1}?$ Доказать, что $dim_K\;B_{P,i}=\text{НОД}(i,n)$.

\smallskip
\textbf{b.} Используя упражнение 56 главы II, доказать, что формула для размерности остается верной, если $P$ --- степень неприводимого мно­гочлена.

\newpage

\newtop{IV {\textit{Некоторые методы алгебраической алгоритмики}}}

\textbf{c.} Доказать, что если $P$ и $Q$ --- два взаимно простых многочлена, то $B_{PQ,i}\simeq B_{P,i}\times B_{Q,i}$. 
Вывести отсюда, что $dim_K\; B_{P,1}$ равна числу неприводимых множителей (различных) многочлена $P$.

\smallskip
\textbf{d.} Пусть $A$ --- матрица Смита порядка $n$, элементами которой явля­ются $\text{НОД}(i,j)$, а $d_i$ --- число неприводимых множителей (различных) $i$-й степени многочлена $P$. 
Доказать, что


$$
\begin{aligned}
(dim\,B_{P,1}, dim\,B_{P,2},\dots , dim\,B_{P,n})=A(d_1, d_2,\dots ,d_n).
\end{aligned}
$$


\medskip
\noindent Обратная матрица к матрице Смита находится по формуле обращения
Чезаро (см. упражнение 56). Вывести отсюда выражение $(d_1, d_2,\dots , d_n)$ через $(dim\,B_{P,1}, dim\,B_{P,2},\dots , dim\,B_{P,n})$.


\bigskip
\noindent\textbf{58. Неприводимые множители $X^n-1$ над конечными полями}

\medskip
Цель данного упражнения --- используя технику упражнения 57, по­
лучить утверждения, касающиеся степеней неприводимых множителей
многочлена $X^n-1$ над $\mathbb{F}_q$, не вычисляя размерностей ядер $Ker(\tau^i-Id)$.
В дальнейшем будем считать, что $n$ взаимно просто с $q$.

\smallskip
\textbf{a.} Рассмотрим многочлен $P(X)=X^n-1$ и базис $B=\{1, X,\dots , X^{n-1}\}$ пространства $\mathbb{F}_q[X]/(X^n-1)$. 
Доказать, что $X^n-1$ не имеет квадратных множителей и что $\tau$ --- автоморфизм.
Вычислить $\tau(X^i)$.
Вывести отсюда, что $\tau$ оставляет без изменения базис $B$ и что действие $\langle \tau \rangle$ на $B$ можно отождествить с действием $\langle q \rangle$ на $\mathbb{Z}_n$ ( $\langle \tau \rangle$ обозначает
подгруппу автоморфизмов, порожденную $\tau$, а $\langle q \rangle$ --- подгруппу группы $Aut(\mathbb{Z}_n)$, порожденную умножениями на $q$).

\smallskip
\textbf{b.}  Если дана подстановка $\sigma$ множества $[l,d]$, то рассмотрим эндо­морфизм $\sigma$ пространства $\mathbb{F}^d_q:\; e_i \mapsto e_{\sigma (i)}$. 
Какова размерность $Ker(\sigma-Id)$ (рассмотреть сначала цикл)? 
Доказать, что $dim\;Ker(\tau^i-Id)$ равна числу орбит в $\mathbb{Z}_n$ при умножении на $q^i$.

\smallskip
\textbf{c.} Объясните, как вычислить число и степени неприводимых мно­
жителей $X^n-1$ над $\mathbb{F}_q$. 
Рассмотреть многочлен $X^{12}-1$ над $\mathbb{F}_q$. 
Что будет, если $q\equiv 1\; (mod\; n)$? 
Предположим, что $n$ --- простое; каково
условие того, что циклотомический многочлен $\Phi_n(X)=(X^n-1)/(X-1)$ неприводим над $\mathbb{F}_q$?

\newpage

\thispagestyle{empty}

\bigskip



$$
\begin{aligned}
\textbf{Решения упражнений}
\end{aligned}
$$



\bigskip
\bigskip
\noindent\textbf{1. Простые сравнения}

\medskip
\textbf{a.} Степени 4 по модулю 9 равны 4,7 и 1. Теперь достаточно прове­
рить исходное сравнение для каждого из этих трех множителей.

\medskip
\textbf{b.} $n$ обратимо по модулю 7, а потому его порядок делит 6.

\medskip
\textbf{c.} Достаточно доказать, что $a^5\equiv a\; (mod\;10)$. 
Последнее сравнение выполняется по модулям 2 и 5.

\bigskip
\noindent\textbf{2. Критерии делимости}

\medskip
Легко доказать, что если $b\equiv 1\; (mod\; d)$, то число, записанное в си­стеме счисления с основанием $b$, делится на $d$ тогда и только тогда,
когда на $d$ делится сумма его цифр (действительно, $\sum n_i b^i\equiv \sum n_i\; (mod\; d)$). В десятичной системе счисления этому условию удовлетво­ряют числа 3 и 9. Простые критерии существуют также для чисел, де­лящих основание (критерий состоит в том, что последняя цифра числа
должна делиться на $d$), и для таких чисел, что $b\equiv -1\; (mod\;d)$ (в этом случае знакопеременная сумма цифр, взятая, начиная с любого конца, делится на $d$). Следовательно, в десятичной системе простые критерии существуют для чисел 2, 3, 5, 9 и 11. Можно расширить это множество, комбинируя критерии и не достигая при этом большой сложности: 6, 4 и 8 (по основанию 100) и т.д.

\bigskip
\noindent\textbf{3. Автоморфные числа}

\medskip
Эти числа удовлетворяют сравнению $x(x-1)\equiv 0\;(mod\;10^n)$. Поэто­му исходная проблема эквивалентна нахождению чисел $x$, удовлетворя­ющих этому сравнению по модулям $2^n$ и $5^n$.

Одно из чисел $x$ и $x-1$ нечетно, а потому обратимо по модулю $2^n$.

\noindent Следовательно,


$$
\begin{aligned}
x(x-1)\equiv 0\, (mod\; 2^n)\Longrightarrow x\equiv 0\; (mod\; 2^n)\;  \text{или}\;  x\equiv 1\; (mod\;2^n).
\end{aligned}
$$


\medskip
\noindent Аналогично, либо $x\equiv0\;(mod\;5^n)$, либо $x\equiv 1\; (mod\; 5^n)$. Следовательно,
существуют четыре потенциальных решения, в зависимости от того,

\newpage

\newtop{IV {\textit{Некоторые методы алгебраической алгоритмики}}}

\noindent чему равна правая часть сравнений. Согласно китайской теореме об остатках, достаточно вычислить один из кандидатов, $x$, другой равен
$10^n-x+1$. Одно из этих двух чисел имеет $n$ десятичных цифр и,
следовательно, является единственно достойным кандидатом.

\bigskip
\noindent\textbf{4. Нумерация Фибоначчи}

\medskip
Пусть $F_{n_1}$ --- наибольшее число Фибоначчи, меньшее $n$. Тогда
$F_{n_1+1}> n\geq F_{n_1}$, откуда $F_{n_1-1}=F_{n_1+1}-F_{n_1}> n-F_{n_1}$; и эту проце­дуру можно повторить. Процесс закончится за конечное число шагов,
поскольку в худшем случае мы придем к первому числу Фибоначчи $(F_2=1)$.

Осталось доказать единственность разложения. Для этого по индук­
ции докажем неравенство $F_n> F_{n-1}+F_{n-3}+F_{n-5}+\dots +F_{2\; \text{или}\; 3}$. 
Оно верно для $F_2$ и $F_3$ и, если оно верно для $F_n$, то


$$
\begin{aligned}
F_{n+2}=F_{n+1}+F_n>F_{n+1}+F_{n-1}+F_{n-3}+\dots
\end{aligned}
$$


\medskip
\noindent Следовательно, выбор $F_{n_1}$, в разложении единственно возможный.

\bigskip
\noindent\textbf{5. Простота чисел $(1\dots1)_b$}

\textbf{a.} Такое число равно $(b^k-1)/(b-1)$, где $k$ --- число его цифр. Если $k$ --- составное, то можно записать всегда верное тождество


$$
\begin{aligned}
\frac{b^{nm}-1}{b-1}=\frac{b^n-1}{b-1}(b^{m(n-1)}+b^{m(n-2)}+\dots+1).
\end{aligned}
$$


\medskip
\textbf{b.} Приведем доказательство, в котором предполагается, что $p\neq 3$.
Пусть $l$ --- порядок 10 по модулю $p$. 
Тогда $10^l-1\equiv 0\;(mod\;p)$. 
Если 3 обратимо по модулю $p$ (т.е. $p\neq 3$) , то последнее сравнение можно разделить на $9=10-1$, а тогда число, состоящее из $l$ единиц, делится на $p$. В качестве $l$ можно взять также число $p-1$, которое не обязатель­но будет порядком 10. Например, 7 делит 111$\,$111, 11 делит все числа, состоящие из четного числа 1, и т.д.

Например, $111\,111\,111 = З^2 \times 37 \times 333\,667$ и 10 имеет порядок 9 по модулю $333\,667$ и порядок 3 по модулю 37. 
Наименьшие простые числа, состоящие из 1, это


$$
\begin{aligned}
11,\;\; 1\,111\,111\,111\,111\,111\,111,\;\; 11\,111\,111\,111\,111\,111\,111\,111.
\end{aligned}
$$


\newpage

\newtop{Решения упражнений}

\noindent\textbf{6. Проблема показателя}

\medskip
Заметим, что $x$ (и $y$) принадлежит $U(\mathbb{Z}_m)$. То, что $x=y^s$, означает, что группа, порожденная $x$, порождена также и $y=x^r$, а последнее эквивалентно тому, что $r$ взаимно простое с порядком $x$. 
Если $s$ обрат­но к $r$ по модулю порядка $x$, то $x^{rs}=x\; (mod\; m)$ и $y^s=x\;(mod\; m)$.
Поэтому достаточно, чтобы $r$ было взаимно просто с порядком $U(\mathbb{Z}_m)$(который равен значению функции Эйлера $\varphi (m)$), и можно взять в ка­честве $s$ обратный к $r$ по модулю $\varphi (m)$.

$m=11\times 211,\; \varphi (m)=(11-1)\times (211-1)=2\,100=2^2\times 3\times 5^2\times 7$ и $r=11\times 13$ взаимно просто с $\varphi (m)$. После вычисления соотношения Безу $1\,307\times 143-89\times 2\,100=1$ можно взять $s=1\,307$.

\bigskip
\noindent\textbf{7. Применения китайской теоремы об остатках}
 
\medskip
Для первой системы достаточно найти частные решения для (1,0,0),
(0,1,0) и (0,0,1). Эти решения соответственно 105, 76 и 120. Следова­
тельно, общее решение есть $105_{x_1}+76_{x_2}+120_{x_3}$ по модулю 420.

Имеем $4^2\equiv 7\;(mod\;9)$ и 4 имеет порядок 3 в $U(\mathbb{Z}_9)$. Значит, $4^n\equiv 7\; (mod\;9)$ эквивалентно тому, что $n\equiv 2\; (mod\;3)$. Аналогично, $2^8\equiv 3\;(mod\; 11)$ и 2 имеет порядок 10 в $U(\mathbb{Z}_{11})$. 
Следовательно, $2^n\equiv 3\; (mod\; 11)$ эквивалентно $n=3\;(mod\; 10)$. Поэтому решение --- это $n\equiv 8\; (mod\; 30)$.

Кольцо $\mathbb{Z}[X]$ не является кольцом главных идеалов, но $1\cdot (3X^3+1)-3X\cdot X^2=1$ и китайская теорема об остатках применима. 
Общее реше­ние есть $(-3X^3)\cdot X^2+(3X^3+1)\cdot (2X+1)=-3X^5+6X^4+3X^2+2X+1$ по модулю $3X^5+X^2$.

\bigskip
\noindent\textbf{8. Восстановление обратных модулярных величин}

\medskip
\textbf{a.} Имеем $1-ab\equiv0\; (mod\;n)$, откуда $(1-ab)^2=1-ab(2-ab)\equiv 0\;(mod\; n^2)$. 
Если применить метод подъема (от модуля $n$ к модулю $n^2$)к корням многочлена $\alpha X-1$, то получим требуемую формулу. Можно напрямую проверить утверждение, касающееся $b^2(a-2ab)$, но можно
вычислить $(b(2-ab))^2$ и заметить, что $(2-ab)^2=3-2ab$ (так как в
кольце из $x^2=0$ следует, что $(1+x)^2=1+2x$, и можно применить это
к $x=1-ab$ в $\mathbb{Z}/n^2\mathbb{Z}$).

\medskip
\textbf{b.} Утверждение, касающееся последовательности $(b_k)_{k\geq0}$, вытекает
из предыдущего пункта.
Для $n=2$ и $a=11\,335$ можно взять $b_0=1$ и
тогда $b_4$ --- обратное к $a$ по модулю $2^{16}$:


$$
\begin{aligned}
b_1=(2-a)\; mod\; 2^2=3,\qquad b_2&=3(2-3a)\; mod\; 2^4=7,\\
b_3=7(2-7a)\; mod\; 2^8=119,\qquad b_4=&119(2-119a)\; mod\; 2^16=48\,503.
\end{aligned}
$$


\newpage

\newtop{IV {\textit{Некоторые методы алгебраической алгоритмики}}}

\textbf{c.} Из того, что $1-ab\equiv 0\;(mod\;n)$ и $1-ab'\equiv 0\; (mod\; n')$, получаем: 


$$
\begin{aligned}
(1-ab)(1-ab')=1-a(b+b'-abb')\equiv 0\; (mod\; nn'),
\end{aligned}
$$


\medskip
\noindent а потому $b+b'-abb'$ --- обратное к $a$ по модулю $nn'$.

\bigskip
\noindent \textbf{9. Арифметика по модулю $2^q-1$}

\medskip
\textbf{a.} Отображение $\mathbb{Z}\in n\mapsto a^n\in U(\mathbb{Z}_{a^q-1})$ --- гомоморфизм с ядром $q\mathbb{Z}$, откуда следует искомый результат.

\medskip
\textbf{b.} Используя то, что $d=\;\text{НОД}(n,m)$ --- линейная комбинация $n$ и $m$, получим:


$$
\begin{aligned}
b\;|\; \text{НОД}  (a^n-1,\; a^m-1)\;&\Longleftrightarrow\; a^n\equiv 1\;(b)\\
 \text{и}  \; a^m\equiv 1\; (b)\;\Longleftrightarrow\; &a^d\equiv 1(b)\;\Longleftrightarrow\; b\;|\; a^d-1,
\end{aligned}
$$


\medskip
\noindent откуда $a^d-1=\text{НОД}(a^n-1,\; a^m-1)$. 
Следовательно, НОД$(2^n-1,\; 2^q-1)$ равен $2^{\text{НОД}(n,q)}-1$ и равен 1 тогда и только тогда, когда НОД$(n,q)=1$.
Если $n'$ обратное к $n$ по модулю $q$, то:


$$
\begin{aligned}
\frac{(2^n)^{n'}-1}{2^n-1}\times (2^n-1)=2^{nn'}-1\equiv 2-1=1\; (mod\; 2^q-1),
\end{aligned}
$$


\medskip
\noindent откуда $1+2^n+2^{2n}+\dots+2^{(n'-1)n}\equiv 1+2^{n\; mod\; q}+2^{2n\; mod\; q}+\dots+2^{(n'-1)n\; mod\; q}\; (mod\; 2^q-1)$ --- обратное к $2^n-1$ по модулю $2^q-1$.

По модулю 32 имеем $2\cdot 13=26,\; 3\cdot 13=7,\; 4\cdot 13=20,\; 5\cdot 13=1$,
а потому $1+2^{13}+2^{26}+2^7+2^{20}=68\,165\,761$ --- обратное к $2^{13}-1$ по
модулю $2^{32}-1$.

\bigskip
\noindent \textbf{10. Перечислительное упражнение}

\medskip
Таких элементов нет ни в $\mathbb{Z}_{25}$, ни в $U(\mathbb{Z}_{40})$, поскольку их порядки не делятся на 10 (порядок $U(\mathbb{Z}_{40})$ равен $\varphi (40)=16$).

В группе $\mathbb{Z}_{10n}$ имеется $\varphi (10)=4$ элемента порядка 10, которые имеют вид $kn$, $1\leq k< 10$ и НОД$(k,\;10)=1$. 
Эти элементы: $n,\; 3n,\; 7n$ и $9n$. 
Это рассуждение можно применить к группам $\mathbb{Z}_{50}$, $\mathbb{Z}_{100}$, а
также к $U(\mathbb{Z}_{31})$, которая является циклической группой порядка 30, и
к $U(\mathbb{Z}_{50}) \simeq U(\mathbb{Z}_{2}) \times U(\mathbb{Z}_{5^2})\simeq U(\mathbb{Z}_{5^2})$ --- циклической группе порядка 20.

Поскольку $2^5\equiv 1\; (mod\; 31)$, то $(-2)^{10}\equiv 1\; (mod\; 31)$, и можно про­верить, что —2 имеет порядок 10. Остальные элементы порядка 10 в
$U(\mathbb{Z}_{31})$ это $(-2)^3=-8$, $(-2)^7=-2^5\cdot 2^2=-4$ и $(-2)^9=-2^5\cdot 2^4=-16$.

\newpage

\newtop{Решения упражнений}

Элемент 2 порождает $U(\mathbb{Z}_{25})$, поскольку, с одной стороны, $2^5\equiv 2\; (mod\; 5)\Rightarrow 2^5\not\equiv 1\; (mod\; 25)$, а с другой,$2^4\not\equiv 1\; (mod\; 25)$. 
Отсюда по­лучаем, что 4 — элемент порядка 10 по модулю 25. Так как $29\equiv 4\;(mod\; 25)$ и $29 \equiv 1\; (mod\; 2)$, то 29 — элемент порядка 10 в $U(\mathbb{Z}_{50})$.

Для любого элемента $x\in \mathbb{Z}_n\times \mathbb{Z}_n$ выполняется $nx=0$. 
Чтобы найти элементы порядка $n$, нужно избавиться от элементов, удовлетворяю­щих равенству $dx=0$ для $d\; |\; n$ и $d<n$. В $\mathbb{Z}$
 имеется $n$ --- $\varphi(n)$ та­ких элементов. 
Следовательно, число элементов порядка $n$ в $\mathbb{Z}^2_n$ равно $n^2-(n-\varphi(n))^2=\varphi(n)(2n-\varphi(n))$.
Для $n=10$ получается 64 элемента.

Группа $U(\mathbb{Z}_{100})$ изоморфна $U(\mathbb{Z}_{4})\times U(\mathbb{Z}_{5^2})$, т.е. $G=\mathbb{Z}_{2}\times\mathbb{Z}_{20}$.
Множество элементов порядка 10 в $G$ это $H-H'$ с $H=\{z\in G\;|\;10z=0\}$ и $H'=\{z\in H\;|\;5z=0$ или $2z=0\}$. 
Очевидно, $H=\mathbb{Z}_2\times2\mathbb{Z}_{20}\simeq \mathbb{Z}_2\times \mathbb{Z}_{10}$
и $|H|=20$. 
Можно проверить, что $|H'|=8$, а пофизм $U(\mathbb{Z}_{100})\simeq U(\mathbb{Z}_{4})\times U(\mathbb{Z}_{5^2})$, используя то, что 2 порождает $U(\mathbb{Z}_{5^2})$, а 3 порождает $U(\mathbb{Z}_4)$:
тому $|H-H'|=12$.
Можно в явном виде построить изомор

$$
\begin{aligned}
51\equiv3\; (mod\;4),\quad 51\equiv1\;(mod\;5^2&),\\
77\equiv 1\;(mod\; 4),\qquad 77&\equiv 2\; (mod\; 5^2),
\end{aligned}
$$


\noindent и, следовательно, $U(\mathbb{Z}_{100})$ есть прямое произведение подгруппы порядка
2, порожденной 51, и подгруппы порядка 20, порожденной 77. Так как $77^2=29\; (mod\; 100)$ и $29^2=41\; (mod\; 100)$, то элементы порядка 10 в $U(\mathbb{Z}_{100})$ суть $51^i\cdot 29^j$ и $41^k$ с $i=0,1,j=1,3,5,7$ и $k=1,2,3,4$.

Элемент 2 имеет порядок 10 в $U(\mathbb{Z}_{1023})$, поскольку $2^{10}\equiv 1\;(mod\; 1023)\;(\text{и}\; 2^i\not\equiv 1\;(mod\; 1023)\;\text{для}\; i<10)$. Но группа $U(\mathbb{Z}_{1023})$ циклическая, так
как $1023=2^{10}-1=(2^5-1)(2^5+1)=31\times 3\times 11$. Используя то, что $U(\mathbb{Z}_{1023})\simeq U(\mathbb{Z}_{31})\times U(\mathbb{Z}_{3})\times U(\mathbb{Z}_{11})\simeq U(\mathbb{Z}_{30})\times U(\mathbb{Z}_{2})\times U(\mathbb{Z}_{10})$, можно изучить эту группу достаточно подробно.

\bigskip
\noindent \textbf{11. Сравнение $\a^{k\varphi(n)+1}\equiv a\;(mod\;n)$ для любого $a$}

\medskip
Если $a$ обратимо по модулю $n$, то свойство верно. Если $p$ --- простое число, делящее $n$, то оно взаимно просто с $n/p$ и $\varphi(n)=\varphi(p)\varphi(n/p)$.
Из того, что $p^m\equiv p\; (mod\; n/p)$ и $p^m\equiv p\;(mod\; p)$, получаем $p^m \equiv p\; (mod\; n)$. Следовательно, сравнение верно для чисел, обратимых по
модулю $n$, и для простых делителей $n$; значит оно верно и для всех целых чисел.

Это сравнение не выполняется для некоторого $n$: например, взяв
$n=p^2$, мы не получим ни для какого $m>1$ сравнение $p^m\equiv p\;(mod\;p^2)$.



\newpage

\newtop{IV {\textit{Некоторые методы алгебраической алгоритмики}}}

\textbf{12. Целые числа, удовлетворяющие модулярным
ограничениям}

\medskip
Используя китайскую теорему об остатках, построим таблицу
$v=[v_0,v_1,\dots,v_{l-1}]$ ($l$ --- произведение мощностей $V_i$), такую, что $v_j\; mod\; m_1\in V_1,\dots,v_j\; mod\; m_r\in V_r$. Упорядочим эту таблицу и по­строим новую $(Cyclic\_Increments)$, индексированную, начиная с 0 (эта
вторая таблица может быть получена из первой), следующим образом:$[v_1-v_0,\dots,v_{l-1}-v_{l-2},v_0-v_{l-1}+m]$, где $m=m_1\dots m_r$ --- произ­ведение модулей. Если теперь рассмотрим последовательность $(n_k)_{k\geq0}$, начинающуюся с $n_0=v_0$, обозначенную как $Initial\_Value$ и определяе­мую через $n_{k+1}=n_k+Cyclic\_Increments(k\; mod\; l)$, то легко проверить, что $n_k\equiv v_{k\; mod\; l}\;(mod\; m)$. Это приводит к алгоритму:

\medskip

\MYvrule$k=0;$\\
$n-Initial\_Value;$\\
$while()$\\
$\{traiter(n);$\\
$n=n+Cyclic\_Increments(k);$\\
$k=fmod(k+1,1);\}$

\medskip

Например, ограничения {1,2,6} по модулю 7 и \{3,5\} по модулю 8
дают 6 ограничений по модулю $7\times8=56$, которые записываются в
таблицу $[v_0<\dots<v_5]=[13,27,29,37,53,51]$. 
Отсюда получаем: 


$$
\begin{aligned}
Cyclic\_Increments=[v_1-v_0,v_2-v_1,v_3-v_2,v_4-v_3,v_5-v_4,v_0-v_5+56].
\end{aligned}
$$


\medskip
Остается объяснить построение таблицы $v_i$ с помощью китайской те­оремы об остатках. 
С очевидными обозначениями моделируется цикл:


$$
\begin{aligned}
&\text{for} (;x_1 \in V_1, \dots, x_r \in V_r;)\\ 
&\{x \longleftarrow chinois_{m_1,\dots ,m_r} (x_1,\dots,x_r);
\text{поместить}\; x\; \text{в}\; v;\}\\
\end{aligned}
$$


\columnseprule = 0.3pt\columnsep=24pt 
\begin{multicols}{2}
\noindent Эффективная реализация этого
цикла требует наличия биекции
интервала $[0,l[$ на произведении
$V_1\times V_2\times\dots\times V_r$
 (напоминаем, что
$l$ --- произведение мощностей $V_i$).
При этом предполагается, что
каждое $V_i$ индексируется числа­
ми от 0 до $l_i-1$, где $l_i=|V_i|$. В
качестве биективного отобра­жения можно использовать систему со 

$$
\begin{aligned}
&\text{for}(k = 0; k < l; ++k)\\ 
&\{q \longleftarrow k;\}\\
&\text{for}(i = 1; i <= r; ++r)\\ 
&\{\\
&x[i] \longleftarrow V_i(q \text{ mod } l_i);q \longleftarrow [q / l_i];\\
&v[k] \longleftarrow chinois_{m_1,\ldots,m_r} (x_1,\ldots,x_r);\\
&\}
\end{aligned}
$$
\end{multicols}
\noindent смешанным основанием $(l_1,l_2,\dots,l_r):$ если $k\in [0,l_1\dots l_r[$, то разделим с остатком $k$ на $l_1$, $k=q_2l_1+k_1$ c $0\leq k_1< l_1$, затем то же самое проделаем с $q_2$, которое принадлежит интервалу $[0,l_2\dots l_r[$, его делим на $l_2$, и т.д. В итоге числу $k\in [0,l_2\dots l_r[$

\newpage

\newtop{Решения упражнений}

\noindent поставим в соответствие $r$-ку $(k_1,k_2,\dots,k_r)$, которая принадлежит ис­
комому произведению. Таким образом, мы приходим к алгоритму, ко­
торый написан справа.

\bigskip
\textbf{13. Пары взаимно простых элементов, сохраняющих НОК}

\medskip
\textbf{a.} Пары $(m,n)$, $(m',n')$ имеют одно и то же НОК, а потому
$m'n'=mn/g$. Равенство $m'=(n/n')(m/g)$ доказывает, что $/g$ делит
$m'$, откуда и следует искомое утверждение.

\smallskip
\textbf{b.} Предыдущий результат показывает, что для определения $m'$ и $n'$
достаточно исключить из $n$ все неприводимые элементы, которые делят
одновременно $n$ и $m/g$. Это в точности те неприводимые, показатель
которых в $n$ меньше, чем в $m$.

Отсюда непосредственно следует алгоритм вычисления $m'$ и $n'$,
\begin{wraptable}{}{0.4\textwidth}
\begin{tabular}{|l|}
\hline
\begin{tabular}[c]{@{}l@{}}
$n'\longleftarrow n $; $g \longleftarrow \text{ НОД} (n,m)$;\\
while()\\ 
\{\\
$d \longleftarrow \text{НОД} (n', m/g)$\\
if (d == 1) break;\\
$n'\longleftarrow n'/d$;\\
\}\\
$m' \longleftarrow nm/gn'$;\\
\end{tabular} \\
\hline
\end{tabular}%
\end{wraptable}
однако его доказательство не­сколько сложнее. Рассмотрим ин­вариант цикла точно перед выхо­дом из цикла. Число $p$, разумеется, неприводимо. Посмотрим сначала, какие можно сделать выводы из этого свойства. Мы заинтересованы в доказательстве его инвариантности.

%%%%%%%%%%%%%%%%%%%%%%%%%%%%%%%%%%%%%%%%%%%%%%%%%
Инвариантное утверждение алгоритма состоит в том, что любой
простой множитель $n/n'$ делит $m/g$. Поскольку при выходе из цикла, $n'$
взаимно просто с $m/g$, то $n/n'$ удовлетворяет этому утверждению, а
следовательно, и все произведение $\frac{nm}{gn'}$. Свойства делимости и сохране­ния НОК тривиальны.

Докажем теперь, что свойство инвариантно. При входе в цикл оно
верно, поскольку $n=n'$. После вычисления НОД имеем $(n'/d\; |\; n$ и,
если $p$ неприводимо, то

\begin{equation*}
p\;|\; \frac{nd}{n'}\Longrightarrow
 \begin{cases}
   p\;|\; \frac{n}{n'}\Rightarrow p\;|\; \frac{m}{g}, \text{по предложению индукции,}\\
   p\;|\; d\Rightarrow p\;|\; \frac{m}{g}, \text{по определению d.}
 \end{cases}
\end{equation*}


\noindent На следующем этапе цикла (если мы еще не вышли из него), заменяя в
предыдущих формулах $n'/d$ на $n'$, снова получим инвариантную фор­мулу.

\bigskip
\textbf{14. Сравнение $P(X)\equiv 0\; (mod\; p^n)$ и целые $p$-адические числа}

\medskip
\textbf{a.} Можно записать $P(Y)-P(Z)=(Y-Z)S_P(Y,Z)$, где $S_P(Y,Z)$ --- многочлен с целыми коэффициентами, удовлетворяющий равенству

\newpage

\newtop{IV {\textit{Некоторые методы алгебраической алгоритмики}}}

\noindent $S_P(X,X)=P'(X)$. Тогда $(y-z)S_P(y,z)\equiv 0\; (mod\; p^n)$. Но $p$ не делит $S_P(y,z)$, так как $S_P(y,z)\; mod p\equiv S_P(y\; mod\; p,z\; mod\; p)\equiv S_P(x,x)=P'(x)\;(mod\; p)$ и, следовательно, $p^n$ делит $y-z$.

\medskip
\textbf{b.} Первые вопросы не трудны. Приведем начальное 7-адическое
разложение $\sqrt{2}$ (здесь таким образом обозначен квадратный корень из 2, сравнимый с 3 по модулю 7):

$$
\begin{aligned}
\sqrt{2}=3+1\cdot 7&+2\cdot 7^2+6\cdot 7^3+1\cdot 7^4+2\times 7^5+1\cdot7^6+2\cdot 7^7+\\
&+4\cdot 7^8+6\cdot 7^9+6\cdot 7^{10}+2\cdot 7^{11}+1\cdot 7^{12}+\dots\\
x_1=3,&x_2=10,x_3=108,x_4=2\,166,\\
x_5=&38\,181,x_6=155\,830,\dots,x_{13}=19\,757\,775\,942.
\end{aligned}
$$

\medskip
Многочлен $X^3-50X^2-2X+100=(X^2-2)(X-50)$ имеет целый
корень 50: переходя к $x_1=1=50\; (mod\; 7)$, получим стационарную
последовательность $x_2=1,x_i=50$ для $i\geq 3$, т.е. $50=1+0\cdot 7+1\cdot 7^2+0\cdot 7^3+0\cdot 7^4+\cdot$. При $x_1=3$ получим такую же последовательность,
как для $X^2\equiv 2\; (mod\; 7^n)$.

Тогда $P(y)\equiv P(x_i)\equiv 0\; (mod\; p^i)$, а потому $P(y)=0$; иначе говоря,
существует $y\equiv x\; (mod\; p)$ такое, что $P(y)=0$.

\medskip
\textbf{c.} Очевидно, это кольцо. Отображение $m\mapsto m\cdot 1=(m,m,m,\dots)$ является инъекцией $\mathbb{Z}$ в $A$, поскольку, если $m\equiv 0\; (mod\; p^i)$ для любого $i$,
то $m$ равно нулю.

\medskip
\textbf{d.} Если $x$ обратимо в $A$, то понятно, что $x_1$ обратимо в $\mathbb{Z}_p$
. Обратно, если $x_1$ обратимо по модулю $p$, то каждое $x_i$ обратимо по модулю
$p^i$ (так как $x_i\equiv x_1\; (mod\; p)$). Если $y_i$ --- обратное к $x_i$- по модулю $p_i$,
то последовательность $y=(y_i)_{i\geq 1}$ принадлежит $A$ (поскольку обрат­ный элемент единствен) и $xy=1$. Имеем: $p^i\; |\; x$ тогда и только тогда,
когда $x_i\equiv 0\; (mod\; p^i)$. Если $x$ не равно нулю, то через $j\geq 0$ обозначим
наименьшее целое число c $p^{j+1}\;\nmid \; x$  . Имеем $x=p^jx'$ и $x'$ обратимо (ибо $p\; \nmid \; x'$). Аналогично, если $y\neq 0$, то $y=p^ky'$ и $y'$
 обратимо. Следовательно, $xy=p^{j+k}x'y'$, не равно нулю.

\medskip
\textbf{e.} Элемент $x_i$ --- решение сравнения $P(x_1)\equiv 0\; (mod\; p)$,
$x_{i+1}=x_i-P'(x_1)^{-1}P(x_i)=x_i-P'(x_i)^{-1}P(x_i)\; (mod\; p^{i+1})$. Так как $P'(X)=(p-1)X^{p-2}$, то $P'(x_i)^{-1}mod\; p=-x_i$ и $x_{i+1}=x_i+x_i(x^{p-1}_i-1)=x^p_i$. Следовательно, $x_i=x^{p^i}_1$. Впрочем, можно проверить, что
$f(x_1)=(x_1,x^p_1,x^{p^2}_1,\dots)$ --- элемент $A$ и что $x_1\mapsto f(x_1)$ --- это инъекция $U(\mathbb{Z})p)$ на $A$.

\newpage

\newtop{Решения упражнений}

\textbf{15. $p$-адические разложения дробей}

\medskip
\textbf{a.} Имеем $1992=(30432)_5$, откуда $1992=2+3\cdot 5+4\cdot 5^2+0\cdot 5^3+3\cdot 5^4$.

$$
\begin{aligned}
-1=(p-1)+(p-1)\cdot p+(p-1)\cdotp^2+(p-1)\cdot p^3+\dots
\end{aligned}
$$

\medskip
\noindent разложение —1.

$$
\begin{aligned}
2+3\cdot 5+1\cdot 5^2+3\cdot 5^3+1\cdot 5^4+3\cdot 5^5+\dots &=\\
=2+5(3+1\cdot 5)(1+5^2+5^4+\dots)=2&+\frac{5\cdot 8}{1-5^2}= 2-\frac{5}{3}=\frac{1}{3}.
\end{aligned}
$$

\medskip
\noindent Если $x=a_0+a_1p+a_2p^2+\dots$, то целое $p$-адическое число $(p-a_0)+(p-1-a_1)p+(p-1-a_2)p^2+\dots$ имеет обычную форму, когда $a_0\neq 0$.
Прибавляя это число к $x$, получим 0, а значит, это $-x$. Если $a_0=0$, то
достаточно провести аналогичное преобразование, начиная с первого
$a_i\neq 0$. В частности:

$$
\begin{aligned}
\frac{-1}{3}=3+1\cdot 5+3\cdot5^2+1\cdot 5^3+3\cdot5^4+1\cdot 5^5+\dots
\end{aligned}
$$

\medskip
\textbf{b.} Произведение частичной суммы $1+1\cdot p^n+\dots +1\cdot p^{in}$сравнимо с $1\; (mod\; p^{n(i+1)})$, а потому:

$$
\begin{aligned}
(1-p^n)\times (1+1\cdot p^n+1\cdot p^{2n}+1\cdot p^{3n}+\dots)=\frac{1}{1-p^n}.
\end{aligned}
$$

\medskip
\noindent Если $p$-адическое разложение числа $x$ почти периодическое, то существуют два таких целых числа $a$ и $b$, что

$$
\begin{aligned}
x=a+bp^{\mu}+bp^{\mu+\lambda}+bp^{\mu+2\lambda}+\dots=a+bp^{\mu}\frac{1}{1-p^\lambda},
\end{aligned}
$$

\medskip
\noindent и число $x$ рационально.

\medskip
\textbf{c.} Было бы соблазнительно для всякой последовательности записать $ba_0\equiv a\; (mod\; p)$, если только $ba_0\equiv a\; (mod\; pA)$, где $A$ обозначает
кольцо целых $p$-адических чисел. Если мы докажем, что $pA\cap \mathbb{Z}=p\mathbb{Z}$,
то получим и наше сравнение в $\mathbb{Z}$. Идеал $pA\cap \mathbb{Z}$ не содержит 1 (в про­
тивном случае $p$ было бы обратимо в $A$), а потому $p\mathbb{Z}\subset pA\cap \mathbb{Z}\subsetneq \mathbb{Z}$ и
максимальность $p\mathbb{Z}$ приводит к нашему утверждению. Тогда:

$$
\begin{aligned}
\frac{a-ba_0}{b}=p(a_1+a_2p+a_3p^2+\dots&),\\
\text{откуда:}\qquad \frac{(a-ba_0)/p}{b}=a_1&+a_2p+a_3p^2+\dots;
\end{aligned}
$$

\newpage

\newtop{IV {\textit{Некоторые методы алгебраической алгоритмики}}}

\noindent теперь получаем:

$$
\begin{aligned}
ba_1\equiv \frac{a-ba_0}{p}(p),\quad ba_2\equiv \frac{\frac{a-ba_0}{p}-ba_1}{p}&(p),\\
ba_3\equiv \frac{\frac{\frac{a-ba_0}{p}-ba_1}{p}-ba_2}{p}&(p),\dots,
\end{aligned}
$$

\medskip
\noindent Положим

$$
\begin{aligned}
y_1=\frac{a-ba_0}{p},\quad y_2=\frac{y_1-ba_1}{p},\quad y_3=\frac{y_2-ba_2}{p},
\end{aligned}
$$

\medskip
\noindent если $ba_i\equiv y_i\; (mod\; p)$ (удобно положить $y_0=a$). Отсюда $a_i=(b^{-1}y_i)\; mod\; p$, где $b^{-1}$ обозначает обратное к $b$ по модулю $p$. На множестве целых чисел определим функцию $f(x)=\frac{x-b((b^{-1}x)\; mod\; p)}{p}$ и тогда $y_{i+1}=f(y_i)$. Имеем $|f(x)|\leq\frac{|x|}{p}+\frac{|b|(p-1)}{p}$. Для того, чтобы из $|x|\leq K$ следовало $|f(x)|\leq K$, достаточно взять такое $K$, что $\frac{K}{p}+\frac{|b|(p-1)}{p}\leq K$, что
выполняется при $K\geq |b|$. Достаточно теперь взять $K\geq max(|a|,|b|)$, чтобы $a\in [-K,K]$ и $f([-K,K])\subset [-K,K]$. Отсюда выводим, что по­следовательность $y_i$ почти периодическая, а значит, таковой является
и последовательность $a_i=(b^{-1}y_i)\; mod\; p$. Опытным путем можно про­верить, что

$$
\begin{aligned}
ba_0+y_1p=a,\quad ba_0+ba_1p+y_2p^2=a,\quad ba_0+ba_1p+ba_2p^2+y_3p^3=a,
\end{aligned}
$$

\medskip
\noindent что дает $b(a_0+a_1p+a_2p^2+\dots)=a$.

Если применить формулы, использованные в упражнении 14 к многочлену $P(X)=bX-a$, то переходя к решению $x_1=b^{-1}a\; (mod\; p)$ и замечая, что $P'(X)=b$ --- постоянный многочлен, получим:

$$
\begin{aligned}
a_i=P'(x_1)^{-1}\frac{-P(x_i)}{p^i}mod\; p=b^{-1}\frac{-P(x_i)}{p^i}mod\; p,\qquad x_{i+1}=x_i+a_ip^i,
\end{aligned}
$$

\medskip
\noindent и можно положить $x_0=0$. Итак, $P(x+y)=P(x)+by$, откуда: 

$$
\begin{aligned}
\frac{-P(x_{i+1})}{p^{i+1}}=\frac{-P(x_i+a_ip^i)}{p^{i+1}}=\frac{-P(x_i)-ba_ip^i}{p^{i+1}}=\frac{-P(x_i)/p^i-ba_i}{p},
\end{aligned}
$$

\medskip
\noindent и получаем те же формулы для $y_i=-P(x_i)/p^i$. Чтобы получить алгоритм, находящий период, используем, например, алгоритм Брента.

\newpage

\newtop{Решения упражнений}

\textbf{16. Поиск образующих элементов циклических групп}

\medskip
Если порядок $x$ равен $n$, то возьмем $x_i=x$ для любого $i$. Обратно, если $z_i=x^{{n/p^{\alpha_i}_i}_i}$, то $z_i^{p_i^{\alpha_i}}=1$ и $z_i^{p_i^{\alpha_i-1}}\neq 1;\; z_i$ имеет порядок $p_i^{\alpha_i}$, а
порядок $z_1\dots z_r$ равен $n$.

Перейдем к алгоритму. В нем функция succ (связанная со струк­
турой порядка группы) позволяет пробежать $G$ и записать $p_i$ и $\alpha_i$ в таблицу. На этом этапе имеем три переменных $g$, $j$ и $x$. Переменная
\begin{wraptable}{}{0.5\textwidth}
\begin{tabular}{|l|}
\hline
\begin{tabular}[c]{@{}l@{}}
$g \longleftarrow e$; $x \longleftarrow \text{succ}(e)$; $j \longleftarrow 1$;\\
while()\\ 
\{\\
	for (i = j; i <= r; ++i)\\ 
	\{ 
		if ($x^{n/{p_i}}$ != $1$)\\ 
		\{
			$g \longleftarrow g \times x^{{n/{p_i}}^{\alpha_i}}$;\\
			if (j == r)  exit = 1;\\
			$(p_i, \alpha_i \longleftrightarrow \longleftrightarrow (p_j, \alpha_j) $;\\
			$j \longleftarrow j + 1$;\\
		\}\\
	\}\\
	if (exit == 1) break;\\
	$x \longleftarrow \text{succ}(x)$\\
\}\\
return $g$;

\end{tabular} \\ 
\hline
\end{tabular}%

\end{wraptable}
$j$ --- это индекс в таблице $[p_1,\dots,p_r]$, указывающий, что эле­менты порядка $p_k^{\alpha_k}$ для $k<j$
были найдены и произведение
этих элементов записано в $g$.
Начиная с $x$, попытаемся по­лучить элементы (вида $x^{n/p_i^{\alpha_i}}$),
имеющие порядок, не полученный ранее. Для этого исполь­зуется второй индекс, пробе­гающий отрезок $[p_j,\dots,p_r]$. В
случае успеха изменяется зна­чение переменной $g$ и ставший
ненужным порядок $(p_i,\alpha_i)$ за­меняется на $(p_j,\alpha_j)$, который еще не был получен.

Несколько улучшений, необходимых для построения алгоритма. В
действительности вычисление $x^{n/p_i}$ включает в себя вычисление $x^{n/p_i^{\alpha_i}}$,
но при этом значение первого числа (вероятно) в дальнейшем не пона­добится. Поэтому процедуру вычисления проводят в два этапа: снача­ла вычисляют $x^{n/p_i^{\alpha_i}}$, а затем возводят результат в степень $p_i^{\alpha_{i-1}}$. На
практике вместо таблиц $[p_1,\dots,p_r]$ и $[\alpha_1,\dots,\alpha_r]$ удобнее использовать
таблицы $[u_1,\dots,u_r]$ и $[v_1,\dots,v_r]$ с $u_i=p_i^{\alpha_{i-1}}$ и $v_i=n/p_i^{\alpha_i}$. Тест $x^{n/p_i}=1$ сводится тогда к присваиванию $y\longleftarrow x^{v_i}$ и проверке $y^{u_i}=1$. Конечно,
присваивание $g\longleftarrow g\times x^{n/p_i^{\alpha_i}}$ заменяется на $g\longleftarrow g\times y$, а команда
$(p_i, \alpha_i) \longleftrightarrow (p_j,\alpha_j)$ на $(u_i, v_i) \longleftrightarrow (u_j,v_j)$ и даже на $(u_i, v_i) \longleftarrow (u_j,v_j)$.

\bigskip
\textbf{17. Вокруг теоремы Лагранжа}

\medskip
\textbf{a.} Предположим, что $G$ коммутативна. Так как $y\mapsto xy$ --- биекция $G$ на себя, то $\prod_{y\in G}y=\prod_{y\in G}xy=x^n\prod_{y\in G}y$, откуда $x^n=1$. В общем

\newpage

\newtop{IV {\textit{Некоторые методы алгебраической алгоритмики}}}

\noindent случае, для произвольной подгруппы $H$, отношение $Hx = Hy$ --- отношение экваивалентности на множестве смежных классов $Hx$ имеющих одну и ту же мощность $\mid H\mid $. Поэтому $\mid H\mid $ делит $\mid G\mid $.

\medskip
\indent\textbf{b}.\quad Группа $G$ обратимых по модулю $a^n - 1$ элементов имеет порядок \\ $\varphi(a^n-1)$. Так как $a^n \equiv 1 ( \mod a^n -1)$ и $ a^i \not\equiv 1 (\mod a^n -1)$ для $0<i<n$, то порядок $a$ равен $n$; значит $n$ делит $\varphi(a^n-1)$.

\bigskip
\noindent\textbf{18. Утверждение, обратное китайской теореме об остатках}

\medskip
\indent \textbf{a.}\quad Пусть $\Omega$ --- мультипликативная группа, а $n$ -- произведение порядков множителей: $n$ --- порядок $\Omega$. Пусть $m$ --- НОК порядков множителей. Тогда $m$ делит $n$ и $\omega^n=1$ для любого $\omega \in \Omega$. Если $\Omega$ --- циклическая, то $m = n$, что доказывает, что порядки множителей взаимно просты. Кроме того, каждый множитель является подгруппой ( или факторгруппой) $\Omega$, а потому цикличен. Обратное утверждение хорошо известно.

\indent \textbf{b.} \quad Для указанных $n$ группы $U(\mathbb Z_n)$ циклические. Обратно, пусть $p_1^{\alpha_1}\ldots p_k^{\alpha_k}$ --- разложение на простые множители числа $n$. Множитель $U(\mathbb Z_{p_i^{\alpha_i}})$ цикличен тогда и только тогда, когда $p_i$ нечетно или $p_i = 2$ и \\$\alpha_i = 1, 2$. Порядок этого множителя $(p_i^{{\alpha_i} - 1}(p_i - 1)$ всегда делится на 2, кроме случая $p_i = 2$ и $\alpha = 1$. Отсюда следует результат ( поскольку порядки множителей должны быть взаимно просты).

\bigskip
\noindent \textbf{19. Замечание, касающееся конечных абелевых групп}

\medskip
\indent Если $\Omega_1$ и $\Omega_2$ --- две \textbf{циклические} группы, то $\Omega_2$ может быть представленна как подгруппа ( или факторгруппа) $\Omega_1$ тогда и только тогда, когда $\mid\Omega_2\mid$ делит $\mid\Omega_1\mid$. Известно, что конечная абелева группа изоморфна произведению циклических групп $\mathbb Z_{n_1} \times \cdots \times \mathbb Z_{n_r}$. Для любого $a \geqslant 2$ арифметическая прогрессия $ka+1, k=1, 2, \ldots,$ содержит простое число $p_1, \ldots, p_r$: 
\begin{align*}
p_1 \equiv 1\quad (mod\quad n&_1),\qquad p_2 \equiv 1 \quad (mod \quad n_2p_1), \\
&\ldots ,\qquad p_r \equiv 1 \quad(mod\quad n_rp_1\ldots p_{r-1}).
\end{align*} 
\noindent Все простые числа $p_i$ различны и $n_i\mid p_i-1$. Циклическую группу $\mathbb Z_{n_i}$ можно представить как подгруппу (или факторгруппу) группы $U(\mathbb Z_{p_i})$, а потому группа $\mathbb Z_{n_1} \times \cdots \times \mathbb Z_{n_r}$ представляется как подгруппа (или  факторгруппа группы $U(\mathbb Z_{p_1}) \times \cdots \times U(\mathbb Z_{p_r}) \simeq U(\mathbb Z_n)$ с $n = p_1\ldots p_r$.

\end{document}
