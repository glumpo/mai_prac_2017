\documentclass{mai_book}

%\usepackage{multirow}

\defaultfontfeatures{Mapping=tex-text}
\setdefaultlanguage{russian}

\usepackage{dsfont}% for mathds and more
%\usepackage{scrextend}

%\clearpage
\setcounter{page}{521} % ВОТ ТУТ ЗАДАТЬ СТРАНИЦУ
%\setcounter{thesection}{5} % ТАК ЗАДАВАТЬ ГЛАВЫ, ПАРАГРАФЫ И ПРОЧЕЕ.
\setcounter{equation}{10}% номер формул
% Эти счетчики достаточно задать один раз, обновляются дальше сами
% \newtop{ЗАГОЛОВОК}  юзать чтобы вручную поменть заголовок вверху страници

\begin{document}
\cleartop
\ \\
\ \\
\ \\
\ \\
\ \\
    \begin{center} \textbf{\LARGE{Упражнения}} \end{center} 

\noindent\textbf{1. Простые сравнения}\\
\ \newline
\hspace*{15pt}\textbf{a.} Доказать, что $6 \cdot 4 \equiv 6$ $(mod\ 9)$ для всякого $n\in \mathbb{N}$.\newline
\hspace*{15pt}\textbf{b.} Доказать, что если $7$ не делит $n$, то оно делит $n^6 - 1$.\newline
\hspace*{15pt}\textbf{c.}~Доказать,~что в десятичной системе счисления числа $n$ и $n^5$
име-\newline ют одну и ту же последнюю цифру.\\
\ \newline
\noindent\textbf{2. Критерий делимости}\\

Доказать, что число, записанное в десятичной системе счисления,\newline делится на 3 тогда и только тогда, когда сумма его цифр делится на 3.\newline Относительно каких еще делителей можно составить подобный про-
\newline стой критерий? Почему? Обобщить на системы счисления по другим \newline основаниям.\\
\ \newline
\noindent\textbf{3. Автоморфные числа}\\

Как найти числа из $n$ десятичных цифр, удовлетворяющие сравне-\newline нию $x^2 \equiv x$ $(mod\ 10^n)$ (число $9376$ таково, поскольку $9376^2 = 87909376)$.\\
\ \newline
\noindent\textbf{4. Нумерация Фибоначчи}\\

Доказать, что всякое целое число записывается единственным обра-\newline зом в системе счисления Фибоначчи, т.е. в виде $n = F_{n_1} + F_{n_2}+ \cdots + F_{n_k}$,\newline где $n_1 \geqslant n_2 + 2$, $n_2 \geqslant n_3 + 2$,$\ldots$,$n_{k-1} \geqslant n_k+2$, $n_k \geqslant 2$.\\
\ \newline
\noindent\textbf{5. Простота чисел ${(1\ldots1)}_b$}\\

\textbf{a.} Доказать, что число записываемое в виде ${(1\ldots1)}_{10}$ по основанию\newline $b$, не может быть простым, если число единиц в его записи есть простое\newline число.\newline
\hspace*{15pt}\textbf{b.} Доказать, что любое простое число, отличное от 2 и 5, делит\newline число ${(11\ldots1)}_{10}$, записанное в десятичной системе счисления. Привести\newline пример.\newline\newpage

\restoretop
\newtoplo{Упражнения}
\newtopre{IV\quad Некоторые методы алгебраической алгоритмики}
\noindent\textbf{6. Проблема показателя}\\

Пусть, например, $m = 2 321, r = 143, x = 2$ и $y = 745$. Проверить, что $y = x_r$ $(mod\ m).$ Можно ли найти такое $s$, $x = y^s$ $(mod\ m),$ и если да, то как?\\
\ \newline
\noindent\textbf{7. Применения китайской теоремы об остатках}\\

Решить следующие системы:
    $$x,x_i\in\mathbb{Z}, x\equiv
    \begin{cases}
    x_1~(mod\ 4), \\
    x_2~(mod\ 15), \\
    x_3~(mod\ 7),
    \end{cases}
    \
    n\in\mathbb{Z},
    \begin{cases}
    4^n \equiv 7~(mod\ 9),\\
    2^n \equiv 3~(mod\ 11).
    \end{cases}
    $$
    $$
    P(X)\in\mathbb{Z}[X], P(X)\equiv
    \begin{cases}
    X^2~(mod\ 3X^3+1),\\
    2X+1~(mod\ X^2).
    \end{cases}
    $$
    Кстати, является ли $\mathbb{Z}[X]$ кольцом главных идеалов?\\
\ \newline
\noindent\textbf{8. Восстановление обратных модулярных величин}\\

\textbf{a.} Пусть $b$ обратно к $a$ по модулю $n$. Проверить, что $b(2-ab)$ обратно\newline к $a$ по модулю $n^2$ и что $b^2(3-2ab)$ обратно к $a^2$ по модулю $n^2$.\newline
\hspace*{15pt}\textbf{b.} Определим последовательность ${(b_k)}_k\geqslant0$ следующим образом:\newline $b_0=b$ и $b_k=b_{k-1}(2-ab_{k-1})\ mod\ n^{2^k}.$ Проверить, что $b_k$ обратно к $a$ по модулю $n^{2^k}.$ Какое число обратно к 11\ 335 по модулю $2^{16}?$\newline
\hspace*{15pt}\textbf{c.} Если известно, что $b$ - обратное к $a$ по модулю $n$, а $b'$ - обратное\newline по модулю к $a$ $n',$ то как найти обратное к $a$ по модулю $nn'?$\\
\ \newline
\noindent\textbf{9. Арифметика по модулю $2^q-1$}\\

\textbf{a.} Пусть $a\geqslant2.$ Изучив отображение $\mathbb{Z} \ni n \mapsto a^n \in U{(\mathbb{Z}}_{a^q-1}),$ доказать, что сравнения $a^n \equiv a^m\ (mod\ a^q-1)$ и $n\equiv m\ (mod\ q)$ эквивалентны.\newline
\hspace*{15pt}\textbf{b.} Чему равен НОД $(a^n-1,a^m-1)?$ Вывести отсюда, что $2^n-1$\newline обратимо по модулю $2^q-1$ тогда и только тогда, когда $n$ обратимо\newline по модулю $q$ и найтиобратное к $2^n-1\ mod\ (2^q-1)$ как функцию от\newline обратного к $n\ mod\ q.$ Обратимо ли $2^{13}-1$ по модулю $2^{32}-1?$\\
\ \newline
    \noindent\textbf{10. Перечислительное упражнение}\\
    
Сколько элементов порядка 10 в следующих группах и каковы они?
    $$
    {\mathbb{Z}}_{25},\quad \mathbb{Z}_{50},\quad \mathbb{Z}_{100},\quad \mathbb{Z}_{10}\times\mathbb{Z}_{10},
    $$    
    
    $$
    U(\mathbb{Z}_{40}),\quad U(\mathbb{Z}_{50}),\quad U(\mathbb{Z}_{100}),\quad U(\mathbb{Z}_{31}),\quad U(\mathbb{Z}_{1023}).
    $$
    
\newpage

\noindent\textbf{11. Сравнение $a^{k\varphi(n)+1}\equiv a\ (mod\ n)$ для любого $a$}\\

Пусть $n$ - свободное от квадрата целое число. Доказать, что из\newline $m\equiv1\ (mod\ \varphi(n))$ следует  $a^m\equiv a\ (mod\ n)$ для некоторого (не обяза-\newline тельно обратимого по модулю $n$) числа $a$. Будет ли это свойство спра-\newline ведливым, если убрать условие свободы от квадрата?\\
\ \newline
\noindent\textbf{12. Целые числа, удовлетворяющие модулярным ограничениям}\\

    Пусть имеется $r$ попарно взаимно простых положительных модулей\newline $m_1,\ m_2,\ \ldots,\ m_r$ и множества $V_1,\ V_2, \ldots,\ V_r$ такие, что $V_1\subset [0,m_1]$,\newline $\ldots,V_r\subset[0,m_r].$ Выяснить, какие значения может принимать перемен-\newline ная $n\in \mathbb{N}$, удовлетворяющая ограничениям: $n\ mod\ p_i\ \in V_i$ для любого\newline $i=1,2,\ldots,r.$\\
\ \newline
\noindent\textbf{13. Пары взаимно простых элементов, сохраняющих НОК}\\

    Для фиксированных элементов $n$ и $m$ факториального кольца опре-\newline делить два взаимно простых элемента $n'$ и $m'$, делящих $n$ и $m$ соответ-\newline ственно и имеющих то же НОК, что и $n$ с $m$, не используя при этом\newline разложение исходных элементов на неприводимые множители.\newline
\hspace*{15pt}\textbf{a.} Доказать, что если $n$, $m$, $n'$ и $m'$ удовлетворяют поставленной\newline задаче и $g=\text{НОД}(n,m)$, то любой неприводимый элемент $p$, делящий\newline $m/g$, делит $m'$,но не делит $n'$.\newline
\hspace*{15pt}\textbf{b.} Вывести отсюда алгоритм (и его обоснование), который находит\newline $n'$ и $m'$, используя только деления и вычисления НОД.\\
\ \newline
\noindent\textbf{14. Сравнение $P(X)\equiv0\ (mod\ p^n)$ и целые $p$-адические числа}\\

    Это упражнение посвящено представлению целых $p$-адических чисел\newline через решение сравнения $P(X)\equiv0\ (mod\ p^n)$. Детали читатель может\newline посмотреть в [28]. Разумеется, $p$ обозначает простое число, а $P(X)$-\newline многочлен с целыми коэффициентами.\newline
\hspace*{15pt}Пусть $x$-решение сравнения $P(x)\equiv0\ (mod\ p)$. Предположим, что\newline $P'(x)\not\equiv0\ (mod\ p)$ и если $P'(x)^{-1}$ обозначает обратный к $P'(x)$ по моду-\newline лю $p$, то, как и в упражнение 34 главы I, построим последовательность\newline $(x_i)_{i\geqslant1}$:
    $$
    x_1=x, x_2=x_1-P'(x)^{-1}P(x_1),
    $$
    
    $$
    x_3=x_2-P'(x)^{-1}P(x_2), x_{i+1}=x_i-P'(x)^{-1}P(x_i).
    $$
Доказать, что $x_i$ - \textbf{единственные} сравнимые с $x$ по модулю $p$, реше-\newline ния сравнений $P(x_i)\equiv 0\ (mod\ p^i).$ Образно говоря, мы преобразовали\newline
       
\newpage

\noindent решение $x_1$ по модулю $p$ в решение $x_2$ по модулю $p^2$ (и это преобразо-\newline вание единственно), затем преобразовали это решение в решение $x_3$ по\newline модулю $p^3$ $\cdots$ В частности, $x_{i+1}\equiv x_i\ (mod\ p^i)$.\newline
\hspace*{15pt}\textbf{a.} Доказать, что если $y\equiv z\equiv x\ (mod\ p)$ и $P(y)\equiv P(z)\equiv 0$\newline $(mod\ p^n)$, то $y\equiv z\ (mod\ p^n)$ (указание: использовать равенство\newline $P(Y) - P(Z)=(Y-Z)Q(Y,Z)).$\newline
\hspace*{15pt}\textbf{b.} Последовательность $(x_i)_{i\geqslant1}$ зависит от выбора $x$ (и обратного к\newline $P'(x)$ по модулю $p$). Проверить, что переходя от $x$ к $x'\equiv x\ (mod\ p)$ по-\newline лучим последовательность $(x'_i)_{i\geqslant1}$ такую, что $x'_i\equiv x_i\ (mod\ p^i)$. Можно\newline ограничить $x_i$ так, что $0\leqslant x_i \leqslant p^i.$ Доказать, что в этом случае $x_i$\newline можно записать в виде $x_i=a_0+a_1\cdot p + a_2\cdot p^2+\cdots + a_{i-1}\cdot p^{i-1}$, где все\newline $a_i$ принадлежат интервалу $[0,p-1]$ и определены однозначно.\newline
\hspace*{15pt} Написать программу построения последовательности $(x_i)_{i\geqslant1}$ и эле-\newline ментов $(a_i)_{i\geqslant0}$. Рассмотреть пример $X^2-3\equiv 0\ (mod\ 7^n)$ и $x_1=2.$ Что\newline можно сказать о примере $X^3+50X^2-2X+100\equiv 0\ (mod\ 7^n)$ при $x_1=1$\newline и $x_1=3$? Более обобщенно, при каких условиях существует $y\in \mathbb{Z}$ такое, что\newline $y\equiv x_i\ (mod\ p^i)$ для любого $i$?\newline
\hspace*{15pt}\textbf{c.} Предыдущие вопросы наводят на мысль об изучении подмноже-\newline ства $A \subset \prod_{i\geqslant 1} \mathbb{Z}/p^i\mathbb{Z}$, состоящего из таких последовательностей $(x_i)_{i\geqslant1}$,\newline что $x_{i+1}\ (mod\ p^i)=x_i$. Доказать, что так определенное подмножество\newline является коммутативным унитарным кольцом $A$, содержащим  $\mathbb{Z}$ (а зна-\newline чит, имеющим нулевую характеристику). Это кольцо называется коль-\newline цом \textbf{ целых $p$-адических чисел.}\newline
\hspace*{15pt}\textbf{d.} Пусть $x \in A$. Доказать, что $x$ обратим тогда и только тогда,\newline когда $x_1\not\equiv 0\ (mod\ p)$. При каких условиях $p^i\mid x$? Вывести отсюда, что\newline кольцо $A$ без делителей нуля.\newline
\hspace*{15pt}\textbf{e.} Доказать, что последовательность $(x_i)_{i\geqslant1}$, соответствующая\newline сравнению  $P(X)\equiv 0\ (mod\ p^i)$ (при подходящих условиях), определяет\newline целое $p$-адическое число $\hat x$ такое, что $P(\hat x)=0$. Пусть $P(X)=X^{p-1}-1$\newline и $x_1\not\equiv 0\ (mod\ p)$. Описать целые $p$-адические числа, соответствующие\newline сравнению $P(X)\equiv0\ (mod\ p^i)$. Доказать, что в этом случае определено\newline инъективное отображение $U(\mathbb{Z}/p\mathbb{Z}) \rightarrow U(A)$.\\
\ \newline
\noindent\textbf{15. $p$-адические разложения дробей}\\

    Пусть $p$-простое число. Любой элемент кольца $p$-адических чисел\newline единственным образом записывается в виде 
    $$
    x=a_0+a_1p+a_2p+a_3p+\cdots ,\quad   0\leqslant a_i\leqslant p-1.   
    $$
\newline
\newpage

    \noindent Посмотрим, как вычислить $a_i$, если $x$-дробь; с одной стороны, это по-\newline зволит доказать, что данная последовательность почти периодическая,\newline а с другой-написать алгоритм поиска периода. До конца упражнения\newline запись $x\ mod\ p$ обозначаеь единственный элемент $y\in
[0,p-1]$ $y\equiv x$\newline $(mod\ p).$\newline
\hspace*{15pt}\textbf{a.} Написать 5-адическое разложение числа 1992. Какого $p$-адическое\newline разложение числа -1? Проверить, что 5-адическое разложение $\frac{1}{3}$ есть\newline $2+3\cdot5+1\cdot5^2+3\cdot5^3+1\cdot5^4+3\cdot5^4+\cdots$ Как выглядит 5-адическое разложение\newline $\frac{-1}{3}$? Допустим, что $p$-адическое разложение $x$ есть $x=a_0+a_1p+a_2p^2+$\newline $a_3p^3+\cdots$ с $a_0\neq 0$. Как выглядит разложение $-x$? А если $a_0=0?$\newline
\hspace*{15pt}\textbf{b.} Чему равно $p$-адическое число $1+1\cdot p^n+1\cdot p^{2n}+1\cdot p^{3n}+\cdots$ Вывести\newline отсюда, что если $p$-адическое разложение числа $x$ почти периодическое,\newline то $x$-рациональное.\newline
\hspace*{15pt}\textbf{c.} Пусть $a\in \mathbb{Z},\ b\in \mathbb{Z}-p\mathbb{Z}.$ Так как $b$ обратимо в кольце целых\newline $p$-адических чисел, то $\frac{a}{b}$ единственным образом записывается в виде\newline $a_0+a_1p+a_2p^2+a_3p^3+\cdots$ с $0\leqslant a_i\leqslant p-1.$ Проверить, что $ba_0\equiv a$\newline $(mod\ p),$ а затем, что 
        $$
        \frac{(a-ba_0)/p}{b}=a_1+a_2p+a_3p^2+\cdots
        $$
        Построить такую функцию $f\ :\ \mathbb{Z}\rightarrow\mathbb{Z},$ что рекуррентная последова-\newline тельность $y_{i+1}=f(y_i), y_0=a$ удовлетворяет сравнению $ba_i\equiv y_i.$ Дока-\newline зать, что $f$ оставляет на месте всякий интервал $[-K,K]$ при достаточно\newline большом $K>0.$ Вывести отсюда, что последовательность $(a_i)_{i\geqslant 0}$ почти\newline периодическая.\newline
\hspace*{15pt}\textbf{d.} Построить алгоритм, позволяющий неограниченно находить ко-\newline эффициенты $a_i$. Сравнить этот алгоритм с алгоритмом упражнения 14.\newline Написать алгоритм, позволяющий вычислять $p$-адическую форму дро-\newline бей.\\
\ \newline
\noindent\textbf{16. Поиск образующих элементов циклических групп}\\

    Пусть $G$-мультипликативная абелева группа и $p_1^{a_1}\ldots p_r^{a_r}$-раз-\newline ложение на простые множители числа $n$. Доказать, что в группе $G$\newline имеется элемент порядка $n$ тогда и только тогда, когда для каждого\newline $p_i$ существует элемент $x_i$ такой, что $x_i^n=1,$ а $x_i^{n/p_i}\neq 1.$ Используя этот\newline результат, построить метод нахождения образующего элемента цикли-\newline ческой группы, разложение порядка которой на простые множители\newline известно.\newline
     
\newpage

\noindent\textbf{17. Вокруг теоремы Лагранжа}\\

        \textbf{a.} Теорема Лагранжа: "в группе $G$ порядка $n$ выполнено равенство\newline $x^n=1$" или так: "если $H$-подруппа группы $G$, то порядок $H$ делит порядок $G$". Конечно эта теорема много раз использовалась в этой\newline книге. Какие классические доказательства следуют из этой теоремы?\newline
\hspace*{15pt}\textbf{b.} Пусть $a\geqslant 2, n\geqslant 1.$ Доказать, что $n\mid \varphi(a^n-1)$, где $\varphi$-функция Эйлера.\\
\ \newline
\noindent\textbf{18. Утверждение, обратное китайской теореме об остатках}\\

	\textbf{a.}Доказать, что прямое произведение абелевых групп циклично\newline тогда и только тогда, когда все множители цикличны и имеют попарно\newline взаимно простые порядки.\newline
\hspace*{15pt}\textbf{b.} Доказать, что $U(\mathbb{Z}_n)$-циклическая тогда и только тогда, когда\newline $n\in \{2,4,p^\alpha,2p^\alpha\}$ и $p$-нечетное простое число.\\
\ \newline
\noindent\textbf{19. Замечание, касающееся конечных абелевых групп}\\

    Доказать, что любая конечная абелева группа реализуется как под-\newline группа (или как факторгруппа) группы $U(\mathbb{Z}_n)$ (мультипликативной\newline группы целых чисел по модулю $n$).\\
    \ \newline
\noindent\textbf{20. Замечание $o\ v_2(n-1)$ при нечетном $n$}\\

    Пусть $n_1,\ldots,n_r$-семейство $r$ \textbf{нечетных} чисел и $n=n_1\ldots n_r.$\newline Положим $k=\inf \{v_2(n_i-1)\}.$ Доказать, что $v_2(n-1)\geqslant k$ и равенство до-\newline стигается тогда и только тогда, когда число индексов $i$ с $v_2(n_i-1)=k$\newline нечетно.\\
\ \newline
\noindent\textbf{21. Простые числа вида $4k+1$}\\

    Было доказано, что -1 является квадратичным вычетом по модулю\newline $p$ тогда и только тогда, когда $p\equiv 1\ (mod\ 4)$. Вывести отсюда беско-\newline нечность множества простых чисел вида $4k+1.$\\
\ \newline
\noindent\textbf{22. Некоторые свойства чисел Ферма}\\

    В этом упражнении $p$ обозначает простой множитель $n$-го числа\newline Ферма $F_n=2^{2^n}+1.$\newline
\hspace*{15pt}\textbf{a.} Доказать, что если $n\neq m$, то числа $F_n$ и $F_m$ взаимно просты.\newline
\hspace*{15pt}\textbf{b.} Доказать, что $p\equiv 1\ (mod\ 2^{n+1}).$ Если $p\neq 3,5),$ то $p\equiv 1$\newline $(mod\ 2^{n+2}).$\newline
\hspace*{15pt}\textbf{c.} Не проводя много вычислений, вывести отсюда, что $F_4$-простое, а $F_5$-нет.
    
\newpage

\noindent\textbf{23. Числа Ферма и критерий Пепина}\\

	\textbf{a.} Пусть $p=2^{2^n}+1$-простое число Ферма с $n\geqslant 2.$ Доказать, что\newline элемент группы $U(\mathbb{Z}_p)$ является порождающим тогда и только тогда,\newline когда он - квадратичный вычет.\newline
\hspace*{15pt}\textbf{b.} Доказать, что 3,5 и 7 порождают $U(\mathbb{Z}_p)$ при $n\geqslant 2.$\newline
\hspace*{15pt}\textbf{c.} Доказать, что $p$ простое тогда и только тогда, когда $3^{\frac{p-1}{2}}\equiv-1$\newline $(mod\ p)$ (критерий Пепина).\\
\ \newline
\noindent\textbf{24. Вычисление квадратных корней}\\

	\textbf{a.} Известен элемент $a$ коммутативного унитарного кольца, удовле-\newline творяющий уравненю $a^4+1=0.$ Вычислить квадратный корень из 2.\newline Применение: вычислить квадратный корень из 2 в кольцах или полях\newline $\mathbb{Z}_n$ с $n$, равными 17 и 257 (простые числа Ферма), 41, 241, 1201, 3281.\newline 
\hspace*{15pt}\textbf{b.} Обобщение: доказать, что в коммутативном кольце с обратимым\newline элементом 2 элементы 2 и -1 являются квадратами тогда и только\newline тогда, когда существует такой $a$, что $a^4+1=0$. В качестве примера\newline найти число корней четвертой и восьмой степеней из 1, а также число\newline квадратных корней из 2 в кольце $\mathbb{Z}_{3281}$.\newline
\hspace*{15pt}\textbf{c.} Каков будет ответ на предыдущий вопрос, если рассматривается\newline поле $\mathbb{Z}_p$? Пусть теперь имеется поле $\mathbb{Z}_p$ с $p=2q+1, q$ нечетно. Как\newline обосновать существование квадратного корня из некоторого элемента\newline и, в случае неудачи, найт этот квадратный корень?\\
\ \newline
\noindent\textbf{25. При каком условии $-1$ является $m$-й степенью?}\\

    Доказать, что $-1$ является $m$-й степенью в группе обратимых по\newline модулю $p^\alpha$ элементов ($p$ - простое число) тогда и только тогда, когда\newline $v_2(m)<v_2(p-1).$\\
\ \newline
\noindent\textbf{26. Квадратичные вычеты по модулю некоторого числа}\\

	\textbf{a.} Пусть $n$ и $m$-два взаимно простых целых числа и $a\in \mathbb{Z}.$ До-\newline казать, что $a$-квадратичный вычет по модулю $nm$ тогда и толко\newline тогда, когда $a$-квадратичный вычет по модулям $n$ и $m$. Какие мож-\newline но сделать выводы относительно квадратичных вычетов по модулю\newline какого-либо целого числа?\newline
\hspace*{15pt}\textbf{b.} Пусть $n$-нечетное целое число и $a$-квадратичный вычет\newline по модулю $n$, \textit{взаимно простой с} $n$. Доказать, что $a$-квадратичный\newline вычет по модулю $n^2.$ (Указание: если $x^2\equiv a\ (mod\ n),$ то квадратичный \newline

\newpage

	\noindent вычет $a$ по модулю $n^2$ можно искать в виде $x+yn$, где $y$ требуется\newline определить.) Вывести отсюда, что $a$-квадратичный вычет по модулю\newline любой степени $n.$\newline
\hspace*{15pt}\textbf{c.} Пусть $a\in \mathbb{Z}$-нечетное целое число. При каком условии $a$ явля-\newline ется квадратичным вычетом по модулю 2? по модулю 4? Доказать, что\newline $a$ является квадратичным вычетом по модулю $2^k,$ $k\geqslant 3,$ тогда и только\newline тогда, когда $a$ есть квадратичныё вычет по модулю 8, и показать, что\newline это эквивалентно тому, что $a\equiv 1\ (mod\ 8).$\\
\ \newline
\noindent\textbf{27. Элементарное вычисление $(\frac{-3}{p})$}\\

    Общий закон квадратичной взаимности дает следующий результат:\newline если $p$ - простое число, отличное от 2 и 3, то 
    $$
    -3~\text{- квадратичный вычет по модулю}~p\ \Longleftrightarrow\ p\equiv 1\ (mod\ 3).
    $$
\noindent Мы же хотим найти более простое доказательство (которое, в частно-\newline сти, не использует закон взаимности).\newline
\hspace*{15pt}\textbf{a.} При каких условиях трехчлен второй степени с коэффициентами\newline в поле $K$ имеет корень в $K$?\newline
\hspace*{15pt}\textbf{b.} Предположим, что -3 - квадратичный вычет по модулю $p$. С\newline помощью трехчлена $X^2+X+1$ и группы обратимых по модулю $p$ до-\newline казать, что $p-1$ кратно 3.\newline
\hspace*{15pt}\textbf{c.} Если $p-1$ кратно 3, то доказать, что группа $U(\mathbb{Z}/p\mathbb{Z})$ содержит\newline элемент порядка 3. Вывести отсюда, что -3 - квадратичный вычет.\\
\ \newline
\noindent\textbf{28. Другой способ вычисления $(\frac{2}{p})$}\\

    Изучив приведенные ниже произведения, вычислить $(\frac{2}{13})$ и $(\frac{2}{23})$:
    $$
    \pi = (2\cdot 4\cdot 6\cdot)\times (8\cdot 10\cdot 12)\ mod\ 13,
    $$
    $$
    \pi = (2\cdot 4\cdot 6\cdot 8\cdot 10)\times (12\cdot 14\cdot 16\cdot 18\cdot 20\cdot 22)\ mod\ 23.
    $$
\noindent Рассмотреть общую ситуацию, различая, при необходимости, случаи\newline $p\equiv 1 (mod\ 4)$ и $p\equiv 3 (mod\ 4).$\\
\ \newline
\noindent\textbf{29. Прямые следствия закона взаимности}\\

    Для таких простых чисел $p$ число 3 является квадратом по моду-\newline лю $p$? Тот же вопрос для 5 и 7.\newline
    
\newpage

\noindent\textbf{30. Подгруппа квадратов по модулю степени простого числа}\\

    Доказать, что множество квадратов по модулю $p^\alpha$ образует под-\newline группу индекса 2 в $U(\mathbb{Z}_{p^\alpha})$, если $p$-нечетное простое число. А если\newline $p=2$?\\
    \ \newline
\noindent\textbf{31. Символ Якоби}\\

    Это упражнение посвящено вычислению символа Лежандра $(\frac{a}{p})$, где\newline $p$-нечетное простое число. Будем считать, что этот символ равен\newline нулю, если $a$ кратно $p$. \newline
\hspace*{15pt}\textbf{a.} Какова сложность вычисления символа Лежандра по формуле\newline $(\frac{a}{p})=a^{\frac{p-1}{2}}\ mod\ p$?\newline
\hspace*{15pt} Другой способ вычисления символа Лежандра состоит в использо-\newline вании закона квадратичной взаимности при условии продолжения сим-\newline вола $(\frac{a}{p})$ на составные значения $p$. Для этого определим символ Якоби\newline $(\frac{a}{p})$ для $a\in \mathbb{Z}$ и \textit{нечетного} целого числа $b$ следующим образом:\newline
        $$
        (\frac{a}{p}) = \prod\limits_{i} (\frac{a}{p_i})\text{, если} \prod_i p_i\text{ - разложение} b\text{на простые множители. }
        $$
\noindent В частности, положим $(\frac{a}{p}) = 0$ тогда и только тогда, когда $a$ и $b$ не\newline взаимно просты.\newline
\hspace*{15pt}\textbf{b.} Доказать, что если $a$ - квадрат по модулю $b$, то $(\frac{a}{p}) = 1$, $a$ и $b$\newline предполагаются взаимно простыми. Верно ли обратное? Описать такие\newline нечетные целые числа $b$, что $(\frac{a}{p}) = 1$ для любого $a\in U(\mathbb{Z}_b)$.\newline
\hspace*{15pt}\textbf{c.} Доказать, что выполняются следующие свойства (напомним, что\newline второй аргумент символа Якоби - нечетное положительное чи-\newline сло $\geqslant$ 3):
        $$
        (\frac{a_1}{b})=(\frac{a_2}{b})\text{, если} a_1\equiv a_2 (mod\ b), (\frac{a_1 a_2}{b}) = (\frac{a_1}{b})(\frac{a_2}{b,})
        $$
        $$
        (\frac{a}{b_1 b_2}) = (\frac{a}{b_1})(\frac{a}{b_2}),
        $$
        $$
        (\frac{-1}{b}) = (-1)^{\frac{b-1}{2}}\text{, иначе говоря,} (\frac{-1}{b})=1 \Longleftrightarrow b\equiv 1 (mod\ 4),
        $$
        $$
        (\frac{2}{b}) = (-1)^{\frac{b^2-1}{8}}\text{, иначе говоря,} (\frac{2}{b})=1 \Longleftrightarrow b\equiv \pm 1 (mod\ 8),
        $$
        $$
        (\frac{a}{b}) = (\frac{b}{a})(-1)^{\frac{(a-1)(b-1)}{4}}\text{ (обобщенный закон взаимности).}
        $$
        \newline
\newpage
\textbf{d.} Используя эти свойства, построить алгоритм вычисления сим-\newline вола Якоби (в частности, символа Лежандра). Применение: выяснить,\newline является ли 713 квадратом по модулю простого числа 1009.\\
\ \newline
\noindent\textbf{32. Символ Якоби: подход Золотарева}\\

	Определим здесь символ Якоби, используя знак подстановки; такой\newline подход был предложен Золотаревым (см. например, [45] или [158]). На-\newline помним, что каждой подстановке $\sigma : I\rightarrow I$ конечного множества $I$\newline можно сопоставить ее знак $\varepsilon(\sigma)\in \lbrace -1, 1\rbrace$ и тем самым определяется\newline гомоморфизм группы подстановок множества $I$ в группу $\lbrace -1, 1 \rbrace$. Это\newline единственный гомоморфизм в группу $\lbrace -1, 1 \rbrace$, который равен -1 хотя\newline бы на одной транспозиции (или на всех транспозициях). Кроме того,\newline если множество $I$ вполне упорядочено и через ${Inv}_\sigma$ обозначить множе-\newline ство инверсий $\sigma$, т.е. таких пар $(i,j)$, что $i<j$, а $\sigma(i)>\sigma(j)$, то
$$
\varepsilon(\sigma)={(-1)}^{\vert{Inv}_\sigma\vert},
\varepsilon(\sigma)=\prod\limits_{i<j}\dfrac{\sigma(j)-\sigma(i)}{j-i}.
$$
\hspace*{15pt}\textbf{a.} Чему равен знак цикла длины $l$? Если $\sigma$ - подстановка множе-\newline ства $I$ и $I_1\cup\ldots\cup I_k$ - разбиение $I$ на множества $I_k$, остающиеся на\newline месте под действием $\sigma$, то доказать, что $\varepsilon(\sigma)=\varepsilon(\sigma_1)\ldots\varepsilon(\sigma_k)$, где $\sigma_k$\newline обозначает сужение $\sigma$ на $I_k$. Если $\sigma : I \rightarrow I, r : J \rightarrow J$ - две подста-\newline новки, то выразить $\varepsilon(\sigma \times r)$ как функцию $\varepsilon(\sigma)$ и $\varepsilon(r)$.\newline
\hspace*{15pt}\textbf{b.} Пусть $I,J$ - два конечных линейно упорядоченных множества\newline мощностей $n$ и $m$ соответственно. Доказать, что знак подстановки,\newline меняющей порядок $I$ на обратный, равен ${(-1)}^{\frac{n(n-1)}{2}}$. Упорядочим $I \times J$\newline с помощью одного из двух возможных лексикографических порядков,\newline что порождает перестановку перехода от одного порядка к другому.\newline Показать, что ее знак равен ${(-1)}^{\frac{n(n-1)}{2}\frac{m(m-1)}{2}}.$\newline
\hspace*{15pt}\textbf{c.} Чему равен знак подстановки $x\mapsto -x$ в $\mathbb{Z}_n$? а в абелевой группе\newline нечетного порядка? Чему равен знак переноса $x\mapsto x+r$ в $\mathbb{Z}_n$?\newline 
\hspace*{15pt} Пусть даны два взаимно простых целых числа $n$ и $m$, $m$ \textbf{нечет-\newline но}. Обозначим через ${\pi}_{n,m}$ умножение на $n$ по модулю $m(\mathbb{Z}_m \ni x \rightarrow$\newline $nx \in \mathbb{Z}_m$ и определим символ Золотарева $(n\vert m)$ как знак подстановки\newline ${\pi}_{n,m}$ (всегда будем считать, что аргументы символа $(.\vert .)$ взаимно про-\newline сты и второй аргумент нечетен). Оставшаяся часть упражнения посвя-\newline щена доказательству того, что этот символ имеет те же свойства, что и\newline символ Якоби. В конце будет доказано, что эти два символа совпадают.\newline
	
\newpage

\textbf{d.} Доказать, что
	$$
	n\equiv n' (mod\ m) \Rightarrow (n\vert m)=(n'\vert m), (nn'\vert m)=(n\vert m)(n'\vert m),
	$$
	$$
	(-1\vert m)={(-1)}^{\frac{m-1}{2}}, (2\vert m)={(-1)}^{\frac{m^2-1}{8}},
	$$
	$$
	(n\vert p)\equiv n^{\frac{p-1}{2}} (mod\ p), (p\text{-простое}).
	$$
\hspace*{15pt}\textbf{e.} Пусть $n, m$ - два нечетных положительных взаимно простых\newline числа. Определим две подстановки $\sigma$ и $r$ множества $[0, n-1] \times [0, m-1]:$
	$$
	\sigma(i,j)=((mi+j)\ mod\ n,j), r(i,j)=(i,(nj+i)\ mod\ m),
	$$
	$$
	0\leqslant i \leqslant n-1, 0\leqslant j \leqslant m-1.
	$$
	Доказать, что $\varepsilon (\sigma)=(m\vert n), \varepsilon (r)=(n\vert m)$. Вывести отсюда, что
	$$
	(n\vert m)(m\vert n)={(-1)}^{\frac{n-1}{2}\frac{m-1}{2}} \text{(закон взаимности).}
	$$
\hspace*{15pt}\textbf{f.} Из закона взаимности получить:
	$$
	(2\vert m)={(-1)}^{\frac{m^2-1}{8}}, (n\vert m)(n\vert m')=(n\vert mm'), (n\vert m)=(\frac{n}{m}) 
	$$
	Доказать, что 2 - примитивный корень по модулю простого числа $p$\newline вида $4q+1$, где $q$ - простое. Тот же вопрос для числа $p$ вида $2q+1$, где\newline $q$ - простое число и $q\equiv 1\ (mod\ 4)$. Доказать,что -2 - примитивный\newline корень по модулю простых чисел $p$ вида $2q+1$, где $q$ - простое и $q\equiv 3$\newline $(mod\ 4)$.\\
	\ \newline
\noindent\textbf{33. Некоторые простые числа, для которых 2 - примитивный\newline корень}\\

	Доказать, что 2 - примитивный корень по модулю простого числа\newline $p$ вида $2q+1$, где  $q$ - простое число с $q\equiv 3\ (mod\ 4)$.\\
	\ \newline
\noindent\textbf{34. Вычисление длины периода: метод Флойда}\\

\textbf{a.} Дана последовательность $x_{n+1}=f(x_n)$, где $f$ - функция на ко-\newline нечном множестве $F$. Доказать, что существует такой индекс $n$, что\newline $x_n=x_2n$. Вывести отсюда алгоритм вычисления длины периода и ин-\newline декса вхождения в период. Доказать, что $x_n$ единственно в том смысле,\newline что если $x_m=x_{2m}$, то $x_n=x_m$.\newline
\hspace*{15pt}\textbf{b.} Доказать, что существует такое $n$, что $x_n=x_{2n}$ \textbf{для любого\newline значения} $x_0$.\newline

\newpage

\noindent\textbf{35. Линейные генераторы: примеры}\\

\textbf{a.} Чему равна длина периода последовательности $u_{n+1}=u_n+b\ mod$\newline $m$ (для какого-нибудь $u_0$)? Чему равен индекс вхождения в период?\newline
\hspace*{15pt}\textbf{b.} Пусть $a$ и $m$ взаимно просты. Рассмотрим последовательность\newline $x_{n+1}=ax_n\ mod m, x_0=1$. Чему равен индекс вхождения? Выразить\newline длину периода через класс $\overline{a}_m$ группы $U(\mathbb{Z}_m)$.\\
\ \newline
\noindent\textbf{36. Линейные генераторы: изучение частных случаев}\\

	Пусть, $\prod_{i=1}^r {p_i}^{a_i}$ - разложение на простые множители числа $m\in \mathbb{N*}$.\newline
\hspace*{15pt}\textbf{a.} Пусть $a\equiv 0\ (mod\ p_i)$ для любого $i=1,\ldots, r$. Доказать, что по-\newline следовательность $x_{n+1}=ax_n+b\ mod m$ стационарна (при любом $x_0$).\newline Что можно сказать об индексе вхождения в период? Изучите последо-\newline вательность $x_{n+1}=6x_n+b mod\ 320$ для различных значений $b$.\newline
\hspace*{15pt}\textbf{b.} Предположим, что $a\not\equiv 0, 1\ (mod\ p_i)$ для $i=1,\ldots, r$ (это экви-\newline валентно тому, что $a-1$ обратимо по модулю $m$). Доказать, что дли-\newline на периода последовательности $x_{n+1}=ax_n+1\ mod\ m, x_0=0$ равна\newline порядку $a$ в $U(\mathbb{Z}_m)$. Чему равен индекс вхождения в период? Дока-\newline зать, что результат не изменится при $x_{n+1}=ax_n+b\ mod m$, если $b$\newline \textit{обратимо по модулю} $m$. Рассмотреть примеры $x_{n+1}=8x_n+1\ mod\ 15,$\newline $x_{n+1}=2x_n+1\ mod\ 27$.\newline
\hspace*{15pt}\textbf{c.} Снова предположим, что $a$ и $a-1$ обратимы по модулю $m$. При-\newline вести примеры последовательностей вида $x_{n+1}=ax_n+1$, у которых\newline длина периода не превосходит порядка $a$ в группе $U(\mathbb{Z}_m)$ ($x_0$ более не\newline равно 0).\newline
\hspace*{15pt}\textbf{d.} Вычислить длины периодов следующих последовательностей:		
		$$
		x_{n+1}=12x_n+1\ mod\ 63, x_{n+1}=2x_n+1\ mod\ 100,
		$$
		$$
		x_{n+1}=4x_n+1\ mod\ 100, x_{n+1}=20x_n+1\ mod\ 360.
		$$\\
		\ \newline
\noindent\textbf{37. Линейные генераторы: частные случаи аффиных\newline преобразований}\\
	
\textbf{a.} Доказать, что рекуррентно заданная последовательность $x_{n+1}=$\newline $ax_n+b$ является образом при аффином преобразовании $y\rightarrow\varphi(y)=$\newline $\alpha y+\beta$ последовательности $y_{n+1}=ay_n+1, y_0=0$.\newline
\hspace*{15pt}\textbf{b.} Пусть $m\in \mathbb{N}$. Найти такое $m'\in \mathbb{N}$, что 
		$$
		\varphi(y)\equiv\varphi(z)\ (mod\ m)\Leftrightarrow y\equiv z\ (mod\ m').
		$$\newline

\newpage

\noindent Используя это, доказать, что определение длины периода (и индекса\newline вхождения в период) произвольной последовательности $x_{n+1}=ax_n+$\newline $b\ mod\ m$ сводится к определению тех же параметров для последователь-\newline ности $y_{n+1}=ay_n+1\ mod\ m', y_0=0$. Посмотреть примеры: $x_{n+1}=$\newline $12x_n+7(45), x_0=1; x_{n+1}=12x_n+3(45), x_0=2; x_{n+1}=11x_n+10(45),$\newline $x_0=1.$\\
\ \newline
\noindent\textbf{38. Порядок элемента $1+p$ в $U(\mathbb{Z}_{p^r})$}\\
	
	Рассмотрим последовательность $x_{n+1}=(1+p)x_n\ mod\ p^r, x_0=1$\newline ($p$ - простое число, отличное от 2), которая является образом при\newline аффинном преобразовании $y\rightarrow \alpha y + \beta$ последовательности $y_{n+1}=$\newline $(1+p)y_n+1\ mod\ m', y_0=0$ (упражнение 37). Доказать, что последо-\newline вательность ${(y_n)}_{n\geqslant 0}$ максимального периода, и этот результат зависит\newline от порядка элемента $1+p$ в $U(
\mathbb{Z}_{p^r})$.\\
\ \newline
\noindent\textbf{39. Последовательности, порожденные сравнениями}\\
	
	Доказать, что длина периода последовательности $x_{k+1}=ax_k+$\newline $+1\ mod\ p^n, x_0=0$ находится по таблице 7.
\begin{table}[h!]
%\centering
\begin{tabular}{|c|c|c|c|c|}
\hline
\multicolumn{2}{|c|}{$p\neq2$}                       & \multicolumn{3}{c|}{$p=2, n\geqslant 2$}                                         \\ \hline
$a\equiv 1 (mod\ p)$ & $a\not\equiv 0,1 (mod\ p)$    & $a\equiv 1 (mod\ 4)$ & \multicolumn{2}{c|}{$a\equiv 3 (mod\ 4)$}                 \\
                     &                               &                      & $a\equiv -1 (mod\ 2^n)$ & $a\not\equiv -1 (mod\ 2^n)$     \\ \hline
$p^n$                & $ord(a, U(\mathbb{Z}_{p^n}))$ & $2^n$                & $2$                      & $2 ord(a, U(\mathbb{Z}_{2^n}))$ \\ \hline
\end{tabular}
\end{table}
\begin{center}
	\textbf{Таблица 7.} Длина периода последовательности, заданной\linebreak линейным сравнением
\end{center}
\ \newline
\noindent\textbf{40. Линейные генераторы: синтез}\\

	Показать, как вычислить длину периода произвольной последова-\newline тельности, заданной линейным сравнением $x_{n+1}=ax_n+b \mod{m}$. Рас-\newline смотреть пример $x_{n+1}=16x_n+100\mod{315}, x_0=2$.\\
	\ \newline
\noindent\textbf{41. Многошаговые линейные генераторы}\\

	В этом упражнени используется много понятий: линейная алгебра,\newline конечное поле, неприводимый многочлен, $k$ - шаговые генераторы почти\newline периодических последовательностей.\newline
\hspace{15pt}\textbf{a.} Скажем, что последовательность ${(x_n)}_{n\geqslant 0}$ элементов множества\newline $F$ генерируется за $k$ шагов, если ее члены получаются по правилу\newline

\newpage

\noindent $x_{n+k}=f(x_n,x_{n+1,\ldots,x_{n+k-1}})$, где $f$ - функция $F^k\rightarrow F$. Если число\newline элементов множества $F$ равно $q$(т.е. $F$ конечно), то эта последователь-\newline ность почти периодическая и длина ее периода не превосходит $q^k$. До-\newline стигается ли эта граница $k$ - шаговым генератором? Предположим, что\newline функция $f:F^k\rightarrow F$ генерирует последовательность с длиной периода\newline $q^k-1$ и что существует такое $a\in F$, что $f(a,a,\ldots,a)=a$. Показать,\newline как при помощи пары $(f,a)$ построить $k$ - шаговый генератор с длиной\newline периода $q^k$.\newline
\hspace*{15pt} Изучим теперь линейные $k$ - шаговые генераторы, т.е. такие реку-\newline рентные последовательности, каждый член которых является линейной\newline комбинацией $k$ предыдущих членов (в поле $K$):
$$
x_{n+k}=a_0x_n + a_1x_{n+1} + \cdots + a_{k-1}x_{n+k-1}.
$$
Обозначим через $P(X)$ многочлен $X^k-(a_{k-1}X^{k-1}+ \cdots +a_1X+a_0)$.\newline Напомним, что многочлену $P$ можно поставить в соответствие матрицу\newline Фробениуса $B=\Phi_p$ размеров $k\times k$, определенную следующим образом\newline (определим также матрицу $A=^tB$):
\begin{multline*}
	\qquad\qquad\quad B=\Phi_p=
	\begin{pmatrix}
	    0 & \ldots & 0 & 0 & a_0 \\
	    1 & ~ & 0 & 0 & a_1 \\
	    \vdots & ~ & ~ & ~ & \vdots \\
	    0 & ~ & 1 & 0 & a_{k-2} \\
	    0 & \ldots & 0 & 1 & a_{k-1} \\
	\end{pmatrix},\\
	A=^tB=
	\begin{pmatrix}
	    0 & 1 & ~ & 0 & 0 \\
	    \vdots & ~ & ~ & ~ & \vdots \\
	    0 & 0 & ~ & 1 & 0 \\
	    0 & 0 & \ldots & 0 & 1 \\
	    a_0 & a_1 & \ldots & a_{k-2} & a_{k-1} \\
	\end{pmatrix}.
	\qquad\qquad\quad
\end{multline*}
\hspace*{15pt}\textbf{b.} Рассмотрим вектор $X_n=(x_n,x_{n+1},\ldots,x_{n+k-1})$. Какова связь\newline между последовательностью $(x_n)$, векторами $X_n$ и матрицей $A$? Обо-\newline значим через $E$ векторное пространство $K^k$, а через $E_B - K[X]$\newline - модуль, ассоциированный с $B$, т.е. определенный следующим образом:\newline $P\cdot x=P(B)(x)$. Доказать, что первый вектор $e_1$ канонического базиса\newline порождает $E_B$, т.е. $E=K\cdot e_1 + K\cdot B(e_1) + K\cdot B^2(e_1) + \cdots$. Чему ра-\newline вен характеристический полином $B$? минимальный многочлен $B$? Чему\newline равны соответствующие многочлены для матрицы $A$? \newline
\hspace*{15pt}\textbf{c.} Показать, что если $F\subset E$ - подпространство, инвариант -\newline ное относительно действия $A$, то существует единственный много-\newline член $Q$, делящий $P$ и такой, что $F=Q\cdot E$ как $K[X]$ - модуль, т.е.\newline

\newpage

\noindent $F = \{Q(A)(x) | x \in E\}$. В дальнейшем будем считать, что $P(X)$ неприводим. Что можно в этом случае сказать о подпространстве пространства $E$, инвариантном относительно действия $A$?

$\:$\newline \indent
\textbf{d.} Допустим, что $X_0$ — ненулевой вектор. Доказать, что $X_i = X_j$ тогда и только тогда, когда $A^{j-i} = Id_E$. Доказать, что $A^\lambda = Id_E$ эквивалентно тому, что $P(X) | X^\lambda - 1$. Пусть $K$ — конечное поле порядка $q$. Доказать, что длина периода равна порядку $\bar{X}$ в расширении $L = K[X]/P$. В частности, она не превосходит $q^k — 1$.

$\:$\newline \indent
\textbf{e.} Показать, что хотя $\bar{X}$ и кажется особым корнем многочлена
$P(X)$, но и любой другой корень $P(X)$ (в подходящем расширении)
имеет те же алгебраические свойства, что и $\bar{X}$ (сформулировать более
точно это высказывание). Показать, что длина периода последователь
ности $(x_n)_{n \geq 0}$ равна $q^k - 1$ тогда и только тогда, когда $P(X)$ имеет
корень порядка $q^k - 1$. Доказать, что такие многочлены существуют
для любого $k$. Назовем такие многочлены примитивными. Не следу
ет путать это понятие (применяемое как правило к многочленам над
конечными полями) с понятием примитивного многочлена в фактори
альных кольцах.

$\:$\newline \indent
\textbf{f.} Используя то, что многочлен $X^3 + 2X^2 + 1$ примитивен над полем
$\mathds{F}_3$, построить 3-шаговые генераторы в множестве $\{0,1,2\}$, имеющие
период длины $26$ и $27$. Для конечного множества $F$ и некоторого $k \geq 1$ доказать, что существует функция $f : F^k \rightarrow F$, которой соответствует
$k$-шаговый генератор с периодом максимальной длины, т.е. $|F|^k$.

$\:$\newline
$\:$\newline
\textbf{42. Псевдопростые числа}

$\:$\newline \indent
Напомним обозначение $S_q(b)=1+b+\dots+b^{q-1}$. В этом упражнении
будем считать, что $q$ нечетно и взаимно просто с $b - 1$.

$\:$\newline\indent
\textbf{a.} Предположим, что $b^{q - 1} \equiv 1\:(\:mod\:q)$. Доказать, что $n = S_q(b)$ нечетно, взаимно просто с $b -1$ и удовлетворяет сравнению $b^{(n- 1)/2} \equiv = 1\:(mod\:n)$. Проверить, что если имеется нечетное псевдопростое число по основанию $b,$ взаимно простое с $b - 1$, то отображение $q \rightarrow S_q(b)$ определяет бесконечно много псевдопростых чисел по основанию $b$. В частности, если $q$ — псевдопростое по основанию $2$, то таковым является и $2^q- 1$.

$\:$\newline\indent
\textbf{b.} Доказать, что существует бесконечное множество простых чисел $q$ таких, что $S_q(b)$ — псевдопростое по основанию $b$. Указание: рассмотреть $b^2$ (Cipolla, $1904\:[154]$).
\newpage
\textbf{43. Числа Кармайкла}
\newline \newline \indent
Напомним, что число Кармайкла это такое целое число $n$, что\newline $b^{n - 1} \equiv 1 (mod n)$ для любого $b$, взаимно простого с $n$.
\newline \indent
\textbf{a.} Пусть число $n \ge 3$ непростое и $n = p_1^{\alpha_1}]_2^{\alpha_2}\dots p_k^{\alpha_k}$. Доказать, что $n$ является числом Кармайкла тогда и только тогда, когда:
\begin{center}
$p_i \ne 2, \;\;\; \alpha_i = 1, \;\;\; p_i - 1 \:| \: n - 1, \;\;\; i = 1, \dots , k,$
\end{center}
и в этом случае $k \ge 3$.
\newline \indent
\textbf{b.} Чему равно наименьшее число Кармайкла? Проверить, что следующие числа являются числами Кармайкла: 
\newline \newline \indent
\hspace{40pt} $1105 = 5 \times 13 \times 17, \; 1729 = 7 \times 13 \times 19, \; 2465 = 5 \times 17 \times 29,$ \newline \indent
\hspace{85pt}$2821 = 7 \times 13 \times 31, \; 6601 = 7 \times 23 \times 41,$
\newline \indent
\hspace{16pt}$29341 = 13 \times 37 \times 61, \; 41041 = 7 \times 11 \times 13 \times 41, \; 172081 = 7 \times 13 \times 31 \times 61, $
\newline \indent
\hspace{55pt}$278545 = 5 \times 17 \times 29 \times 113, \; 825265 = 5 \times 7 \times 17 \times 19 \times 73, $
\newline \indent
\hspace{85pt}$413631505 = 5 \times 7 \times 17 \times 73 \times 89 \times 107.$
\newline \newline \indent
\textbf{44. Числа Кармайкла, являющиеся произведениями \newline \indent
$\;$ трех простых чисел}
\newline \newline \indent
Изучим числа Кармайкла вида $p = p_1 p_2 p_3$, т.е. всевозможные трой
ки чисел $(p_1 , p_2 , p_3)$ со свойствами:\indent
\begin{align}
p_i -  \text{нечетное простое},  \;\;\; p_1 < p_2 < p_3, \;\;\; p_i - 1 \: | \: p_1 p_2 p_3 - 1. 
\end{align} 
\indent
\textbf{a.} Для искомой тройки $(p_1 , p_2 , p_3)$ определим отношение $r = (p_1 p_2 - 1)/(p_3 - 1)$. Доказать, что $r$ --- целое и $1 < r < p_1$. \newline \newline  \indent
\textbf{b.} Доказать, что:
\begin{center}
$(p_1 p_2 - 1 + r)p_1 \equiv r \; (mod \; r(p_2 - 1)), \;\;\; (p_1 p_2 - 1 + r)p_2 \equiv r \; (mod \; r(p_1 - 1))$. \end{center}
Вывести отсюда, что $p_2 \: - 1$ делит $(p_1 - 1)(p_1 + r)\;$ и, следовательно,\newline $p_1 + 2 \le p_2 \le 1 + \\ (p_1 - 1)(p_1 + r)$.
\newline \newline \indent
\textbf{c.} Доказать обратное утверждение. Показать, что множество троек\newline $(p_1 , p_2 ,\\ p_3),$ удовлетворяющих $(8)$ с {\itshape фиксированными} $p_1,$ конечно и написать алгоритм, позволяющий находить эти тройки. \newline \newline \indent
\textbf{d.} Найти все числа Кармайкла вида $3pq, 5pq, 7pq$. Доказать, что $561$ --- наименьшее число Кармайкла.
\newpage
\noindent \textbf{45. Числа, достигающие границы $\frac{1}{4}$ в тесте Рабина}
\newline \newline \indent
Для нечетного целого числа $n$ запишем $n - 1 \equiv 2^k q$ и обозначим через $B_n$ множество обратимых элементов $b$ по модулю $n$ таких, что $n$ — сильно псевдопростое по основанию $b$.
\newline \newline \indent
\textbf{a.} Просмотреть доказательство теоремы Рабина и проверить, что существует лишь два семейства целых чисел, для которых $\phi(n) = 4 \times |В_n|$:
\begin{align*}
n = p_1 p_2,\;\text{для}\;p_2 - 1 = 2(p_1 - 1), p_1\equiv 3 (mod\:4),\text{\hspace{15pt}}\\
n = p_1 p_2 p_3,\;\text{для}\;p_1\equiv p_2\equiv p_3\equiv 3 (mod\:4), p_i - 1 | n - 1.
\end{align*}
Второе семейство состоит из чисел Кармайкла, которые являются\newline произведением трех простых чисел $\equiv 3\:(mod\:4)$. Проверить, что\newline $n \equiv 3\:(mod\:4), B_n = \{b \in U(\mathds{Z}_n) | b^{\frac{n-1}{2}} = \pm 1\}$ и $\phi(n) = 4 \times |B_n|$. Привести примеры чисел, принадлежащих этим двум семействам.
\newline \newline \indent
\textbf{b.} Если $n$ принадлежит первому семейству, то доказать, что $n$ не
псевдопростое, а тем более и не сильно псевдопростое, по основанию $2$.
\newline \newline
\textbf{46. Равенство $a^{\frac{n-1}{2}} = \pm 1$ для любого $a \in U(\mathds{Z}_n)$}
\newline \newline \indent
\textbf{a.} В этом упражнении считаем, что п нечетно. Доказать, что п
простое тогда и только тогда, когда
\begin{align*}
\forall a \in U(\mathds{Z}_n)\:\text{выполняется}\: a^{\frac{n-1}{2}} = \pm 1 \text{\hspace{40pt}}\\
\text{и}\:\exists a \in U(\mathds{Z_n}),\:\text{такое, что}\:a^{\frac{n-1}{2}} = -1.
\end{align*}
\newline \indent
\textbf{b.} Доказать, что равенство $a^{\frac{n-1}{2}}= 1$ для любого $a \in U(\mathds{Z}_n)$ эквивалентно тому, что $n$ — произведение различных нечетных простых чисел $p_i$ с $p_i - 1 | (n — 1)/2$. Среди чисел Кармайкла $n$, рассмотренных в упражнении $43$, найти такие, для которых $a^{\frac{n-1}{2}} = 1$ для любого $a \in U(\mathds{Z}_n)$.
\newline \newline \indent
\textbf{c.} Пусть для п 2 выполняется $a^{\frac{n-1}{2}} = \pm 1$ для всякого $а \in U(\mathds{Z}_n)$. Если $n$ непростое, доказать, что $a^{\frac{n-1}{2}} = 1$ для любого $а \in U(\mathds{Z}_n)$.
\newline \newline \indent
\textbf{47. Нестабильность оснований в тесте Рабина — Миллера}
\newline \newline \indent
Обозначим через $n$ нечетное целое число, а через  $B_n \subset U(\mathds{Z}_n)$ множество таких чисел $b$, что $n$ — сильно псевдопростое по основанию Ь.
\end{document}
