\documentclass{../../template/mai_book}

\defaultfontfeatures{Mapping=tex-text}
\setdefaultlanguage{russian}

\usepackage{dsfont}% for mathds and more
%\usepackage{scrextend}


\begin{document}
\noindent $F = \{Q(A)(x) | x \in E\}$. В дальнейшем будем считать, что $P(X)$ неприводим. Что можно в этом случае сказать о подпространстве про­странства $E$, инвариантном относительно действия $A$?

$\:$\newline \indent
\textbf{d.} Допустим, что $X_0$ — ненулевой вектор. Доказать, что $X_i = X_j$ тогда и только тогда, когда $A^{j-i} = Id_E$. Доказать, что $A^\lambda = Id_E$ эквивалентно тому, что $P(X) | X^\lambda - 1$. Пусть $K$ — конечное поле порядка $q$. Доказать, что длина периода равна порядку $\bar{X}$ в расширении $L = K[X]/P$. В частности, она не превосходит $q^k — 1$.

$\:$\newline \indent
\textbf{e.} Показать, что хотя $\bar{X}$ и кажется особым корнем многочлена
$P(X)$, но и любой другой корень $P(X)$ (в подходящем расширении)
имеет те же алгебраические свойства, что и $\bar{X}$ (сформулировать более
точно это высказывание). Показать, что длина периода последователь­
ности $(x_n)_{n \geq 0}$ равна $q^k - 1$ тогда и только тогда, когда $P(X)$ имеет
корень порядка $q^k - 1$. Доказать, что такие многочлены существуют
для любого $k$. Назовем такие многочлены примитивными. Не следу­
ет путать это понятие (применяемое как правило к многочленам над
конечными полями) с понятием примитивного многочлена в фактори­
альных кольцах.

$\:$\newline \indent
\textbf{f.} Используя то, что многочлен $X^3 + 2X^2 + 1$ примитивен над полем
$\mathds{F}_3$, построить 3-шаговые генераторы в множестве $\{0,1,2\}$, имеющие
период длины $26$ и $27$. Для конечного множества $F$ и некоторого $k \geq 1$ доказать, что существует функция $f : F^k \rightarrow F$, которой соответствует
$k$-шаговый генератор с периодом максимальной длины, т.е. $|F|^k$.

$\:$\newline
$\:$\newline
\textbf{42. Псевдопростые числа}

$\:$\newline \indent
Напомним обозначение $S_q(b)=1+b+\dots+b^{q-1}$. В этом упражнении
будем считать, что $q$ нечетно и взаимно просто с $b - 1$.

$\:$\newline\indent
\textbf{a.} Предположим, что $b^{q - 1} \equiv 1\:(\:mod\:q)$. Доказать, что $n = S_q(b)$ нечетно, взаимно просто с $b -1$ и удовлетворяет сравнению $b^{(n- 1)/2} \equiv = 1\:(mod\:n)$. Проверить, что если имеется нечетное псевдопростое число по основанию $b,$ взаимно простое с $b - 1$, то отображение $q \rightarrow S_q(b)$ определяет бесконечно много псевдопростых чисел по основанию $b$. В частности, если $q$ — псевдопростое по основанию $2$, то таковым явля­ется и $2^q- 1$.

$\:$\newline\indent
\textbf{b.} Доказать, что существует бесконечное множество простых чисел $q$ таких, что $S_q(b)$ — псевдопростое по основанию $b$. Указание: рассмо­треть $b^2$ (Cipolla, $1904\:[154]$).
\newpage
\textbf{43. Числа Кармайкла}
\newline \newline \indent
Напомним, что число Кармайкла это такое целое число $n$, что $b^{n - 1} \equiv 1 \; (mod n)$ для любого $b$, взаимно простого с $n$.
\newline \indent
\textbf{a.} Пусть число $n \ge 3$ непростое и $n = p_1^{\alpha_1}]_2^{\alpha_2}\dots p_k^{\alpha_k}$. Доказать, что $n$ является числом Кармайкла тогда и только тогда, когда:
\begin{center}
$p_i \ne 2, \;\;\; \alpha_i = 1, \;\;\; p_i - 1 \:| \: n - 1, \;\;\; i = 1, \dots , k,$
\end{center}
и в этом случае $k \ge 3$.
\newline \indent
\textbf{b.} Чему равно наименьшее число Кармайкла? Проверить, что следующие числа являются числами Кармайкла: 
\newline \newline \indent
\hspace{40pt} $1105 = 5 \times 13 \times 17, \; 1729 = 7 \times 13 \times 19, \; 2465 = 5 \times 17 \times 29,$ \newline \indent
\hspace{85pt}$2821 = 7 \times 13 \times 31, \; 6601 = 7 \times 23 \times 41,$
\newline \indent
\hspace{16pt}$29341 = 13 \times 37 \times 61, \; 41041 = 7 \times 11 \times 13 \times 41, \; 172081 = 7 \times 13 \times 31 \times 61, $
\newline \indent
\hspace{55pt}$278545 = 5 \times 17 \times 29 \times 113, \; 825265 = 5 \times 7 \times 17 \times 19 \times 73, $
\newline \indent
\hspace{85pt}$413631505 = 5 \times 7 \times 17 \times 73 \times 89 \times 107.$
\newline \newline \indent
\textbf{44. Числа Кармайкла, являющиеся произведениями \newline \indent
$\;$ трех простых чисел}
\newline \newline \indent
Изучим числа Кармайкла вида $p = p_1 p_2 p_3$, т.е. всевозможные трой­
ки чисел $(p_1 , p_2 , p_3)$ со свойствами:\indent
\begin{align}
p_i -  \text{нечетное простое},  \;\;\; p_1 < p_2 < p_3, \;\;\; p_i - 1 \: | \: p_1 p_2 p_3 - 1. 
\end{align} 
\indent
\textbf{a.} Для искомой тройки $(p_1 , p_2 , p_3)$ определим отношение $r = (p_1 p_2 - 1)/(p_3 - 1)$. Доказать, что $r$ --- целое и $1 < r < p_1$. \newline \newline  \indent
\textbf{b.} Доказать, что:
\begin{center}
$(p_1 p_2 - 1 + r)p_1 \equiv r \; (mod \; r(p_2 - 1)), \;\;\; (p_1 p_2 - 1 + r)p_2 \equiv r \; (mod \; r(p_1 - 1))$. \end{center}
Вывести отсюда, что $p_2 \: - 1$ делит $(p_1 - 1)(p_1 + r)\;$ и, следовательно, $p_1 + 2 \le p_2 \le 1 + \\ (p_1 - 1)(p_1 + r)$.
\newline \newline \indent
\textbf{c.} Доказать обратное утверждение. Показать, что множество троек $(p_1 , p_2 ,\\ p_3),$ удовлетворяющих $(8)$ с {\itshape фиксированными} $p_1,$ конечно и написать алгоритм, позволяющий находить эти тройки. \newline \newline \indent
\textbf{d.} Найти все числа Кармайкла вида $3pq, 5pq, 7pq$. Доказать, что $561$ --- наименьшее число Кармайкла.
\newpage
\noindent \textbf{45. Числа, достигающие границы $\frac{1}{4}$ в тесте Рабина}
\newline \newline \indent
Для нечетного целого числа $n$ запишем $n - 1 \equiv 2^k q$ и обозначим через $B_n$ множество обратимых элементов $b$ по модулю $n$ таких, что $n$ — сильно псевдопростое по основанию $b$.
\newline \newline \indent
\textbf{a.} Просмотреть доказательство теоремы Рабина и проверить, что существует лишь два семейства целых чисел, для которых $\phi(n) = 4 \times |В_n|$:
\begin{align*}
n = p_1 p_2,\;\text{для}\;p_2 - 1 = 2(p_1 - 1), p_1\equiv 3 (mod\:4),\text{\hspace{15pt}}\\
n = p_1 p_2 p_3,\;\text{для}\;p_1\equiv p_2\equiv p_3\equiv 3 (mod\:4), p_i - 1 | n - 1.
\end{align*}
Второе семейство состоит из чисел Кармайкла, которые являются про­изведением трех простых чисел $\equiv 3\:(mod\:4)$. Проверить, что $n \equiv 3\:(mod\:4), B_n = \{b \in U(\mathds{Z}_n) | b^{\frac{n-1}{2}} = \pm 1\}$ и $\phi(n) = 4 \times |B_n|$. Привести примеры чисел, принадлежащих этим двум семействам.
\newline \newline \indent
\textbf{b.} Если $n$ принадлежит первому семейству, то доказать, что $n$ не
псевдопростое, а тем более и не сильно псевдопростое, по основанию $2$.
\newline \newline
\textbf{46. Равенство $a^{\frac{n-1}{2}} = \pm 1$ для любого $a \in U(\mathds{Z}_n)$}
\newline \newline \indent
\textbf{a.} В этом упражнении считаем, что п нечетно. Доказать, что п
простое тогда и только тогда, когда
\begin{align*}
\forall a \in U(\mathds{Z}_n)\:\text{выполняется}\: a^{\frac{n-1}{2}} = \pm 1 \text{\hspace{40pt}}\\
\text{и}\:\exists a \in U(\mathds{Z_n}),\:\text{такое, что}\:a^{\frac{n-1}{2}} = -1.
\end{align*}
\newline \indent
\textbf{b.} Доказать, что равенство $a^{\frac{n-1}{2}}= 1$ для любого $a \in U(\mathds{Z}_n)$ экви­валентно тому, что $n$ — произведение различных нечетных простых чисел $p_i$ с $p_i - 1 | (n — 1)/2$. Среди чисел Кармайкла $n$, рассмотренных в упражнении $43$, найти такие, для которых $a^{\frac{n-1}{2}} = 1$ для любого $a \in U(\mathds{Z}_n)$.
\newline \newline \indent
\textbf{c.} Пусть для п 2 выполняется $a^{\frac{n-1}{2}} = \pm 1$ для всякого $а \in U(\mathds{Z}_n)$. Если $n$ непростое, доказать, что $a^{\frac{n-1}{2}} = 1$ для любого $а \in U(\mathds{Z}_n)$.
\newline \newline \indent
\textbf{47. Нестабильность оснований в тесте Рабина — Миллера}
\newline \newline \indent
Обозначим через $n$ нечетное целое число, а через  $B_n \subset U(\mathds{Z}_n)$ мно­жество таких чисел $b$, что $n$ — сильно псевдопростое по основанию Ь.
\end{document}