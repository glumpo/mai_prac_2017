\documentclass{mai_book}

\defaultfontfeatures{Mapping=tex-text}
\setdefaultlanguage{russian}

\begin{document}

\noindent{предварительно вычислить некоторое множество констант ($a_i$), зависящих от элементов матрицы-циркулянта (здесь $a_i$ зависит от $\omega$). При\linebreak этих условиях получаем следующие вычисления:}
$$
\begin{pmatrix}
	b_1 \\ 
	b_2 \\
	b_3 \\
	b_4 \\
	b_5 
\end{pmatrix}
= B^{'}
\begin{pmatrix}
a_1\\
a_2 \\
a_3 \\
a_4 
\end{pmatrix},\,
\begin{pmatrix}
	m_1 \\
	m_2 \\
	m_3 \\
	m_4 \\
	m_5 
\end{pmatrix}
=
\begin{pmatrix}
{\alpha}_1 \times b_1 \\
{\alpha}_2 \times b_2 \\
{\alpha}_3 \times b_3 \\
{\alpha}_4 \times b_4 \\
{\alpha}_5 \times b_5 
\end{pmatrix},\,
\begin{pmatrix}
\hat{a}_1 - \hat{a}_0 \\
\hat{a}_2 - \hat{a}_0 \\
\hat{a}_4 - \hat{a}_0 \\
\hat{a}_3 - \hat{a}_0 \\
\end{pmatrix}
= {}^{t}C^{'}
\begin{pmatrix}
	m_1 \\
	m_2 \\
	m_3 \\
	m_4 \\
	m_5 
\end{pmatrix}.
$$
В этой схеме вычислений заметим, что матрица $B^{'}$ имеет первую  стро-\linebreak ку из одних единиц (это соответствует вычислению многочлена в  точ-\linebreak ке $1$), что дает $\hat{a}_0 = a_0 + a_1 + a_2 + a_3 + a_4 = b_1 + a_0$. Далее заметим,
\begin{wraptable}{r}{0.4\textwidth}
$$
\begin{pmatrix}
\hat{a}_1 \\
\hat{a}_2 \\
\hat{a}_4 \\
\hat{a}_3 
\end{pmatrix}
= ^{t}C^{'}
\begin{pmatrix}
	m_1 + \hat{a}_0 \\
	m_2 \\
	m_3 \\
	m_4 \\
	m_5 \\
\end{pmatrix}.
$$
\end{wraptable}
что матрица $C^{'}$ также имеет первую строку, состоящую целиком из единиц, так как $C^{'}$  совпадает с матрицей $A$ в первой реализации (см. формулу ($9$) на стр. $643$), что также  соответствует вычислению многочлена во многих точках, включая единицу. \\ \par
Следовательно, первый столбец матрицы $^{t}C^{'}$ состоит из единиц,  откуда получаем равенство, приведенное справа. \par
Разумеется, рассуждение, которое мы только что провели для Cc$_4$ и $D_{FT_5}$, легко обобщается на Cc$_{p-1}$ и. Значит, можно дословно сохранить схему вычисления Cc$_{p-1}$ и DFT$_p$, используя два дополнительных 
сложения, вычислить DFT$_p$. Эти два сложения добавлены: одно перед вычислением элементарных произведений ${\alpha}_i b_i$: $m_0 = a_0 + b_1$, а другое --- перед применением $^{t}C^{'}$: $m_1 \leftarrow m_0 + m_1$. Для вычисления $\lambda$DFT$_5$ надо вычислить выражение $m_0 = \lambda (a_0 + b_1)$, требующее одного дополнительного умножения. \\

\textbf{(50) Резюме.} \\

\textit{Схема вычисления $\hat{a} = \lambda$DFT$_p(a)$ получается из схемы вычисления Cc$_{p-1}$ добавлением двух сложений и одного умножения: сначала добавляется операция $m_0 = \lambda \times (a_0 + \sum)$ перед вычислением элементарных произведений (в этой операции $\sum$ обозначает сумму всех $a_i$, при $i \neq 0$, причем эта сумма эффективно вычисляется в процессе циклической свертки), затем осуществляется преобразование $m_1 \leftarrow m_0 + m_1$, в котором $m_1$ обозначает первое элементарное произведение (кратное $\sum$, вычисленное в процессе циклической свертки.} \par
\textit{Если вычисление Cc$_{p-1}$ требует $\mathcal{M}$ умножений и $\mathcal{A}$ сложений, то вычисление Cc$_p$ требует $\mathcal{M}$ умножений и $\mathcal{A}$ сложений, а вычисление $\lambda$DFT$_{p}$ требует $\mathcal{M}+1$ умножений и $\mathcal{A}+2$ сложений.} \par
Применим теперь это исследование к частному случаю DTF$_5$. Для этого вернемся к формулам, приведенным на стр. $648$, позволяющим  вычислять Cc$_4$, заменяя в них ($y_0$, $y_3$, $y_2$, $y_1$) на ($a_1$, $a_2$, $a_4$, $a_3$), ($x_0$, $x_1$, $x_2$, $x_4$) на ($\omega - 1$, ${\omega}^2 - 1$, ${\omega}^4 - 1$, ${\omega}^3 - 1$), и ($z_0$, $z_1$, $z_2$, $z_3$) на ($\hat{a}_1$, $\hat{a}_2$, $\hat{a}_4$, $\hat{a}_3$). \par
Схема, представленная в таблице $5$, позволяет вычислять  преобразование Фурье DTF$_5$ с использованием $17$ сложений и $5$ умножений, а также, вычислять $\lambda$DFT$_p(a)$, используя $17$ сложений и $6$ умножений. \par
Если сопоставить эти результаты с оценками, полученными в предыдущем разделе для сложности вычисления DDTF$_3$, можно заключить, что для вычисления преобразования Фурье порядка $15$, если  допускается предварительное вычисление констант, тогда достаточно $17$ умножений и $81$ сложения в базовом теле! Можно сопоставить этот результат и $196$ умножений и $210$ сложений в наивном методе, однако (внимание: эта сложность квадратична по отношению к размерам данных, тогда как все другие методы дают оценку $\mathcal{O}(n\,log\,n)$ или $76$ умножений и $90$ сложений в методе Кули --- Тьюки, или же $68$ умножений и $90$  сложений в методе Гуда, не рассматривавшемся Виноградом. Замечательно, не правда ли?
\begin{center}
\fbox{
\begin{minipage}{38em}
\begin{align*}
	&s_1 \leftarrow a_1 +a_4, \qquad & s_2 \leftarrow a_2 +a_3, \qquad & s_3 \leftarrow s_1 +s_2, \qquad & s_4~\leftarrow~s_1~-~s_2, \\
	&s_5 \leftarrow a_1 -a_4, \qquad & s_6 \leftarrow a_2 -a_3, \qquad & s_7 \leftarrow s_5 -s_6, \qquad & m_o~\leftarrow~\lambda~\times~(a_0~+~s_3), \\
\end{align*}
\begin{align*}
	&m1 \leftarrow \lambda \frac{-5}{4} \times s_3 &{} m_2~\leftarrow~\lambda\frac{{\omega}^1-{\omega}^2+{\omega}^4-{\omega}^3}{4}~\times~s4, \\
	&{} \qquad &{} m_3~\leftarrow~\lambda\frac{{\omega}^1+{\omega}^2-{\omega}^4-{\omega}^3}{4}~\times~s5, \\
	\end{align*}
\begin{align*}
	&m_4\leftarrow~\lambda\frac{{\omega}^2+{\omega}^4-{\omega}^1-{\omega}^3}{4}~\times~s6, &m_5\leftarrow~\lambda\frac{{\omega}^3-{\omega}^2}{2}~\times~s7, \\
		\end{align*}
\begin{align*}
	&{} &m_1 \leftarrow m_0 + m_1 \\
	&s_8 \leftarrow m_1 + m_2, &s_9 \leftarrow m_3 + m_5, \qquad &s_{10} \leftarrow m_1 - m_2, \qquad s_{11} \leftarrow m_4 - m_5, \\
	&s_{12} \leftarrow s_8 + s_9, &s_{13}\leftarrow s_{10} + s_{11},\qquad &s_{14} \leftarrow s_8 - s_9, \qquad s_15 \leftarrow s_{10} - s_{11}, \\
	&\hat{a}_0 \leftarrow m_0, &\hat{a}_1 \leftarrow s_{12}, \qquad &\hat{a}_2 \leftarrow s_{13}, \hat{a}_3 \leftarrow s_{15}, \qquad \hat{a}_4 \leftarrow s_{14}
\end{align*}
\end{minipage}
}
\end{center}
\begin{center}
\textbf{Таблица 5. } Преобразование Фурье на $5$ точках: $\hat{a} = \lambda\text{DFT}_{5,\omega}(a)$ 
\end{center}

\section{От FFT к тензорному произведению} 

Мы надеемся, что содержание этой главы показывает, что дискретное преобразование Фурье существует не только ради оптимизации произведения многочленов (хотя это и было нашей отправной точкой). \par

На протяжении всего исследования мы встречались почти со всеми понятиями элементарной арифметики, представленными ранее: цикло-томическими многочленами, группами обратимых элементов по  модулю целого числа, примитивными корнями по модулю $p$, китайской  теоремой об остатках, смешанной системой счисления... Мы показали реализацию алгоритма FFT (в которой участвует знаменитая бинарная инволюция), а также коснулись работ Винограда, посвещенных  вычислению билинейных форм (с известными замечательными следствиями о DFT). \par
Мы сознательно отказались от точки зрения, состоящей в определении с самого начала алгебры конечной абелевой группы, в пользу дуальной конечной абелевой группы (группы, образованной  характерами). Под этим формальным углом зрения дискретное преобразование Фурье может рассматриваться как преобразование алгебры абелевой группы в алгебру ее дуальной группы. Тогда появляется соблазн  определить также тензорное произведение алгебр, чтобы доказать теорему Гуда во всей ее общности: алгебра произведения двух групп является тензорным произведением алгебр каждого из сомножителей с  аналогичным свойством, для преобразования Фурье... \par
Однако мы ввели мало-помалу основные инструменты для изучения дискретного преобразования Фурье. DTF$_n$ может рассматриваться как преобразование алгебры $A[T](T^n-1)$) (модель алгебры $A[\mathbb{Z}_n$  циклической группы $\mathbb{Z}_n$). Нам пришлось иллюстрировть наши методы  многочисленными численными примерами (конкретные образы): матрица Вандермонда есть образ DTF, матрица-циркулянт --- образ алгебры циклической свертки...\par
Мы вполне осознаем наивность нашего определения тензорного  произведения, но сколько студентов старших курсов способны понять  абстрактные определения? Наш подход (определение тензорного  произведения двух матриц) позволяет нам лишь обосновать метод Гуда, и мы надеемся, что с помощью конкретных многочисленных примеров, которые мы привели, читатель сможет лучше понять \textit{абстрактное}  тензорное произведение. Мы не против того, чтобы представить  последний аспект так обесславленного тензорного произведения. Для взаимно простых р и q китайская теорема об остатках опять порождает и  устанавливает изоморфизм алгебр $A[\mathbb{Z}_{pq}] \simeq A[\mathbb{Z}_p] \otimes A[\mathbb{Z}_p]$, неуловимый с нашей наивной точки зрения (но не тензорное произведение алгебр). В действительности, и это было нашим желанием, этот изоморфизм  хорошо прочитывается на матрице-циркулянте порядка pq: с точностью до перестановки строк и столбцов он появляется как матрица-циркулянт
\newpage
порядка $p$, состоящая из блоков порядка $q$, каждый из которых является, в свою очередь, циркулянтом.\ 
\hspace{5pt}

Так, для $p$ = 2 и $q$ = 3 имеем схему: 

\[ 
\begin{pmatrix} 
z_0 \\ 
z_1 \\ 
z_2 \\ 
z_3 \\ 
z_4 \\ 
z_5
\end{pmatrix} = 
\begin{pmatrix} 
x_0 & x_1 & x_2 & x_3 & x_4 & x_5 \\ 
x_1 & x_2 & x_3 & x_4 & x_5 & x_0 \\ 
x_2 & x_3 & x_4 & x_5 & x_0 & x_1 \\ 
x_3 & x_4 & x_5 & x_0 & x_1 & x_2 \\ 
x_4 & x_5 & x_0 & x_1 & x_2 & x_3 \\ 
x_5 & x_0 & x_1 & x_2 & x_3 & x_4
\end{pmatrix} 
\begin{pmatrix} 
y_0 \\ 
y_1 \\ 
y_2 \\ 
y_3 \\ 
y_4 \\ 
y_5
\end{pmatrix} \Longleftrightarrow
\] 

\[ 
\Longleftrightarrow
\begin{pmatrix} 
z_0 \\ 
z_1 \\ 
z_2 \\ 
z_3 \\ 
z_4 \\ 
z_5
\end{pmatrix} = 
\begin{pmatrix} 
x_0 \, x_4 \, x_2 & x_3 \, x_1 \, x_5 \\ 
x_4 \, x_2 \, x_0 & x_1 \, x_5 \, x_3 \\ 
x_2 \, x_0 \, x_4 & x_5 \, x_3 \, x_1 \\ 
x_3 \, x_1 \, x_5 & x_0 \, x_4 \, x_2 \\ 
x_1 \, x_5 \, x_3 & x_4 \, x_2 \, x_0 \\ 
x_5 \, x_3 \, x_1 & x_2 \, x_0 \, x_4
\end{pmatrix} 
\begin{pmatrix} 
y_0 \\ 
y_4 \\ 
y_2 \\ 
y_3 \\ 
y_1 \\ 
y_5
\end{pmatrix}.
\] 
Конкретная визуализация тензорного произведения алгебр, не так ли? \ 

\vspace{2pt} В качестве упражнения, и не только для того, чтобы успокоить пу­- \linebreak
ристов (формальные рассуждения проясняют структуру), мы как раз \linebreak
и придерживались абстрактной точки зрения (преобразование Фурье \linebreak
на конечной абелевой группе). При этом выявляются обе алгебраиче­- \linebreak
ские структуры $A[\Omega]$: почленное умножение функций (единственным \linebreak
образом определенное структурой $A$), которое требует $|\Omega|$ операций и, \linebreak
циклическая свертка (существенным образом использующая структуру \linebreak
группы $\Omega$), требующая $|\Omega|^2$ операций; при этом преобразование Фурье \linebreak
используется для перехода от первого преобразования ко второму и \linebreak
обратно. \ 

\vspace{2pt}
Мы заключили пари: можно придти к абстрактной математике че­- \linebreak
рез конкретный способ изложения. Удалось ли нам это? Судить чита­- \linebreak
телю.\ 

\vspace{3pt}\hangindent=29pt
\hspace{28pt} \textit{В наш век пострационализма все большее число работ пишет­- \linebreak
ся на символических языках и все труднее понять почему: какова \linebreak
их действительная цель и в чем необходимость или преимуще­- \linebreak
ство быть окруженным томами, состоящими из банальностей, \linebreak
представленных в символической форме? Кажется, что симво­- \linebreak
лизм становится ценностью в себе и что он должен почитать­- \linebreak
ся за великую точность. Речь идет о новом выражении старых \linebreak
истин, новом символическом ритуале, новой религиозной сути. \linebreak
Однако единственная возможная ценность такого рода вещей,} \linebreak

\newpage
\hangindent=29pt
\hspace{10pt} \textit{единственно возможное извинение за сомнительную претензию \linebreak
на точность, кажется, в этом и состоит. Как только выявляется \linebreak
ошибка или противоречие, всякая словесная увертка становится \linebreak
неуместной: достаточно доказать это.} \ 

\vspace{5pt}\hspace{70pt} \textit{Карл Р. Поппер, Логика научного открытия (1954 [149])}

\newpage
\thispagestyle{empty}
$\newline$
$\newline$
$\newline$
\begin{center}
\LARGE{$\mathbf{Упражнения}$}
\end{center}
$\newline$
$\newline$
$\mathbf{1.}$\, $\mathbf{Наивное}$ $\mathbf{умножение.}$ $\mathbf{Метод}$ $\mathbf{Горнера}$ \ 

\vspace{15pt} $\mathbf{a.}$ Проверить, что наивный метод умножения двух многочленов сте­- \linebreak
пени $n$ $-$ 1 требует в точности $n^2$ умножений и $(n - 1)^2$ сложений. \ 

\vspace{5pt} $\mathbf{b.}$ Записать алгоритм, позволяющий найти значение в точке мно­- \linebreak
гочлена степени $n$ $-$ 1 по методу Горнера, и оценить его сложность как \linebreak
можно точнее. \ 

\vspace{15pt} \noindent $\mathbf{2.}$\, $\mathbf{Интероляция}$ \ 

\vspace{15pt} $\mathbf{a.}$ Пусть семейство $n$ многочленов $P_0(X), P_1(X), \dots, P_{n-1}(X)$ тако- \linebreak
во, что $P_i$ ---  унитарный многочлен степени $i$. Показать, что $\{P_0, P_1, \dots,$ \linebreak
$P_{n-1}\}$ является базисом модуля $A_n[X]$. \ 

\vspace{5pt} $\mathbf{b.}$ Пусть $x_0, x_1, \dots, x_{n-1}$ --- $n$ различных точек. Рассматривая в \linebreak
$A_n[X]$ базис

\begin{center}
1, $X$ $-$ $x_0$, $(X - x_0)(X - x_1)$, $\dots$, $(X - x_0)(X - x_1)$ $\dots$ $(X - x_{n-2})$,
\end{center}

\noindent написать алгоритм, позволяющий (при некоторых условиях, нуждаю­- \linebreak
щихся в уточнении) интерполировать многочлен $P(X)$ степени < $n$, \linebreak
такой, что $P(x_i)$ = $y_i$. Проверить этот алгоритм в $\mathbb Z$ для следующих \linebreak
интерполяционных последовательностей:

\begin{center}
$(x_0, x_1, x_2, x_3)$ = ($-$1, 0, 1, 2), \hspace{10pt} $(y_0, y_1,y_2, y_3)$ = (3, 1, 3, 15) \newline
или \hspace{5pt} $(y_0, y_1,y_2, y_3)$ = (3, 1, 3, 0).
\end{center}

$\mathbf{Произведение\,в}$ $A[X]$ \ 

\hspace{5pt} Пусть $\mathcal{F}$ --- отображение, соответствующее вычислению значений \linebreak
многочлена в $A[X]$ в $n$ точках $x_0, x_1, \dots, x_{n-1}$ из кольца $A$.  Предполо­- \linebreak
жим, что элементы $x_i-x_j$ обратимы в этом кольце при $i$ $\ne$ $j$.  Обозначим \linebreak
тогда $P$ $\star$ $Q$ = $\mathcal{F}^{-1}$ $(\mathcal{F}(P) \times \mathcal{F}(Q))$, где $\mathcal{F}^{-1}$ представляет собой опера­- \linebreak
цию интерполяции. Что представляет собой это произведение на $A[X]$? \newpage
%660
\noindent {\bf 4. Число единиц в матрице Вандермонда, определенной
с помощью корня из единицы }

\medskip

Обозначим $v(n)$ число единиц в матрице $V_w$, элементы которой $w^{ij}$ где $w$ — примитивный корень $n$-й степени из единицы. Иначе говоря,
$v(n)$ — число пар $(i, j) \in \mathbb Z_n \times \mathbb Z_n$, таких, что $ij = 0$ (в $\mathbb Z_n$).

{\bf a.}. Показать, что $v$ — мультипликативная функция, т.е.
$v(nm)=v(n)v(m)$, если $n \wedge m = 1$.

{\bf b.} Проверить, что $v(n) = \sum_{i=0}^{n-1}$ НОД$(i,n) = \sum_{d|n}d \cdot \varphi({n \over d})$, где $\varphi$ — функция Эйлера.

{\bf c.} Вывести из предыдущего, что для простого $p$ и $k \in \mathbb N^* : v(p^k) = p^{k-1}(p+(k-1))$; в частности, $v(p)=2p - 1$.

\medskip

\noindent {\bf 5. $FFT$ на 6 точках}

\bigskip

{\bf a.} Пусть $w$ — корень 6-й степени из единицы. Записав $б = 2 \times 3 = 3 \times 2$ и используя пример $n = 15 = 3 \times 5$ из текста, определить две схемы вычисления значений многочлена степени $< 6$ в степенях $w$.

Сравнить сложность полученной схемы со сложностью обычного метода.

{\bf b.} Тот же вопрос, но в предположении, что $w$ — примитивный
корень 6-й степени из единицы.

\medskip

\noindent{\bf 6. Два этапа в $FFT$}

\smallskip

Рассмотрим принцип вычисления многочлена степени $< 15$, приведенный в разделе 1.2. Объяснить, как второй этап этого вычисления (тот, который выражает $\hat a_j$ в зависимости от $b_j$) может рассматриваться как множество 5 преобразований Фурье на 3 точках. Выразить сложность $FFT_{3 \times 5}$ как функцию от сложностей $DFT_5$ и $DFT_3$.

\medskip

\noindent{\bf 7. Циклическая свертка}

\smallskip

Показать, что множество функций из $[0, n[$ в кольцо $A$ с обычным
умножением и циклической сверткой есть коммутативное унитарное
кольцо.

\medskip

\noindent{\bf 8. Примитивные корни из единицы}

\smallskip

{\bf а}. Пусть $w$ — корень $n$-й степени из единицы такой, что $1 - w^i$ не является делителем нуля при $i = 1,\ldots,n — 1$. Доказать равенство многочленов $X^n -1=(X-1)(X-w)(X-w^2) \ldots (X-w^{n-1})$.\par
\newpage
{\bf b.} Вывести отсюда, что $n = (1 - w)(1 - w^2) \ldots (1 - w^{n-1})$, и
получить характеризацию примитивных корней из единицы, данную в
предложении 4.

{\bf c.} Показать, что свойство из пункта {\bf b}, для произвольных корней из единицы, вообще говоря, не верно.

\bigskip

\noindent{\bf 9. Примитивные корни из единицы (продолжение)}

\bigskip

{\bf a.} Пусть $\varphi : A \rightarrow B$ — морфизм колец. Проверить, что, если $w \in A$ - примитивный корень $n$-й степени из единицы, то это верно и для $\varphi(w)$. Остается ли верным этот результат, если рассматривать только корни из единицы порядка $n$?

\smallskip

{\bf b.} Предположим, что целое число $x \in \mathbb Z$ — примитивный корень из
единицы порядка $n$ по модулю $m$. Показать, что это верно для $x \mod q$ для любого $q$, делящего $m$.

\smallskip

{\bf c.} Показать, что $x \in \mathbb Z$ тогда и только тогда является примитивным
корнем из единицы по модулю $m$, когда для любого \textit{простого} делителя $p$ числа $m$ элемент $x$ является корнем порядка $n$ по модулю $p$.

\bigskip

{\bf 10. Вычисление циклотомических многочленов}

\medskip

Это упражнение устанавливает некоторые соотношения между циклотомическими многочленами, которые позволяют свести, в частности, вычисление произвольного циклотомического многочлена $\Phi_n(X)$ к вычислению $\Phi_m(X)$, где $m$ - произведение нечетных \textit{простых} делителей $n$.

{\bf a.} Показать, что $\Phi_n(X)$ — унитарный многочлен, имеющий только
простые корни (в $\mathbb C$) степени $\varphi(n)$, где $\varphi$ - функция Эйлера (число образующих в циклической группе порядка $n$). Показать также, что
$\Phi_n(0)$ для любого $n \geq 2$.

{\bf b.} Доказать равенство $\Phi_{2q}(X) = \Phi_{q}(-X)$ для целого \textit{нечетного} \\ $q \geq 3$.

{\bf c.} Показать, что
многочлен $\Phi_{nm}(X)$ делит $\Phi_n(X^m)$.

{\bf d.} Пусть $n$ и $m$ — такие два целых числа, что {\bf всякий простой \\
 делитель $m$ является простым делителем $n$} (это будет так, на- \\
пример, если $m$ делит $n$). Доказать равенство $\Phi_{nm}(X) = \Phi_n(X^m)$.

{\bf e.} Вывести из предыдущего, что для целых $n, n_1, n_2, \ldots n_k$ и про- \\ 
стого $p$:

\begin{center}
$\Phi_{n_1^{\alpha_1}n_2^{\alpha_2} \ldots n_k^{\alpha_k}}(X)=\Phi_{n_1n_2 \ldots n_k}(X^{n_1^{\alpha_1-1}}X^{n_2^{\alpha_2-2}} \ldots X^{n_k^{\alpha_k-1}})$,
\end{center}
\newpage
$\Phi_{n^\alpha}(X)=\Phi_n(X^{n^{\alpha-1}})$ для простого $p$, не делящего $n$, $\Phi_{np}(X) = {{\Phi_n(X^p)} \over {\Phi_n(X)}}$;\\
$\Phi_{p^\alpha}(X)=\Phi_p(X^{p^{\alpha-1}})=X^{(p-1)p^{\alpha-1}}+X^{(p-2)p^{\alpha-1}}+\ldots+X^{2p^{\alpha-1}}+X^{p^{\alpha-1}}+1$.

\bigskip

{\bf f.} Пусть $n \geq 2$. Показать, что $\Phi_n(1) \neq 1$ тогда и только тогда, когда $n$ - степень простого числа $p$. Показать, что тогда $\Phi_n(1) = p$

\bigskip


\noindent{\bf 11. Функция Мёбиуса и циклотомические многочлены}

\medskip

Обозначим через $\mu$ функцию Мёбиуса (глава II). Показать, что $\Phi_n(X)=\prod_{d|n}(X^d -1)^{\mu({n \over d})}$.

\medskip

\noindent{\bf 12. Примитивные корни по модулю чисел Ферма и Мерсенна}

\medskip

{\bf a.} Пусть $M_p$ - число Мерсенна, т.е. число $2^p-1$, где $p$ -\textit{нечетное простое} число. Показать, что $-2$ является примитивным корнем порядка $2p$ по модулю $M_p$.

{\bf b.} Пусть $F_q$ - число Ферма $2^{2^{q}}+1$. Показать, что 2 является примитивным корнем порядка $2^{q+1}$ по модулю $F_q$. Вычислить $\sqrt{2}$ по
модулю $F_q$ и показать, что это {\bf примитивный} корень порядка $2^{q+2}$.

\bigskip

\noindent {\bf 13. Частный случай теоремы Дирихле}

\medskip

{\bf a.} Доказать, что для любого целого $n$ существует простое число $p$, такое, что группа обратимых элементов по модулю $p$ содержит элемент порядка $n$.

{\bf b.} Доказать, что это число $p$ имеет вид $kn+1$. Обратно, доказать, что если $p$ - простое число вида $kn+1$, то группа обратимых элементов по модулю $p$ содержит элемент порядка $n$.

{\bf c.} Для фиксированного $n$ показать, что существует бесконечно много простых чисел вида $kn+1$.

\bigskip

\noindent {\bf 14. Загадочная биекция}

\bigskip

{\bf а.} Пусть даны два конечных вполне упорядоченных множества $E$ и $F$ одинаковой мощности, являющейся степенью двойки. Сопоставим им биекцию $\sigma_{E,F}$ множества $E$ на $F$, определяемую рекурсивно следующим образом:

\begin{itemize}
\item{если $E$ и $F$ — два множества из двух элементов, то $\sigma_{E,F}$ - единственный изоморфизм $E$ на $F$,}
\newpage

\item{если $Е$ и $F$ состоят из 2$n$ элементов, то нумеруем их в возрастающем порядке $E = \{e_{1} < e_{2} < \dots < e_{2n}\}, F = \{f_{1} < f_{2} < \dots < f_{2n}\}$, определяем $E^{1} = \{e_{1}, \dots ,e_{n}\}$, $E^{2} = \{e_{n+1}, \dots ,e_{2n}\}$, $F_{1} = \{f_{1}, \dots ,f_{2n-1}\}$, $F_{2} = \{f_{2}, \dots ,f_{2n}\}$ и тогда полагаем $\sigma_{E,F}(x) = \sigma_{E^{1},F_{1}}(x)$ для $x \ni E^{2}$. Определить
эту биекцию не рекурсивным образом и показать, как можно ее
вычислить.}
\end{itemize}

\textbf{b.} Определить аналогичную биекцию для случая, когда число 2 
заменяется на произвольное целое $q \geqslant 3$.

\medskip
\noindent
\textbf{15. Вычисление инволюции, действующей в $F_{FT}$}

\medskip
Пусть $q$ и $k$ --- два фиксированных целых числа, $q \geqslant 2$. Надо составить таблицу для $\widetilde{s}$ при $0 \leqslant s < q^{k}$, где $\widetilde{s}$ ---  число, разложение которого по основанию $q$ является обратным по отношению к разложению по основанию $s$.

\textbf{a.} Как получить разложение по основанию $q$ для $\widetilde{s+1}$, если известно разложение $s$?

\textbf{b.} Пусть $T_{\widetilde{s}}$ --- массив, представляющий разложение $\widetilde{s}$ по основанию $q$. Выразить $\widetilde{s+1}$ и $T_{\widetilde{s+1}}$ как функции $\widetilde{s}$ и массива $T_{\widetilde{s}}$

\textbf{c.} Вывести алгоритм вычисления таблицы значений $\widetilde{s}$.

\medskip
\noindent
\textbf{16. $F_{FT}$ порядка, являющегося степенью целого числа}

\medskip
Доказать предложение 12 на странице 598.

\medskip
\noindent
\textbf{17. Алгоритм $F_{FT^{q^{k}}}$}

\medskip
Выписать явные формулы для $F_{FT_{n}}$, где $n = q^{k}$ (формулы, аналогичные тем, что входят в предложение (5) на стр. 603). Описать в деталях алгоритм  $F_{FT}$ для $n = 3^{k}$.

\medskip
\noindent
\textbf{18. Итерация метода Кули — Тьюки}

\medskip
Пусть имеем целое число $n = n_{0}n_{1} \dots n_{k}$ --- корень $n$-й степени из единицы в кольце $А$ и функцию $f : [0,n[\rightarrow A$. Написать итеративные формулы для вычисления преобразования Фурье $\hat{f}$ для $f$, определенное при $0 \leqslant j < n$ по формуле $\hat{f}(j) = \Sigma_{0\leqslant i<n}f(i) \omega^{ij}$, используя треугольное выражение справа в формуле (8) (см. раздел 4.2.2).
\newpage
    \medskip
\noindent\textbf{19. Метод Кули-Тьюки: пример $12=4\times 3$}\\
\medskip
В тексте метод Кули-Тьюки проиллюстрирован с помощью примера\newline $n=12=3\times 4(p=3, q=4)$. Провести аналогичное исследование для случая $p=4, q=3$ и сравнить эти два метода.\\
\ \newline
\noindent\textbf{20. Метод Гуда для $15=5\times 3$}\\

Детализировать метод Гуда для корня 15-й степени из единицы с разложением $15=5\times 3$\\
 \newline
\noindent\textbf{21.Метод Гуда в терминах многочленов}\\

В этом упражнении $p$ и $q$ обозначают два целых взаимно простых числа и $n=pq$ - их произведение. Пусть $\theta : \mathbb{Z}\rightarrow \mathbb{Z} \times \mathbb{Z}$ и $\theta ' : \mathbb{Z}\rightarrow \mathbb{Z} \times \mathbb{Z}$ индуцируют изоморфизмы \textit{групп} $\mathbb{Z}_n$ на $\mathbb{Z}_p \times \mathbb{Z}_q$ и наоборот. Полагаем $\theta (1)=(r,s), \theta '(1,0)=r'$ и $\theta '(0,1)=s'$.\newline
\hspace*{15pt}\textbf{a.} Напомнить, при каких условиях на $(r,s)$ и $(r',s')\theta$ и $\theta '$ являются взаимными изоморфизмами. Каковы соответствующие значения в китайской теореме об остатках? Показать, что можно выбрать $r'=q$ и $s'=p$.\newline
\hspace*{15pt}\textbf{b.} Показать, как можно определить с помощью $\theta$ и $\theta '$, морфизмы колец $A[X]$ в $A[Y,Z]$ и $A[Y,Z]$ в $A[X]$, индуцирующие посредством перехода к фактору взаимно обратные изоморфизмы:
$$
\theta : A[X]/(X^n-1)\rightarrow A[Y,Z]/(Y^q-1, Z^p-1)
$$
$$
\text{} \theta ' : A[Y,Z]/(Y^q-1, Z^p-1)\rightarrow A[X]/(X^n-1)
$$
\newline
\hspace*{15pt}\textbf{c.} Пусть $P(X)\in A[X]/(X^n-1)$. Используя $\theta$ и $\theta '$, записать равенство в $A[X]/(X^n-1)$, приводящее к методу Гуда. Проиллюстрировать этот метод на примерах.\\
\ \newline
\noindent\textbf{22.Циклическая свертка на 2 точках} \\

\hspace*{15pt}\textbf{a.} Числа, участвующие в этом упражнении - элементы кольца $A$, в котором элемент 2 обратим. Обычно вычисление циклической свертки на 2 точках требует 2 сложений и 4 умножений. Предположим, что матрица-циркулянт \textit{фиксирована} и что свертка вычисляется для $(y_0, y_1)$, где $y_0, y_1$ - \textit{переменные}. Показать, как вычисление
$$
\begin{pmatrix}
	    z_0 \\
	    z_1 \\
\end{pmatrix} = 
\begin{pmatrix}
	    x_0 & x_1 \\
	    x_1 & x_0 \\
\end{pmatrix}
\begin{pmatrix}
	    y_0 \\
	    y_1 \\
\end{pmatrix} = 
\begin{pmatrix}
	    x_0y_0 + x_1y_1 \\
	    x_1y_0 + x_0y_1 \\
\end{pmatrix}
$$
\newpage
\noindent
может быть реализовано с помощью 4 сложений и 2 умножений.

\textbf{b.} Напомним, что $z_{0}+z_{1}T = (x_{0} + x_{1}T)*(y_{0}+y_{1}T) (mod T^{2} - 1)$. Используя соотношение $T^{2} - 1 = (T - 1)(T + 1)$ и китайскую теорему об остатках, доказать другим способом результаты предыдущей задачи.

\medskip
\noindent
\textbf{23. Умножение двух комплексных чисел}

\medskip
Можно перемножить два комплексных числа, используя три вещественных умножения. Каким образом?

\medskip
\noindent
\textbf{24. Дискретное преобразование Фурье на 3 точках}

\medskip
Пусть в кольце $А$ элемент 2 обратим. Пусть $\omega$ — кубический корень
\begin{wrapfigure}{i}{0.4\textwidth}
$\begin{pmatrix}
\hat{a}_{0} \\
\hat{a}_{1} \\
\hat{a}_{2} \\
\end{pmatrix}$ = $\begin{pmatrix}
1&1&1 \\
1&\omega&\omega^{2} \\
1&\omega^2&\omega \\
\end{pmatrix}$ $\begin{pmatrix}
a_0 \\
a_1 \\
a_2 \\
\end{pmatrix}$ 
\end{wrapfigure}
из единицы. Обычный метод вычисления матричного произведения справа требует 6 сложений и 4 умножения, при условии, что 3 х 3-матрица уже вычислена. Применяя упражнение 22 \\
(к двум последним строчкам и столбцам матрицы справа), определить алгоритм вычисления, требующий 6 сложений и 2 умножения.

\medskip
\noindent
\textbf{25. Тензорный ранг семейства элементарных билинейных\\
${}$ ${}$  ${}$ ${}$  ${}$ ${}$ форм}

\medskip
Пусть Е и F — два векторных K-пространства размерностей n и m соответственно.

\textbf{a.} Проверить, что произведения $а \times b, a \in E^{*}, b \in F^{*}$, порождают пространство билинейных форм на Е х F.

\textbf{b.} Для данной билинейной формы h рассмотрим наименьшее r, такое, что h является суммой r элементарных билинейных форм\\
$h = а_1 \times b_1 + a_2 \times b_2 + \dots + a_r \times b_r$. Показать, что $\{ a_1, a_2, \dots ,a_r \}$ --- ба-\\
зис линейного векторного пространства, состоящего из линейных форм\newline
$\{ h(.,у), у \in F \}$. Аналогичный результат дается ниже. Какой получится

\begin{wrapfigure}{i}{0.4\textwidth}
$H = \begin{pmatrix}
11&1&25&3&7\\
8&10&46&3&22\\
9&7&39&4&15\\
-2&-3&-13&-1&-6\\
\end{pmatrix}$
\end{wrapfigure}

результат, если выразить h в базисе $Е \times F$? Как получить в явном виде разложение для минимального r? Обсудить пример билинейной формы, матрица которой H изображена справа.


\textbf{c.} Рассмотрим семейство, состоящее из двух билинейных форм $z_{0},z{1}$ типа $2 \times 2$, определенное через $z_0 + z_1T = (x_0 + x_1T) \times (y_0 + y_1T)(mod T^2 - rT - q)$.\newpage
\indent Выразить $z_0, z_1$. Полагая, что многочлен $P(T) = T^2 - rT - q$ не\\имеет корней в основном кольце, доказать, что тензорный ранг $\{z_0, z_1 \}$ \\ есть 3.

\textbf{Указание.} Для доказательства, что ранг $\leqslant 3$, надо предъявить \\ 3 элементарные билинейные формы, линейными комбинациями кото-\\рых являются $z_0, z_1$. Для доказательства, что ранг >2, предположим\\ противное, т.е. что существуют разложения $z_0 = \alpha a_1 \otimes b_1 + \beta a_2 \otimes b_2, $ \\ $z_1 = \gamma a_1 \otimes b_1 + \delta a_2 \otimes b_2, $, а затем покажем, что матрица $ -\delta z_0 + \beta z_1$ \\ обратима.

\textbf{d.} \quad Доказать, что понятие тензорного ранга зависит от базового \\ кольца: если $H = \{ h_1, h_2, \ldots, h_s \}$ —-семейство $s$ билинейных форм на \\ $K^n \times K^m$ и $K'$ —- надтело тела $K$, то тензорный ранг $H$ над $K$ может \\ отличаться от ранга $H$ над $K'$.\\

\noindent \textbf{26. \quad Тензорный ранг произведения двух многочленов}

\medskip
\indent Пусть $X(T)$ и $Y(T)$ —- два порождающих многочлена степеней $n$ и $m$ \\ соответственно ( с коэффициентами в теле $K$):\\
\begin{center}
$X(T) = x_nT^n + \cdots + x_1T + x_0$ и $Y(T) = y_nT^n + \cdots + y_1T + y_0$
\end{center}
Определим затем семейство $n+m+1$ билинейных форм типа \\ $(n+1) \times (m+1)$ , состоящее из коэффициентов многочлена , являю-\\щегося произведением $X(T)Y(T)$ : \\
\begin{center}
$X(T)Y(T) = z_{n+m}T^{n+m} + z_{n+m-1}T^{n+m-1} + \cdots + z_1T + z_0 $,
\end{center}
где каждое $z_k$-— билинейная форма от $x_i \times y_j$. \\
\indent \textbf{a.} \quad Показать, что билинейный формы $z_0, z_1, \ldots , z_{n+m}$ линейно неза-\\висимы. \\
\indent \textbf{b.} \quad Предположим, что базовое тело $K$ бесконечно. Используя
вы-\\числения в точке, показать, что тензорный ранг этого семейства в \\
точности равен $n+m+1$,\textit{ т.е. для умножения многочлена степени $n$\\
на многочлен степени $m$ используются $n+m+1$ «общих умножений»}
(умножения на элементы $k$ не считаются, учитываются только
умно-\\жения линейных форм с порождающими коэффициентами). Выписать\\
указанные разложения для $n = 2$ и $m = 3$. \\
\indent \textbf{c.} \quad Доказать, что можно найти другое выражение семейства\\
$\{z_0, z_1,\ldots, z_{n+m}\}$ в виде линейной комбинации $n+m+1$ элементарных\\
произведений, записывая:
\begin{align*}
X(T)Y(T) = (&w_0 + q_1T + \cdots + w_{n+m-1}T^{n+m-1}) \\ &+ x_ny_m(T-t_1)(T-t_2)\ldots (T-t_{n+m}),
\end{align*}
\noindent где $t_1, t_1, \ldots, t_{n+m}$ —- $ n + m$ различных точек. Рассмотреть пример для \\ случая $n = 2, m=3$.\\

\noindent\textbf{27 \quad Тензорный ранг циклической свертки порядка 3}

\medskip
\indent Выписать в явном виде реализацию циклической свертки порядка 3 \\ и ранга 4 ( который является тензорным рангом $Cc_3$).\\

\noindent\textbf{28. \quad Циклическая свертка порядка 3 (продолжение)}

\medskip
\textbf{a.}\quad Выразить число операций, необходимых для вычисления цикли-\\ческой свертки на 3 точках $z = x \star_3 y$ , пользуясь обычным методом. \\
\indent \textbf{b.}\quad проверить, что тензор $\{z_0, z_1, z_3\}$ циклической свертки на 3 точ-\\ках :

$z_0 + z_1T + z_2T^2 = (x_0 + x_1T + x_2T^2)(y_0 + y_1T + y_2T^2)(\mod T^3 - 1)$ \\

\noindent допускает следующую реализацию (вытекающую из упражнения 27)\\ ранга 4:
\begin{center}
$ A = \dfrac{1}{3}\begin{pmatrix}
    1 & 1&1\\
    1&0&-1\\
    -1&1&0\\
    0&1&-1
    \end{pmatrix}, B = \begin{pmatrix}
    1 & 1&1\\
    1&0&-1\\
    -1&1&0\\
    0&1&-1
    \end{pmatrix}, C = \begin{pmatrix}
    1 & 1&1\\
    1&1&-2\\
    1&-2&1\\
    -2&1&1
    \end{pmatrix}.$
\end{center}

\textbf{c.} Показать, что 3-линейная форма $h=(h_{ijk})$ типа $n \times n \times n$,  ассоциированная с циклической сверткой на n точках, обладает некоторым свойством <<симметрии>>. С помощью реализации тензора циклической свертки на трех точках из предыдущей задачи, построить другую   реализацию (того же ранга), в которой элементы матрицы С будут 0, ±1. В каком случае эта реализация будет выгодной?

\medskip
\noindent
\textbf{29. Негациклическая свертка порядка 4}

\medskip
Негациклическая свертка порядка п есть произведение в пространстве $A[T]/(T^n +1)$. Реализовать негациклическую свертку порядка 4 $z = х*у$, при помощи 9 умножений и 15 сложений, при  фиксированном х.

\medskip
\noindent
\textbf{30. Произведение двух многочленов степени 3}

\medskip\par
\textbf{а.} Показать, как можно реализовать произведение двух  
многочленов степени 3 с помощью 7 умножений. Что можно сказать о размере постоянных коэффициентов?\newpage
    \textbf{b.} Найти реализацию произведения двух многочленов степени 3, использующую 9 умножений, все участвующие коэффициенты-константы в которой равны $\pm 1$. \par
    \textbf{c.} Найти в явном виде реализацию произведения в $K[T]/(T^4+T^3+T^2+T+1)$, использующую 9 умножений и 16 сложений (если один из сомножителей фиксирован).
    
    \medskip
    \noindent
\textbf{31. Циклическая свертка порядка 5}

\medskip
\textbf{a.} Каково число умножений и сложений, необходимых для наивного вычисления циклической свертки порядка 5, $z=x \star_5 y$?\par
\textbf{b.} Показать, что можно реализовать $z=x \star_5 y$ с помощью 10 умножений и 31 сложения с приемлемыми постоянными коэффициентами.

\medskip
\noindent
\textbf{32.Циклическая свертка порядка 6 и D{\footnotesize FT}$_7$}

\medskip
\textbf{a.} Каково число умножений и сложений, необходимых для наивного вычисления циклической свертки порядка 6: $z=x \star_6 y$?\par
\textbf{b.} Пусть $n$ --- целое нечетное число. Показать, как вычислить циклическую свертку порядка 2$n$, C{\footnotesize C}$_{2n}$, с помощью C{\footnotesize C}$_n$ и C{\footnotesize C}$_2$.\par
\textbf{c.} Вывести отсюда реализацию циклической свертки порядка б, $x \star_6 y$, при фиксированном $x$ с помощью 8 умножений и 34 сложений.

\medskip
\noindent
\textbf{33. Двойственность конечных абелевых групп}

\medskip
\textbf{a.} Дуальным к абелевой конечной группе $\Omega$ является множество $\hat{\Omega}$ характеров, т.~е. гомоморфизмов $\Omega$ в мультипликативную группу $U$ комплексных чисел с модулем, равным 1: $\hat{\Omega} =$ Hom$(\Omega , U)$. Это множество оснащено структурой (абелевой) группы, индуцированной структурой $U$:
\begin{center}
  $\chi_1 \chi_2 : \omega \mapsto \chi_1 (\omega) \chi_2 (\omega),$ $\chi^{-1} : \omega \mapsto (\chi (\omega))^{-1} = \chi(\omega^{-1}),$\par
  где $\chi,\ \chi_1,\ \chi_2 \in \hat{\Omega}.$
\end{center}
\noindent
Сформулировать несколько функторных свойств $\Omega \mapsto \hat{\Omega}$ и доказать, что $\hat{\Omega}$ является группой, изоморфной $\Omega$, где изоморфизм, вообще говоря, не канонический.\par
\textbf{b.} Показать, что те же самые выводы можно получить при замене группы $U$ комплексных чисел с единичным модулем на циклическую группу соответствующего порядка. \par
\textbf{c.} Каждому $ \omega \in \Omega$ сопоставим $\omega^{\#} : \hat{\Omega} \mapsto U$ по правилу $\omega^{\#}(\chi)=\chi (\omega^{-1})$. Показать, что $\omega^{\#}$ --- характер на $\hat{\Omega}$ и что $\omega \mapsto \omega^{\#}$ --- бидуальный изоморфизм на себя.\newpage

\medskip
\noindent
\textbf{34. Преобразование Фурье на конечной абелевой группе}

\medskip
Обозначим через $A[\Omega]$ $A$-модуль, состоящий из всех отображений группы $\Omega$ в кольцо $A$. Если $\Omega$ --- конечная абелева группа и если $U \subset U(A)$ --- циклическая группа, порядок которой делится на экспоненту $\Omega$ (см. определение экспоненты в решении предыдущего упражнения), то можно определить дуальную $\hat{\Omega}$ для $\Omega$ (элементы $\hat{\Omega}$ являются характерами на $\Omega$ со значениями в $U$, см. предыдущее упражнение). Тогда преобразование Фурье $A[\Omega]$ в $A[\hat{\Omega}]$ определяется таким образом:
\begin{center}
D{\footnotesize FT}$_{\Omega}$: $A[\Omega] \ni a \mapsto \hat{a} \in A[\hat{\Omega}],$ где $\hat{a}(\chi)=\sum_{\omega \in \Omega}a(\omega) \chi(\omega).$ 
\end{center}\par
\textbf{a.} Предположим, что $1-u$ обратим в $A$ для любого $u \in U - \{1\}$. Каково преобразование Фурье функции, являющейся константой, равной 1? Доказать, что $| \Omega |$ обратимо в $A$.\par 
\textbf{b.} Определим \textit{обратное} преобразование Фурье $A[\hat{\Omega}] \to A[\Omega]$ следующим образом:
\begin{center}
D{\footnotesize FT}$_{\Omega}^{-1} : A[\hat{\Omega}] \ni a' \mapsto \hat{a'} \in A[\Omega],$ где $\hat{a'}(\omega)= \frac{1}{| \Omega |} \sum_{\chi \in \hat{\Omega}}a'(\chi) \chi^{-1} (\omega).$
\end{center}
\noindent
Доказать, что D{\footnotesize FT}$_{\Omega}$ и D{\footnotesize FT}$_{\Omega}^{-1}$ взаимно обратны.\par
\textbf{c.} Напомним, что групповая алгебра $A[G]$ произвольной группы $G$ есть $A$-модуль с базисом $G$, снабженный следующим ассоциативным умножением: $(\sum\nolimits_{g \in G} \lambda_gg)(\sum\nolimits_{h \in G} \mu_hh)= \sum\nolimits_{k \in G}(\sum\nolimits_{\underset{gh=k}{g,h}} \lambda_g \mu_h k)$; иными словами, произведение двух базисных элементов $g$ и $h$ есть базисный элемент $gh$, вычисляемый в $G$. $A$-модуль $A[\Omega]$ естественным образом наделен двумя структурами $A$-алгебры: с одной стороны, умножением $a \cdot b$ функций (в $A$), а с другой стороны, сверткой $a \star b$, т.е. алгебраическим умножением группы $\Omega$. Что является единичным элементом в алгебре свертки $A[\Omega]$? Доказать, что преобразование Фурье переводит свертку в обычное произведение. То же самое для обратного преобразования Фурье с точностью до множителя $| \Omega |$.

\medskip
\noindent
\textbf{35. Преобразование Фурье на конечной абелевой группе (продолжение)}

\medskip
$U$-циклическая подгруппа $U(A)$ порядка, делящегося на экспоненту рассматриваемой конечной абелевой группы.\newpage
\textbf{a.} Пусть $\mathbb{Z}_n$ --- циклическая группа $\mathbb{Z}/n \mathbb{Z}$. Объяснить, при условии, что фиксирован некоторый элемент порядка $n$ из $U$, почему преобразование Фурье D{\footnotesize FT}$_{\mathbb{Z}_{n}}$ может рассматриваться как преобразование $A [\mathbb{Z}_n]$.\par 
\textbf{b.} Пусть $\Omega = \mathbb{Z}_p \times \mathbb{Z}_q$, где $p,q$ --- произвольные целые числа. Какова дуальная группа для $\Omega$? Зафиксируем в $U$ элементы $\omega_1$ порядка $p$ и $\omega_2$ порядка $q$. Найти преобразование Фурье на $A[\mathbb{Z}_p \times \mathbb{Z}_q]$ и показать, что оно сводится к интерполяции многочлена от двух переменных. Рассмотреть случай $p=4$, $q=2$.

\medskip
\noindent
\textbf{36. F{\footnotesize FT} на конечной абелевой группе}

\medskip
Обозначим через $\hat{\Omega}$ дуальную группу к абелевой конечной группе $\Omega$ порядка $n$, при этом элементы $\hat{\Omega}$ являются характерами со значениями в подгруппе $U$ группы $U(A)$, обратимых элементов кольца $A$. Предположим, что группа $U$ является циклической подходящего порядка (см. предыдущее упражнение). Если $a: \Omega \to A$ есть функция на $A$, то определим $\hat{a}: \hat{\Omega} \to A$ следующим образом: $\hat{a}(\chi)= \sum\nolimits_{\alpha \in \Omega} a_{\alpha}\chi(\alpha), \chi \in \hat{\Omega}.$\par 
\textbf{a.} Пусть функция $a: \Omega \to A$ фиксирована. Найти число умножений для вычисления $\hat{a}(\chi)$ где $\chi$ пробегает $\hat{\Omega}$.\par 
\textbf{b.} Пусть $\Omega_2 \subset \Omega$ --- подгруппа $\Omega$ порядка $q$ и $\Omega_1$ --- система представителей $\Omega$ по модулю $\Omega_2$. Имеем: $n=pq,\ | \Omega_1 |=| \Omega / \Omega_2 |=p,\ | \Omega_2 | =q.$ Записывая каждый элемент $\alpha \in \Omega$ в виде $\alpha = \alpha_1 + \alpha_2$, где $\alpha_1 \in \Omega_1,\ \alpha_2 \in \Omega_2$, показать, как можно вычислить все суммы $\hat{a}(\chi)$ с помощью $n(p+q)$ умножений.\par 
\textbf{c.} Пусть $n=pq$. Используя предыдущий результат для $\Omega = \mathbb{Z}/n \mathbb{Z}$, показать, как можно вывести теорему Кули -- Тьюки.

\medskip
\noindent
\textbf{37. Тензорные произведения и устойчивые схемы}

\medskip
\textbf{a.} В этом упражнении нас интересует сложность вычисления вида $A \cdot x$, где $A$ --- \textit{фиксированная} матрица, а $x$ --- \textit{переменный} вектор.
\begin{wrapfigure}{i}{0.2\textwidth}
  $
  \begin{pmatrix} a & b & c \\ d & a & b \\ e & d & a \end{pmatrix} \begin{pmatrix} x_1 \\ x_2 \\ x_3 \end{pmatrix}
  $
  \end{wrapfigure}
Показать, на примере, что вычисление, приведенное слева ($a,\ b,\ c,\ d$, и $e$ \textbf{фиксированы}, а $x_i$ --- \textbf{переменные}), может быть реализовано с помощью 7 умножений (вместо 9): такая матрица, удовлетворяющая условию $a_{i+1,j+1}=a_{ij}$, называется \textit{тёплицевой}. Сложность вычисления $A \cdot x$ измеряется числом умножений $\mathcal(M)(A)$ и числом сложений $\mathcal{A}(A)$. Предположим для простоты, что $A$ --- квадратная $p \times p$-матрица, а $B$ --- $q \times q$-матрица. Каковы значения $\mathcal{M}(A)$ и $\mathcal{A}(A)$, если вычисление реализуется наивным образом? Доказать неравенство
\end{document}
