%\setcounter{page}{7} задать страницу
% \linebreak - переход на следующую страницу с растяжением до конца строки
% \newline - переход на следующую строку
% \newtop{ЗАГОЛОВОК}  юзать чтобы вручную поменть заголовок вверху страници
% Побелы (по возрастанию) : \thinspace \medspace \thickspace \mspace{18mu}
%выравнивание \begin{center} \\ Текст выравнивается \\ \end{center} 
%                                {flushleft} - по левому краю; {flushright} - по правому краю. 
%\hangindent=3cm \hangafter=2 \noindent
%Вернтикльный отступ \par \par\smallskip \par\medskip \par\bigskip
%\  тут интер. пишем\vspace{1pt} на второй строке чтобы отрегулировать высоту абзаца

%опечатки: 
%строка 81 стр. 571  (не-квадрат) 
%строка 210 стр. 574  (не большие)
%строка 229 стр. 574  (не-квадрат)
%строка 239 стр. 574  (по-другому)
%строка 697 стр. 
% страница 584 неправильное использование точек в матрице
% страница 585 неправильное использование точек в матрице
%\section{571}
%\setcounter{thesection}{571}

\documentclass{mai_book}
\defaultfontfeatures{Mapping=tex-text}
%\setmainfont{DejaVuSerif}
\setdefaultlanguage{russian}
%\usepackage{hhline}

\setcounter{page} {571}

\begin{document}

\lhead{\small{ $\mathit{Решения \thickspace упражнений}$}}
\rhead{\small{571}}
\noindent Для второго семейства $L_{n}$ --- подгруппа квадратов в $U(\mathbb Z_n)$, индекс \linebreak   %uberi prodel v nachale
которой равен 8. Приведем несколько примеров:

\begin{center}
7 $\times$ 19 $\times$ 67, $\thickspace$ 
19 $\times$ 199 $\times$ 271,$\thickspace$ 
31 $\times$ 151 $\times$ 1\,171,$\thickspace$ 
43 $\times$ 127 $\times$ 2\,731,
\smallskip \newline
43 $\times$ 271 $\times$ 5\,827, $\thickspace$
43 $\times$ 631 $\times$ 13\,567, $\thickspace$
43 $\times$ 127 $\times$ 211.
\end{center}

$\mathbf{b.}$ Имеем $n$ $-$ 1 = ($p_{1}-1$)(2$p_{1}$+1) =  $\frac{p_2-1}{2}$(2$p_{1}$+1), а потому $b^{n-1}$ mod \linebreak
$p_{2}$ = $(b^{\frac{p_2-1}{2}})^{2p_1+1}$=$(\frac{b}{p_{2}})$. Если мы хотим, чтобы n было псевдопростым \linebreak
по основанию $b$, то $b$ должно быть квадратичным вычетом по модулю \linebreak
 $p_{2}$, что не выполняется при $b$ = 2, так как $p_{2}$ $\equiv$ 5 (mod 8). 

\paragraph{46.$\medspace$ Равенство $a^\frac{n-1}{2}=\pm1$ для любого $a \in U(\mathbb Z_n)$} $\newline$

$\mathbf{а.}$ Целое число $n$, удовлетворяющее приведенным условиям, является \linebreak
числом Кармайкла, а потому равно произведению различных простых \linebreak
нечетных чисел. Пусть $p$ --- простой делитель $n$, $n$ = $pm$ и выполнено \linebreak
$a^\frac{n-1}{2}$ $\equiv$ $-$1 (mod n). Согласно китайской теореме об остатках, суще­- \linebreak
ствует такое $y$,  что $y$ $\equiv$ $a$ (mod $р$) и $у$ $\equiv$ 1 (mod $m$).  Предположение \linebreak
$y^\frac{n-1}{2}$ $\equiv$ $\pm1$ (mod $n$) и сравнения $y^\frac{n-1}{2}$ $\equiv$ $a^\frac{n-1}{2}$ $\equiv$ $-$1 (mod $p$) доказыва- \linebreak
ют, что $y^\frac{n-1}{2}$ сравнимо с $a$ $-$ 1 по модулю $n$, а значит, и по модулю $m$. \linebreak
Но $y$ $\equiv$ 1 (mod $m$); откуда $m$ = 1 и $n$ --- простое. \ 

\vspace{3pt}$\mathbf{b.}$ Если $n$ удовлетворяет равенству $a^\frac{n-1}{2}$  = 1 для любого  $a$ $\in U(\mathbb Z_n)$, \linebreak
то оно также удовлетворяет равенству $a^{n-1}$ = 1 и, следовательно, явля­-  \linebreak
ется числом Кармайкла. Кроме того, порядок любого элемента в $U(\mathbb Z_n)$ \linebreak
делит ($n$ $-$ 1)/2, а значит, $p_{i}$ $-$ 1 делит ($n$ $-$ 1)/2. Обратное очевидно. \linebreak
Числа $n$ = 1\,729 = 7 х 13 х 19 и $n$ = 2\,465 = 5 х 17 х 29 дают такие \linebreak
примеры.

\paragraph{47.$\medspace$ Нестабильность оснований в тесте Рабина --- Миллера } $\newline$

$\mathbf{а.}$ Сравнение $x^{2}$ $\equiv$ 1 (mod $p^{\alpha}$) приводит к тому, что $х$ $\equiv$ $\pm$ 1 \linebreak
(mod $p^{\alpha}$), а потому, если $b$ удовлетворяет $b^{n-1}$  = 1, то $n$ --- сильно \linebreak
псевдопростое по основанию $b$.\ 

\vspace{1pt}Имеем НОД($n$ $-$ 1,$\varphi(n)$) = НОД($p^{\alpha}$ $-$ 1, ($p$ $-$ 1)$p^{\alpha - 1}$ ) = $р$ $-$ 1. Сле­- \linebreak
довательно, $n$ --- сильно псевдопростое по основанию b тогда и только \linebreak
тогда, когда $b^{p - 1}$  = 1; существует ровно $р$ $-$ 1 таких оснований и отно­- \linebreak
шение равно $1/p^{\alpha - 1}$. \ 

\vspace{2pt}$\mathbf{с.}$ Пусть $p$ --- простой делитель $n$ и $p$ $\equiv$ 3 (mod 4). Из сравнения \linebreak
$b^{2^{i}q}$ $\equiv$ $-$ 1 (mod $n$) следует $b^{2^{i}q}$ $\equiv$ $-$1 (mod $p$), откуда $j$ = 0, так как $-$1 \linebreak
не квадрат по модулю $p$.\newline

\newpage

\lhead{\small{572}}
\rhead{\small{$\mathit{IV \hspace{10pt} Некоторые \hspace{4pt} методы \hspace{4pt} алгебраической \hspace{4pt} алгоритмики}$}}

$\mathbf{d.}$ Если $n$ --- степень простого числа $p$, то $b$ $\in$ $B_{n}$ тогда и только \linebreak
тогда, когда $b^{p-1}$ = 1, а потому $B_{n}$ --- подгруппа. Если $n$ делится на \linebreak
простое число $\equiv$ 3 (mod 4), то $B_{n}$ является подгруппой согласно пре­- \linebreak
дыдущему вопросу. Обратно, пусть $n$ = $p_{1}^{\alpha_{1}}\dots p_{r}^{\alpha_{r}}$ --- разложение на \linebreak
простые множители числа $n$ и $r$ $\geqslant$ 2, $p_{i}$ $\equiv$ 1 (mod 4). Покажем, что $B_{n}$ \linebreak
не подгруппа. Заметим, что для $b$, удовлетворяющего равенству $b^{2}$ = 1, \linebreak
выполняется $b^{q}$ = b, $b^{2q}$ = 1, а значит, $b$ $\notin$ $B_{n}$, если $b$ не равно $\pm$1. \linebreak
Напротив, если $b^{2}$ = $-$ 1, то $b^{2q}$ = $-$1, а потому b $\in$ $B_{n}$. Следователь­- \linebreak
но, достаточно найти такие $b$, $b$', что  $b^{2}$ =  $b'^{2}$ = $-$1 и $bb$' $\ne$ $\pm$1. В \linebreak
циклической группе $U(\mathbb Z_{p_{i}^{\alpha_{i}}})$ элемент $-$1 является квадратом (так как \linebreak
$(-1)^{\frac{\varphi (p^{\alpha_{i}}_{i})}{2}}$ = 1) и существует такой элемент $x_{i}$, что $x^{2}_{i}$ = $-$1 (mod $p^{\alpha_{i}}_{i}$). \linebreak
Достаточно взять следующие $b$ и $b$': \ 

\vspace{6pt} 
\hspace{42pt} $b$ $\equiv$ $b$' $\equiv$ $x_{i}$ \  (mod $p_{i}^{\alpha_{i}}$),\ \ \  $i$ = 2,\dots, $r$, \ 

\vspace{5pt}
\hspace{82pt} $b$ $\equiv$ $x_{1}$ \  (mod $p_{1}^{\alpha_{1}}$),\ \ \   $b$' $\equiv$ $-x_{1}$ \  (mod $p^{\alpha_{1}}_{1}$). \newline

$\mathbf{e.}$ Пусть $n$ = 5 $\times$ 13 = 65. Возьмем $x_{1}$ = 2, $x_{2}$ = 5, затем $b$ = 18, \linebreak
$b$' = 8. Тогда $bb$' = 14 (mod 65) и $18^{2}$ $\equiv$ $8^{2}$ $\equiv$ $-$1 (mod 65), но $14^{2}$ $\equiv$ 1 \linebreak
(mod 65), 14 $\not\equiv$ $\pm$1 (mod 65).

\paragraph{48.$\medspace$ Количество элементов множества $\{b \in U(\mathbb Z_n)$ | $b^{\frac{n-1}{2}}$ = $\pm1\}$} $\newline$

$\mathbf{a.}$ Для того, чтобы сосчитать число элементов $H_{n}$, сведем вопрос \linebreak
к циклическим группам. Если $U_{m}^{r}$ обозначает множество $\{$х $\in$ $U(\mathbb Z_{m} )$ | \linebreak
$x^{r} = 1\}$, то по китайской теореме об остатках

\begin{center}
$H_{n}$ $\backsimeq$ $U^{n-1}_{p^{\alpha_{1}}_{1}}$ $\times$ $\dots$ $\times$ $U^{n-1}_{p^{\alpha_{r}}_{r}}$
\end{center}
\noindent и поскольку $U(\mathbb Z_{p^{\alpha_{i}}_{i}}$ --- циклическая группа ($p_{i}$ --- простое число, от­- \linebreak
личное от 2), то:

\begin{center}
|$U^{n-1}_{p_{i}^{\alpha_{i}}}$| = НОД($n$ $-$ 1,$\varphi$($p^{\alpha_{i}}_{i}$)) = НОД($n$ $-$ 1,$p^{\alpha_{i}-1}_{i}(p_{i} - 1))$ = \ 

\vspace{3pt}
\hspace{97pt} = НОД($n$ $-$ 1,$p_{i}$ $-$ 1).
\end{center}
\noindent Последнее равенство является следствием того, что $p_{i}$ делит $n$,  а значит, \linebreak
не делит $n$ $-$ 1. Аналогичные рассуждения, примененные к $L_{n}$ , дают:

\begin{center}
\ \ \ |$H_{n}$| = $\underset{i=1}{\overset{r}{\prod}}$ НОД($n$ $-$ 1,$p_{i} - 1)$, \ \ \ \ \ \ \  |$L_{n}$| = $\underset{i=1}{\overset{r}{\prod}}$ НОД($\frac{n - 1}{2}$), $p_{i} - 1$). \ \ \ \ \ \ \ \ \ \ 
\end{center}
\noindent Отсюда легко выводится формула для [$H_{n}$ : $L_{n}$]. Наиболее простыми \linebreak
примерами являются, конечно, $n$ = $p^{2m}$ для $t$ = 0 и $n$ = $p^{2m+1}$ для $t$ = $r$. \linebreak

\newpage

\lhead{\small{ $\mathit{Решения \thickspace упражнений}$}}
\rhead{\small{573}}
\noindent Более общо, если $p_{1}$,$\dots$,$p_{r}$, --- $r$ простых чисел и $v_{2}$($p_{i} - 1$) = $v_{2}$($p_{j} - 1$), \linebreak
то для n = $p_{1}^{\alpha_{1}}$ $\dots$ $p_{r}^{\alpha_{r}}$ выполняется t = 0, если $\alpha_{1}$ + $\dots$ + $\alpha_{r}$ четно, и \linebreak
t = r, если $\alpha_{1}$ + $\dots$ + $\alpha_{r}$ нечетно.\ 

\vspace{3pt}$\mathbf{b.}$ Для $x$ $\in$ $U(\mathbb Z_n)$ элемент $х^\frac{n - 1}{2}$ принимает значение $-$1 тогда и \linebreak
только тогда, когда $х^{\frac{n - 1}{2}}$ mod $p_{i}^{\alpha_{i}}$ равно $-$1 для любого $i$. Согласно \linebreak
упражнению 25 это эквивалентно тому, что $v_{2}$(n $-$ 1) $\leqslant$ $v_{2}$($p_{i}$ $-$ 1) для лю­- \linebreak
бого $i$, т.е. $t$ = $r$.  В этом случае отображение $K_{n}$ $\ni$ x $\mapsto$  $х^{\frac{n - 1}{2}}$ $\in$ $\{-1, 1\}$ \linebreak
сюръективно и [$K_{n}$ :  $L_{n}$] = 2. В противном случае [$K_{n}$ :  $L_{n}$] = 1.

\paragraph{49.$\medspace$ О ценка количества чисел $b \in U(\mathbb Z_n)$ таких, что $b^{n - 1} = 1$} $\newline$

Имеем:

\begin{center}
$\frac{|H_{n}|}{\varphi(n)}$ = $\underset{i=1}{\overset{r}{\prod}}$ $\frac{1}{p_{i}^{\alpha_{i}-1}}$ $\frac{НОД(n - 1,p_{i} - 1)}{p_{i} - 1}$ $\leqslant$ $\underset{i=1}{\overset{r}{\prod}}$  $\frac{1}{p_{i}^{\alpha_{i}-1}}$ $\leqslant$ $\frac{1}{p_{i}^{\alpha_{i}-1}}$,
\end{center}
\noindent откуда и следует искомый результат (так как существует $\alpha_{i}$ $\geqslant$ 2). \linebreak
Граница достижима тогда и только тогда, когда $n$ = $3^{2}р_{2}$ $\dots$ $p_{r}$ c\linebreak
$p_{i} - 1$ | $n$ $-$ 1 . Например, 9 = $3^{2}$ , 45 = $3^{2}$ $\times$ 5, 13\,833 = $3^{2}$ $\times$ 29 $\times$ 53, \linebreak
321\,201 = $3^{2}$ $\times$ 89 $\times$ 401, 203\,841 = $3^{2}$ $\times$ 11 $\times$ 29 $\times$ 71.

\paragraph{50.$\medspace$ Псевдопростые числа Эйлера по основанию $b$} $\newline$

$\mathbf{a.}$ Эти включения очевидны и они являются строгими (см. упраж­- \linebreak
нение 51, например). \ 

\vspace{2pt}$\mathbf{b.}$Для доказательства того, что $j$ $\leqslant$ $v_{2}(p_{i} - 1)$, можно считать, \linebreak
что $j$ $\geqslant$ 1. Положим $a$ = $b^{q}$. Тот факт, что $n$ --- сильно псевдопростое \linebreak
по основанию $b$, приводит к $a^{2^{j}} \equiv 1$ (mod $n$) и $a^{2^{j - 1}} \equiv -1$ (mod n). Те \linebreak
же самые сравнения по модулям $p_{i}$ доказывают, что $a$ имеет порядок \linebreak
$2^{j}$ по модулям $p_{i}$, а в частности, $2^{j}$ | $p_{i} -1$. \ 

\vspace{1pt}То, что n --- сильно псевдопростое по основанию b, дает $b^{\frac{n-1}{2}}$ = $\pm$1. \linebreak
Следующие сравнения очевидны:

\begin{center}
$p_{i}$ $\equiv$ 1 (mod $2^{j+1}$), если j  < $v_{2}$($p_{i}$ $-$ 1), \ 

\vspace{3pt}$p_{i}$ $\equiv$ 1 + $2^{j}$ (mod $2^{j+1}$), если j  = $v_{2}(p_{i} - 1)$,
\end{center}
\noindent и доказывают, что $n$ = $p_{i}$ $\dots$ $p_{r}$ $\equiv$ $(1 + 2^{j})^{t}$ (mod $2^{j+1}$). 1 + $2^{j}$ --- эле­- \linebreak
мент порядка 2 в группе $U(\mathbb Z_{2^{j+1}})$. Следовательно, $2^{j+1}$ | $n$ $-$ 1 тогда и\linebreak
только тогда, когда t четно. С другой стороны, $\frac{n-1}{2}$ = $2^{k-1}q$, а потому\linebreak
$b^{\frac{n-1}{2}}$ = 1 тогда и только тогда, когда к $-$ 1 $\geqslant$ j,  т.е. $2^{j+1}$ | $n$ $-$ 1. Таким\linebreak
образом мы установили эквивалентность между тем, что t четно и тем,\linebreak
что $b^{\frac{n-1}{2}}$ = 1. \newline

\newpage

\lhead{\small{574}}
\rhead{\small{$\mathit{IV \hspace{10pt} Некоторые \hspace{4pt} методы \hspace{4pt} алгебраической \hspace{4pt} алгоритмики}$}}

Мы видели, что $2^{j}$ --- порядок $b^{q}$ по модулю $p_{i}$ тогда и только тогда, \linebreak
когда $j$ = $v_{2}$($p_{i}$ $-$ 1), откуда

\begin{center}
$\left(\frac{b}{n} \right)$ = $\left(\frac{b}{n} \right)^{q}$ = $\left(\frac{b^{q}}{n} \right)$ = $\left(\frac{b^{q}}{p_{1}} \right)$ $\dots$  $\left(\frac{b^{q}}{p_{r}} \right)$ = $(-1)^{t}$.
\end{center}

\noindent Итак, доказано равенство (по модулю n) $b^{\frac{n-1}{2}}$ и $\left(\frac{b}{n} \right)$. Обратное неверно \linebreak
(см., например, упражнение 51). \ 

\vspace{3pt}$\mathbf{c.}$ Пусть $n$ --- эйлерово псевдопростое число по основанию n. Так \linebreak
как $n$ $-$ 1 = 2$q$ с нечетным $q$, то $b^{q}$ = $\left(\frac{b}{n} \right)$ = $\pm$1. В любом случае $n$ --- \linebreak
сильно псевдопростое по основанию $b$.

\paragraph{51.$\medspace$ Псевдопростые числа по основанию 2, не больше $10^{4}$} \ 

\vspace{7pt}Не считая простых чисел, имеется 22 псевдопростых числа по осно-­ \linebreak
ванию 2. Среди этих 22-х эйлеровыми псевдопростыми по основанию 2 \linebreak
являются следующие 12:

\begin{center}
561, 1\,105, 1\,729, 1\,905, 2\,047, 2\,465, \ 

\vspace{3pt}3\,277, 4\,033, 4\,681, 6\,601, 8\,321, 8\,481.
\end{center}

\noindent 5 чисел являются сильно псевдопростыми по основанию 2: 2\,047, 3\,277, \linebreak
4\,033, 4\,681, 8\,321.

\paragraph{52.$\medspace$ Тест на простоту Соловея --- Штрассена}$\newline$

$\mathbf{a.}$ Множество $E_{n}$ определяется как множество точек совпадения \linebreak
двух гомоморфизмов, а потому является подгруппой. \ 

\vspace{3pt}$\mathbf{b.}$ Предположим, что $E_{n}$ = $U(\mathbb Z_{n})$. Отсюда $b^{frac{n-1}{2}}$ = $\pm$1, а значит \linebreak
$b^{n-1}$ = 1; это приводит к тому, что $n$ --- произведение различных про­- \linebreak
стых чисел (другими словами, $n$ --- простое или $n$ --- число Кармайкла, \linebreak
см. упражнение 46). Поскольку $n$ --- не квадрат, то символ Якоби ра­- \linebreak
вен $-$1 и упражнение 46 позволяет сделать необходимый вывод. Остав­- \linebreak
шаяся часть упражнения не представляет сложностей.

\paragraph{53.$\medspace$ Числа, достигающие границы $\frac{1}{2}$ в тесте} \ 

\vspace{1pt}\ \ $\mathbf{Соловея}$ --- $\mathbf{Штрассена}$ $\newline$

$\mathbf{a.}$ Согласно упражнению 20, сумма $\sum_{i,k_{i}=k}\alpha_{i}$ нечетна. Докажем, \linebreak
что $L_{n}$ $\subset$ $E_{n}$ . Если $b \in L_{n}$ , т.е. $b^{\frac{n-1}{2}}$ = 1, или, подругому, $(b^{q})^{2^{k-1}}$ = 1, \linebreak
то $(b^{q})^{\frac{p_{j}-1}{2}}$ = 1 , так как $k_{j}$ $\geqslant$ $k$.  По модулю $p_{j}$ это равенство дает \linebreak

\newpage

\lhead{\small{ $\mathit{Решения \thickspace упражнений}$}}
\rhead{\small{575}}
\noindent $\left(\frac{b^{q}}{p_{j}} \right)$ = 1, откуда $\left(\frac{b}{p_{j}} \right)$ = 1 и, наконец, $\left(\frac{b}{n} \right)$ = 1 и, следовательно,  \linebreak
$b \in E_{n}$. Докажем, что отображение $E_{n}$ $\ni$ $b$ $\mapsto$  $\left(\frac{b}{n} \right)$ $\in$ $\{ -1, 1\}$, с ядром  \linebreak
$L_{n}$, сюръективно, т.е. [$E_{n}$ : $L_{n}$] = 2. Для этого выберем $b$ $\in$ $U(\mathbb Z_n)$ \linebreak
такое, что $b$ mod $p_{j}^{\alpha_{j}}$ имеет порядок $2^{k}$ для любого $j$ (это возможно, \linebreak
поскольку $k$ $\leqslant$ $k_{j}$).  Тогда $b^{2^{k-1}}$ = $-$1, а значит, $b^{\frac{n-1}{2}}$ = $-$1. \linebreak
Далее, $\left(\frac{b}{p_{j}} \right)$ = $-$1, если $k_{j}$ = $k$ и $\left(\frac{b}{p_{j}} \right)$ = 1, если $k_{j}$ > $k$.  Поэтому \linebreak
$\left(\frac{b}{n} \right)$ = $(-1)^{\sum_{i,k_{i}=k}\alpha_{i}}$ = $-$1; $b$ принадлежит $E_{n}$ и $\left(\frac{b}{n} \right)$ = $-$1. \ 

\vspace{4pt}Поскольку [$H_{n}$ : $L_{n}$] = $2^{r}$ и [$E_{n}$ : $L_{n}$] = 2, то [$H_{n}$ : $E_{n}$] = $2^{r-1}$. Если \linebreak
$r$ $\geqslant$ 3, то |$E_{n}$| $\geqslant$  $\frac{|H_{n}|}{4}$ $\leqslant$ $\frac{\varphi(n)}{4}$ и равенство выполняется тогда и только \linebreak
тогда, когда $r$ = 3 и $H_{n}$ = $U(\mathbb Z_n)$; в этом случае $n$ --- число Кармайкла \linebreak
$p_{1}p_{2}p_{3}$ с $k_{1}$ = $k_{2}$ = $k_{3}$ (например, $n$ = 7 $\times$ 19 $\times$ 67). \ 

\vspace{4pt}Если $r$ = 1, то $n$ = $p^{\alpha}$ с нечетным $\alpha$, а потому $\alpha$ $\geqslant$ 3, $E_{n}$ = $H_{n}$ и \linebreak
$\frac{|H_{n}|}{\varphi(n)}$ = $\frac{1}{p^{\alpha-1}}$ $\leqslant$ $\frac{1}{9}$. Если $r$ = 2, то [$H_{n}$ : $E_{n}$] = 2 и $H_{n}$ $\ne$ $U(\mathbb Z_n)$ (в против- \linebreak
ном случае $n$ --- число Кармайкла, откуда $r$ $\geqslant$ 3), что дает |$E_{n}$| $\leqslant$ $\frac{\varphi(n)}{4}$. \linebreak
Граница достигается тогда и только тогда, когда [$U(\mathbb Z_n)$ : $H_{n}$] = 2, т.е.
для чисел $n$ = $pq$ с $q$ $-$ 1 = 2(p $-$ 1). (Примеры см. в упражнении 45). \ 

\vspace{3pt}$\mathbf{b.}$ Докажем, что $E_{n}$ $\subset$ $L_{n}$.  Пусть $b$ $\in$ $E_{n}$, т.е. $b^{\frac{n-1}{2}}$ = $\left(\frac{b}{n} \right)$. Для j, \linebreak
такого, что $k$ > $k_{j}$ (а такое $j$ существует), сравнение $b^{\frac{n-1}{2}}$ mod $p_{j}^{\alpha_{j}}$ не \linebreak
может принимать значение $-$1 (так как $v_{2}(\frac{n-1}{2})$ $\geqslant$ $v_{2}(p_{j} - 1))$, а по­- \linebreak
тому $\left(\frac{b}{n} \right)$ = 1 и $b$ $\in$ $L_{n}$. [$L_{n}$ : $E_{n}$] = 2 тогда и только тогда, когда \linebreak
отображение$L_{n}$ $\ni$ $b$ $\mapsto$ $\left(\frac{b}{n} \right)$ $\in$ $\{-1, 1\}$, ядро которого равно $E_{n}$, прини­- \linebreak
мает значение $-$1. Читатель может проверить, что это эквивалентно \linebreak
сформулированному условию. \ 

\vspace{3pt} Напомним, что [$H_{n}$ : $L_{n}$] = $2^{t}$, где $t$ --- число таких индексов $i$, что \linebreak
$k$ < $k_{i}$.  Если $n$ имеет квадратный множитель, то |$E_{n}$| $\leqslant$ |$H_{n}$| $\leqslant$ $\frac{\varphi(n)}{3}$. \linebreak
Если $n$ не имеет квадратного множителя, то [$L_{n}$ : $E_{n}$] = 2, а потому \linebreak
$[H_{n} : E_{n}]$ = $2^{t+1}$. Отсюда |$E_{n}$| $\leqslant$ $\frac{H_{n}}{2}$ $\leqslant$ $\frac{\varphi(n)}{2}$ и равенство достига­- \linebreak
ется тогда и только тогда, когда $t$ = 0 и $H_{n}$ = $U(\mathbb Z_n)$. Речь идет, \linebreak
следовательно, о числах Кармайкла, для которых $k$ > $k_{i}$ для любого \linebreak
$i$, т.е. $b^{\frac{n-1}{2}}$ = 1 для любого $b$, обратимого по модулю $n$ (например, \linebreak
1\,729 = 7 $\times$ 13 $\times$ 19, 2\,465 = 5 $\times$ 17 $\times$ 29).

\paragraph{54.$\medspace$ Построение $n$, $b\hspace{4pt}c\hspace{4pt}b^{\frac{n-1}{2}}$ = $\pm1$, но $b^{\frac{n-1}{2}}$ $\ne$ $\left(\frac{b}{n} \right)$} \ 

\vspace{7pt}Положим $k$ = $v_{2}(p - 1)$ = $v_{2}(q - 1))$. Тогда \ 

\vspace{14pt}$p$,$q$ $\equiv$ 1 + $2^{k}$ (mod $2^{k+1}$) $\Rightarrow$ $n$ = $pq$ $\equiv$ 1 (mod $2^{k+1}$) $\Rightarrow$ $2^{k}$ | $\frac{n-1}{2}$. \linebreak

\newpage

\lhead{\small{576}}
\rhead{\small{$\mathit{IV \hspace{10pt} Некоторые \hspace{4pt} методы \hspace{4pt} алгебраической \hspace{4pt} алгоритмики}$}}
\newlength{\MYwidth} % новый параметр длины 
\def\MYvrule#1\par{ 
\par\noindent 
\MYwidth=\textwidth\addtolength{\MYwidth}{-7pt} 
\hbox{\vrule width 1pt\hspace{5pt}\parbox[t]{\MYwidth}{#1}} 
}

\noindent Пусть $b$ таково, что порядок $b$ по модулю $p$ равен $2^{k}$ и $b$ mod $q$ = 1. \linebreak
Тогда $b$ обратимо по модулю $n$ и удовлетворяет сравнению $b^{\frac{n-1}{2}}$ $\equiv$ 1 \linebreak
(mod $p$) (так как $2^{k}$ | $\frac{n-1}{2}$), а также $b^{\frac{n-1}{2}}$ $\equiv$ 1 (mod $q$), откуда $b^{\frac{n-1}{2}}$ = 1 \linebreak
(mod $n$). Конечно,

\begin{center}
$\left(\frac{b}{pq} \right)$ = $\left(\frac{b}{p} \right)$$\left(\frac{b}{q} \right)$ = $\left(\frac{b}{p} \right)$ = $b^{\frac{p-1}{2}}$ mod $p$= $-$1  \ 

\vspace{5pt}(так как $\frac{p-1}{2}$ не делится на $2^{k}$).
\end{center}

\noindent Например: $(p, q, b)$ = (3, 7, 8) или (5, 13, 27).

\paragraph{55.$\medspace$ Факторизации «${\grave a}$ la Ферма»)} $\newline$

$\mathbf{a.}$ Очевидно, $n$ = $(\frac{p+q}{2})^{2}$ $-$$(\frac{p-q}{2})^{2}$ и, поскольку $p$ и $q$ являются очень \linebreak
близкими числами, то $p$ $-$ $q$ мало. Следовательно, если удастся найти \linebreak
такое $x (=(p+q)/2)$, что $x^{2}$$-$$n$ --- квадрат $y^{2}$ (достаточно маленький), \linebreak
то можно факторизовать $n$ = $(x + y)(x - y)$. \ 

\vspace{1pt}Для того чтобы реализовать этот алгоритм, нужно каким-то обра­- \linebreak
зом выбрать начальное значение $x$.  Но $x^{2}$ > $n$, а потому $x$ > $\lfloor\sqrt{n}\rfloor$, и \linebreak
поиск можно начинать с $\lfloor\sqrt{n}\rfloor$ + 1.

\begin{flushleft}
\MYvrule$x$ $\longleftarrow$ $\lfloor\sqrt{n}\rfloor$ + 1; $y$ $\longleftarrow$ $\sqrt{x^{2}-n}$; \newline
$\mathbf{while}$ $y$ $\notin$ $\mathbb N$ $\{$ \newline
$x$ $\longleftarrow$ $x$ + 1; $y$ $\longleftarrow$ $\sqrt{x^{2}-n}$; \newline
$\}$

\end{flushleft}

Из этого алгоритма получаем $n$ = $(x + y)(x - y)$.  Конечно, можно \linebreak
оптимизировать этот алгоритм и исключить почти все вычисления ква­- \linebreak
дратных корней, заставляя измениться $y$ в соответствии с множеством \linebreak
возможных значений. Идея алгоритма состоит в вычислении величины \linebreak
$x^{2}$ $-$ $y^{2}$ , на каждом шаге проверяя при этом, что $x$ и $y$ не превосходят \linebreak
максимальных значений $\frac{p+q}{2}$ и $\frac{p-q}{2}$ и используя то, что $x$ и $y$ обязательно \linebreak
имеют противоположные четности. \ 

\vspace{3pt}Объясним некоторые изменения, произошедшие с алгоритмом. Пер­-  \linebreak
вое присваивание переменной $y$,  кроме придания ей значения 0 или 1, \linebreak
обеспечивает то, что $y$ и $x$ имеют противоположные четности. При вхо­- \linebreak
де во внутренний цикл г > п и, принимая во внимание неравенства для \linebreak
$x$ и $y$,  нетрудно доказать, что $y$ < $\frac{p-q}{2}$, а это говорит о правомочно­- \linebreak
сти условия, фигурирующего при выходе из цикла. Наконец, понятно, \linebreak
что $\frac{p+q}{2}$ и $\frac{p-q}{2}$ имеют противоположные четности. Следовательно, если \linebreak
$y$ = $\frac{p-q}{2}$ + 1, то $x$ $\ne$ $\frac{p+q}{2}$. Поэтому, если мы находимся в основном цикле \linebreak
после выполнения условия $r$ = $n$,  то имеются две возможности:

\newpage

\lhead{\small{ $\mathit{Решения \thickspace упражнений}$}}
\rhead{\small{577}}

\begin{lstlisting}[mathescape=true, language=Ada]
$x$ $\longleftarrow$ $\lfloor\sqrt{n}\rfloor$ + 1; $y$ $\longleftarrow$ 1 $-$ $x$ mod 2; $r$  $\longleftarrow$ $x^{2}$ $-$ $y^{2}$; 
{
$\hspace{4pt}$ $x$$\leqslant$$\frac{p+q}{2}$ и$\ $$y$$\leqslant$$\frac{p-q}{2}$$\ $имеют$\ $противоположные$\ $четности$\ $и$\ $$r$ = $x^2$ -$y^2$
$\hspace{4pt}$ while $r$ > $n$ }
$\hspace{10pt}$ $y$ $\longleftarrow$ $y$ + 2; $r$ $\longleftarrow$ $r$ - 4$y$ - 4;
$\hspace{4pt}$ }
$\hspace{4pt}$ $x$$\leqslant$$\frac{p+q}{2}$ и$\ $$y$$\leqslant$$\frac{p-q}{2}$ + 1$\ $имеют$\ $противоположные$\ $четности$\ $и$\ $$r$ = $x^2$ - $y^2$
exit$\ $when$\ $ $r$ = $n$;
$\hspace{4pt}$ $x$$\leqslant$$\frac{p+q}{2}$ и$\ $$y$$\leqslant$$\frac{p-q}{2}$ + 1$\ $имеют$\ $противоположные$\ $четности$\ $и$\ $$r$ = $x^2$ - $y^2$
$\hspace{4pt}$ $x$ $\longleftarrow$ $x$ + 1; $y$ $\longleftarrow$ $y$ - 1; $r$ $\longleftarrow$ $r$ + 2$x$ - 2$y$;
}
\end{lstlisting}

\begin{center}
$\mathbf{Алгоритм}$ $\mathbf{13.}$ Факторизация «$\grave a$ la Ферма»
\end{center}

1) $y$ = $\frac{p-q}{2}$ + 1 и, согласно предыдущему свойству, $x$ < $\frac{p+q}{2}$, \ 

\vspace{0pt} 2) $y$ $\leqslant$ $\frac{p-q}{2}$ + 1 и тогда неравенство $x^2$ $-$ $y^2$ < $n$ приводит к $x$ < $\frac{p+q}{2}$. \newline
\noindent В любом случае последующие действия восстанавливают инварианты. \linebreak
Этот алгоритм требует в общем случае $\frac{p-q}{4}$ итерации на внутреннем \linebreak
цикле (чтобы найти хорошее значение $y$) и $\frac{p+q}{2}$ $-$ $\lceil$ $\sqrt n$ $\rceil$ итераций на \linebreak
внешнем цикле (во время которых $y$ может опуститься до 1). Одна­- \linebreak
ко, поскольку фактически значения $y$ возрастают (от 1 или 2), то эти \linebreak
сложности не перемножаются и алгоритм имеет сложность порядка \linebreak
$p$ $-$ $\lceil$ $\sqrt n$ $\rceil$, т.е. порядка $(p - q)/2$.

\begin{lstlisting}[mathescape=true, language=Ada]
$\hspace{3pt}$$x$ $\longleftarrow$ $\lfloor$ $\sqrt n$ $\rfloor$ + 1;
$\hspace{3pt}$for (int $i$ = 1; $i$ <= $r$; ++$i$) { $Res(i)$ $\longleftarrow$ $x$ mod $n_i$; }
$\hspace{3pt}$$\textit{Факторизация}$_$n$ : {
 $\hspace{10pt}$ $\textit{Итерация}$_$x$ : {
 $\hspace{25pt}$ for (int $i$ = 1; $i$ <= $r$; ++$i$) {
 $\hspace{25pt}$ exit $\textit{Итерация}$_$x$ when not $\textit{Carre(Res(i), i);}$
 $\hspace{25pt}$ }
 $\hspace{25pt}$ $y$ $\longleftarrow$ $\lfloor$ $\sqrt x^2 - n$ $\rfloor$ + 1;
 $\hspace{25pt}$ exit $\textit{Факторизация}$_$n$ when $y^2$ = $x^2$ $-$ $n$;
 $\hspace{10pt}$ } $\textit{Итерация}$_$x$;
 $\hspace{10pt}$ $x$ $\longleftarrow$ $x$ + 1;
 $\hspace{10pt}$ for (int $i$ = 1; $i$ <= $r$; ++$i$) {$Res(i)$ $\longleftarrow$ $Res(i)$ + 1 mod $n_i$; }
$\hspace{3pt}$ } $\textit{Факторизация}$_$n$;
\end{lstlisting}
\begin{center}
$\mathbf{Алгоритм}$ $\mathbf{14.}$ Факторизация по Ферма с решетом»
\end{center}

$\mathbf{b.}$ Разумеется, простые числа $n_1$, $\dots$, $n_r$ , которые используются в \linebreak
решете, взаимно просты с $n$. Булева матрица $Carre$ с двумя входами, \linebreak

\newpage

\lhead{\small{578}}
\rhead{\small{$\mathit{IV \hspace{10pt} Некоторые \hspace{4pt} методы \hspace{4pt} алгебраической \hspace{4pt} алгоритмики}$}}

\noindent которая определяется в начале алгоритма, имеет следующее опреде-: \linebreak
ление: $Carre(x, i)$ истинно, если $x^{2}$ $-$ $n$ --- квадрат по модулю $n_{i}$. Ал­- \linebreak
горитм 14 тогда очевиден. Он использует вспомогательную матрицу, \linebreak
содержащую вычеты $x$ по модулю $n_{i}$, чтобы модулярные вычисления \linebreak
не требовали обращения к арифметике повышенной точности (за ис­- \linebreak
ключением инициализации).

\paragraph{56.$\medspace$ Матрица Смита, формула Чезаро} $\newline$

$\mathbf{a.}$ Имеем $(SM)_{ij}$ = $\sum s_{ik}m_{kj}$ = $\sum_{j|k|i} \mu(k/j)$. Если $j$ | $i$, то \linebreak
$(SM)_{ij}$ = 0, в противном случае, осуществляя замену переменной \linebreak
$d$ = $k$/$j$, получим $(SM)_{ij}$ = $\sum_{j|k|i} \mu(k/j)$. Это доказывает, что $SM$ = $Id_{n}$. \linebreak
Аналогичные вычисления доказывают, что $M$ $S$ = $Id_{n}$ (замена перемен­- \linebreak
ной $d$ = $i$/$k$). Эти матричные тождества эквивалентны формуле обрат \linebreak
щения Мёбиуса, поскольку $g$ = $Sh$ означает, что $g$ --- прямая трансфор­- \linebreak
манта $h$,  в то время как $h$ = $Mg$ означает, что $h$ --- обратная транс­- \linebreak
форманта $g$. \ 

\vspace{4pt}$\mathbf{b.}$ $\mathbf{Имеем}$:

\begin{flushleft}
\hspace{15pt} $\hat H_{ij}$ = $\hat h($НОД$(i, j))$ = $\underset{d|НОД(i,j)}\sum h(d)$ = 

\begin{flushright}
= $\underset{d|i и d|j}\sum h(d)$ = $\underset{1\leqslant k \leqslant n}\sum S_{ik}h(k)S_{jk}$ = $(SH^{t}S)_{ij}$ \hspace{15pt}
\end{flushright}

\end{flushleft}

\noindent Равенство $\sum_{d| n} \varphi(d) = n$ доказывает, что единичная функция --- это \linebreak
трансформанта функции $\varphi$. Следовательно, $A$ = $SD^{t}S$,  а отсюда выво­- \linebreak
дится формула Чезаро. Поскольку $M$\hspace{1pt}$S$ = I$d_n$, то $det$ $S$ = $\pm$1, откуда \linebreak
получаем формулу Смита.

\paragraph{57.$\medspace$ Определение степеней множителей многочлена над $\mathbb F_q$} $\newline$

$\mathbf{a.}$ Множество $B_{P,i}$ --- это множество неподвижных точек гомомор­- \linebreak
физма $\tau^{i}$, а потому подалгебра в $K[X]/P$.  Если $n$ = $deg$ $P$, то алгебра \linebreak
$K[X]/P$ есть $K$-алгебра размерности $n$ с базисом $\{1, \bar X, \bar X^{2}, \dots\bar X^{n-1},\}$. \linebreak
Зная многочлен $P$,  можно вычислить матрицу $B$ эндоморфизма $r$: для \linebreak
этого, например, можно использовать равенство $\tau(\bar X^{j})$ = $X^{pj}$ mod $P$,\linebreak
что дает  $j$-й столбец $B$.  Классические методы линейной алгебры (при­- \linebreak
ведение к треугольному виду) позволяют вычислить размерность ядра \linebreak
матриц $B^{i}$ $-$ Id.

\newpage

\lhead{\small{ $\mathit{Решения \thickspace упражнений}$}}
\rhead{\small{579}}

Если $P$ неприводим, то алгебра $K[X]/P$ --- надполе $K$ размерно­- \linebreak
сти $n$. Следовательно, $B_{P, 1}$ = $K$ и $B_{P, n}$ = $K[X]/P$.  Согласно упражне­- \linebreak
нию о конечных полях в I$\hspace{0pt}$I, алгебра $B_{P, i}$ имеет размерность НОД ($i$, $n$). \ 

\vspace{3pt}$\mathbf{b.}$ Согласно цитированному упражнению, каноническая проекция \linebreak
$K[X]/P^{\alpha}]$ $\rightarrow$ $K[X]/P$ сужается до изоморфизма $B_{P^{\alpha}, i}$ на $B_{P, i}$,  откуда \linebreak
следует утверждение, касающееся размерности. \ 

\vspace{3pt}$\mathbf{c.}$ Вытекает из китайской теоремы об остатках. Если ${P^{\alpha_{1}}_{1}}$ $\dots$ ${P^{\alpha_{k}}_{k}}$ \linebreak
--- разложение на простые множители многочлена $P$, то

\begin{center}
dim $B_{P, 1}$ = dim $B_{P^{\alpha_{1}}_{1}, 1}$ + $\dots$ +  dim $B_{P^{\alpha_{k}}_{k}, 1}$ = 1 + $\dots$ + 1 = $k$.
\end{center}

$\mathbf{d.}$ В предыдущих обозначениях имеем: \newline

\begin{flushleft}
\hspace{25pt} dim $B_{P, i}$ = dim $B_{P^{\alpha_{1}}_{1}, i}$ + $\dots$ + dim $B_{P^{\alpha_{k}}_{k}, i}$ = \ 

\vspace{0pt}\hspace{70pt} = НОД($i$, $deg P_{1}$) + $\dots$ + НОД($i$, $deg P_{k}$) = \ 

\vspace{0pt}\hspace{70pt} = НОД($i$, 1)$d_{1}$ + НОД($i$, 2)$d_{2}$ + $\dots$ НОД($i$, n)$d_{n}$ = \ 

\vspace{0pt}\hspace{70pt} = $i$-й коэффициент $A(d_{1}, \dots, d_{n})$.
\end{flushleft}

\paragraph{58.$\medspace$ Неприводимые множители $X^{n} - 1$ над конечными полями} $\newline$

$\mathbf{a.}$ Исследование производной $P'(X)$ = $nX^{n-1}$ при ненулевом $n$ над \linebreak
полем $\mathbb F_q$, доказывает, что многочлен $P(X)$ = $X^{n} - 1$ не имеет квадрат­- \linebreak
ных множителей. Имеем $\tau(X^{i})$ = $X^{qi mod n}$. Оставшаяся часть упраж­- \linebreak
нения вытекает из этого равенства. \ 

\vspace{3pt}$\mathbf{b.}$ Предположим сначала, что $\sigma$ --- цикл. Если $x$ --- вектор с ком­- \linebreak
понентами $x_{i}$, то $\sigma(x)$ = $x$ означает, что $x_{i}$ = $x_{\sigma(i)}$ для любого $i$, а \linebreak
поскольку $\sigma$ --- цикл, то $x_{i}$ = $x_{j}$ для любых $i$ и $j$. Отсюда выводим, что \linebreak
$Кег(\sigma - Id)$ = $K \cdot (e_{1} + \dots + e_{d})$, а потому dim\hspace{1pt}Ker($\sigma$ $-$ Id) = 1. Если $\sigma$ \linebreak
произвольно, то dim\hspace{1pt}Ker($\sigma$ $-$ Id) равна числу циклов в $\sigma$. \ 

\vspace{3pt}$\mathbf{c.}$ Пусть для 1 $\leqslant$ $i$ $\leqslant$ $n$ число $b_{i}$ ---  количество орбит умножения \linebreak
на $q^{i}$ в $\mathbb Z_n$. Тогда $b_{1}$ --- число неприводимых множителей многочлена \linebreak
$X^{n}$ $-$ 1. Более общо, если $A$  обозначает матрицу Смита с элементами \linebreak
НОД($i$, $j$), то $A(d_{1}, d_{2}, \dots, d_{n})$ = $(b_{1}, b_{2}, \dots, b_{n})$ ,  где $d_{i}$ --- число множи­- \linebreak
телей степени $i$. Обращая это равенство, можно вычислить $d_{i}$ через $b_{j}$. \ 

\vspace{1pt} В частности, если $q$ $\equiv$ 1 (mod $n$), то многочлен $X^{n}$ $-$ 1 является \linebreak
произведением многочленов простой степени --- это легко доказать (по­- \linebreak
скольку $n$ | $q$ $-$ 1, группа $\mathbb F^*_q$ содержит $n$ корней $n$-й степени из 1). Вы­- \linebreak
шесказанное означает, что $\Phi_n(X)$ неприводим над  $\mathbb F_q$, если умножение \linebreak
на $q$ имеет две орбиты на  $\mathbb Z_n$. Это эквивалентно тому, что $q$ порож­- \linebreak
дает $U(\mathbb Z_n)$.

\newpage

\lhead{\small{580}}
\rhead{\small{$\mathit{IV \hspace{10pt} Некоторые \hspace{4pt} методы \hspace{4pt} алгебраической \hspace{4pt} алгоритмики}$}}

Имеем $U(\mathbb Z_{12})$ = \{1,5,7,11\} и квадрат каждого элемента равен 1. \linebreak
Ниже приведены орбиты умножений на различные $q$:

\begin{flushleft}
\hspace{15pt} $q$ \small mod 12 = 1: \small\{0\}, \{1\}, \{2\}, \{3\}, \{4\}, \{5\}, \{6\}, \{7\}, \{8\}, \{9\}, \{10\}, \{11\},\ 

\vspace{0pt}\hspace{15pt} $q$ \small mod 12 = 5: \small\{0\}, \{1, 5\}, \{2, 10\}, \{3\}, \{4, 8\}, \{6\}, \{7, 11\}, \{9\}, \ 

\vspace{0pt}\hspace{15pt} $q$ \small mod 12 = 7: \small\{0\}, \{1, 7\}, \{2\}, \{3, 9\}, \{4\}, \{5, 11\}, \{6\}, \{8\}, \{10\}, \ 

\vspace{0pt}\hspace{7pt} $q$ \small mod 12 = 11: \small\{0\}, \{1, 11\}, \{2, 10\}, \{3, 9\}, \{4, 8\}, \{5, 7\}, \{6\}.
\end{flushleft}

Сосчитав орбиты, можно сказать, чему равно число неприводимых \linebreak
множителей $X^{12}$ $-$ 1 в зависимости от значений $q$ по модулю 12:

\begin{flushleft}
\hspace{50pt}12, \hspace{7pt} если $q$ mod 12 = 1; \hspace{7pt} 8, \hspace{7pt} если q mod 12 = 5; \ 

\vspace{5pt}\hspace{100pt}9, \hspace{7pt} если $q$ mod 12 = 7; \hspace{7pt} 7, \hspace{7pt} если $q$ mod 12 = 11.
\end{flushleft}

\noindent Вектор $(b_{1}, \dots, b_{12})$, где $b_{i}$ --- число орбит умножения на $q^{i}$ в $\mathbb Z_12$, равен:

\begin{flushleft}
\hspace{88pt}$q$ mod 12 = 1 : \hspace{5pt} $b$ = (12, 12, 12, $\dots$, 12, 12), \ 

\vspace{0pt}\hspace{88pt}$q$ mod 12 = 5 : \hspace{5pt} $b$ = (8, 12, 8, 12, $\dots$, 8, 12), \ 

\vspace{0pt}\hspace{88pt}$q$ mod 12 = 7 : \hspace{5pt} $b$ = (9, 12, 9, 12, $\dots$, 9, 12), \ 

\vspace{0pt}\hspace{88pt}$q$ mod 12 = 11 : \hspace{5pt} $b$ = (7, 12, 7, 12, $\dots$, 7, 12).
\end{flushleft}

\noindent Равенство $A(d_{1}, \dots, d_{12})$ = $(b_{1}, \dots, b_{12})$  дает вектор $d$, где $d_{i}$ --- число \linebreak
множителей степени $i$:

\begin{flushleft}
\hspace{89pt}$q$ mod 12 = 1 : \hspace{5pt} $d$ = (12, 0, 0, 0, $\dots$, 0, 0), \ 

\vspace{0pt}\hspace{89pt}$q$ mod 12 = 5 : \hspace{5pt} $d$ = (4, 4, 0, 0, $\dots$, 0, 0), \ 

\vspace{0pt}\hspace{89pt}$q$ mod 12 = 7 : \hspace{5pt} $d$ = (6, 3, 0, 0, $\dots$, 0, 0), \ 

\vspace{0pt}\hspace{89pt}$q$ mod 12 = 11 : \hspace{5pt} $d$ = (2, 5, 0, 0, $\dots$, 0, 0).
\end{flushleft}

\noindent Приведем разложения $X^{12}$ $-$ 1 над $\mathbb F_5$, $\mathbb F_7$, $\mathbb F_{11}$ и $\mathbb F_{13}$ соответственно:

\begin{flushleft}
\hspace{60pt} $\mathbb F_{5}$ \hspace{5pt} ($X$ + 1) ($X$ + 2) ($X$ + 3) ($X$ + 4) ($X^{2}$ + $X$ + 1)\ 

\vspace{0pt}\hspace{115pt} ($X^{2}$ + 2$X$ + 4) ($X^{2}$ + 3$X$ + 4) ($X^{2}$ + 4$X$ + 1),\ 

\vspace{0pt}\hspace{60pt} $\mathbb F_{7}$ \hspace{5pt} ($X$ + 1) ($X$ + 2) ($X$ + 3) ($X$ + 4) ($X$ + 5) ($X$ + 6)\ 

\vspace{0pt}\hspace{115pt} ($X^{2}$ + 1) ($X^{2}$ + 2) ($X^{2}$ + 4),\ 

\hspace{55pt} $\mathbb F_{11}$ \hspace{5pt} ($X$ + 1) ($X$ + 10) ($X^{2}$ + 1) ($X^{2}$ + $X$ + 1)\ 

\vspace{0pt}\hspace{115pt} ($X^{2}$ + 5$X$ + 1) ($X^{2}$ + 6$X$ + 1) ($X^{2}$ + 10$X$ + 1),\ 

\hspace{55pt} $\mathbb F_{13}$ \hspace{5pt} ($X$ + 1) ($X$ + 2) ($X$ + 3) ($X$ + 4) $\dots$ ($X$ + 7)\ 

\vspace{0pt}\hspace{115pt} ($X$ + 8) ($X$ + 9) ($X$ + 10) ($X$ + 11) ($X$ + 12).\ 
\end{flushleft}

\newpage
\thispagestyle{empty}
$\newline$
$\newline$
$\newline$
\huge $\mathbf{Глава V}$ \ 

\vspace{18pt}\noindent \Huge $\mathbf{Дискретное}$ \ 

\vspace{3pt}\noindent \Huge$\mathbf{преобразование}$ $\mathbf{Фурье}$ \ 

\vspace{20pt}\noindent \normalsize Во все времена большие числа и, особенно, простые числа, заворажи­- \linebreak
вали людей --- математиков и нематематиков. Одной из первых тео­- \linebreak
рем теории чисел является, конечно же, теорема Евклида, утвержда­- \linebreak
ющая, что множество простых чисел бесконечно. В 1772 г. Эйлер ре­- \linebreak
шил проблему Мерсенна, доказав, что гигантское для той эпохи число \linebreak
$2^{31}$ $-$ 1 = 2\,147\,483\,647 является простым. В настоящее время благода­- \linebreak
ря использованию тестов простоты числа и развитию вычислительной \linebreak
техники, число Мерсенна с показателем степени 31 не доставляет осо­- \linebreak
бых затруднений. Об огромном прогрессе в этом направлении можно \linebreak
судить по следующему факту: в 1979 г. Нельсон и Словинский с ис­- \linebreak
пользованием CRAY-I доказали простоту числа $2^{44\,497}$ $-$ 1 (имеющего в \linebreak
своей записи 13\,395 десятичных цифр). С того времени было открыто \linebreak
много других простых чисел Мерсенна.\ 

\vspace{3pt} Этот прогресс был достигнут не только благодаря возрастающей \linebreak
мощи вычислительных машин. Многие проблемы теории чисел могут \linebreak
рассматриваться с точки зрения их решения на компьютерах\footnote{Это весьма существенно. Например, наивный тест простоты при помощи по­- \linebreak
следовательных делений числа с 200 десятичными цифрами требует (при использо­- \linebreak
вании машины, работающей со скоростью миллион операций в секунду) примерно \linebreak
$10^{86}$ в лет вычислений!}, с од- ­\linebreak
ной стороны, благодаря быстрому выполнению элементарных опера ­\linebreak
ций (умножению и делению больших чисел), а с другой стороны, --- \linebreak
благодаря открытию эффективных алгебраических алгоритмов (тест \linebreak
простоты простых чисел Мерсенна, принадлежащий Лукасу и Леме- \linebreak
ру, вероятностный тест Рабина, «недавний» тест Адлемана и Румли, \linebreak
улучшенный Коэном и Ленстрой $\dots$). \ 

\vspace{3pt} Приведем, к примеру, критерий Лукаса --- Лемера, лежащий в основе

\newpage
\lhead{\small{582}}
\rhead{\small{$\mathit{V-1 \hspace{10pt} Сложность \hspace{4pt} умножения \hspace{4pt} двух \hspace{4pt} многочленов}$}}
\normalsize

\noindent проверки простоты чисел Мерсенна $M_q$ = $2^q$ $-$ 1, где $q$ --- нечетное про­- \linebreak
стое число. Этот критерий предполагает вычисление последовательно­- \linebreak
сти $(L_i)_{i \geqslant 0}$, определенной следующим образом: $L_0$ = 4, $L_{i + 1}$ = $L^2_i$ $-$ 2 \linebreak
mod $M_q$). Он утверждает, что число $M_q$ простое тогда, и только то­- \linebreak
гда, когда  $L_{q - 2}$ = 0. Для того, чтобы доказать простоту числа $M_{132\,049}$ \linebreak
(имеющего 39\,751 десятичных цифр), при использовании теста Лукаса \linebreak
--- Лемера необходимо уметь оперировать с числами, имеющими при­- \linebreak
близительно 40\,000 десятичных цифр. \ 

\vspace{3pt} Другой пример той же природы — тест простоты чисел Ферма \linebreak
 $F_n$ = $2^{2^n}$ + 1. Это тест Пепина, утверждающий, что  $F_n$ простое тогда \linebreak
и только тогда, когда $3^{\frac{F_n - 1}{2}}$ $\equiv$ $-$1 (mod $F_n$). Этот тест (с возможной \linebreak
заменой 3 на 5) предполагает построение последовательности:  $P_0$ = 3, \linebreak
 $P_{i + 1}$ =  $P_i^2$ mod $F_n$; он определяет простоту числа $F_n$ с помощью условия \linebreak
 $P_{2^n - 1}$ $\equiv$ $-$1 (mod $F_n$). \ 

\vspace{3pt} Разработка эффективных алгоритмов стала возможной благодаря \linebreak
работам многих математиков: Кули и Тьюки, Полларда, Винограда, \linebreak
Шёнхаге и Штрассена, Рейдера, Фидуччиа и других. Дискретное пре­- \linebreak
образование Фурье, основы которого будут представлены в этой главе, \linebreak
является ведущей темой большинства этих работ. Мы начнем с задачи \linebreak
перемножения двух многочленов. Действительно, каждое число явля­- \linebreak
ется многочленом в подходящей базе счисления. Поэтому можно ис­- \linebreak
пользовать алгоритм вычисления произведения многочленов, а затем \linebreak
реализовать накопленные переносы. \newline

\paragraph{\large 1 Сложность умножения двух многочленов} \ 

\vspace{15pt}\normalsize \noindent Над произвольным кольцом $А$ традиционный метод вычисления произ­- \linebreak
ведения двух многочленов степени $n$ $-$ 1 требует в точности $n^2$ умн­о- \linebreak
жений и $(n - 1)^2$ сложений. Такой результат можно сформулировать в \linebreak
следующем виде: традиционное умножение двух многочленов степени \linebreak
$(n - 1)$ требует $\mathcal{O}(n^2)$ операций. Идея, лежащая в основе преобразования \linebreak
Фурье, состоит в замене многочлена степени ниже $n$ его значениями \linebreak
в $n$ хорошо выбранных точках. Приведем различные классические ре­- \linebreak
зультаты по вычислению значений в точке и интерполяции.

\paragraph{ 1.1 Интерполяция многочленов над полем} \ 

\vspace{0pt} \hspace{10pt}$\mathbf{и}$ $\mathbf{над}$ $\mathbf{кольцом}$ \ 

\vspace{8pt}\noindent Хорошо известно, что над $\mathbb R$ или $\mathbb C$ многочлен степени, строго меньшей \linebreak
$n$, полностью определяется своими значениями в $n$ различных точках \linebreak

\newpage
\rhead{\small{583}}
\lhead{\small{$\mathit{V-1.1 \hspace{16pt} Интерполяция \hspace{4pt} многочленов \hspace{4pt} над \hspace{4pt} полем \hspace{4pt} и \hspace{4pt} над \hspace{4pt} кольцом}$}}

\noindent $x_0$, $x_1$, $\dots$, $x_{n-1}$, (в действительности можно заменить $\mathbb R$ или $\mathbb C$ любым \linebreak
коммутативным полем). Более точно, отображение

\begin{center}
$K_n[X]$ $\ni$ $P$ $\mapsto$ $( P(x_0), P(x_1), \dots,  P(x_{n-1}))$ $\in$ $K^n$
\end{center}

\noindent является $K$-изоморфиэмом из пространства $K_n[X]$ (состоящего из по­- \linebreak
линомов степени, меньшей $n$, с коэффициентами из $K$)  на пространство \linebreak
 $K^n$, однако, при условии, что точки $x_0, x_1, \dots, x_{n-1}$ различны. \ 

\vspace{0pt}Первый подход состоит в рассмотрении матрицы этого линейного\linebreak
отображения в стандартном базисе с порождающими элементами $x^j_i$. \linebreak
Эта матрица, как известно, является матрицей Вандермонда, опреде­- \linebreak
литель которой есть $\prod_{i<j}(x_j - x_i)$. \ 

\vspace{0pt}Другой метод, более эффективный, состоит в использовании интер­- \linebreak
поляционных полиномов Лагранжа $L_i(X)$, определенных для \linebreak
0 $\leqslant$ $i$ $\leqslant$ $n$ $-$ 1 как

\begin{center}
$L_i(X)$ = $\frac{(X - x_0)(X - x_1) \dots (X - x_{i - 1})(X - x_{i + 1}) \dots (X - x_{n - 1})}{(x_i - x_0)(x_i - x_1) \dots (x_i - x_{i - 1})(x_i - x_{i + 1}) \dots (x_i - x_{n - 1})}$
\end{center}

\noindent и удовлетворяющих свойству $L_i(x_j)$ = $\delta_{ij}$. Эти полиномы позволяют \linebreak
определить $P$ по его значениям $y_0$, $y_1$, $\dots$,  $y_{n - 1}$, в точках  $x_0$,  $x_1$, $\dots$, \linebreak
 $x_{n - 1}$, согласно равенству: $P(X)$ = $\sum y_i L_i(X)$. \ 

\vspace{0pt}Можно видеть, что этот метод вычисления (интерполяции) приме­- \linebreak
ним к произвольному кольцу, в котором элементы $x_i$ $-$ $x_j$, для $i$ $\ne$ $j$, \linebreak
обратимы в этом кольце. Пусть $F$ --- преобразование, ставящее в соот­- \linebreak
ветствие многочлену $P$ его значения в $n$ точках $x_0$,  $x_1$, $\dots$,  $x_{n - 1}$. Это \linebreak
преобразование позволяет перейти от представления многочленов, за­- \linebreak
данных с помощью коэффициентов, к представлению через их значения. \linebreak
В другом смысле преобразование этих представлений есть интерполя­- \linebreak
ция. \ 

\vspace{0pt}Использование указанного метода для вычисления произведения \linebreak
многочленов основано на следующем равенстве: $P$ $\ast$ $Q$ = $\mathcal{F}^{-1}(\mathcal{F}(P)$ $\times$ \linebreak
$\mathcal{F}(Q) )$, где deg($P$ $\ast$ $Q$) < $n$. Умножение слева --- умножение многочленов, \linebreak
умножение справа --- умножение функций из [0, $n$[ в $A$.  Первое требу­- \linebreak
ет $n^2$ операций в основном поле, второе --- $n$. Оставшиеся операции, \linebreak
необходимые для вычисления, сводятся к вычислению значений много- \linebreak
члена в точках и интерполяции многочлена по значениям в $n$ точках. \linebreak
Для вычисления значений обычно используют метод Горнера, который \linebreak
требует $\mathcal{O}(n^2)$ операций для вычисления значений многочлена степени, \linebreak
меньшей $n$, в $n$ точках. Интерполяция имеет аналогичную сложность. \linebreak
Поясним вкратце основную идею метода Горнера вычисления в 1 точке. \linebreak
Например, для многочлена $P(X)$ = $a_3X^3$ + $a_2X^2$ + $a_1X^1$ + $a_0$ степени \linebreak

\newpage
\lhead{\small{584}}
\rhead{\small{$\mathit{V-1 \hspace{10pt} Сложность \hspace{4pt} умножения \hspace{4pt} двух \hspace{4pt} многочленов}$}}
\noindent $\leqslant$ 3 запишем: $P(x)$ = $((a_3 \times x + a_2) \times x + a_1) \times x + a_0$, и по методу Горнера \linebreak
имеем:

\begin{center}
$y$ $\longleftarrow$ $a_3$; $y$ $\longleftarrow$ $y$ $\times$ $x$ + $a_2$; $y$ $\longleftarrow$ $y$ $\times$ $x$ + $a_1$; $y$ $\longleftarrow$ $y$ $\times$ $x$ + $a_0$;
\end{center}

\noindent Поэтому вычисление $y$ = $P(x)$ использует 3 умножения и 3 сложения. \linebreak
Для многочленов степени $n$ $-$ 1 необходимо (для одной точки) $n$ $-$ 1 \linebreak
умножений и $n$ $-$ 1 сложений. \ 

\vspace{0pt}Даже если последовательность $\{x^1, x^2, \dots, x^{n - 1}\}$ раз и навсегда за- \linebreak
табулирована, то обычный метод вычисления значений многочлена тре­- \linebreak
бует $n$ $-$ 1 сложений и $n$ $-$ 1 умножений. Короче, наивный способ вычи­- \linebreak
сления значений многочлена степени, строго меньшей $n$, в $n$ точках \linebreak
требует $n$ х ($n$ $-$ 1) умножений и $n$ х ($n$ $-$ 1) сложений. \ 

\vspace{0pt}Особенность быстрого преобразования Фурье, которое является ни­- \linebreak
чем иным, как быстрым методом вычисления/интерполяции многочле­- \linebreak
на, основана на разумном выборе точек вычисления значений: корнях \linebreak
$n$-й степени из единицы. Таким образом, его сложность понижается до \linebreak
$\mathcal{O}$($n$ long $n$) операций базового кольца $A$ вместо $\mathcal{O}(n^2)$ операций, неиз­- \linebreak
бежных в классическом методе.

\paragraph{ 1.2 Вычисление значений многочленов в корнях} \ 

\vspace{0pt} \hspace{10pt}$\mathbf{из}$ $\mathbf{единицы}$ \  

\vspace{8pt}\noindent Выбор корней $n$-й степени из единицы как точек для вычисления зна­- \linebreak
чений многочленов играет фундаментальную роль для вычисления их \linebreak
произведения по следующим двум причинам:\ 

\vspace{3pt}$\bullet$ \hspace{3pt}он сводит интерполяцию многочлена в корнях из единицы, к вы­- \ 

\vspace{0pt}\hspace{8pt} числению значений многочлена в тех же корнях из единицы, \ 

\vspace{0pt}$\bullet$ \hspace{3pt}он позволяет, если $n$ --- сильно составное число (например, сте­- \ 

\vspace{0pt}\hspace{8pt} пень двойки или степень тройки), реализовать быстрое вычисле­- \ 

\vspace{0pt}\hspace{8pt} ние (и, следовательно, быстрее выполнить интерполяцию).\ 

\vspace{1pt}Пусть $\omega$ --- корень $n$-й степени из единицы. Вычисление значений \linebreak
многочлена степени < $n$ в $n$ точках $\omega^0$, $\omega^1$, $\omega^2$, $\dots$, $\omega^{n-1}$  связано с ли­- \linebreak
нейным отображением, матрица которого $V_\omega$ состоит из порождающих \linebreak
элементов $\omega^{ij}$ Если $P(X)$ = $\sum a_j X^j$ --- многочлен степени меньшей $n$, \linebreak
то вычисление $\hat a_i$ = $P(\omega^i)$ эквивалентно вычислению произведения сле­- \linebreak
дующих матриц:

\[
\begin{pmatrix} \hat a_0 \\ \hat a_1 \\ \vdots  \\ \hat a_{n-1}  \end{pmatrix} = V_\omega 
\begin{pmatrix} a_0 \\ a_1 \\ \vdots  \\ a_{n-1}  \end{pmatrix}, где V_\omega =
\begin{pmatrix} 1 & 1 & \hdots & 1 \\ 1 & \omega^1 & \hdots & \omega^{(n - 1)} \\ \vdots & \vdots & \ddots & \vdots & \\ 1 & \omega^{(n - 1)} & \hdots & \omega^{(n - 1)(n - 1)} \end{pmatrix}.\hspace{5pt} (1)
\] 

\newpage
\rhead{\small{585}}
\lhead{\small{$\mathit{V-1.2 \hspace{16pt} Вычисление \hspace{4pt} значений \hspace{4pt} многочленов \hspace{4pt} в \hspace{4pt} корнях \hspace{4pt} из \hspace{4pt} единицы}$}}

Приведем пример, который позволит читателю убедиться, что ис­- \linebreak
пользование корней из единицы приводит к ускорению вычислений. \linebreak
Рассмотрим задачу вычисления значений многочлена степени 14 в сте­- \linebreak
пенях $\omega$,  корня 15 степени из единицы. A $priori$, наивный подход (как \linebreak
в случае горнеровской схемы, так и в случае вычисления для затабу- \linebreak
лированных значений степеней $\omega$)  для множества из 15 вычислений по \linebreak
формуле (1) требует 15 $\times$ 15 = 225 умножений и 14 $\times$ 15 = 210 сложений. \linebreak
Эти 15 вычислений (1) соответствуют матричному произведению
\[
\begin{pmatrix}
 \hat a_0 \\
 \hat a_1 \\
 \hat a_2 \\
 \vdots  \\
 \hat a_{12} \\
 \hat a_{13} \\
 \hat a_{14} \\  
\end{pmatrix} =
\begin{pmatrix}
 1 & 1 & 1 & \hdots & 1 & 1 & 1 \\
 1 & \omega^1 & \omega^2 & \hdots & \omega^{12} & \omega^{13} & \omega^{14} & \\
 1 & \omega^2 & \omega^4 & \hdots & \omega^9 & \omega^{11} & \omega^{13} & \\
 \vdots & \vdots &\vdots & \ddots & \vdots & \vdots & \vdots & \\
 1 & \omega^{12} & \omega^9 & \hdots & \omega^9 & \omega^6 & \omega^3 & \\
 1 & \omega^{13} & \omega^{11} & \hdots & \omega^6 & \omega^4 & \omega^2 & \\
 1 & \omega^{14} & \omega^{13} & \hdots & \omega^3 & \omega^2 & \omega^1 & \\
\end{pmatrix}
\begin{pmatrix}
 a_0 \\
 a_1 \\
 a_2 \\
 \vdots  \\
 a_{12} \\
 a_{13} \\
 a_{14} \\  
\end{pmatrix}.
\] 
Если не считать умножения на 1 этой матрицы за операцию
\begin{wraptable}{}{0.6\textwidth}
умножения, то их число уменьшается до 180. \linebreak
Но как можно учесть в программировании \linebreak
эти единицы? Тестировать перед умноже­- \linebreak
нием каждый из элементов (значит, реали­- \linebreak
зовать 225 проверок и 180 умножений) мо­- \linebreak
жет оказаться менее выгодным, чем осуще­- \linebreak
ствить 225 умножений. Однако, можно «без \linebreak
поиска единиц внутри» рассмотреть те из \linebreak
них, которые расположены по краям, как в \linebreak
алгоритме слева.
\end{wraptable}
\noindent $\hat a_0$ $\longleftarrow$ $a_0$; \newline
for (int $i$ = 1; $i$ < $n$; ++$i$) \{ \newline
 $\hat a_0$ $\longleftarrow$ $\hat a_0$ + $a_0$; \newline
\} \newline
for (int $j$ = 1; $j$ < $n$; ++$j$) \{ \newline
$\hat a_j$ $\longleftarrow$ $a_0$; \newline
for (int $i$ = 1; $i$ < $n$; ++$i$) \{ \newline
$\hat a_j$ $\longleftarrow$ $\hat a_j$ + $\omega^{ij}a_i$; \newline
\} \} \newline

Несмотря на это, полученный алгоритм всегда имеет сложность \linebreak
$\mathcal{O}(n^2)$: он использует в точности $(n - 1)^2$ умножений и$n(n - 1)$ сло­- \linebreak
жений, так что при $n$ = 15 имеем 196 умножений и 210 сложений. \ 

\vspace{0pt}Другой метод, в справедливости которого читатель может убедить­- \linebreak
ся, реализующий вычисление значений многочлена в тех же точках, \linebreak
с гораздо меньшим числом умножений. Он реализует вычисления в 2 \linebreak
этапа, осуществляя вначале частичные суммирования, дающие 15 чле­- \linebreak
нов $b_j$:
\[
\begin{pmatrix}
 b_0 \\
 b_1 \\
 b_2 \\
 b_3 \\
 b_4 \\
\end{pmatrix} = W
\begin{pmatrix}
 a_0 \\
 a_3 \\
 a_6 \\
 a_9 \\
 a_{12} \\
\end{pmatrix}, \hspace{7pt}
\begin{pmatrix}
 b_5 \\
 b_6 \\
 b_7 \\
 b_8 \\
 b_9 \\
\end{pmatrix} = W
\begin{pmatrix}
 a_1 \\
 a_4 \\
 a_7 \\
 a_{10} \\
 a_{13} \\
\end{pmatrix}, \hspace{7pt}
\begin{pmatrix}
 b_{10} \\
 b_{11} \\
 b_{12} \\
 b_{13} \\
 b_{14} \\
\end{pmatrix} = W
\begin{pmatrix}
 a_2 \\
 a_5 \\
 a_8 \\
 a_{11} \\
 a_{14} \\
\end{pmatrix},
\] 

\newpage
\lhead{\small{586}}
\rhead{\small{$\mathit{V-1 \hspace{10pt} Сложность \hspace{4pt} умножения \hspace{4pt} двух \hspace{4pt} многочленов}$}}

\noindent где $W$ --- матрица Вандермонда, $V_{\omega^3}$, порожденная корнем $\omega^3$. Эти \linebreak
15 членов интерпретируются как вычисления значений многочленов \linebreak
$P_0(Y)$ = $a_0$ + $a_3Y$ + $a_6Y^2$ + $a_9Y^3$ + $a_{12}Y^4$, $P_1(Y)$ = $a_1$ + $a_4Y$ + $a_7Y^2$ + \linebreak
$a_{10}Y^3$ +  $a_{13}Y^4$, и $P_2(Y)$ = $a_2$ + $a_5Y$ + $a_8Y^2$ + $a_{11}Y^3$ +  $a_{14}Y^4$ в следующих \linebreak
точках: 1, $\omega^3$, $\omega^6$, $\omega^9$ и $\omega^{12}$. В этих обозначениях, $P(X)$ = $P_0(X^3)$ + \linebreak
$X \hspace{1pt}P_1(X^3)$ + $X^2 \hspace{1pt}P_2(X^3)$ и, значит, $\hat a_j$ = $P_0(\omega^{3j})$ + $\omega^j \hspace{1pt}P_1(\omega^{3j})$ + $\omega^{2j} \hspace{1pt}P_2(\omega^{3j})$. \linebreak
Поэтому для вычисления элементов $\hat a_j$ можно использовать следующие \linebreak
явные формулы:

\begin{center}
\vspace{10pt}
$\hat a_0$ \hspace{6pt}=\hspace{1pt} $b_0$ +\hspace{1pt} $b_5$ \hspace{18pt}+ $b_{10}$, \hspace{40pt} $\hat a_1$ \hspace{9pt}= $b_1$ \hspace{3pt}+ $b_6\omega^1$ \hspace{5pt}+ $b_{11}\omega^2$, \ 

\vspace{1pt}
$\hat a_5$ \hspace{6pt}=\hspace{1pt} $b_0$ +\hspace{1pt} $b_5\omega^5$ \hspace{8pt}+ $b_{10}\omega^{10}$, \hspace{25pt} $\hat a_1$ \hspace{9pt}= $b_1$ \hspace{3pt}+ $b_6\omega^6$ \hspace{4pt}+ $b_{11}\omega^{12}$,

\vspace{1pt}
$\hat a_{10}$ \hspace{3pt}=\hspace{1pt} $b_0$ +\hspace{1pt} $b_5\omega^{10}$ \hspace{4pt}+ $b_{10}\omega^{5}$, \hspace{28pt} $\hat a_{11}$ \hspace{6pt}= $b_1$ \hspace{3pt}+ $b_6\omega^{11}$ \hspace{1pt}+ $b_{11}\omega^{7}$, \hspace{24pt}
\end{center}

\begin{center}
$\hat a_2$ \hspace{6pt}=\hspace{1pt} $b_2$ +\hspace{1pt} $b_7\omega^2$ \hspace{8pt}+ $b_{12}\omega^4$, \ 

\vspace{1pt}
$\hat a_7$ \hspace{6pt}=\hspace{1pt} $b_2$ +\hspace{1pt} $b_7\omega^7$ \hspace{8pt}+ $b_{12}\omega^{14}$,\ 

\vspace{1pt}
$\hat a_{12}$ \hspace{3pt}=\hspace{1pt} $b_2$ +\hspace{1pt} $b_7\omega^{12}$ \hspace{4pt}+ $b_{12}\omega^{9}$, \hspace{24pt}
\end{center}

\begin{center}
\vspace{10pt}
$\hat a_3$ \hspace{6pt}=\hspace{1pt} $b_3$ +\hspace{1pt} $b_8\omega^3$ \hspace{8pt}+ $b_{13}\omega^{6}$, \hspace{25pt} $\hat a_4$ \hspace{9pt}= $b_4$ \hspace{3pt}+ $b_9\omega^4$ \hspace{4pt}+ $b_{14}\omega^{8}$, \ 

\vspace{1pt}
$\hat a_8$ \hspace{6pt}=\hspace{1pt} $b_3$ +\hspace{1pt} $b_8\omega^8$ \hspace{8pt}+ $b_{13}\omega^{1}$, \hspace{25pt} $\hat a_9$ \hspace{9pt}= $b_4$ \hspace{3pt}+ $b_9\omega^9$ \hspace{4pt}+ $b_{14}\omega^{3}$, \ 

\vspace{1pt}
\hspace{0pt} $\hat a_{13}$ \hspace{3pt}=\hspace{1pt} $b_3$ +\hspace{1pt} $b_8\omega^{13}$ \hspace{4pt}+ $b_{13}\omega^{11}$, \hspace{20pt} $\hat a_{14}$ \hspace{6pt}= $b_4$ \hspace{3pt}+ $b_9\omega^{14}$ \hspace{1pt}+ $b_{14}\omega^{13}$\hspace{1pt}.
\end{center}

Значит этот метод требует 90 сложений и 76 умножений. Его основ­- \linebreak
ной характеристикой является то , что число выражений $P_0(\omega^{3j})$, \linebreak
$P_1(\omega^{3j})$, $P_2(\omega^{3j})$ равно 15 (а не 45, как можно было бы думать a $pri$- \linebreak
$ori$) благодаря тому, что $\omega^15$ = 1. Это пример, который наглядно пока­- \linebreak
зывает эффективные методы вычисления значений, представленные в \linebreak
следующих разделах, позволяет читателю проникнуться идеей исполь­- \linebreak
зования корней из единицы. \ 

\vspace{3pt}Вернемся теперь к тому, что нас здесь интересует, а именно, к \linebreak
обратимости $V_\omega$. Эта матрица Вандермонда $V_\omega$ имеет определитель \linebreak
$\prod_{i<j}(\omega^i - \omega^j)$. Откуда получаем: \ 

\vspace{3pt}$\mathbf{(1)}$\hspace{3pt}$\mathbf{Свойство.}$\ 

\vspace{3pt}Если $\omega$ --- корень $n$-й степени из единицы, то матрица $V_\omega$ обратима \linebreak
тогда и только тогда, когда элементы  1 $-$ $\omega^i$ обратимы в $A$ . \ 

\vspace{6pt}В некотором поле, если $\omega$ --- корень $n$-й степени из единицы, $V_\omega$ обра­- \linebreak
тима тогда и только тогда, когда $\omega$ --- корень порядка $n$. Однако это \linebreak
свойство не переносится на произвольные кольца. Например, в $\mathbb Z$/12$\mathbb Z$ \linebreak
5 --- корень второй степени из единицы, но матрица $V_5$ = $\left(\underset{1}{1} \hspace{2pt}\underset{5}{1}\right)$ име­- \linebreak
ет определитель 4 и потому необратима. В следующем разделе будет \linebreak
установлено, для каких корней $\omega$ матрица Вандермонда $V_\omega$ обратима. 
\end{document}
