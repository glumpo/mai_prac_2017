\noindent Можно видеть, что в любом случае алгоритм должен начинаться с вы-\linebreak
числения НОД и коэффициентов Везу для $a$ и $n$. В случае, когда $a$ не­-\linebreak
обратим и $b$ делится на $d$, каждому решению по модулю $n'$ отвечают\linebreak
$d$ решений по модулю $n$, отличающихся слагаемым, кратным $n'$. Заме­-\linebreak
тим, что этот алгоритм неявно, но конкретно, обрабатывает случай,\linebreak
когда $a=0$.\newline
\\
\hspace*{15pt}\textbf{b.} Для доказательства этого факта с помощью использования двой­-\linebreak
ной индукции заметим, что $(p+1)(p+2)...(p+n-1)(p+n)=$\linebreak
$p\cdot(p+1)(p+2)...(p+n-1)+n\cdot(p+1)(p+2)...(p+n-1)$, где\linebreak
каждое слагаемое, удовлетворяющее тому или иному предположению\linebreak
индукции, делится на $n!$. К тому же, в данном рекуррентном соотно-\linebreak
шении нас интересует целочисленность величины $\left(\begin{smallmatrix}
n+p\\ p\\
\end{smallmatrix}\right).$\newline
\\
\hspace*{15pt}\textbf{c.} Формула Тейлора
$$P(a+tp^\alpha)=P(a)+tp^\alpha P'(a)/1!+...+t^dp^{ad}P^{(d)}(a)/d!$$
доказывает первое свойство, так как вопрос а позволяет показать, что\linebreak
$P^{(i)}$ делится на $i!$. Доказательство необходимого и достаточного усло-\linebreak
вия очень просто, оно опирается на то т факт, что если разделить пре-\linebreak
дыдущую формулу на $p^\alpha$, то получим доказательство в одну сторону,\linebreak
а если умножить полученную формулу на $p^\alpha$, то получится доказатель-\linebreak
ство в другую сторону. Алгоритм получается немедленно:
\begin{lstlisting}[mathescape=true, language=Ada]
$E\longleftarrow\emptyset;$ for $a$ in $\{\text{нули}$ $P$ $\text{по модулю}$ $p^{\alpha}\}$ loop
	for $t$ in $\{ \text{решение}$ $P(a)/p^\alpha+tP'(a)=0\mod{p^1} \}$ loop
		$E\longleftarrow E\cup\{a+tp^\alpha\};$
	end loop;
end loop;
\end{lstlisting}
\hspace*{15pt}\textbf{d.} Предположим, что $P$ имеет единственный корень $a_i$ по модулю $p^i$,\linebreak
который поднимает корень $a$. Сравнение $P(a_i+tp^i)\equiv P(a_i)+tp^i P'(a_i)\equiv$\linebreak
$\equiv P(a_i)+tp^iP'(a)~(\text{mod }p^{i+1})$ доказывает существование и единствен-\linebreak
ность такого $t$ по модулю $p$, что $P(a_i+tp^i)\equiv0 = 0 ~(\text{mod }p^{i+1})$. Кроме\linebreak
того, $tp^i$ дается формулой $tp^i=-P(a_i)P'(a)^{-1}$. Это доказывает, что $P$\linebreak
имеет один и только один корень $a_{i+1}$ по модулю $p^{i+1}$ , поднимающий\linebreak
корень $a$. Дальше достаточно воспользоваться индукцией.\newline
\\
\hspace*{15pt}\textbf{e.} В действительности сформулированное в предыдущем вопросе\linebreak
свойство может быть обобщено и передоказано по модулю $p^{k+\alpha}$ для\linebreak
корней $P$ по модулю $p^\alpha$ для всякого $k\leqslant a$. Необходимое и достаточное\linebreak
условие выглядит тогда так: $P(a+tp^\alpha)\equiv 0~(\text{mod }p^{k+\alpha})$ тогда и только\linebreak
тогда, когда $P(a)/p^\alpha+tP'(a)\equiv0~(\text{mod}~p^k)$. Единственное изменение\linebreak

\pagebreak