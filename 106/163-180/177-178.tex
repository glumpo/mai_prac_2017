\documentclass{mai_book}

\defaultfontfeatures{Mapping=tex-text}
\setdefaultlanguage{russian}
\setcounter{page}{177}

\begin{document}
\chapter{}
\section{Евклид и основная теорема арифметики}
\noindent Вычислить наибольший общий делитель двух целых чисел {\it a} и {\it b}.Такое задание иногда получают лицеисты на уроке математики. Они начинают с разложения чисел {\it a} и {\it b} в произведение степеней простых чисел (используя при необходимости нулевые показатели для степеней простых чисел, чтобы выровнять количество простых чисел в разложениях): 
\[
 a=p_1^{\alpha_1}p_2^{\alpha_2}...p_m^{\alpha_m} (\alpha_i \geqslant 0), \quad b=p_1^{\beta_1}p_2^{\beta_2}...p_m^{\beta_m} (\beta_i \geqslant 0) \quad (p_i  \text{простое}),
\]
\noindent Потом они применяют хорошо известную формулу, которая дает наибольший общий делитель (НОД): 
\[
\text{НОД}(a,b)=p_1^{\inf (\alpha_1,\beta_1)} \times p_2^{\inf (\alpha_2,\beta_2)} \times \cdots \times p_m^{\inf (\alpha_m,\beta_m)}.
\]
Легко используемый в работе с малыми числами (например, $а = 84 = 2^2 х 3 х 7, b = 198 = 2 х З^2 х 11$, что дает НОД(84,198) = 2 x 3 = 6), этот метод быстро становится неприемлемым для больших чисел. Рассмотрим, к примеру $a = 1 100 005 423$ и $b = 1 100 000 077$. Разложение этих двух чисел в произведение простых множителей с помощью обыкновенного калькулятора, требующее примерно $10^4$ делений, убеждает, что приведенная выше формула совершенно бесполезна.

К счастью, 22 века назад греческий математик Евклид открыл эффективный метод вычисления НОД. Этот метод, известный как {\it алгоритм Евклида}, настолько фундаментальный, что слово {\it алгоритм} используется математиками (помимо своего обычного смысла в информатике) для выявления делимости в некоторых кольцах. Можно с полным основанием считать Евклида предшественником алгоритмической алгебры.

Мы продемонстрируем этот алгоритм на примере, рассмотрение которого не привело лицеистов к успеху. 
\newpage

Выполним евклидово деление (вводимое в начальной школе и состоящее в нахождении частного и остатка) числа $a = 1100 005 423$ на число $b = 1 100 000 077$, т.е. запишем $a = bq_1 + r_2$, где $q_1 = 1$ и $r_2 = 5346$. Осуществим аналогичный шаг с $b$ и $r_2$, что приводит к $b = r_2q_2 + r_3$ с $q_2 = 205 761$ и $r_з = 1771$. Продолжим затем таким же образом с $r_2$ и $r_3$ и т.д. В результате получим следующую таблицу евклидовых делений: 

$\hspace*{1cm} r_0 = 1 100 005 423, \hspace*{1cm}r_1 = 1 100 000 077,$\\ 
$\hspace*{1.54cm} r_1 = 1 100 000 077,\hspace*{1cm} r_2 = 5 346, $\\
$\hspace*{1.54cm} r_2 = 5 346,\hspace*{2.08cm} r_3 = 1 771,$ \\
$\hspace*{1.54cm} r_3 = 1 771,\hspace*{2.08cm} r_4 = 33,$ \\
$\hspace*{1.54cm} r_4 = 33,\hspace*{2.4cm} r_5 = 22, $\\
$\hspace*{1.54cm} r_5 = 22,\hspace*{2.4cm} r_6 = 11,$ \\

$\hspace*{1.28cm} 1 100 005 423 = 1 100 000 077\times 1 + 5346 \hspace*{1.54cm}(q_1 = 1),$\\
$\hspace*{1.8cm} 1 100 000 077 = 5 346\times 205 761+ 1 771\hspace*{1.75cm} (q_2 = 205 761),$\\
$\hspace*{2.88cm} 5 346 = 1 771\times 3 +33 \hspace*{2.95cm}(q_3 = 3),$\\
$\hspace*{2.88cm} 1 771 = 33\times 1 +11 \hspace*{3.3cm}(q_4 = 53),$\\
$\hspace*{3.21cm} 33 = 22\times 1 + 11 \hspace*{3.34cm}(q_5 = 1),$\\
$\hspace*{3.21cm} 22 = 11\times 2 + 0 \hspace*{3.5cm}(q_6 = 2).$\\

    Алгоритм за канчивает работу после получения нулевого остатка при делении 22 на 11. Последний полученный ненулевой остаток $r_6 = 11$ является НОД чисел $a$ и $b$. Чтобы убедиться в этом, надо, с одной стороны, увидеть общее равенство (становящееся однородным с обозначением $r_0 = a, r_1 = b, r_7 = 0): r_{i-1} = r_iq_i + r_{i+1}$ для $0\leqslant i\leqslant 6$ и,с другой стороны, воспользоваться свойством натуральных чисел: $d| bq+r \text{ и } d | b$ тогда и только тогда, когда $d | b \text{и} d | r$. Эта последняя эквивалентность приводит, в частности, к равенству НОД$(bq + r,b) = \text{НОД}(b, r)$. Из этого следует, что величина НОД$(r_i,r_{i+1})$ не зависит от $i$. При $r = 0$ она равна НОД$(a, b)$, а при $i = 6 — \text{НОД}(r_6,0) = r_6 = 11$, что подтверждает результат, полученный выше.
    \begin{center}
    \parbox{12cm}{
    Замечание. Бели необходимо применить первый метод вычисления НОД $a = 1100 005 423$ и $b = 1100 000 077$, то надо разложить эти два числа в произведение простых множителей. Поиск простых делителей для разложения а и Ь и найденный общий простой делитель 11 приводят к констатации, что числа $а/11$ и $b/11$ оба являются простыми. Эта последняя проверка требует при использовании наивного метода лицеистов приблизительно $2 х (\sqrt{10^8}/2) = 10^4$ делений (каждое из двух чисел $a/11$ и $b/11$ имеют порядок $10^8$). Разложение на простые множители $a = 11 х 100 000 493, b = 11 х 100 000 007$ снова дает НОД$(a, b) = 11$,}
\end{center}

\end{document}