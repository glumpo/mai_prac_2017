\documentclass{mai_book}

\defaultfontfeatures{Mapping=tex-text}
\setmainfont{DejaVuSerif}
\setdefaultlanguage{russian}

\begin{document}
\noindent Оценим сложность алгоритма: в дальнейшем будем пренебрегать умно­жениями на целые константы и будем рассматривать только операции
над коэффициентами формальных рядов. Очевидно, что для вычисле­ния $i$-го терма ряда $\mathcal{B}$ необходимо $2i$ умножений ($i$ умножений скалярных) и $i-1$ сложений. Знание $i-1$ членов этого ряда и $i$ членов
формального ряда А также необходимы, что дает сложность: $i(i+1)$
умножений и $i(i-1)$ сложений в базовом кольце.

\textbf{c.} При применении к многочленам изученных в двух предыдущих
задачах принципов некоторое упрощение может быть достигнуто для
многочленов с конечным числом членов. Имеется также возможность
реализовать эти алгоритмы для многочленов, представимых массивами
(если иметь дело с плотными или не очень разряженными многочленами) . Обсудим теперь модификации алгоритма возведения в степень в
случае действий с многочленами, с учетом того, что алгоритм возведе­ния многочленов в квадрат был изучен в одном из предыдущих упражнений. В сумме для $i$-го коэффициента, встречающейся при вычислении
многочлена $\mathcal{P}^n$, число слагаемых, которые находятся под знаком суммы, зависит не только от $n$, но и от степени многочлена, над которым
производится действие. Точнее, при $i > 0$:
$$b_{i}=\frac{1}{a_{0}i}\times\sum\limits_{j=1}^{\text{min}(i,\text{deg}\mathcal{P})}((n+1)j-i)a_{j}b_{i-j},$$
и сложность вычисления существенно уменьшается. Действительно,
суммы, позволяющие вычислить коэффициент, имеют самое большее
deg $\mathcal{P}$ членов. Следовательно, вычисление коэффициентов результирующего многочлена требует: $n$ умножений для вычисления $b_{0}$, $i$ умноже­ний в базовом кольце и $i-1$ сложений в базовом кольце, если $i\leqslant\text{deg}\mathcal{Р}$,
deg$\mathcal{P}$, умножений и deg$\mathcal{P}$ сложений в базовом кольце, если $i > \text{deg}\mathcal{P}$.
Таким образом, получаем общую оценку сложности, не превышающую $d^2 n$ умножений и сложений.
\\\\
\noindent\textbf{34. Определение нулей многочдена по модулю $p^{\alpha}$}\\

\textbf{а.} Решение этого уравнения тривиально в случае, когда а обратим
по модулю $n$. Если $a$ не обратим, то вычислим $d = \Nod(a,n)$ и рассмотрим два случая: в лучшем $b$ делится на $d$, в этом случае разделим уравнение на $d$ и решим уравнение $a'x = b'$ по модулю $n'$, с обратимым $a'$.
\begin{lstlisting}[frame=none, mathescape=true]
\\ Пусть $d\geqslant 0$, $u$ и $\;v$ такие, что $\;ua+vn=d=\Nod(a,n)$
if (b % d != 0) return NULL;
else return (ub/d+kn/d) \\ $0\leqslant k<d$
\end{lstlisting}
(Если $b$ не делится на $d$, то уравнение не имеет решений.)
\end{document}