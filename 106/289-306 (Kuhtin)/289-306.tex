%Я это еще доделываю, так что оставьте комменты %

\documentclass{mai_book}

\defaultfontfeatures{Mapping=tex-text}
\setmainfont{DejaVuSerif}
\setdefaultlanguage{russian}

\setcounter{page}{289} % ВОТ ТУТ ЗАДАТЬ СТРАНИЦУ
%\setcounter{thesection}{5} % ТАК ЗАДАВАТЬ ГЛАВЫ, ПАРАГРАФЫ И ПРОЧЕЕ.
%\setcounter{equation}{10}% номер формул

%=================================================================================================================================
%289
\begin{document}
\noindent Это очевидно, если $b \leqslant 5$. Если $b > 5$,  то $5 - b$  представляет 5 по моду­\linebreak
лю $b$\ldots

Можно заметить, что мы располагаем парами $(a,b)$, где $b$ --- дели-\linebreak
тель $a$  с евклидовыми делениями $a = bq + r$, где $r$ уже не нуль! (Возь­\linebreak
мите, например, $40 = 5 \times 6 + 10$.)\\
\\
\noindent\textbf{27. Самый малый алгоритм Евклида на $\mathbb{•}{Z}$}\\
\\
\hspace*{15pt}То, что $\varphi$ ---  алгоритм Евклида, содержится в примере (раздел 3.1)\linebreak
из этого курса. Пусть $\psi$ --- другой алгоритм Евклида на $Z$  со значени­\linebreak
ями в $N$. Докажем с помощью индукции по $n =\psi(b)$, что $\varphi(b) \leqslant \psi(b)$.\\
\hspace*{15pt}Это верно для $n = 0$, так как из $\psi(b) = 0$ следует, что $b = 0$\linebreak
(см. лемму 29). Предположив, что это выполняется для всех $b$, таких,\linebreak
что $\psi(b) < n$,  докажем, что оно верно и для $\psi(b) = n$.\\
\hspace*{15pt}Подвергнем $\psi$-евклидову делению $[|b|/2]$ на число $b:[|b|/2] = bq+r,$\linebreak
где $\psi(r) < \psi(b)$. Так как r  представитель $[|b|/2]$ по модулю  b, то \linebreak
$|r| \geqslant [|b|/2]$. Отсюда имеем: 
$$\varphi(r) \geqslant \varphi([|b|/2]) = \varphi(b) - 1.$$
Так как $\psi(r) < n$, то по индукции $\psi(r) = \varphi(r)$ и поэтому:
$$\varphi(b) \leqslant \varphi(r) + 1 = \psi(r) + 1 \leqslant \psi(b),$$
что заканчивает доказательство.\\
\\
\noindent\textbf{30. Конечные поля, порядок которых есть квадрат\linebreak
простого числа}\\
\\
\hspace*{15pt}Согласно изложенному выше (раздел 1.3, предложение 8), $\rho$ --- не­\linebreak
приводимо в $Z[i]$ и потому порождает максимальный идеал. Итак,\linebreak
$Z[i]/(p)$ ---  поле, так как:
$$x + iy \equiv 0 ~(mod~ \rho) \Longleftrightarrow x, y \equiv 0 ~(mod~ \rho ),$$
\textbf{аддитивная} группа $\mathbb{Z}[i]/(p)$  изоморфна $\mathbb{Z}/(p)$ $\times$ $\mathbb{Z}/(p)$, откуда порядок\linebreak
поля, о котором идет речь, равен $p^{2}$.\\
\\
\noindent\textbf{31. Несколько конечных полей порядка, являющегося\linebreak
степенью двойки}\\
\\
Многочлены $P(X)$ следующие:
			$$\begin{array}{llll}
							 X^2 + X + 1,   & X^3 + X + 1,   & X^4 + X + 1,\\ 
			                          & X^5 + X^2 + 1, & X^6 + X + 1, & X^7 + X + 1, 
			\end{array}$$
\newpage
%======================================================================================================================
%290
\noindent неприводимые по модулю 2 (для каждой данной степени $n$ --- это пер-\linebreak 
вые неприводимые многочлены степени n, в смысле лексикографичес-­\linebreak
кого порядка, о чем читатель, конечно, догадался). Поля $\mathbb{Z}_2[X]/(P)$\linebreak
отвечают на вопрос задачи.\\
\\
\noindent\textbf{32. Получение всех коэффициентов Безу}\\
\\
\hspace*{15pt}Так как $(a, b)$ можно всегда заменить на $(a/d, b/d)$, то будем пред­\linebreak
полагать, что $d = 1$, т.е. $a$ и $b$ взаимно просты. Тогда имеем: $(u_0 - u)a =$ \linebreak
$(v - v_0)b \Rightarrow a \mid (v - v_0)b$. К тому же $a$ и $b$  взаимно просты. Отсюда $a$\linebreak
делит $v - v_0$. Следовательно, существует $k \in \mathbb{Z}$ такое, что $k = u_0v - v_0u$,\linebreak
так как $a(u_0v - v_0u) = v - v_0$.\\
\\
\noindent\textbf{33. Единственность ограниченных коэффициентов Безу}\\
\\
\hspace*{15pt}С помощью замены $(a, b)$ на $(a/d, b/d)$  можно всегда предполагать,\linebreak
что $d = 1$.\\
\\
\hspace*{15pt}\textbf{a.} Согласно предыдущему упражнению, существует такое $k \in \mathbb{Z}$,\linebreak
что $u' = u - kb$ и $v' = v+ka$. Следовательно, число $u'$ есть представитель\linebreak
по модулю $b$ числа $u$, отличный от $u$. Неравенство для $|u|$ показывает,\linebreak
что $|u'| \geqslant |b|/2$. Равенство возможно только для четного $b$. Но $a$ и $b$ не\linebreak
могут быть оба четными, так как они взаимно просты.\\
\\
\noindent\textbf{34. Мажорирование коэффициентов Безу в $K[X]$}\\
\\
\hspace*{15pt}Достаточно сымитировать доказательство, касающееся коэффици-­\linebreak
ентов Безу в $\mathbb{Z}$ из раздела 7.2. Алгоритм создает последовательность\linebreak
$Q_1, Q_2, ..., Q_n$  с одной стороны, $U_0, U_1, ..., U_{n+1}$ и $V_0, V_1, ..., V_{n+1}$ с:
		$$\begin{array}{ccccc}
			U_0 = 1,& U_1 = 0, & U_{i+1} = U_{i-1} - Q_iU_i, &                               & \\
								& V_0 = 0, & V_1 = 0,                    & V_{i+1} = V_{i-1} - Q_iU_i, & 1 \leqslant i \leqslant n,
		\end{array}$$
и легко проверить, что для  $2 \leqslant i < n~ deg(Q_i) > 0$ и:
		$$\begin{array}{cccc}
			deg(U_2) <  & deg(U_3) < ... < & deg(U_{n+1}),     & \\
							      & deg(V_2) <       & deg(V_3) < ... <  & deg(V_{n+1}),
		\end{array}$$
\noindent
что дает искомый результат, так как $U_n$ и $V_n$ ---  коэффициенты Безу\linebreak
и $U_{n+1} \sim B, V_{n+1} \sim A$.\\

\pagebreak
%===================================================================================================================
%291
\noindent\textbf{35. Дерево Штерна — Броко}\\

\textbf{a.}  Свойство $p'q - pq' = 1$ верно для всякой пары последовательных\linebreak
дробей, полученных с помощью описанной процедуры. Это доказывает\linebreak
взаимную простоту числителей и знаменателей полученных дробей и\linebreak
что $\frac{p}{q} < \frac{p + p'}{q + q'} < \frac{p'}{q'}$.\\

\hspace*{0pt}Если $\frac{p}{q} < \frac{b}{c} < \frac{p'}{q'}$, то из неравенств $bq - cp \geqslant 1$ и $cp' - bq' \geqslant 1$ получаем,\linebreak
что
$$q + q' \leqslant q(cp' - bq') + q'(bq - cp) = c.$$\\
\hspace*{15pt}\textbf{b.} Рассматривая во время процесса деления дробь $\frac{a}{b}$, можно заме­-\linebreak
тить, что не может быть получена дробь со знаменателем, большим,\linebreak
чем $b$,  легко вывести алгоритм 8, в котором используются первообраз-\linebreak
ные уравнения, введенные в упражнении 30 главы I.\\

  %-------------------------------
\begin{lstlisting}[mathescape = true, caption = {Прохождение ряда Фарея}]
	float List(float n){
		float Auxiliaire, Resultat;
			if (n == 1){
				return{$[\frac{0}{1}, \frac{1}{1}]$};
				}
			else{
				Auxiliaire $\leftarrow$ Farey(n-1); Resultat $\leftarrow$ NULL;
				x $\leftarrow$ Head(Auxiliaire); Auxiliaire $\leftarrow$ Tail{Auxiliaire};
					while(Auxiliarie != null){
						Resultal $\leftarrow$ Construct(z, Resultat);
						у $\leftarrow$ Head(Auxiliaire); Auxiliaire $\leftarrow$ Tail(Auxiliaire);
						z $\leftarrow$ x $\oplus$ y;
						if (Denominateur(z) <= n){
							Resultat $\leftarrow$ Construct(z, Resultat);}
						x $\leftarrow$ y;}
						Resultal $\leftarrow$ Construct(x, Resultat);
					return Reverse(Resultat);}}}
			\end{lstlisting}
	%------------------------------
	\ \newline
\hspace*{15pt}Список-результат построен в обратном порядке (что не очень важно­\linebreak
само по себе). Перед окончанием выполнения его порядок еще раз\linebreak 
меняют. В этом алгоритме знак $\oplus$  обозначает странную операцию, ко-\linebreak
торую применяют к дробям в процессе Штерна — Броко.

\pagebreak
%======================================================================================================================
%292

\textbf{c.} Если две последовательные дроби в $\mathcal{F}_N$ не удовлетворяют соот-­\linebreak
ношению $q + q' > N$, то медиана принадлежит $\mathcal{F}_N$ и должна находиться\linebreak
между двумя простыми, что является противоречием

Начиная с соотношения $p'q - pq' = 1$ и $p''q' - p'q'' = 1$,  получаем
		$$\begin{array}{cccc}
				p'(p''q - pq'') = p + p' & \quad\text{и}\quad & q'(p{''}q - pq{''}) = q + q'',\\
		\end{array}$$
что доказывает первое из свойств. Отправляясь от тех же соотношений,\linebreak
можно доказать, что 
$$p''q - pq'' = \frac{q + q''}{q'} = \frac{q + N}{q'} - \frac{N - q''}{q'}.$$
\noindent Согласно свойству знаменателей, доказанному выше, последний терм\linebreak
меньше 1 и, следовательно, второе свойство доказано.

Для формул, позволяющих выразить член ряда Фарея с помощью\linebreak
двух предшествующих дробей, достаточно проверить, что два выра-\linebreak
жения для $p''$ и $q''$  взаимно просты и что их отношение эффективно\linebreak
дает $a''$. В этом случае запись алгоритма очевидна.\\
\\
\hspace*{15pt}\textbf{d.} Предположим, что $p$ и $q$ отличны от нуля и единицы и что $p < q$.\linebreak
Свойство, сформулированное в первой задаче, позволяет доказать, что
обе родительские дроби для $\frac{p}{q}$ дают коэффициенты Безу для $p$ и $q$.
Кроме того, свойства
		\begin{itemize}
				\item всякая дробь $\frac{p}{q}$ на — Броко с помощью\linebreak
двух дробей, расположенных на той же ветви дерева,
				\item на ветви дерева, начиная с уровня 2, знаменатели строго возра­\linebreak
стают
\end{itemize}
позволяют доказать, что одна из этих пар коэффициентов минимальна\\
\\
\noindent\textbf{36. Вычисление коэффициентов Безу n-ки целых чисел}\\
%-------------------------------
\begin{lstlisting}[mathescape = true, caption = {Коэффициенты Безу}]
$u_1 \leftarrow 1; d \leftarrow a_1$;
for(i = 0; i == 2; i++){
  //d = НОД($a_1,\ldots,a_{i-1}$) = $U_{i}a_{i}+\ldots+u_{i-1}a_{i-1} и j>i:U_j = 0$
  $(\alpha, \beta) \leftarrow$ Bezout$(d, a_i); d \leftarrow \alpha d + \beta a$;
  for(j = 0; j == (i-1)){
    $u_2 \leftarrow \alpha u_j;$}
  $u_i \leftarrow \beta;$
if (|d| == 1) break;
}
\end{lstlisting}
%--------------------------------
\newpage
%======================================================================================================================================================================
%293
Пусть требуется вычислить коэффициенты Безу n-ки целых чисел $\overline{a}$\linebreak
Алгоритм 9 находит эти коэффициенты и помещает их $\overline{u}$.\\
\\
\noindent\textbf{39. Сложность центрированного деления}\\
\\
\hspace*{15pt}\textbf{a.} Первые члены последовательности $(G_n)$ выглядят следующим\linebreak
образом: 0, 1, 2, 5, 12, 29, 70, 169, ...Имеется точно $n$ центрирован­\linebreak
ных делений в алгоритме Евклида, примененном к паре $(a, b)$, ибо:

		$$\begin{array}{ccccc}
				&    & a = 1 \times G_n = G_{n-1},        & |G_{n-1}| \leqslant |G_n|/2,\\
				&		 & G_n = 2 \times G_{n-2} + G_{n-2},  & |G_{n-2}| \leqslant |G_{n-1}|/2,\\
				&    & \vdots                             & \\
				&    & G_3 = 2 \times G_2 + G_1,          & |G_1| \leqslant |G_2|/2,\\
				&    & G_2 = 2 \times G_1 + 0.            & \\
		\end{array}$$\\

\textbf{b.} Запишем деления в виде $r_{i-1} = r_iq_i, + r_{i+1}$ с $|r_{i+1}| \leqslant |r_i|/2$ для\linebreak
$1 \leqslant i \leqslant n$ и $|r_1| < |r_0|$. Отсюда следует, что $|q_i| \geqslant 2$ для $2 \leqslant i \leqslant n.$\linebreak
Чтобы доказать неравенство $|r_{i-1}| \geqslant 2|r_i| + |r_{i+1}|$ ((для $2 \leqslant i \leqslant n$),\linebreak
рассмотрим два случая:
\begin{itemize}
	\item $|q_i| < 3 : |r_{i-1}| = |r_iq_i + r_{i+1}| \geqslant |q_i||r_i| - |r_{i+1}| \geqslant 3|r_i| - |r_{i+1}| \geqslant$\linebreak
$2|r_i| + |r_{i+1}|.$
	\item $|q_i| < 3$ лишь при  $q_i = \pm 2.$ Из того, что $|\pm 2r_i + r_{i + 1}| = |r_{i-1}| \geqslant 2|r_i|,$\linebreak
следует $|r_{i-1}| = |\pm 2r_i + r_{i+1}| = 2|r_i| + |r_{i+1}|$ (если выполняется $|u + v| \geqslant |u|$\linebreak
и $|v| \leqslant |u|,$ то$|u + v| = |u| + |v|).$
\end{itemize} 
\hspace*{15pt}Из неравенств $|r_{i-1}| \geqslant 2|r_i| + |r_{i+1}|$ выводим:

		$$\begin{array}{ccccc}
				|r_n| \geqslant 1 = G_1, |r_{n-1}| \geqslant 2 = G_2,\\
				|r_{n-2}| \geqslant 2|r_{n-1}| + |r_n| \geqslant 2G_2 + G_1 = G_a,\\
				\ldots                                                                                        & \\
				|r_1| \geqslant 2|r_2|+ |r_3| \geqslant 2G_{n-2} = G_n, |r_0| \geqslant |r_1| + |r_2| \geqslant G_n + G_{n-1}.\\
		\end{array}$$
Оставшаяся часть упражнения трудностей не представляет.\\
\\
\noindent\textbf{40. Порождения простых чисел с помощью многочленов}\\
\\
\hspace*{15pt}\textbf{a.} Предположим, что существует такой многочлен. Нетрудно пока­\linebreak
зать, что $P(n)$ делит $P(n + kP(n))$ для любого целого $k$ (достаточно\linebreak
расписать $P(n + kP(n))$). Так как $P$ не постоянен и принимает только\linebreak
простые значения, то приведенное свойство означает, что для всяко­\linebreak
го $k \in Z$, $P(n) = \pm P(n + kP(n)).$  Следовательно, один из многочленов\linebreak

\pagebreak
%===============================================================================================================================================================================================
%294
\noindent$P - P(n)$ или $P + P(n)$ имеет бесконечное множество корней, что про­\linebreak
тиворечит тому, что Р не постоянен.

Замечание: натуральное число $n$ называется эйлеровым, если много­\linebreak
член $X^{2} + X + n$ принимает простые значения для всякого целого числа\linebreak 
из интервала $[0, n - 1]$. Доказано, что существует только шесть (уже\linebreak
известных) чисел Эйлера: 2, 3, 5, 11, 17, 41 (см. Ле Лионне [113]).\\
\\
\hspace*{15pt}\textbf{b.} В действительности мы докажем, что это неравенство почти\linebreak
всегда точное. Согласно постулату Бертрана $p_{n+i} < 2p_n$,  и если име­\linebreak
ется точное неравенство для $p_n$ то оно имеется и для $p_{n+1}$. Доста­\linebreak
точно теперь заметить, что точное неравенство будет при $n = 3$ (так\linebreak
как $7 < 5 + 3 + 2$).\\
\\
\hspace*{15pt}\textbf{c.} Постулат Бертрана позволяет утверждать, что n-е простое число\linebreak
не меньше $2^{n}$ и, следовательно, $10^{n}$. Тогда, если через $K$ обозначим\linebreak
сумму ряда $\sum_{k\geqslant1}10^{-k^{2}}$, то это и будет та постоянная, которую мы\linebreak
ищем. Действительно,
		$$\begin{array}{ccc}
				10^{n^{2}}K = 10^{n^{2}-1}p_1 + 10^{n^{2}-4}p_2 +\ldots+10^{2n-1}p_{n-1} + p_n + 10^{-2n-1}p_{n+1}\\
				+ 10^{-4n-4}p_{n+2} +\ldots\equiv p_n + x ~(mod~ 10^{2n-1}), где x\in ]0,1[.\\
		\end{array}$$
		
Так как $p_n < 10^{n}$, то утверждение верно.\\
\hspace*{0pt}Однако, с практической точки зрения, приведенная выше формула\linebreak
числа $K$ неэффективна: чтобы посчитать по ней число $K$ , необходимо\linebreak
прежде всего знать последовательность простых чисел, а затем еще и\linebreak
просуммировать ряд $\ldots$ приблизительное значение этой суммы 0,2003.\linebreak
Вообще, приемлемая формула, позволяющая находить простые произ­\linebreak
вольно большие числа, не известна.\\
\\
\noindent\textbf{41. Соотношение Безу для многочленов}\\
\\
\hspace*{15pt}Если $P$ и $Q$ взаимно простые многочлены из $A[X]$, то они принад­-\linebreak
лежат и $K[X]$. Из равенства Безу $UP + VQ = 1$, освобождаясь от зна­-\linebreak
менателей, появляющихся в $U$ и $V$, получаем соотношение, требуемое в\linebreak
упражнении. Расширенный алгоритм Евклида показывает, что можно\linebreak
выбрать многочлены $U$ и $V$, удовлетворяющие условиям, наложенным\linebreak
на степени (см. упражнение 34).\\
\\
\textbf{42. Сложность алгоритма Евклида в} $\textit{K}[X]$\\
\\
\hspace*{15pt}\textbf{a.} Пусть $(R_i)_{0 \leqslant i \leqslant n+1}$ --- последовательность остатков в алгоритме\linebreak
Евклида, примененном к паре $(\textit{A}, \textit{B})$, с  $R_0 = \textit{A}, R_1 = \textit{B}, \textit{R}_{n+1} = 0$ и\linebreak 
$R_n \neq 0$; тогда $0 \leqslant deg R_n < deg R_{n-1} < \ldots < deg R_1$ и, следовательно,\linebreak
$n - 1 \leqslant deg(R_1)$.

\pagebreak
%=======================================================================================================================================================================================================
%295
\textbf{b.} Подходящей является последовательность $F_0 = 1, F_1 = X + 1$ и\linebreak
$ F_{n+2} = XF_{n+1} + F_n.$\\
\\
\noindent\textbf{43. Оптимальность алгоритма для $K[X]$ (теорема Лазара)}\\
\\
\hspace*{15pt}\textbf{a.} Ищем $\mu \leqslant \varphi$. Запишем оптимальное деление в виде $A = B \times Q + R$\linebreak
и евклидово деление $A = B \times Q' + R'$ и $deg(R') < deg(B)$. Это дает\linebreak
$\mu(A, B) = \mu(B, R) + 1$ и $\varphi(A, B) = \varphi(B, R') + 1.$
		\begin{enumerate}
				\item Предположим, что $deg(R) > deg(B).$ Тогда из двух евклидовых\linebreak
делений $B = 0 \times R + B$ и $R = (Q' - Q) \times B + R'$ выводим:\\
\\
				$\begin{array}{lccccl}
							\mu(\textit{A, B}) & = & \mu(\textit{B, R}) + 1  & = & \varphi(B, R) + 1 & (\text{\footnotesize определение, затем рекуррентность}).\\
																	 & = & \varphi(\textit{R, B}) + 2 & = & \varphi(B, R) + 3 & (\text{\footnotesize 1-е, затем 2-е евклидово деление}).\\
																	 & = & \varphi(\textit{A, B}) + 2 &   &                  & (\text{\footnotesize по определению}).\\
				\end{array}$\\
\\
\noindent что противоречит $\mu(A, B) < \varphi(A, B)$.
				\item Предположим, что $deg(R) = deg(B)$. Тогда $deg(R') < deg(R)$, от­-\linebreak
куда $deg(R' - R) = deg(R)$ и равенство:
		\end{enumerate}
          $$\begin{array}{ccr}
B(Q - Q') = R' - R, & deg(R' - R) = deg(R) = deg(B), & (13)\\
					\end{array}$$
приводит к тому, что $Q - Q'$ постоянный (не нулевой). Если $k$ обозна-\linebreak
чает  $(1/Q - Q')$, то имеем евклидово деление $B$ на $R$: $B = -kR + kR'$ с 
$deg(kR') < deg(R)$,  откуда:\\
\\
			$\begin{array}{cccccl}
\mu(A, B) & = & \mu(B, R) + 1       & = & \varphi(B, R) + 1        & (\text{\footnotesize определение,}\\
            &   &                       &   &                         & \text{\footnotesize затем рекуррентность}).\\
            & = & \varphi(R, kR') + 2 & = & \varphi(-kR, R') + 2  & (\text{\footnotesize евклидово деление} \text{(13)},\\
						&		&			                  &   &                         & \text{\footnotesize затем вопрос а}).\\
						& = & \varphi(B, R') + 2  & = & \varphi(A, B) + 1        &(B \equiv - kR~(mod~R')\\
						&   &                       &   &                         & \text{\footnotesize затем определение,}).\\
				\end{array}$\\
\\
что противоречит $\mu(A, B) \leqslant \varphi(A, B)$. Поэтому единственно возможный\linebreak
случай $deg(R) < deg(B)$,  и, согласно единственности обычного частного\linebreak
Евклида, $R  = R'\ldots$\\
\\
\noindent\textbf{44. К теореме Лазара}\\
\\
\hspace*{15pt}Последовательность $(C_n)_{n \geqslant i}$  строго возрастающая и, следовательно,\linebreak
состоит из натуральных чисел; ее первые члены 0, 1, 3, 5, 18, 31, 111,\linebreak
191... Нетрудно проверить неравенства:\\
			$$\begin{array}{cc}
						C_{2i-1} < (1 - \alpha)C_{2i} < 0,5 \times C_{2i} & 0,5 \times C_{2i-1} < C_(2i-2) < \alpha C_{2i-1}
			\end{array}$$
\pagebreak

%==========================================================================================================================================================================================
%296
\noindent константа а введена в теореме Лазара ($\alpha \approx 0,6180339$ --- обратное к\linebreak
золотому сечению).\\
\hspace*{0pt}Предположим, что первый элемент при центрированном делении\linebreak
имеет четный индекс, скажем $C_{2n}$  Ниже приведена последовательность\linebreak
центрированных делений $C_{2n}$  на $C_{2n-1}$ (записанных в виде $a = qb + r$):
			$$\begin{array}{rcl}
						C_{2n}              & = & 4C_{2n-1} + (C_{2n-2} - C_{2n-1}),\\
						C_{2n-1}            & = & (-2)(C_{2n-2} - C_{2n-1}) + C_(2n-3),\\
						C_{2n-2} - C_{2n-1} & = & (-3)C_{2n-3} + (C_{2n-3} - C_{2n-4}),\\
						C_{2n-3} - C_{2n-4} & = & 3C_{2n-5} + (C_{2n-6} - C_{2n-5}),\\
						C_{2n-5}            & = & (-2)(C_{2n-6} - C_{2n-5}) + C_(2n-7),\\
																	& \vdots & \\
						C_3                 & = & (-2)(-2) + 1 \text{ или } C_3 = 2 \cdot 3 + (- 1),\\
		        \pm 2        				& = & (\pm 2)(\pm 1) + 0\\
			\end{array}$$
(имеется точно $2n - 1$ делений). Можно задать алгоритм Евклида, тре-\linebreak
бующий $2n - 1$ делений, из которых $n - 1$ нецентрированных (записанных\linebreak
в виде $a = qb + r$):
$$\begin{array}{rccl}
C_{2n}    & = & 3C_{2n-1} + C_{2n-2}       & (\text{\small нецентрированное}),\\
C_{2n-1}  & = & 2C_{2n-2} + (-C_{2n-3})    & (\text{\small центрированное}),\\
C_{2n-2}  & = & (-3)(-C_{2n-3}) + C_{2n-4} & (\text{\small нецентрированное}),\\
-C_{2n-4} & = & (-2)C_{2n-4} + C_{2n-5}    & (\text{\small центрированное}),\\
C_{2n-4}  & = & 3C_{2n-5} + C_{2n-6}       & (\text{\small нецентрированное}),\\
            & \vdots & \\
C_{2}     & = & (\pm 3)(\pm C_{1}) + C_{0} & C_0 = 0 (\text{\small центрированное}, \\
            &   &                              & \text{\small остальное нулевое}).\\
\end{array}$$

Используя теорему Лазара, легко проверить, что первая или вторая\linebreak
последовательности оптимальны по длине.\\
\\
\noindent\textbf{45. (Неэффективное) решето Эратосфена для многочленов}\\

\begin{wrapfigure}{r}{0.4\textwidth}
\begin{lstlisting}[mathescape=true, frame=l]
for(i = 2; i == n; i++){
  if ($Cribble(i)$){
    for(j = i; n == (n/i)); j++){
      $Cribble$($i \ast j$) $\leftarrow False$;
    }
  }
}
\end{lstlisting}
\end{wrapfigure}
\ \newline
\hspace*{15pt}\textbf{а.} Пусть $Crible(2 .. n$) — последова­-\linebreak
тельность булевых переменных со зна­-\linebreak
чениями «истинно». Решето Эратосфена\linebreak
останавливается на очередном значении\linebreak
$i с CriWe(i)$ истинным и вычеркивает все\linebreak
кратные $i * j$ числа $i (Crible(i * j) \leftarrow$\linebreak
$False)$, а затем повторяет эту процедуру\linebreak
для следующего $i$. Таким образом, простые числа от 2 до $n$ в точности те $i$, для которых $Crible(i)$ истинно.

%-------------------------------------
\pagebreak
%==============================================================================================================================================================================================
%297
\noindent
\textbf{b.} Пусть $A$ — множество унитарных непостоянных многочленов над\linebreak
$\mathbb{Z}/p\mathbb{Z}$ и В — объединение  $\bigcup^{\infty}_{k=1} \times [0, p^{k}]$ Построим биективное ото­\linebreak
бражение А на В следующим образом: если $P(X) = X^{k} + a_{k-1}X^{k-1} +\linebreak
\ldots +a_1X + a_0 $---  многочлен из $A(k \geqslant 1)$,  то поставим ему в соответствие\linebreak
пару  $(k,a_{k-1}p^{k-1}+\ldots+a_{1}p + a_0)$ из $B$.
  %-----------------------------------------
\begin{lstlisting}[mathescape = true, caption = {Решето Эратосфена}]
$for(i= 0; i == B_n-1; i++)${
  $if (Crible(i))${
    $for(j = i; j == B_n - 1)${
      $Crible(\sigma^{-1}(\sigma(i) \ast \sigma(j))) \leftarrow False$;
    }
    $exeption if(Overflow $\Rightarrow$ null);$
  }
}
\end{lstlisting}
	%-----------------------------------------

Отобразим теперь $B$ на $\mathbb{N}$, сопоставляя паре ${1} \times [0, p]$ отрезок $[0, p]$,\linebreak
паре ${2} \times [0, p^{2}]$  отрезок $[p, p + p^{2}]$  и так далее $B_n = p + p^{2} + \ldots +p^{n}$,\linebreak
то построим следующую биекцию $B$ на $\mathbb{N} = [B_0, B_1]\cup [B_1, B_2] \cup [B_2, \linebreak
B_3] \cup \ldots$. Очевидно, $B_n$ --- число унитарных непостоянных многочленов\linebreak
степени $ \leqslant n$. Пусть $n$ — целое число, $\sigma$ --- взаимнооднозначное отобра­\linebreak
жение интервала $[0, B_n]$ на множество унитарных непостоянных мно­\linebreak
гочленов степени $\leqslant n$. Теперь получаем алгоритм 10, в котором $Q * R$,\linebreak
обозначает произведение многочленов Q и R (при помощи Overflow этот \linebreak
алгоритм игнорирует исключения, при которых $deg(Q) + deg(R) > n$).\linebreak
По окончании работы этого алгоритма определяются все неприводи­
мые многочлены степени  $\leqslant n$: это в точности те многочлены $\sigma(i)$, для\linebreak
которых Crible(i) истинно. Разумеется, этот алгоритм неэффективен \linebreak
по времени и объему! Кроме того, он позволяет получить список не­\linebreak
приводимых многочленов по модулю 2 степени  $\leqslant 10$ или по модулю 3\linebreak
степени $\leqslant 8$.\\

\hspace*{15pt}\textbf{d}. Ниже приведен краткий список результатов, которые должна\linebreak
давать программа, реализующая предшествующий алгоритм:
			$$\begin{array}{llccl}
						\text{степень 2} : & X^{2} + X + 1,                     &    & \\
						\text{степень 3} : & X^{3} + X + 1,                     &    & X^{3} + X^{2} + 1,\\
						\text{степень 4} : & X^{4} + X + 1,                     &    & X^{4} + X^{3} + 1,\\
												& X^{4} + X^{3} + X^{2} + X + 1,     &    & \\ 
			\end{array}$$
\pagebreak
%=============================================================================================================================================================
%298XX
			$$\begin{array}{llccl}
						\text{степень 5} : & X^{5} + X^{2} + 1,                 &    & X^{5} + X^{3} + 1,\\
												& X^{5} + X^{3} + X^{2} + X + 1,     &    & X^{5} + X^{4} + X^{2} + X + 1,\\
												& X^{5} + X^{4} + X^{3} + X + 1,     &    & X^{5} + X^{5} + X^{4} + X^{2} + 1,\\
						\text{степень 6} : & X^{6} + X + 1,                     &    & X^{6} + X^{3} + 1,\\
												& X^{6} + X^{4} + X^{2} + X + 1,     &    & \\
												& X^{6} + X^{4} + X^{3} + X + 1,     &    & X^{6} + X^{5} + 1,\\
												& X^{6} + X^{5} + X^{2} + X + 1,     &    & \\
												& X^{6} + X^{5} + X^{3} + X^{2} + 1, &    & X^{6} + X^{5} + X^{4} + X + 1,\\
												& X^{6} + X^{5} + X^{5} + X^{2} + 1. &    & \\
			\end{array}$$
\\
\noindent \textbf{46. Многочлены $X^{n} - 1$}\\
\\
\hspace*{15pt}\textbf{a.} Равенства\\
$X^{un+vm}-1 = \begin{cases}
						(X^{un} - 1) - X^{un+um}(X^{-vm - 1}),&  если  u \geqslant 0, v \leqslant 0,\\
						(X^{un} - 1)X^{vm} + (X^{um} - 1),&     если  y \geqslant 0, v \geqslant 0,
			\end{cases}$\\
доказывают, что $X^{d} - 1$ --- линейная комбинация $X^{n} - 1$ и $X^{m} - 1$.\linebreak
Для доказательства первой части упражнения заметим, что $d$ положи­\linebreak
тельно и меньше $n$ и $m$, а поэтому $d = un + vm$ с, например,  $u \geqslant 0$.\linebreak
$v \leqslant 0$  Наконец, $d$ делит $n$ и $m$, а потому $X^{d} - 1$ делит $X^{n} - 1$ и $X^{m} - 1$.\linebreak
Чтобы перейти к произвольному кольцу, достаточно повторить для не­\linebreak
го предыдущие рассуждения.\\

\textbf{b.} Конечно, нет: рассмотреть $f(X) = X$ и $g(X) = X + 2$.\\

\textbf{c.} $X^{n} - 1 mod X^{m} - 1 = X^{n mod m} - 1$.\\
\\
\noindent\textbf{47. Неприводимые многочлены над} $\mathbb {Z}$\\

Для $P_k$ осуществим замену переменных $X \rightarrow X + 1$ и получим\linebreak
$P_k(X + 1) = 2 + \sum^{2k}_{i=1}(^{2^{k}}_i) $ Теперь можно применить критерий Эйзен­\linebreak
штейна. Заметим, что $P_k$ это циклотомический многочлен $\Psi_{2^{k+1}}(X)$;\linebreak
в самом деле, для корня $\alpha$ многочлена $P_k$ проверяется, что $\alpha^{2^{k}} = -1$,\linebreak
$\alpha^{2^{k} + 1} = 1$. . Значения $P_k(2)$ ---  это числа Ферма, числа, из которых толь­\linebreak
ко 5 считаются простыми.\\
\hspace*{15pt}Для $Q_n$  можно также применить критерий Эйзенштейна или дока­
зать их неприводимость непосредственно (что сводится к изменению
доказательства критерия в данном конкретном случае). Предположим,
что
$$Q_n = (a_{p}X^{p} +\ldots+a_0)(b_{r}X^{r}+\ldots+ b_0).$$

Тогда $a_0$ или $b_0$ равно 2; допустим, что это $a_0$ Теперь без труда доказы­
вается (используя то, что разложение нетривиально), что все $a_i$, четные,\linebreak

\pagebreak
%===============================================================================================================================================================
%299
\noindent а это противоречит тому, что коэффициент при старшем члене много­\linebreak
члена $Q_n$ есть 1. Замечание, аналогичное предыдущему: многочлены\linebreak
$X_n — 1$ не являются неприводимыми, хотя самые большие известные\linebreak
простые числа — числа вида $2^{n} — 1$ (числа Мерсенна).\\
\\
\noindent\textbf{48. Оценка числа неприводимых множителей}\\
\\
\hspace*{15pt}\textbf{a.} Запишем $X^{n} - a = (b_{q}X^{q}+\ldots+ b_0)(c_{r}X^r+\ldots+c_0) с q > 0,$\linebreak
$r > 0$ Нетрудно показать, что $p$ делит все $b_i$ или все $c_i$, а это противо­\linebreak
речит тому, что коэффициент при старшем члене у произведения равен\linebreak
1 (можно также применить критерий Эйзенштейна).\\
\hspace*{15pt}\textbf{b.} Пусть $(Q = b_{q}X^{q}+\ldots+b_0 c q< n)$ --- делитель $P$. $A/I[X]$ ---\linebreak
область целостности (так как $I$ --- простой) и в нем
$$a_{n}X^{n} = (b_{q}x^{q}+\ldots+b_0)(c_{r}X^{r}+\ldots+c_{k}X^{k}),$$
c $k < n$ и $c_k \neq 0.$ Однако из $a_k = 0 = b_{0}c_k$ следует, что $b_0 = 0 $ т.е. в $A$\linebreak
$b_0 \in I.$\\
\hspace*{15pt}\textbf{b.}Применим результаты предыдущего пункта с $I = (p):$ всякий\linebreak
делитель $P$ имеет свободный член, кратный $p$, и, следовательно, если\linebreak 
r --- число неприводимых сомножителей $P,$ то $p^r | a_0$\\
\\
\indent\textbf{49. Неприводимый многочлен, приводимый по модулю
всякого простого числа}\\
\\
\hspace*{15pt}\textbf{a.} Для $p = 2$ можно проверить, что $X^{4} + 1 = (X + 1)^{4}$ в $\mathbb{Z}/2\mathbb{Z}[X]$\\
\hspace*{15pt}\textbf{b.} Если $p \equiv ~(mod~ 4)$ то —1 есть квадрат по модулю $p$ и
$$X^{4} + 1 = (X^{2} + \sqrt{-1})(X^{2} - \sqrt{-1}),\quad\text{в }\mathbb{Z}/p\mathbb{Z}[X]$$\\
\\
$\sqrt{-1}$ обозначает квадратный корень из —1 по модулю $p$. Также, ес­\linebreak
ли 2 --- квадрат (квадратичный вычет) по модулю $p$, то $X^{4} + 1 = $\linebreak
$(X^{2} + \sqrt{2}X + 1)(X^{2} - \sqrt{2}X + 1)$и, конечно, если —2 квадратичный вычет\linebreak
по модулю $p$, то $X^{4} + 1 = (X^{2} + \sqrt{2}X - 1)(X^{2} - \sqrt{2}X - 1)$ Однако\linebreak
одно из трех чисел —1,2, —2, наверняка, квадрат по модулю $p$: действи­\linebreak
тельно, если —1 и 2 не-квадраты, то таковым является $—2 = —1 \times 2.$\linebreak
Которое из этих чисел квадрат, можно узнать с помощью следующих\linebreak
критериев:
$$-1\text{ квадрат }\Leftrightarrow p \equiv 1 (4), 2\text{ квадрат } \Leftrightarrow p \equiv \pm 1 (8),$$
$$-2\text{ квадрат }\Leftrightarrow p \equiv \pm 1 (4)$$

\pagebreak
%===============================================================================================================================================================================================================
%300
\noindent Приводим несколько примеров:
$$X^{4} + 1 = (X^{2} - 2 )(X^{2} + 2) ~(mod~ 5),$$
$$X^{4} + 1 = (X^{2} + 5X + 1 )(X^{2} - 5X + 1) ~(mod~ 23),$$
$$X^{4} + 1 = (X^{2} + 3X + 1 )(X^{2} - 3X + 1) ~(mod~ 11),$$
\\
\hspace*{15pt}\textbf{d.} Сравнение $-1 \equiv 7^{2}$ ~(mod~ 25) дает разложение:
$$X^{4} + 1 = (X^{2} +7) \times (X^{2} - 7) ~(mod~ 25).$$
Если рассмотреть разложение $X^{4} + 1$ по модулю 15 в произведение двух\linebreak
унитарных непостоянных многочленов $X^{4} + 1 = P(X)Q(X) ~(mod~ 15),$\linebreak
то легко проверить, что
$$X^{4} + 1 = (X^{2} + 2 )(X^{2} - 2) в Z/5Z[X],$$
$$X^{4} + 1 = (X^{2} + X - 1 )(X^{2} - X - 1) в Z/3Z[X],$$
последние --- разложения на неприводимые элементы, а значит, имеем: 
$$P(X) \equiv X^{2} + 2 ~(mod~ 5), Q(X) \equiv X^{2} - 2 ~(mod~ 5)$$
и
$$P(X) \equiv X^{2} + X + 1 ~(mod~ 3), Q(X) \equiv X^{2} - X + 1 ~(mod~ 3)   (14)$$
или
$$P(X) \equiv X^{2} - X + 1 ~(mod~ 3), Q(X) \equiv X^{2} + X + 1 ~(mod~ 3)   (15)$$
Между тем, многочлены $P_{0}(X) = X^{2} + 10X + 7   Q_{0}(X) = X^{2} + 5X + 13$\linebreak
удовлетворяют (14), откуда в силу единственности $P(X) \equiv P_{0}(X)$\linebreak
$~(mod~ 15)$ и $Q(X) \equiv Q_0(X) ~(mod~ 15)$ Аналогично, многочлены $P_{1(X)} =$\linebreak
$X^{2} + 5X + 7$ и $Q_{1}(X) = X^{2} + 10X + 13$ удовлетворяют (15) и, следователь­\linebreak
но,$P(X) \equiv P_{1}(X) ~(mod~ 15)$ $Q(X) \equiv Q_{1}(X) ~(mod~ 15)$ Однако, можно\linebreak
проверить, что ни $P_0Q_0$ ни $P_1Q_1$  не равны $X^{4} + 1$ по модулю 15\ldots\\
\\
\noindent\textbf{50. Нули многочленов с целыми коэффициентами}\\
\\
\hspace*{15pt}\textbf{a.}  Запишем $0 = a_0+\sum^{n-1}_{i=1}a_{i}(p/q)^{i} + (p/q)^{n}$. Освобождаясь от знаме­
нателя, получаем $0 = p^{n}+\sum a_{i}p^{i}q^{n-i}+a_{0}q^{n}.$ Отсюда получаем сначала, 
что $q$ делит $p$, что приводит к $q = 11,$ а потом, что $p$ делит $a_0$.
\hspace*{15pt}\textbf{b.} Как и ранее, приходим к выражению $0 = a_{n}p^{n} + \sum^{n-1}_{i=1}a{i}p^{i}q^{n-1}+$\linebreak
$a_{0}q^n$ ") что доказывает ($p$ и $q$ взаимно просты), что $p$ $| a_0 и q | a_n$. Это по­\linebreak
зволяет “находить” все рациональные корни многочлена, так как число\linebreak
делителей $a_0 и a_n $конечно. При необходимости это позволяет частично\linebreak
обосновать примитивность $P$.\\
\newpage
%======================================================================================================================================================================================================================
%301 
\noindent\textbf{51. Расширенный критерий Эйзенштейна}\\
\\
\hspace*{15pt}\textbf{a.} a. Рассмотрим разложение $P$ в произведение неприводимых много­\linebreak
членов. Так как $p | а_0 и p^{2} |а0$, то он делит свободный член лишь одного\linebreak
из неприводимых факторов, скажем  $Q = \sum b_{i}X^{i}. B A/2(p)[X]$ имеем
$$a_{n}X^{n}+\ldots+a_{k}X^{k} = (b_{q}X{q}+\ldots+b_0)(c_{r}X^{r}+\ldots+c_0),$$
\\
\noindent и, по только что доказанному, $b_0 = 0, с_0 \neq 0.$ Раскрывая скобки и\linebreak
сравнивая, без труда получим, что все коэффициенты $Q$ нулевые, за ис­\linebreak
ключением коэффициента при $k$-й степени $X$ . Следовательно, степень\linebreak
$Q$ не меньше $k$. Следует также отметить, что свободный член рассмо­\linebreak
тренного неприводимого многочлена делится на $p$\\
\hspace*{0pt} Когда $k$ равно $n$, степени многочлена $P$, получаем критерий Эйзен­\linebreak
штейна\\
\hspace*{15pt}\textbf{b.} Если $P$ разложим, то он может быть записан в одной из форм\linebreak
(так как ни 5, ни —5 не являются его корнями): $P = X^{4} + 3X^{3} + 3X^{2} -$\linebreak
$5 = (X^{2} + aX \pm 5)(X^{2} + bx \mp 1)$ Это приводит к системе уравнений \linebreak
$а + b = 3$ и $а - 5b = 0$, которая не имеет целых решений. Следовательно,\linebreak
$P$ неприводим.\\
\\
\hspace*{15pt}Что касается многочлена $Q$, то в зависимости от четности а воз­
можны два случая. Если $a$ четно, то сразу можно применить критерий
Эйзенштейна и $Q$ неприводим.
\hspace*{0pt} Если а нечетно и $Q$ разложим, то один из его неприводимых множи­\linebreak
телей имеет степень, не меньшую 2, и его свободный член делится на 2.\linebreak
Если $Q$ имеет целый корень, то, следовательно, это может быть только 1\linebreak
или —1; действительно, при $а = —3$ имеем $Q = (X - 1)(5X^3 - X^2 + 2X - 2),$\linebreak
а при $а = 17$ $Q = (b^X{2} + cX \pm)(dX^{2} + eX^{2} \pm 1)$ с четными $c$ и $e$ , откуда\linebreak
достаточно быстро приходим к противоречию.\\
\\
\noindent\textbf{52. Неприводимость циклотомических многочленов}\\
\\
\hspace*{15pt}\textbf{a.} Запишем $X^{n} - 1 = PQ $ и предположим, что $P(\alpha^p) \neq 0$ Тогда\linebreak
$Q(\alpha^{p} = 0$, и, следовательно, многочлен $Q(X^{P})$ имеет корень $\alpha$. Тогда\linebreak
$\alpha$ ---  корень НОД$(Р, Q(X^{P}))$. Но $P$ неприводим, а значит $P | Q(X^{P})$.\\

Перейдем к $Z/pZ[X]$. В этом кольце имеем формальное тождество\linebreak
$Q(X^{P}) = Q(X)^{P}$ . Если $\pi$ --- неприводимый делитель $P$, то $\pi$ также не­\linebreak
приводимый делитель $Q$, а значит, $\pi^{2}$ делит $X^{n} — 1 в Z/pZ[X]$, что невоз­\linebreak
можно, так как $X^{n} — 1$ не имеет кратных множителей даже в $Z/pZ[X]$\\
\newpage
%=====================================================================================================================================================================================
%302
\noindent (рассмотреть для доказательства производную $X^{n} — 1$ и учесть, \linebreak
что $p|n$). Полученное противоречие показывает, что $\alpha^{p}$ корень $P$.\\
\hspace*{15pt}\textbf{b.} Пусть $P$ — неприводимый множитель $\varphi_n$ Тогда он обращается\linebreak
в нуль на некотором примитивном корне из 1 степени $n$, а значит, и на\linebreak
всех остальных тоже. Действительно, пусть $\alpha$ — примитивный корень,\linebreak
тогда все остальные примитивные корни имеют вид $\alpha^{k}$ $c НОД(k, n) = 1$.\linebreak
Следовательно, $P = \text{Ф}_n$\newline
\\
\noindent\textbf{53. Идеалы в $K[[X]]$}\\
\\
\hspace*{15pt}Всякий формальный ряд, свободный член которого ненулевой, обра­\linebreak
тим и, следовательно, порождает $K[[X]]$. Свободные идеалы $K[[X]]$ со­\linebreak
стоят из формальных рядов с нулевым свободным членом.\\
\hspace*{0pt} Пусть $I$ идеал в $K[[X]]$ и $f \in I$ многочлен с наименьшим среди\linebreak
многочленов $I$ показателем $n$ таким, что $f = X^{n}\sum a_{i}X{i} c a_0 \neq 0$. Тогда\linebreak
$f \in (X^{n})$ и, соответственно, $X^{n} \in I$. Следовательно, $I = (X^{n}).$ Итак,\linebreak
идеал из K[[X]] исчерпываются идеалами ${0}, (X), (X^{2}),\ldots, (X^{n} ), \ldots$ и\linebreak
K[[X]].\\
\\
\noindent\textbf{54. О факториальных кольцах}\\
\\
\hspace*{15pt}\textbf{a.} a. Отображение $\varphi$ котором идет речь, с одной стороны, биектив­\linebreak
но, а с другой стороны, является морфизмом (моноидов): $\varphi(\epsilon\epsilon',v + v') =$\linebreak
$\varphi(\epsilon,v)\varphi(\epsilon', v')$ В частности, любой элемент $a \in А — {0}$ единственным,\linebreak
образом записывается в виде $\epsilon(a) \prod_{p\in P}p^{v_{p}(a)}$ и $a | b$ эквивалентно тому,\linebreak
что $v_p(a) \leqslant v_{p}(b)$ для всякого $p \in P$.\\
\hspace*{0pt}Любая теория делимости в А может быть сведена к вычислени­\linebreak
ям в $\mathbb{N^{P}}$ Например, проверим, что $d = \prod_{p \in P} p^{min(v_{p}(a),v_{p}(b))}$ являет­\linebreak
ся НОД $(а, b)$. Действительно, если заданы $v, w \in \mathbb{N^{P}}$ то множество \linebreak
$U = min(v_p, w_p)_{p \in P}$ --- наибольшее со свойством $u_p \leqslant v_p и u_p \leqslant w_p$\linebreak
для любого $p \in6 P$ . Поэтому и множество главных идеалов нётерово \linebreak
по включению, так как $Аа \subset АЬ$ эквивалентно $v(а) \geqslant v(b), a \mathbb{N^{P}}$\linebreak
нётерово по отношению к упорядочению $\geqslant$ (отображение, которое ка­\linebreak
ждому $v \in \mathbb{N^{P}}$ ставит в соответствие $\sum_{p}v_p \in \mathbb{N}$ строго возрастает).\\
\hspace*{0pt} Отметим, что отображение $\varphi$ устанавливает изоморфизм группы\linebreak
$U(А) \times \mathbb{Z^{P}} на K*$ ($K$ — поле частных $A$).\\
\hspace*{15pt}\textbf{b.} Данные свойства делимости в факториальных кольцах\linebreak сводятся
к вычислениям в $\mathbb{N}$ Например, $cНОД(а, b) = НОД (ca,cb)$ сводится к\linebreak
проверке равенства $\gamma+min(\alpha, \beta) = min(\gamma+\alpha, \alpha+\beta),$ которое очевидно\linebreak
В кольцах главных идеалов эти свойства можно проверять, опираясь на\linebreak
соотношения Везу: например, если $Ad = Аа+АЬ$, то $Acd = Аса+АсЬ$\ldots

\pagebreak
%==========================================================================================================================================================================================
%303
\textbf{с.} Единственная нетривиальная импликация заключается в доказа­\linebreak
тельства того, что если два элемента имеют НОД, то любой неприво­\linebreak
димый элемент $p$ прост: если $a$ и $b$ такие, что $p | ab$, то нужно доказать,\linebreak
что $p$ делит $a$ или $Ь$. Рассмотрим для этого НОД $d$ элементов $ap$ и $ab$\linebreak
(НОД существует по предположению). Элементы $p$ и $a$ делят $ap$ и $ab$ и, \linebreak
следовательно, $p | d$ и $а | d$. Поэтому элемент $d/a$ определен и делит $p$\linebreak
(ибо $d | ap$). Так как $p$ неприводим, то $1 ~ d/a$, а тогда $d ~ а \Rightarrow=> р | а$\linebreak
или $p ~ d/a$ и в этом случае $ар ~ d \Rightarrow ар | ab \Rightarrow р | Ь.$\\
\\
\noindent\textbf{55. Условия, при которых факториальное кольцо является\linebreak
кольцом главных идеалов}\\
\\
\hspace*{15pt}\textbf{a.} Пусть $P$ --- простой идеал $A$. Так как нужно доказать, что $P$ глав­\linebreak
ный, то можно считать, что $P \neq {0}$ и $P \neq А$. Пусть\linebreak
$x \in Р — {0}, x$ необратим и записывается в виде произведения непри-\linebreak
водимых $x = p_i\ldots p_k$. Идеал $P$ простой, а поэтому содержит один из \linebreak
$p_{i}: (p_i) \subset Р .$ В силу максимальности идеала $(p_i,)$ получаем необходимое\linebreak
утверждение.\\
\hspace*{15pt}Докажем, $что Ax/d + Ay/d = А.$ Предположим, что $Ax/d + A y/d \subseteq А,$\linebreak
тогда идеал $Ax/d + Ay/d$ содержится в некотором максимальном идеале\linebreak
(теорема Крулля), а следовательно, в простом и главном, скажем $Ар.$\linebreak
Тогда элемент $p$ делит как $x/d,$ так и $y/d$, которые взаимно просты ---\linebreak
противоречие! Значит $Ax/d + Ay/d = A$ и $Ах + Ay = Ad$.\\
\hspace*{15pt}Теперь ясно, что всякий конечно порожденный идеал --- главный. Но\linebreak
так как $A$ факториально, то множество главных идеалов $A$ нётерово.\linebreak
Кольцо, в котором множество конечно порожденных идеалов нётерово\linebreak
(по включению), само нётерово (наличие неконечнопорожденного идеа­\linebreak
ла позволяет построить строго возрастающую последовательность ко­\linebreak
нечно порожденных идеалов). Так как всякий конечно порожденный\linebreak
идеал $A$ главный, то $A$ --- кольцо главных идеалов.\\
\hspace*{15pt}\textbf{b.} Пусть $x$ и $y$ --- два элемента $A$. Если $d = НОД(x,y)$, то $x/d$\linebreak
и $y/d$ взаимно просты, и следовательно, $1 \in Ax/d + Ay/d,$ откуда\linebreak
$Ах + Ay = Ad$. Отсюда следует искомое утверждение (действительно,\linebreak
если выполняется пункт ($Ь$), то выполняется и ($a$)).\\
\hspace*{15pt}\textbf{c.}Для доказательства рассмотрим следующие умножения $y \mapsto ху:$\linebreak
эти отображения инъективны при $x \neq 0$ и биективны в силу конечно­\linebreak
сти. Рассмотрим последний случай (самый сложный). Пусть $x \in R —{0},$\linebreak
по предположению $K[x] э у \mapsto ху \in К[x]$ инъективно, но это отобра­\linebreak
жение $K$-линейно и $K[х]$ — конечномерно (так как х алгебраичен над\linebreak
К), а, стало быть, сюръективно, и существует такой элемент $x'$, что\\
\newpage
%===================================================================================================================================================================================
%304
\noindent $xx' = 1 $ Аналогично, существует $x''$ с $x''x = 1,$ но известно, что в та­\linebreak
ком случае $x' — x''$ . Так как всякий элемент $x$ обраим в R, то R ---\linebreak
поле\\
\hspace*{15pt}\textbf{d.}  Можно считать, что А не является полем. Идеал $I\bigcap\mathbb{Z}$ — простой\linebreak
идеал в $\mathbb{Z}$, и осталось доказать, что он ненулевой. I содержит нену­\linebreak
левой элемент x, и x является корнем унитарного многочлена $\sum a_{i}X^{i}$\linebreak
с целыми коэффициентами, причем можно считать, что его степень\linebreak
не ниже $n$. Тогда $a_0 \neq 0$ (иначе после вынесения x получили бы уни­\linebreak
тарный многочлен с целыми коэффициентами степени $< n$), и поэтому\linebreak
$a_0 = - \sum^{n}_{i = 1}a_{i}x^{i} \in I \cup \mathbb{Z}$ Значит, идеал $p\mathbb{Z}$ ненулевой и имеет вид pZ.\linebreak
Инъективное отображение $\mathbb{Z} \rightarrow А$ определяет инъекцию $\mathbb{Z}/I\cup\mathbb{Z} \rightarrow A/I$\linebreak
и алгебра $A/I$ является алгеброй над полем $\mathbb{Z}/p\mathbb{Z}$, а так как она без\linebreak
делителей нуля и алгебраическая над $\mathbb{Z}/p\mathbb{Z}$, то является полем по пре­\linebreak
дыдущему пункту. Тогда I --- максимальный идеал.\\
\\
\noindent\textbf{56. Произведение упорядоченных нётеровых множеств}\\
\\
\hspace*{15pt}\textbf{a.} Достаточно это сделать для двух упорядоченных множеств, что\linebreak
не вызывает особых проблем.\\
\hspace*{15pt}\textbf{b.} Если $Е = {0,1}$, то упорядоченное множество $E^{\mathbb{N}}$ изоморфно\linebreak
множеству всех подмножеств $\mathbb{N}$ с отношением вложения (при помо­\linebreak
щи характеристической функции: $\chi_A, \chi_B$ тогда и только тогда, ко­\linebreak
гда $А \subset В$). Рассматриваемое подпространство S образовано конеч­\linebreak
ными подмножествами или конечными дополнениями и, очевидно, не\linebreak
нётерово: $0 \subseteq {0} \subseteq {0,1} \subseteq {0,1,2} \subseteq\ldots$ В общем случае для сравни­\linebreak
мых между собой $a, Ь \in E с a < Ь,$ например, имеется строго возраста­\linebreak
ющая последовательность в подпространстве S:

			$$\begin{array}{ccccc}
						\text{(а, а, а, а, а, а ,}\ldots) & < & \text{(а, b, а, а, а, а ,}\ldots) & < & \\
																				 & < & \text{(а, b, b, а, а, а ,}\ldots) < \text{a, b, b, b, a, a}\ldots) & < & \ldots \\
			\end{array}$$
которая доказывает, что S не нётерово, а $E^{\mathbb{N}}$ тем более.\\
\hspace*{15pt}\textbf{c.} Доказательство аналогично доказательству, касающемуся $A[X]$\linebreak
(стр. 233).\\
\hspace*{15pt}\textbf{d.} Если $Е_M$ — упорядоченное по включению множество идеалов из\linebreak
М , то, как показано в курсе, имеется строго возрастающее отображе­\linebreak
ние $Е_M в E_N \times E_{M/N}$ - Впрочем, отображение, которое идеалу I кольца\linebreak
$A[X]$ ставит в соответствие $dom_n(I)_{n \in N} \in E^{\mathbb{N}}_A$ , — строго возрастающее\linebreak
отображение со значениями в множестве возрастающих последователь­\linebreak
ностей идеалов.

\pagebreak
%================================================================================================================================================================================================================
%305
\noindent\textbf{57. Конечные (коммутативные) поля}\\
\\
\hspace*{15pt}\textbf{a.} Для доказательства первой части использовать то, что фактор-\linebreak
кольцами $\mathbb{Z}$ без делителей нуля являются лишь $\mathbb{Z}/p\mathbb{Z}$, где $p$ --- про­\linebreak
стое, и само $\mathbb{Z}$. Для доказательства второй части заметим, что включе­\linebreak
ние $K \subset K'$ наделяет $K'$ структурой векторного пространства над $K$\linebreak
(а значит, $|K'| = |K| dim_K^{K'}$ ). Коммутативность не использовалась.\\
\hspace*{15pt}\textbf{b.} Отметим сначала, что корни многочлена $X^{q} - X$ простые, так\linebreak
как его производная —1! Для первой части, если К поле из q элементов,\linebreak
то $K^{*}$ --- мультипликативная группа, состоящая из $q — 1$ элементов,\linebreak
откуда $x^{q-1} = 1$ для всякого $x \in K^{*}$, а стало быть, $x^{q} = x$ для любого\linebreak
$x \in K$. Иначе говоря, K содержится среди корней многочлена $X^{q} — X$ и,\linebreak
следовательно, совпадает с этим множеством из соображений порядка.\linebreak
Все корни многочлена $X^{q} — X$ лежат в К , а следовательно, и в $\Omega$. Это\linebreak
доказывает единственность в $\Omega$ поля из q элементов.\\
\hspace*{15pt}Обратно, предположим, что корни $X^{q} — X$ содержатся в $\Omega$. Так как\linebreak
они различны, то их точно q. Обозначим через $\sigma : \Omega \rightarrow \Omega$ отображение,\linebreak
задаваемое как $\sigma(x) = x^{p}$. Так как $(x + y)^{p} = x^{P} + y^{P}$, то $\sigma$ --- изоморфизм\linebreak
$\Omega$ \textit{на себя.} Если $q = p^{n}$, то отображение $r : x \mapsto x^{q}$, равное $\sigma^{n}$, также\linebreak
изоморфизм. Отсюда следует, что $\mathbb{F_q}$, подполе $\omega$ с $q$ элементами.\\
\hspace*{15pt}\textbf{c.} Если $K \subset H \subset K', то |H|$ --- степень числа $|K| и |K'|$ ---степень\linebreak
числа $|H|$, а значит, $H$ — поле из $q^{d}$ элементов с $d | n$. Обратно, если\linebreak
$d | n$, то все корни многочлена $X^{q^{A}} - X$ лежат в $K'$ (поскольку он\linebreak
делит многочлен $X^{q^{m}} - X$, чьи корни образуют $K'$) и, следовательно,\linebreak
${x \in K' | x^{q^{A}} = x}$ --- промежуточное поле из $q^d$ элементов.\\
\hspace*{15pt}Отображение, которое делителю d числа n ставит в соответствие
поле $\mathbb{F}_{q^{d}} = {x \in K' | x^{p^{d}}  = x}$, является изоморфизмом упорядоченных\linebreak
множеств в том смысле, что $d_1 | d_2$ эквивалентно $\mathbb{F}_{q^{d_1}} \subset \mathbb{F}_{q^{d_1}}$\\
\hspace*{15pt}Обозначим через $\tau$ изоморфизм x $\mapsto x^{q}$. Тогда:
$$\tau^{s}(x) = x и \tau^{r}(x) = x \Longleftrightarrow \tau^{НОД(s, r)}(x) = x$$
(рассмотреть соотношение Безу между $НОД(s,r), r, s)$. Этого доста­\linebreak
точно, чтобы ответить на последний вопрос.\\
\hspace*{15pt}\textbf{d.} Так как $|K'| = p^{n}$ , то многочлен $X^{p^n} - X$ обнуляется на $K'$ и\linebreak
можно рассмотреть многочлен $Q = (X^{p^{n}} — Х )/P$ (по теореме 35 мно­\linebreak
гочлен $_P$ целит: $X^{p^{n}} — X$ ). Так как $degQ < p^{n}$ , то существует $y \in K'$\linebreak
такой, что $Q(y) \neq 0$, и тогда $y$ — корень $P$. Оставшаяся часть вопроса \linebreak
затруднений не представляет.\\
\hspace*{15pt}\textbf{e.} Ясно, что $r^{n} = Id_{K'}$ обратно, если $r^{m} = Id_{K'}$, то множество\linebreak
$K'$ порядка $q^{n}$ содержится в множестве корней многочлена $X^{q^{m}} — X$ и,\linebreak

\pagebreak
%============================================================================================================================================================================================
%306
\noindent значит, $n \leqslant m$ Так как $K'$ изоморфно полю $\mathbb{F_P}[X]/(Q)$, то существует\linebreak
$x \in K'$, который образует $K'$ над $\mathbb{F_p}$, а тем более над К, т.е. $K' = K[x].$\linebreak
Элементы $x, \tau(x),\ldots,\tau^{n-1}(x)$ --- это все различные корни минимально-\linebreak
го многочлена $P$ элемента $x$ над $K$ . Автоморфизм $\sigma$, постоянный над
$K$ , не изменяет и коэффициенты $P$, а следовательно, переводит $x$, ко-\linebreak
рень $P$, в другой корень $\tau^{i}(x)$. Так как морфизмы $\tau^{i}$ и $\sigma$ совпадают на\linebreak
$x$, то они совпадают всюду.\\
\\
\noindent\textbf{Уравнение $x^{p^{k}} = x$ в локальном кольце характеристик и $p$}\\
\\
\hspace*{15pt}\textbf{a.} Существуют такие $y$ и $z$, что $1 — xy \in I,~ 1 — xz \in J.$ Тогда\linebreak
$(1 — xy)(1 — xz) = 1 — x(y + z — xyz) \in IJ$ , что доказывает обратимость
$x$ по модулю $I J$.\\
\\
\hspace*{15pt}\textbf{b.} Перепишем уравнение $x(x^{q-1} - 1) \equiv 0 ~(mod~ I^{n})$. Если $x \in I$, то\linebreak
$x^{q-1} \in I$ и $x^{q-1} \notin I$ (в противном случае сумма $x^{q-1}$ и $1 — x^{q-1}$ , рав-\linebreak
ная 1, принадлежала бы $I$). Так как $I$ максимален, то $1 - x{q-1}$ обратим
по модулю $I$, а, следовательно, и по модулю $I^{n}$ и $x \equiv 0 ~(mod~ I^{n})$. На-\linebreak
против, если $x \notin I$, то $x$ обратим по модулю $I$ и $I^{n}$, а значит, $x^{q-1} \equiv 1$\linebreak
$~(mod~ I^{n})$.\\
\\
\hspace*{15pt}\textbf{c.} Так как $q$ --- степень характеристики $p$, то отображение $x \rightarrow x^{q}$
есть морфизм алгебр и, следовательно, $S_1$ и $S_n$ --- подалгебры, очевидно
удовлетворяющие $\pi(S_n) \subset S_1$. Осталось доказать, что если $x$ является
решением $X^{q} \equiv X ~(mod~ I^{n})$, удовлетворяющим $x \equiv 0 ~(mod~ I)$, то $x \equiv 0$
$~(mod~ I^{n})$. Но это следует из пункта \textbf{b}.\\
\hspace*{15pt}\textbf{d.} Так как $a \equiv 0 ~(mod~ I)$, то существует такое $i$, что для $j > i$\linebreak
$a^{q^{i}} \equiv0 ~(mod~ I^{n})$ Положим $s = a + a^{q} + a^{q^{2}} +\ldots+ a^{q{i}}$. Ясно, что $s \in I,$\linebreak
а кроме того, $s^{q} = a^{q}+ a^{q^{2}} +\ldots+ a^{q^{i+1}}$ откуда $s^{q} = s - a + a^{q^{i+1}} \equiv s - a$\linebreak
$~(mod~ I^{n})$\\
\hspace*{15pt}Доказательство того, что $\pi(S_n) = S_1$ сводится к доказательству
того, что решение $x^{q} \equiv x ~(mod~ I)$ “продолжается” до решения у по
модулю $I^{n}$, т.е. существует $x \equiv x ~(mod~ I)$ и $y^{q} \equiv ~(mod~ I^{y} )$. Приме­
ним предыдущее рассуждение при $a = x^{q} - x$ (которое лежит в $I$) и
получим элемент $a \in I$ такой, что $s — s^{q} \equiv x^{q} — x ~(mod~ I^{n})$, но это
можно переписать в виде $x + s \equiv (x + s)^{q} ~(mod~ I^{n})$ и взять $x = x + s$\\
\hspace*{15pt}\textbf{e.}Если $\pi : A/I^{n} \rightarrow A/I$, то $Ker\pi = I/I^{n}$ --- максимальный иде­
ал в $A/I^{n}$ если J — максимальный идеал в А, содержащий $I{n}$, то он
содержит I и, следовательно, J = I. Поэтому $I/I^{n}$ --- единственный
максимальный идеал в В. Используя то, что В — локальное кольцо,
легко доказать, что $x^{q} = x$ в В эквивалентно $x = 0$ или $x^{q-1} = 1$. Что­
бы получить конечный результат, нужно предположить, что некоторая
степень $\mathcal{M}$ равна нулю (в случае, когда $\mathcal{M} — I/I^{n}$, имеем $\mathcal{M} = 0)$.
\newpage
%==========================================================================================================================================================================
%307
%------- X _----
\end{document}