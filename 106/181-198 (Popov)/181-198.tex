\documentclass{../../template/mai_book}

\defaultfontfeatures{Mapping=tex-text}
\setmainfont{DejaVuSans}
\setdefaultlanguage{russian}

%\clearpage
%\setcounter{page}{7} % ВОТ ТУТ ЗАДАТЬ СТРАНИЦУ
%\setcounter{thesection}{5} % ТАК ЗАДАВАТЬ ГЛАВЫ, ПАРАГРАФЫ И ПРОЧЕЕ.
% Эти счетчики достаточно задать один раз, обновляются дальше сами
% \newtop{ЗАГОЛОВОК}  юзать чтобы вручную поменть заголовок вверху страници

\begin{document}
\setcounter{page}{181}
\newtop{II-1.2 Что такое факториальное кольцо?}

\noindent \textit{нуля, то оба элемента равны } \textbf{с точностью до обратимого элемента}\textit{,
т.е. существует $\text{\textepsilon}\in\:U(A)$, такой, что $a=\text{\textepsilon}\:\!b$.}

\textit{($iii$) Элемент р $\in$ А называется} \textbf{неприводимым}\textit{, если он не является ни нулевым, ни обратимым и если его единственными делителями
являются 1 и р. Более точно: р неприводим тогда и только тогда, когда d | р $\Rightarrow$ d \textasciitilde$\:$ 1 или d \textasciitilde$\:$ р. Это свойство равносильно тому, что идеал Ар является максимальным в множестве} \textbf{однопорожденных}\textit{ (т.е глав­ных) идеалов в А, отличных от А и \{0\}.}

\textit{($iv$) Если кольцо А без делителей нуля, то р — неприводимый эле­мент тогда и только тогда, когда из р = de следует, что d или е — обратимый элемент в А.}

\section{\large 1.2 Что такое факториальное кольцо?}

Теперь можно дать точное определение структуры колец, обобщающей
в какой-то мере структуру кольца Z целых чисел.

\noindent\textbf{(4) Определение }(факториальность). 

\textit{Кольцо А называется факториальным, если оно не имеет делите­
лей нуля и если оно удовлетворяет основной теореме арифметики, что,
формально, выражается следующим образом:
\newline\indent($i$) Наличие разложения на неприводимые множители:}
\[
\forall a \in A^* \ U(A),\,\exists p_1 ,p_2 ,\dots,p_n , \text{неприводимые} \\ 
\text{и такие, что}\: a = p_1 p_2 \dots p_n . 
\]

\textit{($ii$) Единственность такого разложения:}
\[
a
\]

В дальнейшем мы рассмотрим критерии, позволяющие распознать
факториальность кольца. В настоящий момент читатель должен пове­
рить, что это понятие не «бессодержательное», т.е. класс факториаль­
ных колец сравнительно широк. Прежде, чем продолжать дальнейшее
изучение делимости, укажем одно арифметическое приложение факто-
риальности.

\pagebreak

1

\end{document} 