\documentclass{../../template/mai_book}

\defaultfontfeatures{Mapping=tex-text}
\setmainfont{DejaVuSerif}
\setdefaultlanguage{russian}

\begin{document}

\lhead{\small\textit{Решения упражнений}}
\rhead{145}

\noindent
[1, $n$] в [1, $n$], при этом коэффициент $\mu_\alpha$ определяется суммой

\begin{equation*}
\mu_{\alpha} = \sum_{\epsilon_i\in\{0,1\}} (-1)^{\epsilon_1+...+\epsilon_n} \epsilon_{\alpha(1)}...\epsilon_{\alpha(n)}.
\end{equation*}

\noindent
Если $\alpha$ — перестановка, коэффициент $\mu_\alpha$ равен $(—1)^n$, поскольку единственный ненулевой член суммы — это тот, для которого все $\epsilon_i$, равны 1. В противном случае можно разложить $\mu_\alpha$ следующим образом:

\begin{equation*}
\sum_{\epsilon_i, i \neq k} (-1)^{\epsilon_1+...+\epsilon_n} \sum_{\epsilon_k=0,1} (-1)^{\epsilon_k} \epsilon_{\alpha(1)}...\epsilon_{\alpha(n)},
\end{equation*}

\noindent
где $k$ — элемент из [1, $n$], который не принадлежит образу $\alpha$, и хорошо видно, что внутренний член (сумма по $\epsilon_k$) равен нулю; это доказывает, что $\mu_\alpha$ в этом случае равно нулю. Второе искомое выражение (суммы которого записываются на множествах) просто выводится из первого, исключением ненулевых $\epsilon_i$, появляющихся в сумме.

\setcounter{equation}{6}

\begin{equation}
{\sum}_{E'} = {\sum}_E + (-1)^{|E'|} \prod_{1 \leq i \leq n} S_i(E').
\end{equation}

\noindent
Вклад наименьшего элемента 0̸  в эту сумму — нулевой; значит, начнем с его последователя, который может быть \{1\}, \{$n$\} или какое-нибудь одноэлементное множество в соответствии с порядком, заданным на [1, $n$]. Можно заметить, что $S_i(E') = S_i(E) \pm a_{ij}$, где $\{j\} = E \triangle E'$.

\pagebreak

\lhead{146}
\rhead{\small\textit{$I$ \quad Алгоритмика и программирование на языке Ада}}

\noindent
В этой записи, как в алгоритме 12, $\pm$ должен пониматься как $+$, если $E \subset E'$ и $-$, если $E' \subset E$.

Чтобы оценить ${\sum}_{E'}$, исходя из ${\sum}_E$ с использованием 
соотношения (7), нужно осуществить $n$ сложений (для подсчета каждого $S_i (E')$), затем $n - 1$ перемножений: $S_1(E') \times \cdots \times S_n(E')$, и, наконец, 1 сложение. Имеем $2^n - 2$ операций для осуществления (7), при этом первый член $\sum = -{\prod}_{1 \leq i \leq n} \alpha_{in}$ требует $n - 1$ перемножений; это доказывает сформулированный результат о сложности. Сложность \textit{O}($n2^n$) значительна, но остается того же порядка, что и сложность, индуцированная определением ($n!(n - 1)$ перемножений и $n! - 1$ сложений). Формула Стирлинга позволяет сравнить эти два значения сложности: $ n \cdot n!/(n \cdot 2^n) \approx (n/2e)^n \sqrt{2\pi n}$.

\paragraph{22. Перманент матрицы (продолжение)}

\subparagraph{a.} Правая часть может рассматриваться как многочлен (от переменных $a_{ij}$), равный ${\sum}_\alpha \mu_\alpha a_{1 \alpha(1)} ... a_{n \alpha(n)}$, где сумма распространяется на все отображения [1, $n$] в [1, $n$]. Коэффициент $\mu_\alpha$ задан формулой

\begin{equation*}
\mu_\alpha = \sum_\omega \mu_\alpha(\omega) \quad \text{с} \quad \mu_\alpha(\omega) = \omega_1 \text{ ... } \omega_{n - 1} \omega_{\alpha(1)} \text{ ... } \omega_{\alpha(n)},
\end{equation*}

\noindent
в которой полагаем $\omega_n = 1$. Если $\alpha$ — перестановка, то каждое $\mu_\alpha(\omega)$, присутствующее в сумме $\mu_\alpha$, равно 1 и, следовательно, $\mu_\alpha = 2^{n - 1}$. Напротив, если $\alpha$ не является перестановкой, то сумма $\mu_\alpha$ — нулевая. Действительно, образ $\alpha$ отличен от [1, $n$], и различаем два случая: \newline
\indent ($i$) $\exists$ $k < n$, не принадлежащий образу $\alpha$, \newline
\indent ($ii$) $\exists$ $k < n$, дважды полученный из $\alpha$.

В обоих случаях члены $\mu_\alpha(\omega)$, присутствующие в сумме, группируются попарно, один соответствуя $\omega_k = 1$, другой — $\omega_k = -1$, и взаимно уничтожаются (в случае ($ii$) $\mu_\alpha(\omega) = \omega_k$).

\subparagraph{b.} Формула пункта \textbf{a} может быть записана в следующем виде:

\begin{equation*}
\frac{per A}{2} = \sum \omega_1 \text{ ... } \omega_{n - 1} \prod_{1 \leq i \leq n} (a_{in} + \omega_1 a_{i1} + \cdots + \omega_{n - 1} a_{in - 1})/2.
\end{equation*}

\noindent
Как и в предыдущем упражнении, вычисление перманента получается генерированием перебора при линейной упорядоченности на $\{-1, 1\}^{n - 1}$, в которой два последовательных элемента отличаются только одной компонентой. Если для $\omega \in \{-1, 1\}^{n - 1}$ и $i \leq n$ положить

\begin{equation*}
S_i(\omega) = (a_{in} + \omega_1 a_{i1} + \cdots + \omega_{n - 1} a_{in - 1})/2,
\end{equation*}

\pagebreak

\end{document} 
