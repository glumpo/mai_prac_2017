\documentclass{book}

\usepackage{mathtext, parcolumns}
\usepackage[cp1251]{inputenc}  %% 1
\usepackage[T2A]{fontenc}      %% 2
\usepackage[english,russian]{DejaVuSerif}
\defaultfontfeatures{Mapping=tex-text}
\usepackage{graphicx}
\usepackage[pdftex]{graphicx}

\begin{document}
\newpage
\setcounter{thesection}{289}
Это очевидно, если $b \leq 5$. Если $b > 5$, то $5 - b$ представляет 5 по
модулю b...


Можно заметить, что мы распологаем парами $(a,b)$, где $b$ "--- делитель
$a$ с эвклидовыми делениями $a = bq + r$, где $r$ уже не нуль! (Возьмите,
например, $40 = 5 \times 6 + 10$.)\\
\\
\textbf{27. Самый малый алгоритм Евклида на $Z$}
\\

То, что $\phi$ --- алгоритм Евклида, содержится в примере (раздел 3.1)
из этого курса. Пусть $\psi$ --- другой агоритм Евклида на $Z$ со значениеями
 в $N$. Докажем с помощью индукции по $n =\psi(b)$, что $\phi(b) \leq \psi(b)$.


Это верно для $n = 0$, так как из $\psi(b) = 0$ следует, что $b = 0$
(см. лемму 29). Предположив, что это выполняется для всех b, таких,
что $\psi(b) < n$, докажем, что оно верно и для $\psi(b) = n$.


Подвергнем $\psi$-евклидовому делению $[|b|/2]$ на число $b:[|b|/2] = bq+r$,
где $\psi(r) < \psi(b)$. Так как r представитель $[|b|/2]$ по модулю b, то
$|r| \geq [|b|/2]$. Отсюда имеем:\\
\\
															$\phi(r) \geq \phi([|b|/2]) = \phi(b) - 1.$\\
\\
Так как $\psi(r) < n$, то по индукции $\psi(r) = \phi(r)$ и поэтому:\\
\\
															$\phi(b) \leq \phi(r) + 1 = \psi(r) + 1 \leq \psi(b)$,\\
\\
что заканчивает доказательсвто.\\
\\
\textbf{30. Конечные поля, порядок которых есть квадрат\\
простого числа}\\

Согласно изложеному выше (раздел 1.3, предложение 8), $\rho$ "--- неприводимо 
в $Z[i]$ и потому порождает максимальный идеал. Итак,
$Z[i]/(p)$ "--- поле, так как:\\
\\
									\textit{$x + iy \equiv 0$} (mod $\rho$) $\Longleftrightarrow$ \textit{$x, y$ $\equiv 0$} (mod $\rho $),\\
\\
аддитивная группа $Z[i]/(p)$ изоморфна $Z/(p)$ х $Z/(p)$, откуда порядок
поля, о котором идет речь, равен $р^2$.\\
\\
\textbf{31. Несколько конечных полей порядка, являющегося\\
степенью двойки}\\

Многочлены $\Rho(X)$ следующие:\\
\\
			\begin{array}{llll}
			$X^2 + X + 1$, & $X^3 + X + 1$, & $X^4} + X + 1$,\\ 

			        &$X^5 + X^2 + 1$, & $X^6 + X + 1$, & $X^7 + X + 1$,\\ 
			\end{array}
\newpage
неприводимые по модулю 2 (для каждой данной степени n --- это 
первые неприводимые многочлены степени n, в смысле лексикографического 
порядка, о чем читатель, конечно, догадался). Поля $Z_2[X]/(P)$\\
отвечают на вопрос задачи.\\
\\
\textbf{32. Получение всех коэффициентов Безу}
\\

Так как $(а, Ь)$ можно всегда заменить на $(a/d, Ь/d)$, то будем предполагать,
что $d = 1$, т.е. $а и b$ взаимно просты. Тогда имеем: $(u_0 - u)а =
(v — v_0)b \Rightarrow а | (v — v_0)b$. К тому же а и b взаимно просты. Отсюда а
делит $v — v_0$. Следовательно, существует $k \in Z$ такое, что $k = u_0v — v_0u$,
так как $a(u_0v — v_0u) = v — v_0$.\\
\\
\textbf{33. Единственность ограниченных коэффициентов Безу}\\

С помощью замены $(а, Ь)$ на $(а/d, b/d)$ можно всегда предполагать,
что $d = 1$.\\

\textbf{a.} Согласно предыдущему упражнению, существует такое $k \in Z$,
что и $u^{'} = u — kb$ и $v^{'} = v+ka$. Следовательно, число $u^'$ есть представитель
по модулю b числа u, отличный от u. Неравенство для $|u|$ показывает,
что $|u^{'}| \geq |b|/2$. Равенство возможно только для четного b. Но а и b не
могут быть оба четными, так как они взаимно просты.\\
\\

\textbf{34. Мажорирование коэффициентов Безу в $K[Х]$}\\

Достаточно сымитировать доказательство, касающееся коэффициентов
Безу в Z из раздела 7.2. Алгоритм создает последовательность
$Q_1, Q_2, ..., Q_n$ с одной стороны, $U_0, U_1, ..., U_{n+1}$ и $V_0, V_1, ..., V_{n+1}$ с:\\
\\
                 \begin{array}{ccccc}
								$U_0 = 1$,&$U_1 = 0$,&$U_{i+1} = U_{i-1} - Q_iU_i$,\\
								   & $V_0 = 0$, & $V_1 = 0$, & $V_{i+1} = V_{i-1} - Q_iU_i$, &       $1 \leq i \leq n$,\\
								 \end{array}\\
\\
и легко проверить, что для $2 \leq i < п deg(Q_i) > 0$ и:\\
\\
							\begin{array}{cccc}
							$deg(U_2) < $ & $deg(U_3) < ... <$ & $deg(U_{n+1})$,\\
							               & $deg(V_2) < $ & $deg(V_3) < ... < $ & $deg(V_{n+1})$,\\
							\end{array}\\
\\
что дает искомый результат, так как $U_n$ и $V_n$ --- коэффициенты Безу\\
и $U_{n+1} ~ В, V_{n+1} ~ А$.\\
\newpage
\textbf{35. Дерево Штерна --- Броко}\\

\textbf{a.} Свойство $p^{'}q — pq^{'} = 1$ верно для всякой пары последовательных
дробей, полученных с помощью описанной процедуры. Это доказывает
взаимную простоту числителей и знаменателей полученных дробей и
что $\frac{p}{q} < \frac{p + p^{'}}{q + q^{'}} < \frac{p^{'}}{q^{'}}$.

Если $\frac{p}{q} < \frac{b}{c} < \frac{p^{'}}{q^{'}}$, то из неравенств $bq — ср \geq 1$ и $cp^{'}—bq^{'} \geq 1$ получаем,
что\\
												$q + q^{'} \leq q(cp^{'} - bq^{'}) + q^{'}(bq - cp) = c$.\\
\\
\textbf{b.} Рассматривая во время процесса деления дробь $\frac{a}{b}$, можно заметить,
что не может быть получена дробь со знаменателем, большим,
чем Ь, легко вывести алгоритм 8, в котором используются первообразные
уравнения, введенные в упражнении 30 главы I.\\

  **********ТУТ ДОЛЖНА БЫТЬ ВСТАВКА ПРОГРАММЫ НА АДЕ ВЫПО-Я НА СИ************\\
\\
                                                            \begin{alltt} 
                \textbf{Алгоритм 8.}  Порождение ряда Фарея \end{alltt}\\


Список-результат построен в обратном порядке (что не очень важно
само по себе). Перед окончанием выполнения его порядок еще раз
меняют. В этом алгоритме знак $\oplus$ обозначает странную операцию, которую
применяют к дробям в процессе Штерна — Броко.\\
\newpage
\textbf{c.} Если две последовательные дроби в $F_N$ не удовлетворяют соотношению
 $q + q^{'} > N$ , то медиана принадлежит $F_N$ и должна находиться
между двумя простыми, что является противоречием.\\


Начиная с соотношения $p^{'}q — pq^{'} = 1$ и $p^{''}q^{'} — p^{'}q^{''} = 1$, получаем\\

\begin{array}{cccc}
    & $р^{'}(р^{''}q - pq^{''}) = p + p^{'}$ & и & $q^{'}(p{''}q - pq{''}) = q + q^{''}$,\\
\end{array}\\
\\
что доказывает первое из свойств. Отправляясь от тех же соотношений,
можно доказать, что\\


$$p^{''}q - pq^{''} = \frac{q + q^{''}}{q^{'}} = \frac{q + N}{q^{'}} - \frac{N - q^{''}}{q^{'}}.$$\\
Согласно свойству знаменателей, доказанному выше, последний терм
меньше 1 и, следовательно, второе свойство доказано.
Для формул, позволяющих выразить член ряда Фарея с помощью
двух предшествующих дробей, достаточно проверить, что два выражения 
для $р^{''}$ и $q^{''}$ взаимно просты и что их отношение эффективно
дает $а^{''}$. В этом случае запись алгоритма очевидна.\\


\textbf{d.} Предположим, что р и q отличны от нуля и единицы и что $р < q$.
Свойство, сформулированное в первой задаче, позволяет доказать, что
обе родительские дроби для $\frac{p}{q}$ дают коэффициенты Безу для р и q.
Кроме того, свойства
\begin{itemize}
	\item всякая дробь $\frac{p}{q}$ получена в дереве Штерна --- Броко с помощью
двух дробей, расположенных на той же ветви дерева,
	\item на ветви дерева, начиная с уровня 2, знаменатели строго
возрастают
\end{itemize}
позволяют доказать, что одна из этих пар коэффициентов минимальна.\\
\\
\textbf{36. Вычисление коэффициентов Безу n-ки целых чисел}\\
*******************VSTAVKA NA CCCCCCCCCCCCCCC*********\\
\\
                                                         \begin{alltt}                  
                  \textbf{Алгоритм 9. Коэффициенты Безу} \end{alltt}\\
\newpage
Пусть требуется вычислить коэффициенты Безу n-ки целых чисел $^{-}a$
Алгоритм 9 находит эти коэффициенты и помещает их в n-ку $^{-}u$.\\
\\
\textbf{39. Сложность центрированного деления}\\

\textbf{a.} Первые члены последовательности $(G_n)$ выглядят следующим
образом: 0, 1, 2, 5, 12, 29, 70, 169, ...Имеется точно n центрированных 
делений в алгоритме Евклида, примененном к паре $(а, b)$, ибо:\\

\begin{array}{ccccc}
    &    & $a = 1 \times G_n = G_{n-1},$& $|G_{n-1}| \leq |G_n|/2,$\\
		&		 & $G_n = 2 \times G_{n-2} + G_{n-2},$  & $|G_{n-2}| \leq |G_{n-1}|/2,$\\
		&    & $\vdots$ & \\
		&    & $G_3 = 2 \times G_2 + G_1,$ & $|G_1| \leq |G_2|/2,$\\
		&    & $G_2 = 2 \times G_1 + 0.$ & \\
\end{array}\\

\textbf{b.} Запишем деления в виде $r_{i-1} = r_iq_i, + r_{i+1}$ с $|r_{i+1}| \leq |г_i|/2$ для
$1 \leq i \leq n$ и $|r_1| < |г_0|$. Отсюда следует, что $|q_i| \geq 2$ для $2 \leq i \leq n.$
Чтобы доказать неравенство $|r_{i-1}| \geq 2|г_i| + |г_{i+1}|$ (для $2 \leq i \leq n$),
рассмотрим два случая:
\begin{itemize}
	\item $|q_i| < 3 : |r_{i-1}| = |r_iq_i + r_{i+1}| \geq |q_i||r_i| - |r_{i+1}| \geq 3|r_i| - |r_{i+1}| \geq$\\
$2|r_i| + |r_{i+1}|.$
	\item $|q_i| < 3$ лишь при $q_i = ±2.$ Из того, что $|±2r_i + r_{i + 1}| = |r_{i-1}| \geq 2|г_i|,$
следует $|r_{i-1}| = |±2r_i + r_{i+1}| = 2|г_i| + |r_{i+1}|$ (если выполняется $|u + v| \geq |u|$
и $|v| \leq |u|,$ то $|u + v| = |u| + |v|).$

Из неравенств $|г_{i-1}| \geq 2|r_i| + |r_{i+1}|$ выводим:\\
\end{itemize}\\
\begin{array}{ccccc}
     $|r_n| \geq 1 = G_1,$ $|r_{n-1}| \geq 2 = G_2,$\\
		 $|r_{n-2}| \geq 2|r_{n-1}| + |r_n| \geq 2G_2 + G_1 = G_a,$\\
		 $\ldots$ & \\
		 $|r_1| \geq 2|r_2|+ |r_3| \geq 2G_{n-2} = G_n$,$ |r_0| \geq |r_1| + |r_2| \geq G_n + G_{n-1}.$\\
\end{array}
\\
Оставшаяся часть упражнения трудностей не представляет\\
\\
\textbf{40. Порождения простых чисел с помощью многочленов}\\

\textbf{a.} Предположим, что существует такой многочлен. Нетрудно показать,
что $Р(n)$ делит $Р(n + kP(n))$ для любого целого k (достаточно
расписать $Р(n + kP(n))$). Так как Р не постоянен и принимает только
простые значения, то приведенное свойство означает, что для всякого
$k \in Z$, $Р(n) = ±Р(n + kР(n)).$ Следовательно, один из многочленов
$Р — Р(n)$ или $Р + Р(n)$ имеет бесконечное множество корней, что противоречит
тому, что Р не постоянен.
\newpage

Замечание: натуральное число n называется эйлеровым, если многочлен 
$X^{2} + Х + n$ принимает простые значения для всякого целого числа
из интервала $[0, n — 1]$. Доказано, что существует только шесть (уже
известных) чисел Эйлера: 2, 3, 5, 11, 17, 41 (см. Ле Лионне [113]).\\

\textbf{b.} В действительности мы докажем, что это неравенство почти
всегда точное. Согласно постулату Бертрана, $p_{n+i} < 2р_n$, и если имеется
точное неравенство для $р_n$ то оно имеется и для $p_{n+1}$. Достаточно
теперь заметить, что точное неравенство будет при $n = 3$ (так
как $7 < 5 + 3 + 2$).

\textbf{c.} Постулат Бертрана позволяет утверждать, что n-е простое число
не меньше $2^{n}$ и, следовательно, $10^{n}$. Тогда, если через $\textit{K}$ обозначим
сумму ряда $\sum_{k\geq1}10^{-k^{2}}$, то это и будет та постоянная, которую мы
ищем. Действительно,\\
\\
\begin{array}{ccc}
		$10^{n^{2}}K = 10^{n^{2}-1}p_1 + 10^{n^{2}-4}p_2 +\ldots+10^{2n-1}p_{n-1} + p_n + 10^{-2n-1}p_{n+1}$\\
		$+ 10^{-4n-4}p_{n+2} +\ldots\equiv p_n + x (mod 10^{2n-1}),$ где x\in [0,1]\\
\end{array}
Так как $р_n < 10^{n}$, то утверждение верно.

Однако, с практической точки зрения, приведенная выше формула
числа $\textit{K}$ неэффективна: чтобы посчитать по ней число $\textit{K}$, необходимо
прежде всего знать последовательность простых чисел, а затем еще и
просуммировать ряд ... приблизительное значение этой суммы 0,2003.
Вообще, приемлемая формула, позволяющая находить простые произвольно
большие числа, не известна.\\
\\
\textbf{41. Соотношение Безу для многочленов}
\\

Если $\textit{P}$ и $\textit{Q}$ взаимно простые многочлены из $\textit{A}[Х]$, то они принадлежат 
и $\textit{K}[Х]$. Из равенства Безу $\textit{UP} + \textit{VQ} = 1$, освобождаясь от знаменателей, 
появляющихся в $\textit{U}$ и $\textit{V}$, получаем соотношение, требуемое в упражнении. 
Расширенный алгоритм Евклида показывает, что можно выбрать многочлены 
$\textit{U}$ и $\textit{V}$, удовлетворяющие условиям, наложенным
на степени (см. упражнение 34).\\
\\
\textbf{42. Сложность алгоритма Евклида в} $\textit{K}[Х]$\\

\textbf{а.} Пусть $(R_i)_{0 \leq i \leq n+1}$ --- последовательность остатков в алгоритме
Евклида, примененном к паре $(\textit{A}, \textit{В})$, с $R_0 = \textit{A}, R_1 = \textit{В}, \textit{R}_{n+1} = 0$ и
$R_n \leq 0$; тогда $0 \leq deg R_n < deg R_{n-1} < \ldots < deg R_1$ и, следовательно,
$n - 1 \leq deg(R_1)$.\\
\newpage

\textbf{b.} Подходящей является последовательность $F_0 = 1, F_1 = X + 1$ и 
$ F_{n+2} = XF_{n+1} + F_n.$\\

\textbf{43. Оптимальность алгоритма для $K[Х]$ (теорема Лазара)}\\

\textbf{a.} Ищем $\mu \leq \phi$. Запишем оптимальное деление в виде $A = В \times Q + R$
и евклидово деление $А = В х Q^{'} + R^{'}$ с $deg(R^{'}) < deg(B)$. Это дает
$\mu(A, В) = \mu(В, R) + 1$ и $\phi(А, В) = \phi(В, R^{'}) + 1.$

\begin{enumerate}
	\item Предположим, что $deg(R) > deg(B).$ Тогда из двух евклидовых
делений $В = 0 х R + В$ и $R = (Q^{'} — Q) х В + R^{'}$ выводим:\\
\\
\begin{array}{lccccl}
$\mu(\textit{A, B})$ & = & $\mu(\textit{B, R}) + 1$ & = &$\phi(B, R) + 1$ & (_{определение, затем реккурентность}).\\
                     & = & $\phi(\textit{R, B}) + 2$ & = &$\phi(B, R) + 3$ & (_{1-е, затем 2-е евклидово деление}).\\
                     & = & $\phi(\textit{A, B}) + 2$ &   &                 &(_{по определению}).\\
\end{array}\\
\\
что противоречит $\mu(A, B) < \phi(A, B)$.\\
	\item Предположим, что $deg(R) = deg(B)$. Тогда $deg(R^{'}) < deg(R)$,
откуда $deg(R^{'} — R) = deg(R)$ и равенство:\\
\end{enumerate}
\\
\begin{array}{ccr}
$B(Q - Q^{'}) = R^{'} - R,$ & $deg(R^{'} — R) = deg(R) = deg(B),$ & (13)\\
\end{array}\\
\\
приводит к тому, что $Q — Q^{'}$ постоянный (не нулевой). Если k обозна­
чает $(1/Q — Q^{'})$, то имеем евклидово деление В на R: $B = —kR + kR^{'}$ с
$deg(kR^{'}) < deg(R)$, откуда:\\
\\
\begin{array}{cccccl}
$\mu(A, B)$ & = & $\mu(B, R) + 1$       & = & $\phi(B, R) + 1$        & (_{определение,}\\
            &   &                       &   &                         &  _{затем рекуррентность}).\\
            & = & $\phi(R, kR^{'}) + 2$ & = & $\phi(-kR, R^{'}) + 2$  &(евклидово деление (13),\\
						&		&			                  &   &                         &затем вопрос а).\\
						& = & $\phi(B, R^{'}) + 2$  & = & $\phi(A, B) + 1$        &($B \equiv - kR (mod R^{'})$\\
						&   &                       &   &                         &затем определение).\\

\end{array}\\
\\
что противоречит $\mu(A, В) \leq \phi(А, В)$. Поэтому единственно возможный
случай $deg(R) < deg(B)$, и, согласно единственности обычного частного
Евклида, $R  = R^{'}\ldots$\\
\\
\\
\textbf{44. К теореме Лазара}\\
\\


Последовательность $(C_n)_n \req i$ строго возрастающая и, следовательно,
состоит из натуральных чисел; ее первые члены 0, 1, 3, 5, 18, 31, 111,
191... Нетрудно проверить неравенства:\\
\\
\begin{array}{cc}
   $C_{2i-1} < (1 - \alpha)C_{2i} < 0,5 \times C_{2i}$ & $0,5 \times C_{2i-1} < C_(2i-2) < \alpha C_{2i-1}$
\end{array}
\newpage

константа а введена в теореме Лазара ($\alpha \approx 0,6180339$ — обратное к
золотому сечению).


Предположим, что первый элемент при центрированном делении
имеет четный индекс, скажем $C_{2n}$ Ниже приведена последовательность
центрированных делений $С_{2n}$ на $C_{2n-1}$ (записанных в виде $а = qb + r$):\\
\\

\begin{array}{rcl}
     $C_{2n}$              & = & $4C_{2n-1} + (C_{2n-2} - C_{2n-1}),$\\
		 $C_{2n-1}$            & = & $(-2)(C_{2n-2} - C_{2n-1}) + C_(2n-3),$\\
		 $C_{2n-2} - C_{2n-1}$ & = & $(-3)C_{2n-3} + (C_{2n-3} - C_{2n-4}),$\\
		 $C_{2n-3} - C_{2n-4}$ & = & $3C_{2n-5} + (C_{2n-6} - C_{2n-5}),$\\
		 $C_{2n-5}$            & = & $(-2)(C_{2n-6} - C_{2n-5}) + C_(2n-7),$\\
		                       & \vdots & \\
		 $C_3$                 & = & $(-2)(-2) + 1$ или $C_3 = 2 \cdot 3 + (- 1),$\\
		 $\pm 2$               & = & $(\pm 2)(\pm 1) + 0$\\
\end{array}\\
\\
(имеется точно $2n — 1$ делений). Можно задать алгоритм Евклида, тре­
бующий $2n — 1$ делений, из которых $n —1$ нецентрированных (записанных
в виде $а = qb + r$):\\
\\
\begin{array}{rccl}
$C_{2n}$    & = & $3C_{2n-1} + C_{2n-2}$       & (нецентрированное),\\
$C_{2n-1}$  & = & $2C_{2n-2} + (-C_{2n-3})$    & (центрированное),\\
$C_{2n-2}$  & = & $(-3)(-C_{2n-3}) + C_{2n-4}$ & (нецентрированное),\\
$-C_{2n-4}$ & = & $(-2)C_{2n-4} + C_{2n-5}$    & (центрированное),\\
$C_{2n-4}$  & = & $3C_{2n-5} + C_{2n-6}$       & (нецентрированное),\\
            & \vdots & \\
$C_{2}$     & = & $(\pm 3)(\pm C_{1}) + C_{0}$ & $С_0 = 0$ (центрированное, \\
            &   &                              & остальное нулевое).\\
\end{array}\\
\\
Используя теорему Лазара, легко проверить, что первая или вторая
последовательности оптимальны по длине.\\
\\

\textbf{45. (Неэффективное) решето Эратосфена для многочленов}\\
\\
\renewcommand{\baselinestretch}{1.5}\normalsize
\begin{parcolumns}[nofirstindent]{2}
\colchunk[1]{а. {\scripsize Пусть Crible(2 .. n) — последовательность булевых переменных со значениями «истинно». Решето Эратосфена останавливается на очередном значении i с Crible(i) истинным и вычеркивает все кратные i * j числа г (Crible(i * j) <— False), а затем повторяет эту процедуру для следующего i. Таким образом, про}}
\colchunk[2]{\renewcommand{\baselinestretch}{1.1}\footnotesize {\scripsize for I in 2 .. п loop if Crible(i) then for j in i .. n/i loop Crible(i * j) <— False', end loop; end if; end loop;} \par}
\colplacechunks
\end{parcolumns}
\newpage
стые числа от 2 до n в точности те i, для которых Crible(i) истинно.

\textbf{b.} Пусть A — множество унитарных непостоянных многочленов над
$\mathbb{Z}/p\mathbb{Z}$ — объединение $\bigcup^{\infty}_{k=1} \times [0, p^{k}]$ Построим биективное ото­
бражение А на В следующим образом: если $P(Х) = Х^{k} + a_{k-1}X^{k-1} +
\ldots +a_1X + a_0 $— многочлен из $А(k \req 1)$, то поставим ему в соответствие
пару $(k,a_{k-1}p^{k-1}+\ldots+a_{1}p + a_0)$ из В.\\
\\
        VSTTTTTTTTAVKAAAAAAAAAAAAAA!!!!!\\
\\
\textbf{Алгоритм 10.} Решето Эратосфена\\
\\

Отобразим теперь В на \mathbb{N}, сопоставляя паре ${1} х [0, р]$ отрезок $[0,р]$,
паре ${2} х [0, р^{2}]$
[ отрезок [р, р+ р2
[ и так далее. Если Вп
 = р + р2 Н-----|-рп
,
то построим следующую биекцию В на N = [Во, Bi[U[Bi, Вг[и[Вг,
Вз[и • • •. Очевидно, Вп
 — число унитарных непостоянных многочленов
степени п. Пусть п — целое число, <г — взаимнооднозначное отобра­
жение интервала [О, В„[ на множество унитарных непостоянных мно­
гочленов степени п. Теперь получаем алгоритм 10, в котором Q * В,
обозначает произведение многочленов Q и R (при помощи Overflow этот
алгоритм игнорирует исключения, при которых deg(Q) + deg(B) > n).
По окончании работы этого алгоритма определяются все неприводи­
мые многочлены степени <С п: это в точности те многочлены сг(г), для
которых Crible(i) истинно. Разумеется, этот алгоритм неэффективен
по времени и объему! Кроме того, он позволяет получить список не­
приводимых многочленов по модулю 2 степени <С 10 или по модулю 3
степени 8.
d. Ниже приведен краткий список результатов, которые должна
давать программа, реализующая предшествующий алгоритм:


\end{document}