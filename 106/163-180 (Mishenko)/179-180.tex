\begin{center}
\parbox{11cm}
{
но как это далеко от итераций алгоритма Евклида! Зато первый
метод дает те сведения, которые не дает второ
}
\end{center}

Вот второй  аргумент,  доказывающий,  что  $r_{6}$ = 11 есть НОД$(a,b)$. 
Хотя он очень похож на первый, приведенный выше аргумент, но имеет 
одно преимущество.  Он  выделяет  незамеченное  в  прошлом  понятие,  а именно \textit{соотношение Безу}.
Заметим, что $bq + г$ и $b$ — линейные комби­нации $b$ и $r$
  и наоборот  (речь идет о линейных комбинациях с коэффи­
циентами из $\mathbb{Z}$).

Если обозначить через $\mathbb{Z}(bq+r)$ +  $\mathbb{Z}b$ множество линейных комбина­
ций $bq + г$ и $b$ с коэффициентами из $\mathbb{Z}$, то получим равенство множеств 
$\mathbb{Z}(bq + r)$ +$\mathbb{Z}b$ = $\mathbb{Z}b+\mathbb{Z}r$.
 Множества $\mathbb{Z}_{{r}_{i}} + \mathbb{Z}_{{r}_{i+1}}$  являются одними и теми 
же. В частности, имеем $\mathbb{Z}_{{r}_{i}}a Za + \mathbb{Z}_{{r}_{i}}b=\mathbb{Z}_{{r}_{i}} r$,
  что соответствует отношениям 
$r_{6}$ | $a$, $r_{6}$ | b и $r_{6}$ $\in$ $\mathbb{Z}a$ $\mathbb{Z}b$
$\mathbb{Z}a + \mathbb{Z}b$.
 Используя существование  целых чисел $u$ и $v$
(которые мы и не старались вычислить),  получаем $r_{6} = ua + vb$.
  Легко 
проверить,  применяя последнее  равенство,  что $\delta$  | $r_{6}$  равносильно $\delta$ | $a$ и 
$\delta$ | $b$.
 Это снова доказывает,  что $r_{6}$ = НОД$(a,b)$
 \begin{center}
 \parbox{11cm}
 {
 	\textbf{Замечание}.  Сложение оставляет на месте множество $\mathbb{Z}b$+$\mathbb{Z}r$. То
же верно для умножения на элементы из $\mathbb{Z}$.
 Математики называ­ют такое множество идеалом. Это основное понятие, к которому
мы будем неоднократно обращаться в дальнейшем.
 }
 \end{center}
 Надо отметить  последний  важный  пункт,  который  позволяет  убе­
диться,  что все лицеисты  Франции  и  Наварры  находят тот  же  самый 
результат,  когда  вычисляют  НОД  с  помощью  первого  метода.  Пра­
вильность их  метода основывается  в  действительности  на следующем 
результате  (и это еще надо доказать):

\noindent \textbf{(1)Теорема} основная теорема арифметики

\textit{Всякий элемент из $\mathbb{N}$* разлагается на 
простые множители.  Это раз­
ложение однозначно с точностью до порядка простых сомножителей.}
\newline

Теперь  декорации готовы. В  последующих сценах мы  введем и свя­
жем различные концепции: евклидово деление, НОД, соотношение Везу, 
разложение на простые  множители,  идеалы...
\section{Обобщение арифметики целых чисел}
\noindent Мы  коснемся  теперь  общих  понятий  теории  делимости.  Это  предпо­
лагает  введение  точных  определений  основных  понятий,  без  которых 
математик не может работать, и выявление их основных свойств.  Что­
бы избежать появления длинного списка определений/утверждений, мы
\pagebreak
выбрали  в  этом  разделе  конкретный  пример  для  изучения  — кольцо 
целых чисел Гаусса, сообщая предварительно минимальное количество 
сведений,  позволяющих  работать с  этим  объектом.  Другие  результа­
ты, относящиеся к свойствам делимости, будут изложены в следующем 
разделе.

Сразу же  уточним, что определения,  которые последуют,  ориенти­
рованы  на теорию  делимости,  отдающую предпочтение  элементам.  В 
общих чертах будут рассмотрены  алгебраические  структуры,  в  кото­
рых 
\textbf{основная  теорема  арифметики} 
справедлива для  их 
\textbf{элемен­тов}. 
Существуют  и  другие  теории  делимости  (кольца Дедекинда),  в 
них основная теорема арифметики справедлива для 
\textbf{идеалов}.
\subsection{Делимость и неприводимые элементы}

\noindent \textbf{(2) Определение}
Элемент  \textit{х}   коммутативного  и   унитарного\footnote{В  этой  книге  (за исключением беглого упоминания других ситуаций) все рас­
сматриваемые кольца предполагаются коммутативными и унитарными (т.е. 
с
 еди­
ницей. — 
Прим. ред).
 Добавим, что для «хорошей» теории делимости кольца долж­
ны быть целостными. Можно было бы попытаться ограничится только целостными
кольцами, но такая точка зрения чересчур стеснительна в отношении таких поня­
тий,  как  нётеров характер, простой идеал или  максимальный идеал...  Использо­
вание  свойства целостности  (без  делителей  нуля)  или  необязательно целостности
будет уточняться по мере необходимости.}
  кольца  А   называется
\textbf{единицей} 
А   и л и  обратимым в   А ,  если  найдется  такой элемент  $у \in А$,
что $ху  =  ух$ =  1. 
Множество  всех единиц в   А   является
 \textbf{мультиплика­тивной группой} 
и  обозначается $U(A)$
\newline

\noindent \textbf{(3) Определение}

$(i)$  Элемент  $а$  кольца  $А$   делит  $b$  (в  $A$ ),  если  существует  $c$  $\in$   $A$   та­
кой,  что  $Ь  =  са$.  Будем   говорить  также,  что  $b$  кратно  $a$,  и   отмечать
этот  факт  в   виде  $a$  |  $b$;  или,  если  хотим  уточнить  кольцо  $A$ ,  $a|_{A}b$.
Это  свойство делимости может быть эквивалентным образом выраже­
но  в   терминах  идеалов  (обратим  внимание  на  перевернутость  «$\subset$»  по 
отношению к «I») 
следующим  образом: $a$  |  $b$ эквивалентно $Ab \subset Aa$.

$(ii)$ Заметим 
относительно свойства делимости «|»,  что два элемента
$a$  и  $b$,  удовлетворяющие равенству $Aa=Ab$,  неразличимы .  Это отноше­
ние между $a$ и  $b$ является эквивалентностью. В  этом случае мы говорим,
что $a$  и  $b$
 ассоциированные 
элементы,  и  применяем запись $a ~ b$ ,   или,
если требуется уточнить кольцо $A$ ,  
$a \sim_{A}b$ .  Если кольцо $A$  без делителей

