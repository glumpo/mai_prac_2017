\noindent Можно доказать, что в этом алгоритме являются цифрами ввиду
соотношений: $0\leqslant carry < b\;aux\leqslant(b —1)b < b$. Этот алгоритм легко распространяется на (наивное) умножение двух чисел произвольной
длины.
\def\tmpsuka{Вычисление $v\times(u_{m}\ldots u_{0})_b,0\leqslant v<b$}

\begin{lstlisting}[mathescape=true,caption=\tmpsuka]
carry=0;
for (i=0; i <= m; ++i){
	//$\;\;(w_{i-1}+carry\times b'=(u_{i-1}\ldots u_{0})_{b}\times v$
	aux = v*u$_{i}$+carry;$\;$w$_{i}$=aux%b;$\;$carry=aux/b;
}
w$_{m+1}$=carry;
\end{lstlisting}

\textbf{d.} Остаток $r = (u_{n},\ldots,u_{0})_{b}$ mod $v$, где $v$ — цифра, вычисляется по
алгоритму Горнера, где полагается $r_{n+1} = 0$ и $r_{i} = (r_{i+1}\times b + u_{i})$ mod $v$
при $i = n,n—1,\ldots,0$. Искомое значение есть $r_{0}$, что доказывается
с помощью $r_{i}=(u_{n}\ldots u_{i})_{b}$ mod $v$. Можно видеть, что алгоритм 22-A
вычисляет частное $\lfloor u/v\rfloor$, а алгоритм 22-B вычисляет остаток $u$ mod $v$.
Распространение на числа произвольной длины
\begin{table}[h!]
\centering
\begin{tabular}{|l|l|}
\hline
\underline{\textbf{A.} Вычисление частного} & \underline{\textbf{B.} Вычисление остатка} \\
{\begin{lstlisting}[mathescape=true, frame=none]
r = 0;
for (j = n;j >= 0; j--){
	aux = r*b + u$_{n}$;
	q$_{j}$ = aux/v;
	r = aux % v;
}	
\end{lstlisting}} & 
{\begin{lstlisting}[mathescape=true, frame=none]
r = 0;
for (j = n; j >= 0; j--){
	r = (r*b + u$_{j}$) % v;
}
\end{lstlisting}}	
\\ \hline

\multicolumn{2}{c}{{\large Деление $(u_{m}\ldots u_{0})_b$ на $v$ при $0\leqslant v<b$}}
\end{tabular}
\end{table}
этих двух операций не является таким простым, как в случае умноже­ния. В действительности, это тема дальнейших упражнений.
\\\\
\noindent\textbf{37. Деление в арифметике повышенной точности}
\\

Предположим, что $v$ имеет \textit{нормализованную } запись (т.е. $v_{m} \neq 0$)
и, кроме того, $n \geqslant m$ (запись не обязательно нормализованная). Надо
найти частное и остаток с помощью массива $q = (q_{n-m}\ldots q_{1}q_{0}$ и $r = (r_{m}\ldots r_{1}r_{0})$. Заметим, что условие $u/v < b$ можно записать также в
виде $(u_{m+1}\ldots u_{1})_{b}<(v_{m}\ldots v_{1}v_{0})_{b}$ В самом деле,
$$(u_{m+1}\ldots u_{0})/v<b \Leftrightarrow (u_{m+1}\ldots u_{0})/b<v \Leftrightarrow \lfloor(u_{m+1}\ldots u_{0})/b\rfloor<v$$
\newpage
\noindent и $\lfloor(u_{m+1}\ldots u_{0})/b\rfloor=(u_{m+1}\ldots u_{1})$. Вот общий алгоритм вычисления пары (частное, остаток):

\begin{leftbar}
\begin{lstlisting}[frame=none, mathescape=true]
$(r_{n+1}r_{n}\ldots r_{1}r_{0}\;=\;(0u_{n}\ldots u_{1}u_{0}$;
for (j = n-m; j >= 0; j--){
	\\$\;\;u=(q_{n-m}\ldots q_{j+1})\times vb^{j+1}+r\;\;\text{и}\;\;r<vb^{j+1}$
	q$_{j}$=$(r_{m+j+1}r_{m+j}\ldots r_{j})$/v;
	$(r_{m+j+1}\ldots r_{j}$ = $(r_{m+j+1}\ldots r_{j})$ - $q_{j}v$;
}	
\end{lstlisting}
\end{leftbar}
Частное, используемое в цикле, является частным двух чисел, удо­влетворяющих порождающему условию (5) (и, следовательно, каждое
вычисляемое $q_{j}$ является цифрой!). Чтобы убедиться в этом, достаточно заметить, что перед присваиванием $q_{j}$ имеем:
$$ r<vb^{j+1}\;\;\;\Longleftrightarrow\;\;\;(r_{m+j+1}\ldots r_{j+1})_{b}<v,$$
что, по предыдущему замечанию, равносильно $(r_{m+j+1}\ldots r_{j})/v<b$
\\\\
\noindent\textbf{38. Вычисление частного методом проб и ошибок}
\\

\textbf{a.} Предположения (5) включают неравенство $(u_{m+1}\ldots u_{1})_{b}\;<$\linebreak
$<\;(v_{m}\ldots v_{1}v_{0})_{b}$, откуда u$u_{m+1}\leqslant v_{m}$. Читатель может проверить, что
тогда:
$$\text{min}(\lfloor u_{m+1}b+u_{m})/v_{m}\rfloor,\;b-1)=\left\lbrace\begin{array}{ll}
b-1,&\text{если}\;\;u_{m+1}=v_{m},\\
\lfloor(u_{m+1}b+u_{m})/v_{m}\rfloor&\text{иначе,}
\end{array}\right.$$
и проверка для определения наименьшего из двух чисел сводится к про­верке $u_{m+1}=v_{m}$

Докажем сначала, что $q\leqslant\widehat{q}$. Так как $q\leqslant b — 1$ (предположение (5)),
можно очевидно предположить, что $\widehat{q}=\lfloor(u_{m+1}b+u_{m})/v_{m}\rfloor$. Запишем:
$$u=u_{m+1}b^{m+1}+u_{m}b^m+\alpha,\;\;\;\;v=v_{m}b^m+\beta,$$
с $\beta<b^m$ и $\alpha<b^m$. Тогда получим $u-\widehat{q}v=(u_{m+1}b+u_{m}-\widehat{q}v_{m})b^m+$\linebreak
$\alpha-\widehat{q}\beta$. По определению $\widehat{q}\;u_{m+1}b+u_{m}<v_{m}(\widehat{q}+1)$ и, следовательно, $u_{m+1}b+u_{m}-u_{m}\widehat{q}\leqslant v_{m}-q$, что влечет
$$u-\widehat{q}v\leqslant(v_{m}-1)b^m+\alpha-\widehat{q}\beta\leqslant(v_{m}-1)b^m+\alpha\leqslant v_{m}b^m+(\alpha-b^m)<v_{m}b^m\leqslant v,$$
и неравенство $u-\widehat{q}v<v$ дает $q\leqslant\widehat{q}$.

Для доказательства второго неравенства введем $\delta=\widehat{q}-q$. Используя неравенство для $\widehat{q}$: $\widehat{q}\leqslant\frac{u_{m+1}b+u_{m}}{v_{m}
}\leqslant\frac{u}{v_{m}b^m}$ и для $q$: $q>\frac{u}{v}-q$,
