\documentclass{mai_book}

\defaultfontfeatures{Mapping=tex-text}
\setmainfont{DejaVuSerif}
\setdefaultlanguage{russian}

%\clearpage
\setcounter{page}{270} % ВОТ ТУТ ЗАДАТЬ СТРАНИЦУ
%\setcounter{thesection}{5} % ТАК ЗАДАВАТЬ ГЛАВЫ, ПАРАГРАФЫ И ПРОЧЕЕ.
\setcounter{equation}{10}
% Эти счетчики достаточно задать один раз, обновляются дальше сами
% \newtop{ЗАГОЛОВОК}  юзать чтобы вручную поменть заголовок вверху страници

\begin{document}


\begin{center} %Решения упражнений
\ \newline\ \newline
\Large \textbf{Решения упражнений}\\
\ \newline
\end{center}
\noindent\textbf{1. Десятичные цифры простых чисел}\\
\ \newline
\hspace*{15pt}Пусть $l$ -- длина $B$, т.е. $10^{l-1}\leqslant B <10^l$\: и \:$k\geqslant 0$. Тогда числа вида\linebreak
$n\cdot10^{k+1} + B\cdot10^k + c$ являются числами десятичная запись которых\linebreak
содержит последовательность $В$, если $ 0 \leqslant c < 10^k.$ Если $k > 0$, то\linebreak
можно взять $с$ взаимно простым с 10 и тогда числа $10^{k+1}$ и $B\cdot10^k + c$\linebreak
будут взаимно просты. Поэтому можно применить теорему Дирихле. В\linebreak действительности оказывается, что существует бесконечно много про­-\linebreak
стых чисел, содержащих фиксированную последовательность цифр $B$ в\linebreak
заранее заданных позициях.\\
\ \newline
\noindent\textbf{2. Вычисление НОД}\\

\hspace*{0pt} Допустим, что $m>n$. Тогда $a^m - b^m = (a^n-b^n)a^{m-n}+(a^{m-n}-b^{m-n}b^n)$.\linebreak 
Это тождество и условия взаимной простоты $a$ и $b$ дает\newline
$$\text{НОД}(a^m-b^m,\ a^n-b^n)=\text{НОД}(a^{m-n}-b^{m-n},\ a^n-b^n)=$$
	$$=\text{НОД}(a^{m\ mod\ n}-b^{m\ mod\ n},\ a^n-b^n),$$
Повторяя это соотношение как в алгоритме Евклида, получим искомый результат.
\ \newline

\noindent\textbf{3. Алгоритм Евклида и непрерывные дроби}\\
\\
\hspace*{10pt} \textbf{a.}\ Пусть $a/b$ — рациональное несократимое число (с $b > 0$). Тогда\linebreak 
последовательность частных, полученных в алгоритме Евклида, при­-\linebreak
мененном к $а$ и $b$, дает разложение $а/b$ в непрерывную дробь. К тому\linebreak
же все частные, за исключением, быть может, первого, положительны,\linebreak
если используемое деление является обычным делением Евклида.\linebreak

 \textbf{b.}\ Рассмотрим непрерывные дроби $s_i = [c_i,c_{i+1},..,c_m]$ и\linebreak
$t_i=[d_i,d_{i+1},...,d_n]$. Тогда, очевидно, имеем $s_i=c_i+1/s_{i+1}$ и \linebreak
аналогичное соотношение для $t$ и $d$. Отсюда получаем $s_i>1$ для $i<m$,\linebreak
и, следовательно, $[s_i] = c_i$. кроме того, по предположению, $s_0=$\linebreak
$=[c_0,c_1,...,c_m]=[d_0,d_1,...,d_n]=t_0$. Используя предыдущее соотношение,\linebreak
получаем $c_0=d_0$ (целые части $s_0$ и $t_0$) и $s_1=t_1$. Затем постепенно\linebreak \newpage

%===============================================================
% 271 страница

%\newtop{Решения упражнений} %!!!!!!!!!!!!!!!!!!!!

\noindent доказываем, что $c_i=d_i$. Этот процесс заканчивается на наименьшем \linebreak
из чисел $m$ и $n$. Допустим, что это $m$. Тогда $s_m=t_m$ и $c_m=d_m$ через\linebreak
предшествующую рекуррентность. Кроме того, $s_m=c_m$ по определе­нию $s$.\linebreak
В результате\:\: $c_m=d_m$\:\: и\:\: $m=n$.\newline
\ \newline
\hspace*{15pt}\textbf{c.} Достаточно рассмотреть последние деления алгоритма Евклида:\nolinebreak $$r_{n-2}=r_{n-1}a_{n-1}+1\quad\text{ и }\quad r_{n-1}=1\times r_{n-1} \quad\text{ с\: } r_{n-1}>1.$$
Последнее деление можно заменить на следующее:\:\: $r_{n-1}=1\times (r_{n-1}-1)+1$ \linebreak
без появления нулевого частного и закончить деление на $1=1\times1$.\linebreak
Это означает, что $$[c_0,c_1,c_2,...,c_n]=[c_0,c_1,c_2,...,c_n-1,1],\quad\text{если\: } c_n > 1.$$

Это эквивалентно факту, что любое целое число $a$ допускает ровно два
разложения в непрерывную дробь: $[a]$ и $[a-1,1]$.\newline\\
\hspace*{15pt}\textbf{d.} Вот часть ответа (развиваемая потом в упражнении 6):
\begin{gather*}
	[a_0]=a_0,\quad [a_0,a_1]=\frac{a_0a_1+1}{a_1},\quad
	[a_0,a_1,a_2]=\frac{a_0a_1a_2+a_0a_1+a_1a_2}{a_1a_2+1}...\text{\:,}\\
	F_4/F_3=[1,1,2]\quad \text{и}\quad F_5/F_4=[1,1,1,2].
\end{gather*}
\\
\noindent\textbf{4. Многочлены континуаты}\\
\\
\hspace*{15pt}\textbf{a.} Легко видеть, что $F_n=K_n(1,...,1)$. Это означает, что в $K_n$\linebreak
имеется $F_n$ слагаемых.\\
\\
\hspace*{15pt}\textbf{b.} Нетрудно проверить, что свойство верно для $n=$ —1, 0, 1.\linebreak
Рассмотрим моном $X_1...X_n$ . Можно разделить исключения примыкающих\linebreak
пар на две категории: те, которые исключают пару $X_nX_{n-1}$, и те,\linebreak
которые оставляют $X_n$. По предположению индукции, первая категория\linebreak
дает все одночлены от $K_{n-2}$, а вторая — все одночлены от $K_{n-1}$,\linebreak
умноженные на $X_n$. Отсюда результат (обнаруженный впервые Эйлером).\linebreak
\\
\hspace*{15pt}\textbf{c.} Следовательно, континуанты обладают зеркальной симметрией:\linebreak
$K_n(X_1,...,X_n)=K_n(X_n,...,X_1)$. Можно определить последователь­-\linebreak
ность $K_n$ следующим образом:
$$K_n=X_1K_{n-1}(X_2,...,X_n)+K_{n-2}(X_3,...,X_n).$$
Соотношение, которое будет использоваться в следующих упражне­ниях.\newpage

%========================================================================
%272
Требуемое тождество доказывается легко. Оно указывает, что в\linebreak
по­ле рациональных дробей
$$ \frac{K_n(X_1,...,X_n)}{K_{n-1}(X_2,...,X_n)}=[X_1,X_2,...,X_n]=X_1+\frac{1}{X_2+...}.$$
Если $r_{i-1}=r_ia_i+r_{i+1}$, где $r_n=1$ и $r_{n+1}=0$ есть последовательность\linebreak
делений алгоритма Евклида, примененного к целым взаимно простым\linebreak
числам, то $r_i=K_{n-i}(a_{i+1},...,a_n)$.\newline
\\
\noindent\textbf{5. Континуаты (продолжение)}\\
\\
\hspace*{15pt}\textbf{a.} Нетрудно доказать, используя индукцию, что рассматриваемое\linebreak
произведение равно матрице
$$\begin{pmatrix}
	K_n(X_1,...,X_n)& K_{n-	1}(X_1,...,X_{n-1})\\
	K_{n-1}(X_2,...,X_n)& K_{n-2}(X_2,...,X_{n-1})
\end{pmatrix}$$
\noindent Этот результат дает другое доказательство зеркальной симметричности\linebreak
многочленов континуант.\newline
\\
\hspace*{15pt}\textbf{b.} Вычислив определитель, находим искомое равенство (с точно­стью\linebreak
до индексов). Следовательно, значения $K_n(a_1,...,a_n)$ и\linebreak
$K_{n-1}(a_2,...,a_n)$ дают взаимно простые целые числа.\\
\\
\hspace*{15pt}\textbf{c.} Простая индукция показывает, что искомое произведение равно
$$\begin{pmatrix}
	K_{n+2}(X_1,...,X_{n+2})& K_{n+1}(X_2,...,X_{n+2})\\
	K_n(X_1,...,X_n)& K_{n-1}(X_2,...,X_n)
\end{pmatrix}$$
\noindent Определитель этой матрицы (что и требовалось доказать) равен
\begin{eqnarray*}
	X_{n+2}\Big( K_{n-1}(X_2,...,X_n) K_{n+1}(X_1,...,X_{n+1})-\quad\\
	-K_n(X_1,...,X_n)K_n(X_2,...,X_{n+1}) \Big),
\end{eqnarray*}
т.е.\:\: $(-1)^{n+1}X_{n+2}$\:\: по предыдущему пункту\\
\\
\noindent\textbf{6. Разложение в непрерывную дробь}\\
\\
\hspace*{15pt}\textbf{a.} Это решение уже намечено в упражнении 3:
$$[a_0]=a_0,\quad [a_0,a_1]=\frac{a_0a_1+1}{a_1},\quad [a_0,a_1,a_2]=\frac{a_0a_1a_2+a_0a_1+a_1a_2}{a_1a_2+1}...$$
\newpage

%===================================================================
%273
\noindent Формула выглядит так:
$$[a_0,a_1,...,a_n]=\frac{K_{n+1}(a_0,a_1,...,a_n)}{K_n(a_1,...,a_n)},$$
и доказывается по индукции. Попутно использовалось соотношение\linebreak
$[a_0,...,a_n]=[a_0,[a_1,...,a_n]]$ и равенство, выражающее зеркальность\linebreak
многочленов континуант.\\
\\
\hspace*{15pt}\textbf{b.} Требуемое соотношение вытекает прямо из определения \linebreak
последовательностей $(a_i)$ и $(x_i)$. Запишем
\begin{equation}
	\begin{split}
		|x-[a_0,...,a_n]|&=|[a_0,...,a_n,x_{n+1}]-[a_0,...,a_n]|=\\
		&=\bigg|\frac{ K_{n+2}(a_0,...,a_n,x_{n+1} }{ K_{n+1}(a_1,...,a_n,x_{n+1} }
		-\frac{ K_{n+1}(a_0,...,a_n) }{K_n(a_1,...,a_n)} \bigg|= \\
		&=\frac{1}{ K_n(a_1,...,a_n)K_{n+1}(a_1,...,a_n,x_{n+1}) }
	\end{split}
\end{equation}
Очевидно, $x_{n+1}$ положительно (по построению) и континуанты возра­стают\linebreak
(хорошо видно, каким образом ), следовательно, получаем
\begin{equation}
	\big|x-[a_0,...,a_n]\big|\leqslant \frac{1}{K_n(a_1,...,a_n)^2}<\frac{1}{2K_n(a_1,...,a_n)},
\end{equation}

\noindent Соотношение, которое впоследствии понадобится. Это доказывает\linebreak 
схо­димость $x$. Можно также констатировать, что подходящие\linebreak веществен­ные числа дают очень хорошие аппроксимации указанного чис-\linebreak
ла. В действительности, как увидим в дальнейшем, они даже наилучшие.\newline
\\
\hspace*{15pt}\textbf{c.} Установленная в пункте \textbf{(a)} формула дает значение в неприводи­мой\linebreak 
форме, ибо, согласно упражнению 5, вводимые туда числа взаимно\linebreak
просты. Остается их только вычислить, что и сделаем, применяя рекур­-\linebreak
рентное определение континуант и значений многочленов. Приводим\linebreak
совмещенные вычисления числителя и знаменателя, чтобы\linebreak 
минимизи­ровать необходимые операции.
\begin{lstlisting}[mathescape=true]
$K\longleftarrow1;$ $K'\longleftarrow a_n$;
for (int i=n-1; i>=0; i=i-1){
	$(K',K)\longleftarrow(a_iK'+K,K')$;
}
\end{lstlisting}
%\newline
В начале итерации для значения $i$ имеем инвариантное отношение
$$ \frac{K'}{K}=[a_{i+1},...,a_n] $$
\newpage

%==================================================================
%274
\noindent Этот алгоритм(который оперирует с коэффициентами разложения),\linebreak
разумеется, не должен использоваться для вычисления последователь-­\linebreak
ности подходящих дробей, так как каждый промежуточный этап не\linebreak
дает первые подходящие дроби.\newline
\hspace*{15pt}Если необходимо вычислить последовательность подходящих дро­бей,\linebreak
то можно использовать потоковый алгоритм разложения. Для это­го \linebreak
используют две рекуррентные последовательности, определяемые через\nolinebreak
\begin{center}
	$P_0=1,\quad P_1=a_0,\quad P_n=a_nP_{n-1}+P_{n-2},$\\
	$Q_0=0,\quad Q_1=1,\quad Q_n=a_nQ_{n-1}+Q_{n-2}.$
\end{center}

\noindent Итак, определяемые последовательности удовлетворяют соотношению\linebreak
$[a_0,...,a_n]=P_{n+1}/Q_{n+1}.$ \quad Это нас приводит к алгоритму 5.
\begin{center}
\begin{lstlisting}[mathescape=true][xrightmargin=1cm]
	int a;
	scanf("%d", &a);
	$P\longleftarrow1;$ $P'\longleftarrow a;$ $Q\longleftarrow 0;$ $Q'\longleftarrow 1;$
	printf("%d$\text{ }$%d\n", $P',\text{ }Q'$);
	while(){
		scanf("%d", &a);
		$(P',P)\longleftarrow (aP'+P,P');$ $(Q',Q)\longleftarrow(aQ'+Q,Q');$
		printf("%d$\text{ }$%d\n"$,$ $P',\text{ }Q'$);
	}
\end{lstlisting}
\textbf{Алгоритм 5.} Вычисление текущих подходящих дробей
\end{center}
\ \linebreak
\hspace*{15pt}\textbf{d.} Число золотого сечения удовлетворяет уравнению $x=1+1/x$.\linebreak
Следовательно, его разложение в непрерывную дробь есть $[1,1,1,1,...]$.
Последовательность подходящих дробей, согласно пункту \textbf{(a)} упражне­ния 4\linebreak
и предшествующему пункту \textbf{(a)} упражнения 6, есть отношения\linebreak
последовательных чисел Фибоначчи: $F_2/F_1,F_3/F_2,F_4/F_3,...$\newline
\\
\noindent\textbf{7. Аппроксимация вещественных чисел с помощью \newline \hspace*{13pt}непрерывных дробей}\\
\\
\hspace*{15pt}\textbf{a.} Выражение $p_{n+1}q_{n-1}-q_{n+1}p_{n-1}=K_{n+2}K_{n-1}-K_{n+1}K_n$ есть знак\linebreak 
$(-1)^{n+1}$, откуда следует результат. Выражение $p_{n+1}q_n-q_{n+1}p_n=$\linebreak
$K_{n+2}K_n-K_{n+1}K_{n+1}$ есть знак $(-1)^n$. Факт, что $x$ находится между двумя сходящимися последовательностями показывает, что эти последовательности сходятся к $x$.\newline
\\
\hspace*{15pt}\textbf{b.} Равенство (11) утверждает, что
$$\Bigg|x-\frac{p_n}{q_n}\Bigg| = \big|x-[a_0,...,a_n]\big|=\frac{1}{K_n(a_1,...,a_n)K_{n+1}(a_1,...,a_n,x_{n+1})}.$$\newpage


%================================================================
%275
\noindent
Используя двустороннюю оценку $a_{n+1}<x_{n+1}<a_{n+1}$, а также, что каждая\linebreak
континуата --- возрастающая функция любой из этих переменных,\linebreak получаем
$$\frac{1}{q_nK_{n+1}(a_1,...,a_n,a_{n+1}+1)}<\bigg|x-\frac{p_n}{q_n}\bigg|<\frac{1}{q_nq_{n+1}}$$
Но $K_{n+1}(X_1,...,X_n,X_{n+1}+1)$. Действительно, первая континуанта, по\linebreak
определению, равна $K_{n+2}(X_1,...,X_{n+1},1)$, тогда как вторая равна по\linebreak
определению $K_{n+1}(X_1,...,X_{n+1})+K_n(X_1,...,X_n)$, \:\:т.е. второй континуанте.\linebreak
Следовательно,
$$ K_{n+1}(a_1,...,a_n,a_{n+1}+1)\leqslant K_{n+2}(a_1,...,a_{n+1},a_{n+2})=q_{n+2},$$
откуда и получается требуемая оценка. Другие неравенства выводятся\linebreak
немедленно.\newline
\\
\hspace*{15pt}\textbf{c.} Зная, что $|x-p_n/q_n|<1/2q_n$ (соотношение (12)), и что\linebreak
$|p_n/q_n-p/q|\geqslant1/q_n$ (так как $q=q_n$), получаем, что $|x-p/q|>1/2q_n$,
что и дает искомое неравенство.\newline
\\
\hspace*{15pt}\textbf{d.} Матричные соотношения показывают, что матрица 
$\begin{pmatrix}
	p_n&  p_{n-1}\\
	q_n& q_{n-1}\\
\end{pmatrix}$\linebreak
обратима над $\mathbb{Z}$. Следовательно, можно записать $p=\alpha p_n+\beta p_{n-1}$ и \linebreak
$q=\alpha q_n+\beta q_{n-1}$, где $\alpha$ и $\beta$ — целые. Однако $q<q_n$, откуда следует,\linebreak
что $\beta$ — не нуль и что $\alpha$ и $\beta$ имеют противоположные знаки. Кроме\linebreak
того, согласно доказанному в вопросе (а), $p_n-xq_n$ и $p_{n-1}-xq_{n-1}$ тоже\linebreak
имеют противоположные знаки. Отсюда $\alpha(p_n-xq_n)$ и $\beta(p_{n-1}-xq_{n-1}$\linebreak
одного\:\: знака,\:\: а\:\: потому
$$|p-xq|=|\alpha(p_n-xq_n)+\beta(p_{n-1}-xq_{n-1})|\geqslant|\beta|\cdot|p_{n-1}-xq_{n-1})|>|p_n-xq_n|$$
\\
\noindent\textbf{8. Факториал и простые числа} \newline 
\\
\hspace*{15pt}\textbf{a.} Пусть $n=p_1^{\alpha_1}...p_r^{\alpha_r}$  — разложение $n$ на простые множители. \linebreak
Предположим сначала, что $r>1$. Тогда $p_i^{\alpha_i}<n$ и, следовательно, делит\linebreak
$(n-1)!$. Если $r=1$, то так как $n$ — непростое, это приводит к тому,\linebreak
что $\alpha_1>1$. Если $\alpha_1=2$, то числа $p_1$ и $2p_1$ встречаются в разложении\linebreak
$(n-1)!$, кроме случая $p_1=2$. Если $\alpha_1>2$, то числа $p_1$ и $p_1^{\alpha_1}$ меньше $n$\linebreak
и, следовательно, появляются в разложении $(n-1)!$.\newline
\\
\hspace*{15pt}\textbf{b.} Действительно, очевидно, что всякое целое число $k$ из интервала\linebreak
$[2, n+1]$ делит $(n+1)!+k$. Отсюда мы нашли те целые числа, которые\linebreak
искали.\newpage

\end{document}
