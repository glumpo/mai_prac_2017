\documentclass{mai_book}

\defaultfontfeatures{Mapping=tex-text}
\setmainfont{DejaVuSerif}
\setdefaultlanguage{russian}

%\clearpage
\setcounter{page}{270} % ВОТ ТУТ ЗАДАТЬ СТРАНИЦУ
\setcounter{thesection}{5} % ТАК ЗАДАВАТЬ ГЛАВЫ, ПАРАГРАФЫ И ПРОЧЕЕ.
% Эти счетчики достаточно задать один раз, обновляются дальше сами


% \newtop{ЗАГОЛОВОК}  юзать чтобы вручную поменть заголовок вверху страници

\begin{document}
\newtop{Решения упражнений}
%\chapter*{\Large\hfill Решения упражнений\hfill}
\begin{center} %Решения упражнений
\ \newline\ \newline
\Large \textbf{Решения упражнений}\\
\ \newline
\end{center}
\noindent\textbf{1. Десятичные цифры простых чисел}\\
\ \newline
\hspace*{15pt}Пусть $l$ -- длина $B$, т.е. $10^{l-1}\leqslant B <10^l$\: и \:$k\geqslant 0$. Тогда числа вида\linebreak
$n\cdot10^{k+1} + B\cdot10^k + c$ являются числами десятичная запись которых\linebreak
содержит последовательность $В$, если $ 0 \leqslant c < 10^k.$ Если $k > 0$, то\linebreak
можно взять $с$ взаимно простым с 10 и тогда числа $10^{k+1}$ и $B\cdot10^k + c$\linebreak
будут взаимно просты. Поэтому можно применить теорему Дирихле. В\linebreak действительности оказывается, что существует бесконечно много про­-\linebreak
стых чисел, содержащих фиксированную последовательность цифр $B$ в\linebreak
заранее заданных позициях.\\
\ \newline
\noindent\textbf{2. Вычисление НОД}\\

\hspace*{0pt} Допустим, что $m>n$. Тогда $a^m - b^m = (a^n-b^n)a^{m-n}+(a^{m-n}-b^{m-n}b^n)$.\linebreak 
Это тождество и условия взаимной простоты $a$ и $b$ дает\newline
$$\text{НОД}(a^m-b^m,\ a^n-b^n)=\text{НОД}(a^{m-n}-b^{m-n},\ a^n-b^n)=$$
	$$=\text{НОД}(a^{m\ mod\ n}-b^{m\ mod\ n},\ a^n-b^n),$$
Повторяя это соотношение как в алгоритме Евклида, получим искомый результат.
\ \newline



\noindent\textbf{3. Алгоритм Евклида и непрерывные дроби}\\
\\
\hspace*{10pt} \textbf{a.}\ Пусть $a/b$ — рациональное несократимое число (с $b > 0$). Тогда\linebreak 
последовательность частных, полученных в алгоритме Евклида, при­-\linebreak
мененном к $а$ и $b$, дает разложение $а/b$ в непрерывную дробь. К тому\linebreak
же все частные, за исключением, быть может, первого, положительны,\linebreak
если используемое деление является обычным делением Евклида.\linebreak

 \textbf{b.}\ Рассмотрим непрерывные дроби $s_i = [c_i,c_{i+1},..,c_m]$ и\linebreak
$t_i=[d_i,d_{i+1},...,d_n]$. Тогда, очевидно, имеем $s_i=c_i+1/s_{i+1}$ и \linebreak
аналогичное соотношение для $t$ и $d$. Отсюда получаем $s_i>1$ для $i<m$,\linebreak
и, следовательно, $[s_i] = c_i$. кроме того, по предположению, $s_0=$\linebreak
$=[c_0,c_1,...,c_m]=[d_0,d_1,...,d_n]=t_0$. Используя предыдущее соотношение,\linebreak
получаем $c_0=d_0$ (целые части $s_0$ и $t_0$) и $s_1=t_1$. Затем постепенно\linebreak \pagebreak


% 271 страница
доказываем, что $c_i=d_i$. Этот процесс заканчивается на наименьшем \linebreak
из чисел $m$ и $n$. Допустим, что это $m$. Тогда $s_m=t_m$ и $c_m=d_m$ через\linebreak
предшествующую рекуррентность. Кроме того, $s_m=c_m$ по определе­нию\linebreak 
$s$. В результате $c_m=d_m$ и $m=n$.\newline





\end{document} 
