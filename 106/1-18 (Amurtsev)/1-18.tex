\hspace{3.4in}АЛГЕБРАИЧЕСКАЯ

\hspace{3.4in}АЛГОРИТМИКА
\newpage
\hspace{3.0in} LOGIQUE MATHEMATIQUES

 \hspace{3.1in} \textbf{I N F O R M A T I Q U E}
\\\\\\
{\Huge \textbf {ALGORITMIQUE}}
\\
\newline {\Huge \textbf {ALGEBRIQUE}}
\\\\
\Large{ \textbf{Avec exercices corriges}}
\\\\\\\\\\
\normalsize {Patrice NAUDIN}
\newline \textit {\normalsize {Maitre de conferences a I'universite de Bordeaux I}}
\\\\
\normalsize {Claude QUITTE}
\newline \textit {Maitre de conferences a I'universite de Poitiers}
\\\\
\normalsize {Preface de Francis SERGERAERT}
\\\\\\\\\\\\\\\\\\\\\\\\\\\\\\\\\\\\\\\\\

\hspace{2.0in} MASSON Paris Milan Barcelone Bonn 1992
\newpage
\large {Патрис НОДЕН},    \large {Клод КИТТЕ}
\\\\\\\
{\Huge \textbf {АЛГЕБРАИЧЕСКАЯ}}
\\
\newline {\Huge \textbf {АЛГОРИТМИКА}}
\\\\\
\textbf{С упражнениями и решениями}
\\\\\\
\newline Предисловие Франсиса Сержераера
\\\\\\\
Перевод с французского
\newline \textit{В.А. Соколова}
\\\\
под редакцией
\newline \textit {Л.С. Казарина}
\\\\\\\\\
\small {Рекомендовано Научно-методическим советом по прикладной математике
\newline Учебно-методического объединения университетов для использования в учеб-
\newline ном процессе для студентов вузов по специальности «Прикладная математи-
\newline ка» и «Прикладная математика и информатика»}
\\\\\\\\\\\\\\\\\\\\\\\\
\includegraphics{2.png}

\large {МОСКВА «МИР» 1999}
\newpage

УДК 512+519.6

   ББК 22.14+22.19

\hspace{1.0cm} Н72
\\

\hspace{1.0cm} \textbf{Ноден П., Китте К.}

Н72 \hspace{0.5cm} \small {Алгебраическая алгоритмика (с упражнениями и решениями):} 

\hspace{1.0cm} \small {Пер. с франц. — М.: Мир, 1999. — 720 с., ил.}

\hspace{1.5cm} ISBN 5-03-003318-1

\hspace{1.5cm} \small {Книга известных французских математиков — это по существу эн-

\hspace{1.0cm} циклопедия алгоритмов алгебры и теории чисел от Евклида и до на-

\hspace{1.0cm} ших дней. В ней прослеживается общая идея — представить основные

\hspace{1.0cm} алгебраические структуры и концепции в виде объектов, поддающих-

\hspace{1.0cm} ся машинной обработке. Главными для авторов являются два вопроса:

\hspace{1.0cm} что значит вычислить математический объект и как его вычислить

\hspace{1.0cm} наиболее эффективно.

\hspace{1.5cm} Изложение отличается методическими достоинствами: тщательный 

\hspace{1.0cm} отбор материала, многочисленные замечания теоретического и исто-

\hspace{1.0cm} рического характера, большое число упражнений с решениями в конце

\hspace{1.0cm} каждой главы.

\hspace{1.5cm} Для математиков-прикладников, для всех изучающих и применяю-

\hspace{1.0cm} щих компьютерную алгебру и информатику как учебное и справочное 

\hspace{1.0cm} пособие.}

\hspace{4.0in} ББК 22.14+22.19
\\

\hspace{2.4in} \includegraphics{1.png}
\\

\hspace{2.0cm} Издание осуществлено при поддержке Российского фонда 

\hspace{2.3cm}фундаментальных исследований по проекту 97-01-14001
\\\\

\hspace{2.5cm} Издание осуществлено в рамках программы помощи 

\hspace{2.9cm} издательскому делу «Пушкин» при поддержке 

\hspace{3.2cm} Министерства Иностранных Дел Франции

\hspace{4.0cm} и Посольства Франции в России
\\\\
 
\hspace{2.0cm} \textit {Редакция литературы по математическим наукам}
\\\\

ISBN 5-03-003318-1 (русск.) \hspace{2.1cm} © Masson, Paris, 1992 

 ISBN 2-225-82703-6 (франц.) \hspace{2.0cm} © перевод на русский язык,
 
\hspace{2.8in} «Мир», 1999
\newpage

\hspace{2.0cm} {\Large \textbf {Предисловие переводчика}

\hspace{2.6cm} {\Large \textbf {и редактора перевода}.
\\\\

    \normalsize {Со времени появления монографии Г. Биркгофа и Т. Барти «\textit {Современная прикладная алгебра}» (28 лет тому назад) сочетание прилагательных «\textit {прикладной}» и «\textit {абстрактный}» в применении к алгебре перестало шокировать или изумлять читателей. Прилагательное «\textit {современный}», разумеется, должно исчезнуть по мере разрастания теории и завоевания все новых территорий (переведенная недавно книга Р. Лидла и П. Пильца тому свидетельство). Более интересным является сопоставление компьютерной математики и алгебры, появившееся сравнительно недавно, а уж алгебраическая алгоритмика и вовсе должна вызвать ассоциацию с «\textit {маслом масляным}», ибо корни обоих понятий (алгебра и алгоритм), как известно, одни и те же. Тем не менее, несмотря на близкое родство, алгоритмика (читай, информатика) и алгебра длительное время развивались как бы обособленно, хотя слово «\textit {метод}» в учебниках по высшей алгебре вытеснялось постепенно словом «\textit {алгоритм}», а в информатику проникли модули и тензорные произведения. Вопросы сложности вычислений, явившиеся краеугольным камнем для информатики, позволили поставить весьма интересные чисто алгебраические задачи. Стало ясно, что в алгебре можно вычислять, даже если исходные вопросы касаются таких субстанций, как модуль, кольцо, группа. А раз можно вычислять, то нужно вычислять как можно быстрее, ибо уже студенту-первокурснику понятно, что «\textit грубая сила» машины не сможет выручить при решении некоторых совсем простых, на первый взгляд, задач.
    
   Многие алгоритмы алгебры, с которыми нам приходилось сталкиваться в жизни, становились известными из математического фольклора, другие изобретались самостоятельно (ибо посмотреть было негде). Потом появился изумительный Д. Кнут (Искусство программирования), пугающий своей фундаментальностью, куда все-таки приходилось заглядывать, невзирая на известные трудности. А что делать студенту? Книга «\textit {Алгебраическая алгоритмика}» явилась нам неожиданно во всем своем блеске благодаря знакомству с ее авторами, работавшими в Пуатье. Начав работать, мы оба поняли, что этой-то книги нам не хватало и как преподавателям, и как математикам-профессионалам.}
\newpage 
\setcounter{page}{6}% ВОТ ТУТ ЗАДАТЬ СТРАНИЦУ
С одной стороны, — это новый взгляд на вещи, казавшиеся устоявшимися. Например, что может быть проще алгоритма Евклида или возведения в степень? С другой, некоторая систематизация, азбука информатики (как мы мечтали 30 лет тому назад о каталоге полиномиальных алгоритмов!). Наконец, это философия программирования, изложенная на небольшом пространстве и без присущего сугубым профессионалам высокомерия.
В то же время книга читается (мы отмечаем это с некоторым ужасом) как художественная, беллетристическая литература. Чего стоит одно дерево Штерна — Броко из задачи 35 в главе II!

   Основной материал книги сопровождается огромным количеством упражнений (их около 300, но некоторые из упражнений содержат в себе от 3 до 5 задач), примеров, иллюстраций. Студента несомненно порадует то обстоятельство, что большинство упражнений приводится с решением (не сразу, а через несколько страниц). Пересечение с другими известными нам книгами по алгоритмике невелико.
Книга П. Нодена и К. Китте восполняет пробел, существовавший в литературе на русском языке (и по математике, и по информатике), и может быть использована для чтения специальных и основных курсов как для студентов-математиков, так и для информатиков. Кроме того, она, будучи использована в качестве дополнительного источника, окажется несомненно полезной при чтении курсов «Алгебра и теория чисел» и «Алгебра и геометрия».

   Следует отметить, что терминология, применяемая французскими математиками, не вполне совпадает с принятой в нашей литературе. Так, натуральный ряд начинается с нуля, а не с 1, как обычно принято у нас. Отсюда появление обозначения N*. Приведение матрицы к ступенчатому виду осуществляется по столбцам, а не по строкам, причем имени Гаусса авторы в этой связи не упоминают. Есть и другие мелкие детали, которые читатели, несомненно, заметят. Что ж, это заставляет читателя быть внимательным!

  Приятно отметить, что в подготовке к изданию перевода книги активное участие приняли наши ученики и студенты С.В. Монахов, С.М. Медведев (подготовившие оригинал-макет русского перевода), И.А. Сагиров, Л.В. Гудкова, А.В. Хлямков, А.Е. Скрипкин. Мы им выражаем нашу глубокую благодарность. Мы также признательны заведующему лабораторией теоретико-групповых методов профессору А.Л. Онищику за предоставление технических средств для верстки этой книги. Мы благодарны издательству MASSON, давшему согласие на передачу права на издание перевода книги издательству «Мир», и Посольству Франции в Москве включение русского издания книги
\newpage 
в программу «Пушкин», а также Российскому фонду фундаментальных исследований, поддержавшему проект перевода книги издательским грантом.

   В заключение мы хотим прямо сказать, что наш замысел вряд ли осуществился, если бы не помощь и активное сотрудничество наших французских друзей и коллег из университета Пуатье — авторов книги Патриса Нодена и Клода Китте. Они дали согласие на безвозмездное издание перевода их книги, любезно предоставили в наше распоряжение файлы с ее электронным вариантом и постоянно, на протяжении всего времени работы над переводом, поддерживали с нами связь, консультировали нас, полностью пересмотрели текст французского издания и внесли в него много исправлений и улучшений, которые учтены в русском издании.
   
   Только благодаря этому сотрудничеству наш совместный проект успешно завершен.
   
   Надеемся, что перевод этой примечательной книги будет интересен и полезен нашим читателям.


 \hspace{3.1in} \textit {Л.С. Казарин, В.А. Соколов,} 
   
\hspace{3.0in} \textit {Ярославль, 10 августа 1998 г.}
\newpage
\hspace{1.4cm} {\Large \textbf {Предисловие к русскому изданию}}
\newline\\
\hspace*{15pt}
\hspace*{15pt}
\\\\
Проект перевода Львом Казариным и Валерием Соколовым нашей кни-
ги «Алгебраическая алгоритмика», возникший более двух лет тому на-
зад, подошел к завершению. Мы особенно рады этому событию, ес-
ли учесть, что пройденный путь, вероятно, не был гладким. Основные очертания проект приобрел в результате нашего путешествия в Россию, которое мы предприняли весной 1996 года по приглашению Валерия Соколова от имени Ярославского государственного университета. Во время нашего пребывания в Ярославле Лев и Валерий оказали нам, как гостям университета, самый радушный прием. Лев сопровождал нас во всех поездках, которые Ярославский университет организовал для нас, чтобы мы открыли для себя маленький кусочек России — это то, что выходит за рамки профессиональной деятельности, но является, тем не менее, приятным (и необходимым) элементом любой поездки за границу. Благодаря Льву мы открыли ряд исторических уголков, входящих в «Золотое кольцо»; он опекал нас на первых шагах в нашей повседневной жизни иностранцев в России, что порой было очень непросто: вспомним автобус, который отвозил нас на вокзал в пятницу в 6 часов утра, когда все киоски были еще закрыты! «Автобус... А как насчет билетов?» А в Москве Аркадий Онищик, Сергей и Ирина Струнковы доказали нам на непостижимом для представителей Запада уровне, что российское гостеприимство вовсе не легенда. Обо всех этих встречах и людях мы сохраним самые волнующие воспоминания.

   Прервем здесь дифирамбы, хотя это вступление в действительности лишь в небольшой степени передает наше душевное состояние. Нелегкое, все-таки, это дело — сочинять предисловие к переводу книги, которую сами написали! Поговорим, однако, об идеях, которые мы хотели изложить в этой книге.

   В течение трех лет работы над ней (1989-91 гг.) нашей целью — амбициозной и одновременно наивной — было представить с одной конкретной точки зрения некоторые разделы алгебры, преподаваемые на старших курсах университета. В то время нам хотелось (и сейчас нам этого хочется более, чем когда-либо) изменений в изложении математики: не отказываясь от абстрактных понятий, использовать их так, чтобы гармонично сочетать теорию и практику.
\newpage
   Любой студент-математик знаком с основными понятиями, которые встречаются в арифметике: целые числа, сравнимость чисел по модулю п, многочлены и т.д. Но, чаще всего, эти объекты, по самой своей сути предназначенные для практики, определяются очень формально и абстрактно. Так ли уж необходимо вводить фактор-группу некоторой группы по отличной от нее подгруппе для изучения целых чисел, сравнимых по модулю п? Что усвоят наши студенты при таком подходе? Так ли уж неразумно опираться, когда это возможно, на практику? Ведь освоив вычисления по модулю целого числа (или по модулю многочлена), уже гораздо позже можно рассмотреть общий случай вычислений по модулю идеала, «более естественно» приводящий к понятию фактор-кольца. В самом деле, нет недостатка в конкретных проблемах, где может с успехом использоваться вычислительная практика: тест на простоту чисел Лукаса — Лемера, критерий Пепина, RSA-метод в криптографии, генератор случайных чисел и т.д. Мы рассматриваем все эти темы в нашей книге и убеждены, что они лучше подходят для обучения, чем «теорема о факторизации (переход к фактор-множеству) некоторого отображения».

   Что же изменилось с 1992 года, времени выхода нашей книги? В области математической алгоритмики на рынке появилось некоторое количество монографий, но это в основном узко специализированные книги научного характера, оказывающие весьма слабое влияние на преподавание математики. Что же касается информатики как таковой, то единственным плодом ее непрерывного бурного развития в данной области является увеличение числа систем формальных вычислений — часто настолько же мощных, насколько и непоследовательных, неприспособленных для научной работы из-за недостаточной строгости. Нынче можно осуществлять сложные вычисления в теории Галуа, в алгебраической геометрии, в теории чисел, в теории групп и т.д., но все это, как правило, недоступно для начинающего студента-математика. И против всех ожиданий, области, в которых системы формальных вычислений уже хорошо обкатаны — это, как ни странно, разделы курса алгебры — всегда излагаются в том же виде, в каком они были десятилетия назад. Использование этих прикладных программ осталось вне преподавания математики и касается, в основном, лишь научных исследований.

   Сообществу математиков, кажется, грозит опасность раскола на два чуждых друг другу мира: «теоретиков» и «алгоритмистов». Эта ситуация, на наш взгляд, была бы достойна сожаления. В то же время, во Франции — особенно в преподавании математики — нередко шли на отказ от догматических воззрений в пользу нового хорошего подхо-
\newpage
да к объяснению того или иного понятия. Преподавание (равно, как и научное исследование) может лишь обогатиться благодаря многообразию способов изложения материала: достаточно хотя бы раз в жизни прочитать студентам курс лекций по алгоритмике, чтобы убедиться в этом.

\hspace{3.8in} П. Ноден, К. Китте,
        
\hspace{3.3in} Пуатье, 2 февраля 1999 г.
\newpage
\hspace{1.4in} {\huge Предисловие}
\\\\\\\\
   Патрис Ноден и Клод Китте попросили меня написать предисловие к их книге, и я начну с благодарности им: быть вовлеченным в работу в приятной компании по случаю выхода в свет такого прекрасного произведения — это счастье.

   История математики очень увлекательна. Для сюжета, который нас занимает сегодня, существенны три персонажа. Евклид пишет свои «Начала» за три столетия до нашей эры, и с этой даты большинство историков ведут отсчет появления той математики, которую мы знаем сегодня, имеющей характер некой игры, заключающейся в том, чтобы делать интересные выводы из множества хорошо подобранных аксиом. Природа этих аксиом и соблюдаемые правила игры с самого начала не были достаточно ясны, и математика XIX века познала такое развитие, что вскоре основы здания были едва способны выдерживать всю конструкцию. Итогом этого было то, что называют кризисом оснований математики.

   Но тут появился Гильберт; он понял и талантливо объяснил, что математика может быть формализована. Гильберт описал работу математиков, как чисто формальную игру, манипулирующую множеством аксиом (комбинаторных объектов) с помощью доказательств (чисто комбинаторной техники) и приводящую к теоремам (также комбинаторным объектам). На самом деле так говорить — неправильно: Гильберт мог таким образом описать результаты, полученные его предшественниками, но никоим образом — с помощью какой процедуры они их открыли. Это главное различие для того сюжета, которым мы интересуемся. Гильберт знал это и вскоре поставил перед сообществом хорошие вопросы на эту тему: может ли аксиоматическая система быть полной, или, другими словами, всякое ли утверждение, имеющее смысл в этой системе, истинно или ложно? Кроме того, существует ли алгоритм, позволяющий определять, истинно или ложно то или иное утверждение (или же невыводимо в случае неполноты)?

   После Евклида и Гильберта третьим является Гёдель. Он доказал, что всякая аксиоматическая достаточно «интересная» теория обязательно будет неполной (или противоречивой). Этот явно негативный результат странным образом является одним из наиболее позитивных.
\newpage
Непосредственно вдохновленные доказательством Гёделя логики Чёрч и Тьюринг вскоре доказали (1936), независимо друг от друга и разными способами, что второй вопрос Гильберта также имеет отрицательный ответ: не может существовать алгоритм, даже чисто теоретический, грозящий математикам безработицей. Этот результат все еще отрицательный, но — терпение, и мы увидим, что на самом деле Чёрч и Тьюринг стоят у истоков фантастической — и позитивной — революции в математике. Ибо им пришлось очень тщательно обдумать, что же такое алгоритм. Первый из них изобрел для уточнения этого понятия A-исчисление, второй — свою знаменитую теоретическую машину. И это стало рождением целой новой ветви в математике, называемой алгоритмикой, и книга, которую я предваряю предисловием, частично относится к этой области. В этом ее новизна.

   Благодаря фон Нейману идеи Тьюринга вскоре привели к концепции и реализации того, что теперь называется компьютерами. Вначале в них видели лишь инструмент, предназначенный для того, чтобы использовать результаты труда математиков-теоретиков в различных приложениях. Этот взгляд, все еще распространенный, когда инфор- матик ниже по течению принимает эстафету от математика, является сегодня совершенно неверным. Больше уже не считают, что математические теории, представление о которых существенно изменилось, находятся в верховьях по отношению к положению алгоритмического инструмента, и ставить эту ветвь в подчинение другим так же наивно, как пытаться выяснить в алгебраической топологии, что главнее: топология по отношению к алгебре или наоборот.

   Математики все еще часто и довольно нелепо считают за честь получить результаты, не использующие новые алгоритмические средства. Это обычное явление, сопровождающее значительное и быстрое развитие в какой бы то ни было сфере жизни: подобное развитие порождает кризис, разрешающийся более или менее удачно. Когда возникла алгебраическая топология, многие строптивые топологи старательно отделяли топологические доказательства, не использующие алгебраическую топологию, считая их более хорошими, от других, незаконнорожденных. Но в конце концов алгебраическая топология утвердилась. В рамках алгебраической топологии в один прекрасный день появились спектральные последовательности', в это время доказательства, использующие такую технику, еще рассматривались как «грязные», и их, по-возможности, избегали. Мало-помалу, спектральные последовательности стали обычным делом. Совсем недавно новые исследования заново изучили вопрос, связанный со спектральными последовательностями, для того, чтобы сделать из них алгоритмический инструмент;
\newpage
я знал редактора одного прекрасного математического журнала, сердито требовавшего представить аргумент, «убеждающий», что новые результаты, полученные таким способом, были бы недоступны при использовании «других средств» (sic) и не иначе, без чего представленная статья на эту тему была бы, очевидно, отвергнута. S.O.S. — расизм!

   Книга Патриса Нодена и Клода Китте представляет классическую алгебру — ту, которая преподается на уровне второго цикла университета, но в свете новой алгоритмической идеологии. Для рассматриваемой темы алгебра подходит лучше всего: алгоритмика по самой своей природе комбинаторна, а среди всех основных математических дисциплин алгебра является, несомненно, «самой комбинаторной». Преподавать студентам алгебру таким способом — прекрасная идея с разных точек зрения. Очень полезно, что значительно усиливается конкретный характер теории, результатов, упражнений, отдельных тем; это облегчает понимание, требует хорошей точности, но и придает характер игры занятию, которое нередко воспринимается скучным и отталкивающим.

   Но есть еще более важное обстоятельство. Богатство, присущее математике, чаще всего требует, как, впрочем, и в любой сфере деятельности, использования сразу нескольких средств очень разной природы. Самые большие успехи в математике почти всегда могут быть описаны именно так. В другой области, теоретической физике, показательна работа Луи де Бройля, объясняющая, что правильно понять свойства материи можно, лишь рассматривая ее одновременно с волновой и корпускулярной точек зрения. Эйнштейн поступил аналогично в отношении массы и энергии, Гильберт — с геометрией и функциональным анализом и т.д. Так и с момента возникновения алгоритмики всякая попытка пересмотреть тот или иной раздел математики с точки зрения алгоритмики может быть встречена с интересом и надеждой.

   Так устанавливаются взаимные связи, природа которых богата и разнообразна. В некоторых случаях алгоритмика оказывается на службе у обычной математики, являясь тогда инструментом прикладной математики. Часто использование средств алгоритмики порождает новые области математических исследований. Проблемы сложности вычислений, составляющие значительную часть данного труда, — это как раз тот самый случай. Замечу, кстати, что именно к этой области принадлежит самая важная открытая проблема современной математики — сравнение сложностей Р и NP; если, основываясь на сравнительном подобии, распространить математику на все науки, то проблемы Римана, Пуанкаре, «обратная теорема Галуа» окажутся где-то недалеко от химии, тогда как проблема $P \neq N P$ не выйдет за пределы математики.
\newpage
   Иной раз алгоритмические проблемы порождают новый сюжет, который может быть затем полностью отделен от источника своего происхождения и превратиться в раздел «чистой» математики. Хоть исторически это и не вполне корректно, но таким образом можно представлять студентам гармонический анализ. Одна из глав этой книги посвящена дискретному преобразованию Фурье. Даже если бы Фурье не существовал и даже если бы никто с тех пор не мог его заменить, все равно алгоритмика умножения многочленов обязывает обнаружить так называемый анализ Фурье, его интерес, его богатство и его эффективность. А потом уже нетрудно, переходя к пределу, получить ряды Фурье и преобразование Фурье, исходя из дискретного преобразования Фурье; наиболее просто это сделать, используя нестандартный анализ — но берегись инквизиторов: сочетая две ереси в одном курсе, рискуешь сломать себе шею, если не быть осторожным или попросту скромным!

  Другой интересный случай, представляющий, несомненно, очень богатый сюжет: для некоторой ранее существовавшей математической теории ее пересмотр может привести к таким точкам зрения, которые требуют иных методов, имеющих свой собственный интерес, не уменьшая, однако, тем самым, интерес к исходным теоретическим методам. Хорошим примером такого сорта является матричное исчисление. Классическая теория определителей, на базе знакопеременных полилинейных форм, непосредственно не используется при вычислении определителей. На самом деле ситуация даже еще лучше. Наши самые далекие предшественники, несомненно, были знакомы с сущностью так называемого метода Гаусса решения систем линейных уравнений. Однажды появились теоретики и вообразили, что лучше это делать с помощью определителей. Потом алгоритмисты снова поправили стрелки часов: для машины метод Гаусса намного лучше, чем метод Крамера, что, однако, вовсе не лишает интереса к этому методу, который с успехом используется для теоретического анализа первого. Этот и другие связанные с ним вопросы изящно изложены в данной книге; содержащаяся в ней обширная библиография и комментарии позволят прилежным читателям быстро добраться до современных проблем, о богатстве которых они вначале и не подозревали.

   Информатики сразу распознают одного из духовных наставников авторов: Дона Кнута. Отметим лишь составление макета книги, полностью реализованное самими авторами (без Postscript’а!) с помощью TEX'a. Мне не известна официальная градация в сообществе TEX пер-тов, но, несомненно, Патрис Ноден и Клод Китте в его первом эшелоне.
\newpage
   Вошло в поговорку безграничное богатство анализа всех аспектов проблемы в книгах Кнута; то же самое мы обнаруживаем здесь. Для читателя, раполагающего достаточным временем, эта книга — неисчерпаемый кладезь возможностей рассматривать отдельную тему с множества точек зрения. Когда, говоря о программе, авторы решаются перейти к следующей теме, они не забывают заранее указать полезную библиографию; часто это даже библиография в квадрате, содержащая цитируемые тексты, которые, в свою очередь, также содержат обширную библиографию — изящный пример использования подпрограмм. Как и у Кнута, в этой книге имеется обширный набор упражнений разного уровня — от простых примеров, предназначенных проверить понимание того или иного понятия, до самых сложных задач, приближающихся к темам научных исследований. И, как и у Кнута, решения упражнений также включены в книгу!

  Для изучения алгоритмов требовалась техническая поддержка в виде языковых средств информатики. Авторы выбрали язык Ада — безусловно, один из лучших на сегодня языков. И пусть потенциальные пользователи не волнуются по поводу выбора языка, все еще не слишком распространенного во французских университетах: перевод Ада- программ в их любимый язык является несложным техническим упражнением; при этом строгость документации программ просто поразительна и может служить примером, так что на этот счет нет никаких опасений. Такой перевод, требующий ясно отличать алгоритмическую часть от реализации, очень поучителен. Кроме того, авторы зачастую предварительно дают описание алгоритмов в чистом виде в соответствии с аксиоматикой Хоара.

   Это превосходная книга, которую я ставлю в моей библиотеке на полку со «священными» книгами, где находятся уже два «евангелия» — алгоритмика по Вирту и алгоритмика по Кнуту. Теперь я обладаю третьим «евангелием»; кто же напишет четвертое? Но ведь эта книга потребовала коллективной работы двух авторов — еще одно примечательное свойство, так что может быть моя коллекция «евангелий» уже полна? Нет: это противоречило бы Гёделю, но это уже совсем другая история.
   
\hspace{3.6in} Франсис Сержераер,

\hspace{3.1in} Мейлан, 27 сентября 1991 г.
\newpage
\hspace{1.5in} {\huge От авторов}
\\\\\\\\
   Вот и наступил момент, когда, закончив книгу, автор начинает свое введение и ищет подходящие формулировки, которые убедили бы читателя, что все дальнейшее было спланировано, обосновано... с самого начала. Мы не будем отступать от этого правила. Однако, скажем откровенно, наша точка зрения изменялась многократно с того момента, когда мы принялись за этот труд.

   Вопрос «Возможно ли вычислять математические объекты?» является лейтмотивом этой работы; как можно убедиться, эта книга не дает окончательного ответа, но, надеемся, показывает, что этот вопрос не лишен интереса. В особенности в той области, которую затрагиваем здесь, а именно, в алгебре, мы убеждены, эффективное вычисление объектов — вручную или с помощью компьютера — проливает свет на используемые понятия. Но, начиная с таких элементарных понятий, как кольцо целых чисел по модулю п, и до более абстрактных структур, как модули в основных кольцах, что же все-таки можно вычислять и как? В начале работы мы были совсем уже готовы уступить намерению рассматривать алгебру без каких бы то ни было формальных средств, а только в чисто вычислительном аспекте; но изучение преобразования Фурье — и, в особенности, работы Ш. Винограда — убедительно подтолкнули нас к тому, чтобы избежать этого соблазна. В результате проект, сознательно ограниченный вначале, стал более полновесным и содержательным; мы даже позволили себе сочинить полную главу, посвященную дискретному преобразованию Фурье, с включением в нее билинейных форм и тензорного произведения; добавили раздел, излагающий классический подход к факторизуемости колец многочленов...

   Однако мы не отказались от главной идеи, которая составляет основу этой книги: можно вполне конкретно представить понятия элементарной алгебры, т.е., опираясь на учебные примеры и приводя, по мере возможности, конструктивные доказательства.
   
\hspace{0.8in} \small  \textit {Возьмем для примера теорему Цермело, согласно которой}

\hspace{0.6in} \small  \textit {пространство можно преобразовать во вполне упорядоченное}

\hspace{0.6in} \small  \textit {множество; приверженца Кантора будут очарованы строгостью}

\hspace{0.6in} \small  \textit {доказательства — действительной и кажущейся; прагматики}

\hspace{0.6in} \small  \textit {им возразят: Вы утверждаете, что можете преобразовать про-}
\newpage

\hspace{0.6in} \small \textit {странство во вполне упорядоченное множество; ну что ж, пре-}

\hspace{0.6in} \small  \textit {образуйте его! — Это слишком долго. — Тогда покажите нам }

\hspace{0.6in} \small  \textit {по крайней мере, кого-то, кто нашел бы достаточно времени и }

\hspace{0.6in} \small  \textit {терпения и кто смог бы выполнить это преобразование. — Нет, }

\hspace{0.6in} \small  \textit {мы это не можем, потому что число необходимых для этого опе-}

\hspace{0.6in} \small  \textit {раций бесконечно, оно даже превосходит Ко- —А можете ли вы }

\hspace{0.6in} \small  \textit {показать, как можно было бы выразить конечным числом слов }

\hspace{0.6in} \small  \textit {то правило, которое позволило бы упорядочить пространство?} 

\hspace{0.6in} \small  \textit {— Нет — и прагматики делают вывод, что эта теорема или }

\hspace{0.6in} \small  \textit {лишена смысла, или неверна, или же, по меньшей мере, недока-}

\hspace{0.6in} \small  \textit {зуема.}


\hspace{1.8in} \small  \textit {Анри Пуанкаре, Последние мысли (1913 [161])}
\\\\

   Это идет вразрез с общепринятой практикой, которая заключается в преподавании математики на абстрактном уровне, без опоры на эксперимент, как в других научных дисциплинах. Скольких хлопот избегают при этом! Гораздо легче, например, доказать существование для любого простого числа р примитивного по модулю р многочлена, нежели построить процедуру, порождающую за 5 шагов в периоде длины 32 примитивный по модулю 2 многочлен $X^5 + X^2 + 1$.

   Эта книга рискует удивить читателя; в ней иногда объясняются очень простые вещи, а иногда быстро проходят через более сложные понятия. Часто используются понятия группы, модуля, кольца, тела и т.п., но ни одно из этих понятий не определяется; существуют прекрасные книги по алгебре, содержащие эти определения, и нам хотелось не перефразировать их, а скорее попытаться взглянуть на эти структуры с точки зрения эффективности. Всякий раз, когда нам это представлялось возможным, мы стремились заменить традиционную манеру изложения темы и показать многосторонние связи между различными понятиями (например, различные типы колец в теории делимости, сходство между конечными абелевыми группами и редукцией эндоморфизмов и т.д.). Выбор рассматриваемых тем не бесспорен. Мы мало внимания уделяли многочленам, которые, однако, являются фундаментальными объектами в алгебре; в то же время сравнительно много внимания посвящено дихотомическому алгоритму возведения в степень и алгоритму Евклида. Цель, которую мы перед собой ставили, — изучить небольшое количество методов, относящихся к области, называемой ныне алгебраической алгоритмикой, стараясь связать их между собой и с классическими понятиями алгебры. Этот вывод привел нас к сознательному отказу от рассмотрения таких сюжетов как, например,
\newpage
теория Галуа или теория групп, которые очевидным образом допускают алгоритмический подход. Таким образом, мы стремились предоставить необходимые средства для углубленного изучения этой дисциплины, оседлавшей математику и информатику. В действительности, существо описанных в данном курсе методов сконцентрировано в четырех основных алгоритмах: дихотомический алгоритм возведения в степень, алгоритм Евклида, китайская теорема об остатках и быстрое преобразование Фурье; они образуют основу для любой системы формальных вычислений. Тем не менее, в упражнениях содержатся введения в различные темы: от пошаговой трассировки до символа Якоби, включая р-адические числа, непрерывные дроби, многоразрядную арифметику...

   Введение в предмет составляют основы информатики, необходимые для того, чтобы недвусмысленно рассуждать об алгоритмах, об их обоснованиях, о программах. Объекты информатики, определяемые здесь, очень разнообразны: от начал программирования очень быстро переходим к таким понятиям, как модулярность, шифрование информации, порождаемость..., затрагивая весьма тонкую технику программирования на языке Ада. И вот великое слово произнесено! На самом деле большинство реализаций алгоритмов, которые мы даем, написано на языке Ада. Многие удивлялись, что не был выбран Лисп (и мы готовы спорить), Паскаль или Си (в данном случае спор с нами бесполезен). Ада обладает той строгостью, которая, как нам кажется, вполне гармонирует с математической строгостью; мы также широко пользовались богатством систем контроля в языке Ада как для программирования, так и для описания алгоритмов.
\\\\

\hspace{1.0in} \small  \textit {Программисту на языке Лисп известно значение всего, но он} 
   
\hspace{0.8in} \small  \textit {никогда не знает, сколько это стоит.}
\\\\

\hspace{1.5in} \small  \textit {Алан Дж. Перлис, Программистские эпиграммы (1978)}
\\\\

   Пусть сторонников языка Лисп не смущает эта цитата. Мы далеки от того, чтобы быть невосприимчивыми к этому языку. И если мы по достоинству ценим определенные аспекты языка Ада, то мы всегда сожалели о хронической нехватке некоторых базовых средств в этом языке (арифметика высокой точности — лишь один тому пример).

   Выбор материала по информатике для включения в эту книгу был труден: она адресована в первую очередь не информатикам, а математикам, чьи познания в информатике не всегда достаточно основательны. Тем не менее, в этой книге мы не предполагаем, что читателю
\newpage

