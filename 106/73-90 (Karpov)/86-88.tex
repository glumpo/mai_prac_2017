\documentclass{mai_book}

\setdefaultlanguage{russian}
\setcounter{page}{86}

\begin{document}

\begin{mynotice}
Существуют другие способы, позволяющие определить
линейный порядок на декартовом произведении данных
%все, что выше, находится на странице 85
множеств. Один из них, в частности, упорядочивает перестановки,
чередуя их сигнатуры, чего не делает лексикографический порядок.
Заинтересованный читатель может обратиться к упражнениям
17, 23 и 24 в конце главы.
\end{mynotice}

Рассмотрим перестановку $\sigma$ множества $E$, представленную пока таблицей
из двух строк, и отыщем другую перестановку $\sigma'$, непосредственно
превосходящую $\sigma$ в лексикографическом порядке, индуцированном на второй строке таблицы.

$$
\sigma = \begin{pmatrix}
a_1 & a_2 & \cdots & a_{n-1} & a_n \\
b_1 & b_2 & \cdots & b_{n-1} & b_n \\
\end{pmatrix}
\text{и}\: \sigma' = \begin{pmatrix}
a_1 & a_2 & \cdots & a_{n-1} & a_n \\
b'_1 & b'_2 & \cdots & b'_{n-1} & b'_n \\
\end{pmatrix}
$$

\noindent выражения, в которых элементы $a_i$ и $b_i$ из $E$ удовлетворяют для всякого
 $i \in [1, n - 1]$ условию $a_i < a_{i+1}$.

Поскольку элементы $a_i$ расположены в порядке возрастания, они не
несут никакой полезной информации и могут быть подразумеваемы,
что приводит нас к представлению перестановок в форме $n$-ок.
Множество, которое мы пытаемся пронумеровать, есть, следовательно,
множество всех $n$-ок различных элементов из $E$ (мощности $n$). Предыдущее
определение лексикографического порядка без труда распространяется
на множество $n$-ок различных элементов из $E$, и задача теперь становится
такой: считая данной $n$-ку $b = (b_1,b_2, \cdots ,b_n)$ элементов из $E$, найти следующую
за ней $n$-ку $b' = (b'_1, b'_2,..., b'_n)$, удовлетворяющую условиям:\newline

(\textit{i})$\left\{ b_1,b_2,\cdots, b_n \right\} =\left\{ b'_1,b'_2, \cdots, b'_n\right\}$,

(\textit{ii})$\b = (b_1, b_2, \cdots, b_n) <_lex (b'_1, b'_2, \cdots, b'_n) = b'$,

(\textit{iii})$c > b \Rightarrow c \geq b'$, что можно также записатьдля линейной
упорядоченности через $(b,\: b') = \emptyset$.\newline

Прежде чем вычислять последователя какой-либо такой $n$-ки
 в общем случае, можно поинтересоваться существованием этого последователя.

\begin{property}
\hspace*{15pt}Пусть $E$ — линейно упорядоченное множество. Тогда единственная строго убывающая
(соответственно, возрастающая) последовательность, образованная из всех элементов $E$,
представляет наибольшую (соответственно, наименьшую) перестановку $E$ и не имеет,
таким образом, последователя (соответственно, предшественника).
\end{property}

Это очень простое свойство позволяет применить рекурсию для построения следующего
элемента для какой-либо $n$-ки в общем случае.
\newpage

\noindent Действительно, имея определеgние лексикографического порядка на $E^n$ 
для вычисления последователя какого-либо элемента нужно попытаться 
вычислить последователя в $E^{n-1}$ для $(n - 1)$-ки, образованной  
последними составляющими первоначальной $n$-ки; и этот процесс редукции 
может повторяться до тех пор, пока не достигнем набора из $(n - k)$ 
элементов, который не имеет последователя. Предыдущее свойство  
позволит нам охарактеризовать такой набор из $(n - k)$ элементов: он  
образован убывающей последовательностью элементов. Таким образом,  
первый этап в вычислении последователя для $n$-ки элементов из $E$ есть  
поиск финальной убывающей максимальной части этой $n$-ки; затем  
нужно переупорядочить элементы, и следующее свойство указывает, каким 
образом.

\begin{property}[последователя для $n$-ки различных элементов]

\hspace*{0.55cm}Пусть $E$ — множество из $n$ элементов и пусть $b = (b_1,b_2,\cdots ,b_n)$. 
Предположим, что существует такое $l \geq 1$, что для всякого $i$ из $(l, n — 1]$ 
имеют место соотношения $b_i > b_{i+1}$ и $b_l < b_{l+1}$; тогда последователь 
$n$-ки в множестве $n$-ок различных элементов из $E$ определяется как
$$
(b_1, b_2,\cdots,b_{l-1},\quad b_k,\quad b_n,b_{n-1},\cdots,b_{k+1},\quad b_l,\quad b_{k-1},\cdots,b_{l+1}),
$$
где $k$ — наибольший индекс, превосходящий $l$ и такой, что $b_k > b_l$. 
Другими словами, чтобы вычислить следующий за $b$ элемент,  
достаточно обратить финальную убывающую часть $b$, затем поменять элемент, 
предшествовавший этой части, с его последователем в этой части. 
\end{property}
\begin{myproof}
\begin{itemize}
\item Прежде всего, должно быть ясно, что если индекс $l$ не существует, 
то мы будем находиться в ситуации, выраженной в свойстве 9: $n$-ка 
не имеет последователя. 
\item Обозначим $b'$ новую $n$-ку, определяемую этим свойством. Тогда 
ясно, что $b' >_{lex} b$. Действительно, эти две $n$-ки совпадают на их 
$(l—1)$ элементах, и их элементы с индексом $l$ удовлетворяют условию 
$b'_l = b_k > b_l$. 
\item Остается показать, что $b$' является наименьшей мажорантой $b$ в 
лексикографическом порядке; а это можно сделать, показав, что  
открытый интервал $(b,b')$ пуст (в силу линейного лексикографическо- 
го порядка). Предположим, что существует такое $c$, что $b < c < b'$:
\end{itemize}
%тут должна быть таблица

Заметим прежде всего, как это подсказывает схема, что части  
слева от вертикальной черты равны во всех строках, следовательно, 
нет нужды учитывать их в дальнейшем. Рассмотрим только  
элементы каждой перестановки, чьи индексы превосходят или равны $l$, 
Следующее свойство, которое мы не будем повторять по ходу  
доказательства, является основным: \textit{части трех перестановок,  
расположенные справа от вертикальной черты, образованы из одних 
и тех же элементов, различных между собой.}\newline 
Теперь мы покажем, что если предположить, что $b < c < b',$ 
то $b_l < c_l < b'_l$. Действительно, предположим $b_l = c_l$;  
поскольку $b < c$, то из этого следует, что в лексикографическом порядке 
$b_{l+1...n} < c_{l+1...n}$, но часть $b_{l+1...n}$, будучи строго убывающей,  
максимальна в лексикографическом порядке, а так как она образована 
из тех же элементов, что и часть $c_l+1...n$, то не может быть меньше 
последней. Следовательно, $b_l \neq c_l$ и, значит, $b_l < c_l$.  
Рассматривая части $c_{l+1...n}$ и $b'_{l+1...n}$, где последняя является минимальной в 
лексикографическом порядке, аналогичным образом получаем, что 
$c_l < b'_l$. Итак, $b'_l = b_k$, и выбор $b_k$ делает невозможным найти среди 
элементов $b_{l+1},\: \cdots,\: b_n$ элемент $c_l$, заключенный строго между $b_l$ и 
$b'_l$, что противоречит первоначальному предположению  
(существует $c$, удовлетворяющее $b < c < b'$). Значит, $b'$ есть последователь 
для $b$. 
\end{myproof}

\begin{mynotice}
Предыдущее свойство позволит нам эффективно 
вычислять последователя $n$-ки различных элементов из  
множества $E$ мощности $n$. 
Кроме того, как будет видно из записи алгоритма — который 
уже должен просматриваться из этого рассуждения — нетрудно 
вычислить отношение между сигнатурами исходной $n$-ки и новой. 
\end{mynotice}

\end{document}