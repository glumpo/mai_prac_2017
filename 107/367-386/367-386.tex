\documentclass{../template/mai_book}

\defaultfontfeatures{Mapping=tex-text}
\setmainfont{DejaVuSerif}
\setdefaultlanguage{russian}

\clearpage
\setcounter{page}{368}
\setcounter{thm}{19}
\setcounter{section}{5} % ТАК ЗАДАВАТЬ ГЛАВЫ, ПАРАГРАФЫ И ПРОЧЕЕ.
% Эти счетчики достаточно задать один раз, обновляются дальше сами
\setcounter{subsection}{2}



\usepackage{amsmath}
\begin{document}


\pagebreak

\sectiontop
\section{Беглый обзор} 

\medskip

\noindent  Начав с очень простой леммы исключения, с помощью которой мы 
бы-\\стро вывели несколько конкретных применений (вычисление 
опреде-\\лителей, вычисление базиса пространства, порожденного конечным 
се-\\мейством векторов, и др.), к концу этой главы мы добрались до 
клас-\\сических результатов теории модулей над кольцами главных идеалов: \\
инвариантные множители, классификация модулей конечного типа (в\\ 
частности, абелевых групп конечного типа). Была предпринята 
попыт-\\ка установить каждый раз привилегию алгоритмической точки зрения\\ 
и реализацию некоторых алгоритмов на языке Ада, однако без 
отре-\\чения от понятий линейной алгебры. Мы отдавали себе отчет в том,\\
что эффективность линейной алгебры «конечного типа» над кольцом\\ 
главных идеалов обуславливается вычислением коэффициентов Безу в\\ 
этом кольце главных идеалов. 


Читатель, желающий получить классический обзор, может 
обра=\\титься к одной из глав прекрасной работы Самуэля: «Алгебраическая\\ 
теория чисел». Там он найдет компактное изложение (2 страницы!), 
ка-\\сающееся модулей над кольцами главных идеалов. Эта замечательная\\
небольшая работа может быть рекомендована любому 
интересующему-\\ся теорией чисел. Однако в литературе существуют и другие
изложе-\\ния: например, очень полное у Бурбаки, книга II «Алгебра», глава 7\\ 
«Модули над кольцами главных идеалов\footnote{Раздел книги Бурбаки называется «Вычисление инвариантных множителей», но 
думал ли, Бурбаки, что действительно можно вычислить инвариантные множители матрицы с помощью миноров этой матрицы? }, или книга Ленга, Algebra.\\ 
Последняя работа Симса, «Algebra, A Computational Approach», 
пред-\\ставляет очень хорошее введение в эту теорию. 


Однако в этой главе имеется большой пробел: отсутствует 
примене-\\ние теории модулей над кольцами главных идеалов к приведению 
эндо-\\морфизмов векторного пространства конечной размерности — 
клас-\\сификация с точностью до подобия\footnote{2Две матрицы А и В подобны, если существует обратимая матрица Р, 
удовлетворяющая равенству А = РВР$^{-1}$, это влечет отношение эквивалентности, классы 
которой называются классами подобия. Это понятие нельзя путать с понятием 
эквивалентных матриц: А и В эквивалентны, если существуют две обратимые матрицы 
L и Я, удовлетворяющие условию А = LBR.}, теория, в которой кольцо главных\\ идеалов $К[Х]$ играет основную роль. Приведем перечень (без 
доказа-\\тельства) основных результатов.

\medskip

{\noindent \bf Приведение эндоморфизмов} 

\medskip
Если u — эндоморфизм векторного $K$-пространства $Е$ конечной раз-


\pagebreak

\noindent мерности, соответствующая ему структура $K[X]$-модуля на $E$, 
обозна-\\чаемая $E_u$, определяется следующим образом: 

\medskip

$P\cdot x = P(u)(x)$\;\;для\;\;$P\in K[X]$\;\;и\;\;$x\subset E$. 

\medskip

\noindent Аналогично, если $ U \in M_n(K)$, присоединяют к нему структуру\\ 
$K[X]$-модуля на $K^n$, обозначаемую $K_U^n$. Тогда любой результат на\\ 
$K[X]$-модуле $E_u$(или $K_U^n$) интерпретируется как результат на 
клас-\\се подобия и. Ниже приводятся самые важные результаты.

\medskip

($\mathit{i} $) Подмодули $E_u$ являются в точности векторными 
подпростран\\ствами Е, стабильными по отношению к u. Многочлен Р обнуляет $E_u$ \\тогда и только тогда, когда $Р(u)$ = 0. Он обнуляет х тогда и только\\ тогда, когда $Р(u)(x)$ = 0. 

\medskip

($\mathit{ii} $) Два модуля $E_u$и и $F_u$ $K[X]$-изоморфны тогда и только тогда,\\ когдаи u и $\upsilon$ сопряженные, т.е. если существует K-изоморфизм $\varphi$ из\\
$Е$ на $F$ такой, что $\varphi\circ\upsilon = \upsilon\circ\varphi$. Этот результат может быть 
так-\\же выражен в терминах матриц. $K[X]$-модули $K^n_U$ и $K^n_V$ изоморфны\\
тогда и только тогда, когда $U$ и $V$ подобны, т.е. $U = PVP^{-1}$, где\\ 
$P \in GL_n(K)$. Доказано также, что это эквивалентно тому, что 
матри-\\цы  $XId_n - U$ и $XId_n - V$ из $M_n$($K[X]$) эквивалентны:

\medskip

$XId_n - U = L\times (XId_n - V)\times R$\;\;где\;\;$L,R\in GL_n(K[X])$. 

\medskip

($\mathit{iii} $) Пусть U, $U_1$, $U_2$, ..., $U_k$ — квадратные матрицы порядков  
$n,\\n_1, n_2,..., n_k $ соответственно, где $n=n_1+n_2+...+ n_k $. Сказать, что\\
$ K^n_U\simeq K^{n_1}_{U_1}\times K^{n_2}_{U_2}\times...\times K^{n_k}_{U_k} $, рассматриваемые как $A'[X]$-модули, тоже 
самое, что сказать, что матрицы $U$ и diag($U_1, U_2,..., U_k$) подобны.

\medskip

($\mathit{i}\upsilon $) Элемент $x\in E$ является образующим в $K[X]$-модуле $E_u$ тогда\\
и только тогда, когда Е =$ K_x + K_{u(x)} + K_{u^2(x)}+...++ K_{u^i(x)+...} $\\ 
Тогда говорят, что векторное пространство $Е$ циклично под $u$. В этом 
случае, если $P(X)$ — минимальный полином от $u$ и если $n$ = $deg(P)$, то 
легко доказать, что $\{x, u(x), u^2(x),..., u^{n-1}(x)\}$ — базис $E$. Выражение\\
и в этом базисе является матрицей Фробениуса, соответствующей $P$ \\
(или сопутствующей матрицей, соответствующей $P$): 

\medskip

$$J_{\lambda} = \begin{pmatrix}
0 & 0 & \; & 0 & -\alpha_0 \\
1 & 0 & \; & 0 & -\alpha_1 \\
0 & 1 & \; & 0 & -\alpha_2 \\
\vdots & \vdots & \ddots & \vdots & \vdots \\
0 & 0 & 0 & 1 & -\alpha_{n-1}
\end{pmatrix}
$$


$$\text{если} \;\;PX) = X^n+\alpha_{n-1}X^{n-1}+... +\alpha_{-1}X+\alpha_0$$.

\pagebreak

($\upsilon$)Понятно, что для каждого унитарного многочлена Р имеется 
мо-\\дель циклического пространства с минимальным многочленом Р. Для\\ 
этого достаточно рассмотреть факторкольцо $K[X]/(P)$, которое 
явля-\\ется векторным $\mathit{K}$-пространством размерности deg(P), и оснастить его \\эндоморфизмом-умножением на $\bar{X}$. В каноническом базисе $K[X]/(P)$\\
этот эндоморфизм имеет для матрицы Фр свой минимальный (или 
ха-\\рактеристический) многочлен $P$. Если $E_u$ определено как и ранее, то\\ 
имеет место $\# [Х]$-изоморфизм из $K[X]/(P)$ на $E_u$ ($\bar{l}$ 
преобразовыва-\\ется в х). Итак, класс подобия циклических эндоморфизмов 
характе-\\ризуется минимальным многочленом (впрочем, равным 
характеристи-\\ческому многочлену).

\medskip

($\upsilon\mathit {i} $) Китайская теорема формулируется следующим образом: если\\ $P_1$ и $P_2$ — Два взаимно простых многочлена, а $P=P_1P_2$ — их произве-\\дение, тогда $ \Phi_P$ подобна $diag( \Phi_{P_1},\Phi_{P_2}$), что влечет китайский изомор-\\физм между $K[X]$модулями: 

\medskip

$$K[X]/(P) \times X[X]/(P_2) \simeq КХ]/(P_1P_1).$$ 

\medskip

($\upsilon\mathit{ii} $) На алгебраически замкнутом поле матрицы Фробениуса свя-\\заны с матрицами Жордана. Действительно, если $P(X)$ = ($X — \lambda)^n$,\\простое изменение базиса показывает, что Фр подобна матрице 
Жор-\\дана $J_{\lambda}$:

$$J_{\lambda} = \begin{pmatrix}
\lambda & 1 & 0 & \; & 0 \\
0 & \lambda & 1 & \; & 0 \\
0 & 0 & \ddots & \; & \vdots \\
\vdots & \vdots & 0 & \lambda & 1 \\
0 & 0 & 0 & 0 & \lambda
\end{pmatrix}
$$

\medskip

\noindent Китайская теорема доказывает, что матрица Фробениуса $\Phi_P$ 
подоб-\\на блочной диагональной матрице, блоками которой будут жордановы\\ матрицы (пишут $P(X) = (X-\lambda_1)^{n_1}...(X - \lambda)^{n_k}$).

\medskip

($\upsilon\mathit{iii} $) Можно рассмотреть матрицы Фробениуса как кирпичи, с по-\\мощью которых можно построить любой эндоморфизм и подобие. Дей-\\ствительно, так как $E_u$ — $\mathit{K}$[Х]-модуль кручения конечного типа (в\\силу теоремы Гамильтона — Кэли характеристический многочлен об-\\нуляет $E_u$), то теорема об инвариантных множителях формулируется\\ следующим образом: существует такой базис E, что матрица и в этом

\pagebreak

\noindent базисе будет выражаться следующим образом: 

$$\begin{pmatrix}
\Phi_{P_1} & 0 & \; & 0 \\
0 & \Phi_{P_2} & 0 & \vdots  \\
\vdots & \vdots & \ddots & 0  \\
0 & 0 & 0 & \Phi_{P_k} 
\end{pmatrix},
$$

\smallskip
 
\noindent где $P_i$ — многочлены, удовлетворяющие условию $P_1|P_2|P_3...|P_k$,\\ 
и к тому же единственны. Они являются инвариантами подобия 
эн-\\доморфизма u, а матрица называется нормальной формой Фробениуса.\\ 
Можно также интерпретировать этот результат в виде:


$$ E = E_1\oplus E_2\oplus...\oplus E_k $$,



\noindent где $E_i$ — стабильные (по отношению к u) и циклические под и 
пространства, минимальный многочлен $u|E_i$ делит минимальный 
многочлен $u|E_{i+1}$ Инварианты подобия характеризуют класс подобия и: два 
эндоморфизма и и v подобны тогда и только тогда, когда они имеют 
одни и те же инварианты подобия. 

\medskip

($\mathit{i} \upsilon $) Многочлен $P_k$, последний инвариант подобия, является в
точ-\\ности аннулятором $Eu$. Следовательно, $P_k$ — минимальный многочлен\\ 
для u, а произведение $P_1P_2... P_k$ инвариантов подобия не что иное, как 
характеристический многочлен для u. 


Отметим аналогию между векторным пространством конечной 
раз-\\мерности, оснащенным эндоморфизмом u (который скрывает 
струк\\туру $K[X]$-модуля кручения) и конечной абелевой группой (которая\\ 
тоже скрывает $Z$-модуль кручения). Таким образом, 
характеристиче\\ский многочлен соответствует порядку группы, и теорема Гамильтона\\ 
— Кэли приближена к теореме Лагранжа (которая утверждает, что\\ 
$nх = О$ в любой группе порядка п); целое число, соответствующее 
ми-\\нимальному многочлену, является наименьшим общим кратным 
поряд-\\ков элементов группы (существует элемент группы, порядок которого\\
является наименьшим общим кратным порядков, существует элемент\\ 
векторного пространства, минимальный многочлен которого является\\ 
минимальным многочленом пространства $Е$ в целом)...

\medskip

Имеются по крайней мере две причины, по которым мы не стали\\ 
рассматривать приведение эндоморфизмов: первая причина 
заключа-\\ется в том, что реализовать простой перевод недостаточно с 
педагоги-\\ческой точки зрения (основная теорема, касающаяся представления в\\ виде блоков Фробениуса, может быть доказана более простым способом

\pagebreak

\noindent с помощью структуры сопутствующего векторного $K$-пространства).\\ 
Однако, было бы крайне желательно хорошее изложение некоторых тем\\ 
(алгоритмы вычисления характеристического многочлена, 
минималь-\\ного многочлена, определение инвариантов подобия, представление в\\ виде блоков Фробениуса... сопровождаемые многочисленными 
приме-\\рами), которое существенно удлинило бы эту главу. Нужно отметить,\\ 
что такие алгоритмы используют специальные методы для матриц —\\ 
не вычисляют, например, характеристический многочлен как 
опреде-\\литель с коэффициентами в $K[X]$, а просто пользуются вычислениями\\ 
в $K$ и приспособленными для этого методами. Этот подход 
использу-\\ется в литературе: можно, например, обратиться к диссертации Озел-\\ ло [140], в которой описываются алгоритмы вычисления нормальной\\ 
формы Фробениуса матрицы $A \in M_n(K)$. Сложность этих 
алгорит-\\мов полиномиальна от n в случае, когда $К$ = $Q$ или $К$ = $Z/pZ$. В ее \\
библиографии можно найти множество других рекомендуемых работ,\\ 
касающихся явного приведения эндоморфизмов.

\pagebreak

\cleartop
\rule{0pt}{10pt}

\begin{center}

{\large\bf Упражнения}
\end{center}

\rule{0pt}{10pt}

{\noindent \bf 1. Евклидовы возрастающие алгоритмы} 

\medskip

{\bf a}. Дать примеры евклидовых делений $a = bq + r$, в которых $b | a$ \\
и $r\ne 0$. 

\medskip

{\bf b}. Предположим, что $\varphi$ — евклидов возрастающий алгоритм над\\ 
кольцом А, т.е. такой, что если а делит $b$ и $b$ не нуль, то $\varphi(a)  \leqslant \varphi(b)$. 
Доказать, что если $b | a$ и если $b\ne 0$, то евклидово деление по 
возрастающему евклидову алгоритму $a$ на $b$ дает в результате нуль. 

\medskip

{\bf c}. Убедиться, что большинство известных евклидовых алгоритмов\\ 
возрастающие. 

\medskip

{\noindent \bf 2. Явное выражение подмодуля} 

\medskip

{\bf a}. Если $А$ — кольцо главных идеалов, то доказать, что любой 
под-\\модуль в $А$, имеет вид $ImX$, где $X$ — некоторая $n \times m$-матрица. 
Дока-\\зать, что существует подмодуль, который нельзя представить в виде\\ 
$КеrХ$. 

\medskip

{\bf b}. Доказать, что над кольцом главных идеалов любой модуль 
ко-\\нечного типа может быть определен m образующими $f_1,f_2,...,f_m$ и\\ 
n соотношениями: 

\begin{equation}
\left\{\begin{gathered}
a_{11}f_1 + a_{12}f_2 + ... + a_{1m}f_m = 0\\
a_{21}f_1 + a_{22}f_2 + ... + a_{2m}f_m = 0\\
\vdots3\
a_{31}f_1 + a_{32}f_2 + ... + a_{3m}f_m = 0\\
\end{gathered}\right.
\end{equation}

\medskip

{\bf c}. Сформулировать несколькими способами явное описание 
подмодуля в $A^n$. 

\medskip

{\noindent \bf 3. Система образующих для $SL_2(Z)$} 

\medskip

Чтобы выразить в явном виде систему образующих группы $SL_2(Z)$,\\
— группы $2 \times 2$-матриц с целыми элементами и с определителем 1, -

\pagebreak


вводят следующие матрицы в $SL_2(\mathit{Z})$: 

$$A = \begin{pmatrix}
0 & -1\\ 1 & 0
\end{pmatrix},\;\;\;
B_q = \begin{pmatrix}
1 & q\\ 0 & 1
\end{pmatrix}, \;\;\; B=B_1,
$$

$$C_q = \begin{pmatrix}
0 & -1\\ 1 & q
\end{pmatrix},\;\;\;C = C_{-1}.$$

\medskip

{\bf a.} Вычислить $A^2, B_qB_{q'}, B_q^{-1}.$ Выразить $C_q$ через А и $B_q$. Каков\\ 
порядок $А$? $В$? $С$?

\medskip

{\bf b.} Пусть $а,b \in \mathit{Z},$ где $b\ne 0$. Доказать, что существует такое $q \in \mathit{Z}$,\\ 
что 

{\begin{center}
$\begin{pmatrix}
0 & -1\\ 1 & q
\end{pmatrix}\begin{pmatrix}
a\\ b
\end{pmatrix}
=\begin{pmatrix}
a'\\ b'
\end{pmatrix},$\;\;\; где |b'|<|b|.
\end{center}}

 
Доказать, что существует матрица М, равная произведению матриц \\
$C_q$ и удовлетворяющая условию: М $
\begin{pmatrix}
a\\ b
\end{pmatrix}
=\begin{pmatrix}
a'\\ 0 
\end{pmatrix}.
$ 

\medskip

{\bf c.} Доказать, что $SL_2(\mathit{Z})$ порождена матрицами $A$ и $B$, а также 
ма-\\трицами $A$ и $C$ (в частности, $SL_2(\mathit{Z})$ порождена элементом порядка 4\\ 
и элементом порядка 3). 

\medskip

{\noindent\bf 4. Метод приведения к ступенчатому виду для вектора }

Что даст метод приведения к ступенчатому виду, примененный к\\ 
вектору $ (x_1,x_2,...,x_m) $ как к $1 \times m$-матрице? 

\medskip

{\noindent\bf 5. Необходимое и достаточное условие того, что Ах является \\
прямым слагаемым в $A^n$}

\medskip

Пусть А обозначает кольцо главных идеалов, а $x \in A^n  $.

\medskip

{\bf a.} Дать необходимое и достаточное условие того, что х входит в 
часть базиса $A^n$. Доказать, что это условие эквивалентно тому, что Ах 
является прямым слагаемым в $A^n$.

\medskip

{\bf b.} В случае, когда х удовлетворяет этому условию, дать алгоритм, 
позволяющий вычислить базис $A^n$, содержащий $x$. Привести пример 
базиса $\mathit{Z}^4$, содержащего вектор $x = (30,42,70,105)$. 

\medskip

{\noindent\bf 6. Определители и линейная независимость
над коммутативным кольцом }

\medskip

Пусть А — унитарное коммутативное кольцо, an — положительное 
целое число. Для $I \in \{1,2,..., n\}$ обозначим через $p_I$ проекцию из $A^n$

\pagebreak

\noindent на $A^I$, а через $p_i$, проекцию на $\mathit{i}$-ую компоненту. Это упражнение до-\\казывает следующую теорему: $m$ векторов $ w1, w2,..., wm$ из $A^n$ линейно\\ 
зависимы тогда и только тогда, когда существует такое $\alpha \in A — \{0\}$,\\ что для любого подмножества $I$ из $[1,n]$ с кардинальным числом m,\\ 
имеет место равенство $\alpha det(p_I(w_1),p_I(w_2),]dots ,p_I( wm)) = 0$. В 
частно-\\сти, $n$ векторов из $A^n$ линейно зависимы тогда и только тогда, когда\\ 
их определитель является делителем 0 в $A$.

\medskip

{\bf a.} Пусть $w1, w2,... wm$ — m векторов из $A^n, J\subset 1,2,..., n]$ с 
кар-\\динальным числом $m$ — 1. Пусть дан индекс $j$, и полагают $I = J \cup \{j\}. $\\
Доказать, что

\medskip

$$
\displaystyle\sum^m_{i=1}(-1)^i det(p_j w1,...,p_j w{i-1},p_j w{i+1},...,p_j wm)\cdot p_j( wi)=\\$$


$$
\begin{cases}
0, $если$ j\in J,\\
\pm det(p_j w1... p_j wm), $если$ j \notin J.
\end{cases}
$$
\medskip

{\bf b.} Доказать теорему (например, по индукции). 

\medskip

{\bf c.} Доказать, что в $A^n$, и вообще в любом модуле, порожденном n 
элементами, $n$ + 1 векторов линейно зависимы. 


{\noindent\bf 7. Орбита вектора под действием $GL_n(\mathit{Z})$}

\medskip

Рассмотрим отношение эквивалентности, определенное на $\mathit{Z}$, следу-\\ 
ющим образом:

\medskip

$ \begin{pmatrix}
a\\b
\end{pmatrix}\equiv \begin{pmatrix}
a'\\b'
\end{pmatrix}, $\;\;\;если, по определению, $\exists A \in GL_2(\mathit{Z})$ 

\begin{center}
такая, что $ A \begin{pmatrix}
a\\b
\end{pmatrix}= \begin{pmatrix}
a'\\b'
\end{pmatrix}. $
\end{center}

\medskip

\noindent Охарактеризовать более простым способом это отношение 
эквивалент-\\ности. (Указание: доказать существование в любом классе 
эквивалент-\\ности вектора, вторая координата которого нулевая.) Обобщить на\\ 
размерность $n$.

\medskip

{\noindent\bf8. Определитель матрицы $M$ и порядок $\mathit{Z}^n$/ImM} 
Пусть дана квадратная матрица $M$ порядка п с целыми 
элемента-\\ми, нужно вычислить элементарным способом порядок факторгруппы\\ 
$\mathit{Z}^n/ImM$.

\pagebreak

{\bf a.}Предполагается, что $M$ имеет порядок 2 и верхнюю треугольную\\ 
форму. Какое условие нужно наложить, чтобы группа $\mathit{Z}^2$/ImM была\\ конечной? В этом случае каков ее порядок? Указание: можно начать,\\ например, со случая диагональной матрицы. 

\medskip

{\bf b.} Предполагается, что $M$ имеет порядок 2. Показать, что 
существует такая матрица $U \in GL_2\mathit{Z}$, что $UM$ имеет верхнюю 
треугольную форму, затем ответить на вопросы из пункта $a$. 

\medskip

{\bf c.} В общем случае показать, что факторгруппа $\mathit{Z}^n/ImM$ конечна \\
тогда и только тогда, когда $det(M)\ne 0$, и в этом случае факторгруппа \\
$\mathit{Z}^n$/ImM является группой, имеющей порядок, равный абсолютному 
значению $det(M)$.

\medskip

{\noindent\bf 9. Нормы и число элементов частного}

\medskip

Пусть $\theta$ — целое квадратичное число, т.е. комплексное число, 
удо-\\влетворяющее унитарному уравнению второй степени с целыми 
коэф-\\фициентами. Предположим, что $\theta$ — истинное квадратичное целое 
чи-\\сло, т.е. не элемент из $\mathit{Z}$. Трехчлен, в котором в является корнем, 
бу-\\дет тогда единственным, а $\tilde{\theta}$ обозначает второй корень. Тогда 
коль-\\цо $A$ = $\mathit{Z}[\theta]$ оснащено автоморфизмом $\sigma$, удовлетворяющим условию\\ 
$\sigma(\theta) = \tilde{\theta}$ (этот автоморфизм единственный, инволютивный и оставляет\\ 
фиксированным $\mathit{Z}$), и алгебраической нормой N, определенной 
равенством $N(Z) = Z\sigma(Z)$. 


Для $Z \in А$ доказать, $что N(Z) = det(m_Z)$, где $m_Z$ обозначает 
умножение на $Z$. Для $Z \in А*$ вывести, что факторкольцо $A/(Z)$, где $(Z)$ 
обозначает идеал в $A$, порожденный $Z$, является кольцом, имеющим $|N(Z)|$ 
элементов. 

\medskip

{\noindent\bf 10. Вычисление ядра}



{\bf a.}Доказать, что ядро матрицы $X$ име-
ет ранг 2 (предъявить базис).
\begin{wrapfigure}{r}{0.5\textwidth}
$X=\begin{pmatrix}
1 & -2 & 3 & 1 & 2\\
2 & 1 & 4 & -1 & 1\\
1 & -1 & 2 & 1 & 1
\end{pmatrix}$  
\end{wrapfigure}

Вычислить базис $\{R_1, R_2, R_3, R_4, R_5\}$ из $\mathit{Z}^5$, удовлетворяющий условию:

$$ \mathit{Z}^5=E\oplus KerX,\;\;\;\;E =\mathit{Z}R_1\oplus\mathit{Z}R_2\oplus\mathit{Z}R_3,\;\;\; Ker X= \mathit{Z}R_4\oplus\mathit{Z}R_5 $$
 
{\bf b.} Выразить в явном виде линейные формы ($(\lambda_i(x))_{1\leqslant i\leqslant 5},$ 
удовлетворяющие условию: $ x = \lambda_1(x)R_1+\lambda_2(x)R_2+...+\lambda_5(x)R_5 $


\pagebreak

{\noindent\bf11. Поиск образа и ядра}
\begin{wrapfigure}{r}{0.3\textwidth}
$X=\begin{pmatrix}
1 & 2 & 3 & 4 \\
5 & 6 & 7 & 8\\
9 & 10 & 11 & 12
\end{pmatrix}$  
\end{wrapfigure}

{\bf a.} Применить алгоритм приведения к сту-\\пенчатому виду к матрице $X$.
 
\bigskip

{\bf b.} Предъявить базис $ \beta $ образа $X$. Определить отношения между\\
столбцами $X$, выражение этих столбцов в В и выражение векторов $ \beta $\\ 
через эти столбцы. 

\medskip

{\bf c.}Предъявить базис ядра X и разложение $Z^4=Ker X \oplus E$. 

\medskip

{\noindent\bf 12. Несколько предупреждений}

\medskip

Пусть $F$ — подпространство из $Z^4$, порожденное тремя векторами:\\ 
$Х_1 = (6,12, -12,18), Х_2 = (15,0,30,15), Х_3 = (10,10,0,20).$

\medskip

{\bf a.} Выразить базис $ F$ (какова его размерность?) и отношение между 
векторами $Х_i$. 

\medskip

{\bf b.} Доказать, что $\{X_1,X_2\}$ является множеством линейно 
независимых элементов $F$ и что $Х_з$ не принадлежит подпространству, 
порожденному $X_1,X_2$. Доказать, что $\{X_1,X_2\}$ не может быть дополнено до 
базиса $F$. Тот же вопрос и для перестановки векторов $X_1,X_2$ и $X_3$. 

\medskip

{\noindent\bf13. Решить, является ли вектор линейной комбинацией 
других векторов}

\medskip

Рассмотрим векторы $Х_1 = (1,-1,2,1), Х_2 = (3,1,2,1)$ и 
$Х_3$ = (3,5,2,3) из $\mathit{Z}^4$. Является ли вектор $B$ = (5,3,4,3) линейной 
комбинацией векторов $X_1,X_2$2 и $X_3$? 

\medskip

{\noindent\bf14. Матричное уравнение $B = AX$}

\medskip

Рассмотрим следующие матрицы: 


$$А =\begin{pmatrix}
2 & 1 & 2 \\
3 & 2 & 0 \\
1 & 1 & 1 \\
1 & 2 & 0 
\end{pmatrix},
B =\begin{pmatrix}
5 & 3 & 10 \\
5 & 1 & 7 \\
3 & 1 & 6 \\
3 & -1 & 5 
\end{pmatrix}$$

\medskip

{\bf a.} Доказать, что А является инъекцией из $\mathit{Z}^3$ в $\mathit{Z}^4$. 

\medskip

{\bf b.} Доказать, что $ImВ \subset ImА$ и построить матрицу С размеров 
3x3, удовлетворяющую условию В = А х С. 

\medskip

{\bf c.} Показать, что фактормодули Im A/Im В и $\mathit{Z}^3$/ImC изоморфны. 
Каков порядок фактора ImA/ImB

\pagebreak

{\noindent\bf 15. Проекторы}

\medskip

Пусть $R$ и $S$ — две квадратные взаимно обратные п х п-матрицы. 
Любое разложение $n = p + q$ индуцирует горизонтальное разложение $R$
и вертикальное разложение $S$: 

\medskip

\begin{center}
 $R =\bigl(R'\big|R"\bigr),\;\;S=\bigl(\frac{S'}{S"}\bigr),\;\text{где}\;\;\begin{cases}
R': n\times p, R":n\times q,  \\
S': p\times n, S":q\times n
\end{cases}$
\end{center}


\medskip
 
Доказать, что $R'S'$ и $R"S"$ являются двумя проекторами (т.е. двумя \\
идемпотентами) суммы $Id_n$ и образов $ImR', ImR"$ соответственно. 

\medskip

{\noindent\bf 16. Над кольцом главных идеалов любой подмодуль 
свободного модуля свободен}

\medskip

Пусть $(E_i)+{i\in I}$ — семейство $A$-модулей ($A$ — произвольное 
унитарное коммутативное кольцо) такое, что любой подмодуль $Е_i$ свободен. 
Нужно доказать следующий результат: любой подмодуль М из $ \oplus_{i \in I} E_i$\\ 
изоморфен прямой сумме $ \oplus_{i \in I} M_i$,-, где $M_i \subset E_i$ и, в частности, что $M$\\ 
свободен. Доказательство использует трансфинитную индукцию по $I$\\
(если $I$ конечно, то речь идет об обычной индукции). Для этого
снаб-\\жают $I$ структурой корректного порядка, т.е. для которого любое 
непустое подмножество  имеет наименьший элемент. Положим также: 

\medskip

$$\displaystyle L_i=\bigoplus_{j\leqslant i}E_j, \dot{L}_i=\bigoplus_{j<i}E_j \rightarrow\;\;\; L_i=\dot{L}_i \oplus E_i.$$

\medskip

{\bf a.} Доказать, что существует $M_i \subset L_i \cup M $, изоморфный подмодулю \\
$E_i$ и удовлетворяющий условию: $L_i \cup M = (\dot{L}_i \cup M)\oplus M_i$. 

\medskip

{\bf b.} Показать, что$M = \bigoplus_{i \in I}M_i$. 

\medskip

{\bf c.} Доказать, что над кольцом главных идеалов любой подмодуль \\
свободного модуля свободен. 

\medskip

{\noindent\bf 17. Китайские формулы для 3 модулей}

\medskip

Предъявить в явном виде изоморфизм между $ Z_4 \times Z_9 \times Z_25$ и $Z_{900}$ 
Можно найти 3 целых числа $a_1,a_2,a_3,$ удовлетворяющих условию: 

$$a_1 \equiv 
\begin{cases}
1\;\;(mod 4),  \\
0\;\;(mod 9),  \\
0\;\;(mod 25),  
\end{cases}
\;\;\;\;\;\;
a_2 \equiv 
\begin{cases}
1\;\;(mod 4),  \\
0\;\;(mod 9),  \\
0\;\;(mod 25),  
\end{cases}
\;\;\;\;\;\;
a_3 \equiv 
\begin{cases}
1\;\;(mod 4),  \\
0\;\;(mod 9),  \\
0\;\;(mod 25).  
\end{cases}
$$
\pagebreak
{\noindent\bf 18. Нормализация произведения циклических групп}

\medskip

Пусть $(a_1, a_2,..., a_n)$ и $(b_1, b_2,..., b_n)$ — две последовательности 
це-\\лых чисел (или, в общем случае, элементов кольца главных идеалов).\\ 
По определению, они эквивалентны, если:

$$\exists \varphi:Z^n\tilde{\bar{\rightarrow}} Z^n,\;\;\;\;\;\varphi(a_1Z\times a_2Z\times...\times a_nZ)=b_1Z\times b_2Z\times...\times b_nZ.$$

\noindent Обозначение $(a_1, a_2,..., a_n) \sim(b_1, b_2,..., b_n)$. Эквивалентность двух
по-\linebreak
следовательностей влечет изоморфизм между факторгруппами 
$ Z_{a_1}\times Z_{a_2}\times ...\times Z_{a_n}\;\;\text{и}\;\;Z_{b_1}\times Z_{b_2}\times ...\times Z_{b_n} $, через частный 
изомор-\linebreak
физм $\varphi$. То, что это соответствие истинно, не очевидно: это будет 
до-\linebreak
казано в упражнении 36. Для начала речь пойдет о выражении 
един-\linebreak
ственной нормализованной последовательности $(b_1, b_2,..., b_n)$, которая\linebreak
эквивалентна $(a_1, a_2,..., а_n)$, через $a_i$. 



{\bf a.} Доказать, что $ (p^{\alpha}a,p^{\beta}\;\;)\sim(\;\;p^{\beta}a,p^{\alpha}) $, где р — простое число, не 
делящее ни а, ни b. Указание: можно доказать, что (а, b) ~ $(а', b')$как\\linebreak 
только $a\wedge b = a'\wedge b'$ и $ab = a'b'$. 

\medskip


{\bf b.} Для простого числа р обозначим через $ wp(x)$ показатель р в 
раз\\ложении на простые множители числа х. Запишем: 


$$a_1=p^{ wp(a_1)}a_1'\;\;\;a_2=p^{ wp(a_2)}a_2',\;\;\;...,\;\;\;a_n=p^{ wp(n_1)}a_n'$$ 
$$\text{где р не делит}\;\;a_i'. $$



Затем упорядочим последовательность $ wp(a_1),..., wp(a_n)$ в 
возрастаю-\\
щую последовательность $ \alpha^1_p\leqslant...\leqslant\alpha^n_p $. Доказать, что $(a_1, a_2,..., а_n)\sim$\\
$(p^{\alpha_p^1}a_1',p^{\alpha_p^2}a_2',...,p^{\alpha_p^n}a_n') $. Предъявить нормализованную 
последова-\\тельность, эквивалентную (3 600,960,40). 

\medskip

{\bf c.} Повторить предыдущую операцию для каждого р в случае, когда: 
$$\displaystyle a_1=\prod_pp^{ wp(a_1)},\;\;\;...,\;\;\; a_n=\prod_p p^{ wp(a_n)}$$
$$\displaystyle \text{и,полагая}\;\;\;b_1=\prod_pp^{\alpha_p(b_1)},\;\;\;...,\;\;\; b_n=\prod_p p^{\alpha_p(b_n)} $$ 
доказать, что последовательность $(b_1, b_2,..., b_n)$ нормализованная и 
эквивалентная данной последовательности $(a_1, a_2,..., а_n)$. Сравнить 
этот метод с методом, описанным в этой главе. 

\medskip

{\bf d.} Для $1\leqslant k\leqslant n$ показать, что 

$$ b_1b_2... b_k = \text{НОД}\{a_{i_1}a_{i_2}... a_{i_k}|1\leqslant i_1<i_2<... <i_k\leqslant n\}. $$

\pagebreak

{\noindent\bf19.Какие группы изоморфны?} 


Среди следующих абелевых групп (порядка 48) найти изоморфные 
группы. 
$$Z_{48},\;\;\;Z_2 \times Z_{24},\;\;\;Z_4 \times Z_{12},\;\;\;Z_8 \times Z_6,\;\;\;Z_{16} \times Z_3, $$
$$Z_2 \times Z_2 \times Z_{12}\;\;\; Z_2 \times Z_4 \times Z_6,\;\;\;Z_2 \times Z_8 \times Z_3, $$
$$Z_2 \times Z_2 \times Z_2 \times Z_6,\;\;\;Z_2 \times Z_2 \times Z_4 \times Z_3,\;\;\; Z_2 \times Z_2 \times Z_2 \times Z_2 \times Z_3, $$


{\noindent\bf20. Вычисление обратной матрицы к квадратной матрице} 


Пусть $X — n \times n$-матрица с элементами в кольце главных идеалов. 
Написать алгоритм, позволяющий вычислить обратную матрицу к $X$ 
(если $X$ не обратима, то алгоритм должен этот случай исключить). 
Применить этот алгоритм к следующим примерам: 


$$\begin{pmatrix}
6 & -7 & 1 \\
-95 & 111 & -16 \\
15 & -18 & 4 
\end{pmatrix},\;\;\;\;
\begin{pmatrix}
3 & 6 & 7 & 4 \\
1 & 3 & 2 & 1\\
2 & 6 & 9 & 1\\
4 & 3 & 2 & 0
\end{pmatrix}.$$


{\noindent\bf21. Вычисление полного прообраза}


Пусть А — кольцо главных идеалов, и : $A^m \longrightarrow A^n \text{и} \upsilon : A^p \longrightarrow A^n 
$\\- два морфизма. Придумать метод, позволяющий вычислить базис 
подмодуля $u^{-1}(Im \upsilon) \text{в} A^n$. Можно рассмотреть морфизм (u | v), 
определенный на $A^m \times A^p$ соотношением:


$$ A^m \times A^p \ni (x,y)\to u(x) + v(y)\in A^n, $$


и проекцию $ р_1 : A^m \times A^p \rightarrow A^m $ на первый множитель. Рассмотреть 
следующий пример: 


$$U=\begin{pmatrix}
1 & 4 & 2 & 4 & 7 \\
2 & 3 & 1 & 4 & 4\\
3 & 2 & 2 & 5 & 6\\
4 & 0 & 2 & 5 & 5
\end{pmatrix},\;\;\;\;
V=\begin{pmatrix}
1 & 0 & 0  \\
2 & 1 & 0 \\
3 & -1 & 0 \\
4 & -2 & 1 
\end{pmatrix}.$$ 


{\noindent\bf22. Вычисление пересечения двух подмодулей }


Пусть $E$ и $F$ — два подмодуля в $A^n$ ($А$ — кольцо главных 
идеа-\\лов) , порожденные соответственно векторами $ X_1,X_2,...,X_m\;\;\text{и }Y_1 $,

\pagebreak

\noindent$ Y_2,...,Y_p. $ Придумать алгоритм, вычисляющий систему образующих 
пересечения $E\cap F$. Рассмотреть следующий пример: 


$$ E=ImX,\;\;\;\;F=ImY,\;\;\;\;\text{где} $$ 
$$X=\begin{pmatrix}
1 & 2 & 3  \\
-1 & 2 & -3 \\
2 & 1 & 1 \\
1 & 1 & -3 
\end{pmatrix},\;\;\;\text{и}\;\;\;
Y=\begin{pmatrix}
1 & 3 \\
1 & -2  \\
0 & 3 \\
-2 & 1  
\end{pmatrix}.$$ 

{\noindent\bf23. Система сравнений} 

\medskip

{\bf a.} Объяснить, как найти все решения $ (x_1,x_2,x_3,x_4 \in Z^4) $ системы: 


$$ \begin{cases}
 a_{11}x_1 +a_{12}x_2 + a_{13}x_3 + a_{14}x_4\equiv 0\;\; (mod q_1),\\
 a_{21}x_1 +a_{22}x_2 + a_{23}x_3 + a_{24}x_4\equiv 0\;\; (mod q_2),\\
 a_{31}x_1 +a_{32}x_2 + a_{33}x_3 + a_{34}x_4\equiv 0\;\; (mod q_3),
 \end{cases}$$ 

\noindent где $a_{ij}$ и $q_i$ принадлежат $Z$ (некоторые $q_i$ могут быть нулевые). 

\medskip

{\bf b.} Решить систему линейных уравнений, а затем следующую 
систему сравнений: 


 $$\begin{cases}
 4x_1 + 3x_2 + 2x_3 + \;\;x_4=0\\
 5x_1 + 6x_2 + 7x_3 + 8x_4=0,\\
 12x_1 + 11x_2 + 10x_3 + 9x_4=0,
 \end{cases}\;\;\;\;\;\;\;\;\;\;\;\;;$$
 $$ \begin{cases}
4x_1 + 3x_2 + 2x_3 + \;\;x_4=0\;\; (mod 8),\\
 5x_1 + 6x_2 + 7x_3 + 8x_4=0\;\; (mod 2),\\
 12x_1 + 11x_2 + 10x_3 + 9x_4=0\;\; (mod 6),
 \end{cases}$$ 


{\noindent\bf24. Подмодули максимального ранга} 

\medskip

Пусть $A$ - кольцо, $L$ - подмодуль $A^n$. 

\medskip

{\bf a.} Предположим, что $L$ - свободный подмодуль ранга $n$, и 
обозначим через $d$ определитель базиса $L$ в базисе $A^n$ (нужно отметить, 
что все определители этих базисов равны с точностью до 
обратимого элемента). Доказать, что d не является делителем 0 и что d$A^n \subset L$ 
(можно дать два доказательства, из которых одно будет верно лишь 
для главного случая). 

\medskip

{\bf b.} Предположим, что $A$ - кольцо главных идеалов. Доказать 
эквивалентность rang(L)$ = n\Longleftrightarrow \exists d \in A - \{0\}, \;\;A^n \subset L .$

\pagebreak


{\noindent\bf25. Дополнение и замыкание }

\medskip

Рассмотрим три модуля$ M \subset N \subset L$ над кольцом А. 

\medskip

{\bf a.} Если М является прямым слагаемым в L, показать, что М - 
прямое слагаемое в N. 

\medskip

В дальнейшем в этом упражнении предполагается, что А - кольцо 
главных идеалов, a L - свободный модуль конечного типа. 

\medskip

{\bf b.} Доказать, что если М - прямое слагаемое в L и если М и N 
имеют один и тот же ранг, то М = N. 

\medskip 

{\bf c.} При каком условии М = $\bar{M}$? Напомним: \\
$\bar{M}= \{x \in L|\exists a\in A - \{0\},\;\;ax\in M\}  $. 

\medskip

{\bf d.} Предположим $М \subset N$. Доказать, что $ \bar{M}=\bar{N}$ эквивалентно $rang M = rang N$. \\

\medskip

{\noindent\bf26. Ортогональность и дополнения}

\medskip

Если $х,у\in Z^n $ будем полагать, что $(x,y)=sum^{n}_{i=1}x_iy_i $ и 
$M^{\bot} = \{x \in Z^n|\forall y\in M, (x,y)=0\}$, где М - подмножество в $Z^n$. 

\medskip

{\bf a.} Пусть $( f_1,f_2,\ldots,f_n )$ - семейство из $n$ элементов из $Z^n$. При 
каком условии матрица $(f_i,f_j)_{i,j}$ будет обратимой? 

\medskip

{\bf b.} Пусть М - подмодуль $Z^n$ ранга m. Доказать, что $M^{\bot}$- 
под-\\модуль в $Z^n$ ранга m - n. Указание: можно сначала предположить, что 
М - прямое слагаемое, затем заметить, что М1 = М . 

\medskip

{\bf c.} Доказать, что  $M^{\bot\bot} = \bar{M}$ и что, в частности,  $M^{\bot\bot}$ =  $M$ тогда и\\
только тогда, когда М является прямым слагаемым в $Z^n$. 

\medskip

{\noindent\bf27. Идеалы кольца целых квадратичных} 

\medskip

Пусть $\theta$ - целое квадратичное $(\theta\notin Z)$ и $A = Z[\theta] $. Доказать, что\\
любой идеал I в А порожден двумя элементами. Кроме того, I = An +\\ 
Az, где n - целое число, определенное из равенства nZ = $I\cap Z$. 

\medskip

{\noindent\bf28. Формула Пика} 

\medskip

Рассмотрим квадратную сетку на комплексной плоскости, 
постро-\\енную из точек с целыми координатами, и многоугольник P с 
верши-\\нами в точках этой квадратной сетки. Обозначим через п число точек

\pagebreak

\begin{wrapfigure}{r}{0.5\textwidth}
\center{\includegraphics[width=0.8\linewidth]{Image}}
\end{wrapfigure}

квадратной сетки, находящихся строго внутри\\ 
P, а через f число точек квадратной сетки, 
нахо-\\дящихся на границе P. Формула Пика выражает\\ 
площадь P: площадь(P) = $\frac{f}{2} + n - 1$. 

Например, для приведенного справа 
много-\\угольника p n = 13, f = 11 и по формуле 
Пи-\\ка площадь(p) = $\frac{f}{2} - 1$ = 17,5, что можно\\ 
легко проверить. 

{\bf a.} Доказать формулу Пика для n = 0 и f = 3 (многоугольник 
то-\\гда будет треугольником и 3 вершины треугольника являются 
един-\\ственными точками, принадлежащими треугольнику). Указание: 
рас-\\смотреть две стороны треугольника и показать, что векторы, 
соот\\ветствующие этим сторонам, образуют базис $Z^2$. 

{\bf b.} Доказать формулу Пика с помощью индукции.

{\noindent\bf29. Теорема о ядре} 

Пусть М - модуль над кольцом главных идеалов А. Для $a \in A$ 
обозначим через $Ker_M$ а подмодуль М, определенный равенством $Ker_M a =$ 
$ \{z \in M| ax=0\} $. 

{\bf a.} Пусть$a_1,a_2,...,a_k$ - взаимно простые элементы А, и пусть \\
а = $a_1a_2... a_k$ - их произведение. Доказать теорему о ядре, которая\\ 
гласит, что  $Ker_M a = Ker_M a_1 \oplus...\oplus Ker_M a_k и,$ кроме того, каждый\\ 
проектор $Ker_Ma \to Ker_M a_i $ является гомотетией. 

{\bf b.} Перенести предыдущий результат на случай, когда М - 
векторное u-пространство, и - эндоморфизм в М, и М снабжено следующей 
структурой К[Х]-модуля: 

$$ P\bullet x=P(u)\cdot x = \alpha_nu^n(x)+\cdots+\alpha_1u^n(x)+\alpha_0(x), $$
$$\displaystyle \text{где}\;\;\;x\in M\;\; \text{и}\;\;  P(X) = \sum^n_0\alpha_iX^i\in K[X] $$

{\bf c.} Пусть М - векторное С-пространство, состоящее из бесконечно\\ 
дифференцируемых функций из R в С, на котором действует 
опера-\\тор дифференцирования $\frac{d}{dt}$. Тогда М наделено следующей структурой\\ 
С[t]-модуля: 
$$ P\bullet f=P(\frac{dt}{t})(f)=a_n\frac{d^nf}{dt^n}+\cdots+a_1\frac{d^nf}{dt^n}+\alpha_0f$$
$$\displaystyle f \in M,\;\;\;\;P(t)=\sum^n_0\alpha_it^i\in C[t]. $$

\pagebreak

Вычислить $(X - \lambda) \bullet e^{\lambda t}f$; $(X - \lambda)^n \bullet e^{\lambda t}f;$ $Ker (Х - \lambda)n.$

\medskip

Пусть $Р(t) = t^n+\alpha_{n-1}t^{n-1}+...+\alpha_1t+\alpha_0$ - унитарный многочлен из \\
$\mathbb{C}[t]$ степени $n$, которому соответствует дифференциальное уравнение 

\medskip

$$ \frac{d^{n}f}{dt^{n}}+\alpha_{n-1}\frac{d^{n-1}f}{dt^{n-1}}+...+\alpha_{1}\frac{df}{dt}+\alpha_{0}f = 0.$$

\medskip

Если $P(t) = \prod^k_{i=1}(t-\lambda_i)^{r_i}$, доказать, что $\{t^je^{\lambda_it}| 0 \leqslant j < r_i, 1 \leqslant i \leqslant k\} $\linebreak
- базис пространства решений этого дифференциального уравнения. 

\medskip

{\noindent\bf30. Примерные компоненты модуля кручения} 

\medskip

Пусть М - модуль \textbf{кручения} над кольцом главных идеалов А. \\
Для максимального элемента тг (т.е. экстремального) из A определим\\ 
тг-примарную компоненту модуля М: Мn = $\{i \in M \;|\; \exists\; n \in N :$\\ 
$\pi^nx = 0\}$. Если $\prod$ - система, представляющая максимальные 
(экстре-\\мальные) элементы А, то доказать, что имеет место каноническое 
раз-\\ложение: $ M = \bigoplus_{\pi\in\prod} M_{\pi} $ и что для $ x \in M $-компонента $x_\pi$ элемента х\\ 
является кратным х. (Указание: использовать теорему о ядре.)

\medskip

{\noindent\bf31. Примерные модули над кольцом главных идеалов (задача)}

\medskip

В этом упражнении А обозначает кольцо главных идеалов, $\pi$ - 
экс-\\тремальный элемент кольца А. Нас интересуют тг-модули над А, т.е.\\ 
A-модули M, обнуляющиеся степенью $\pi: \pi^kM = 0$ для некоторого к.\\ 
Цель упражнения - доказать следующий результат: Любой $\pi$-модуль\\ 
М над кольцом главных идеалов является прямой суммой однопоро-\\ 
жденных подмодулей. Доказательство, приведенное ниже, является 
не-\\посредственным следствием основного материала, помещенного в главе\\ 
и не использует никакого предположения о конечности. 

\medskip

{\bf a.} Доказать следующие утверждения: 
($i$) если N - подмодуль М, удовлетворяющий условию М = N + $\pi$M,\\ 
то N = М,\\ 
($ii$) если $(\bar{x_i})_{i\in I}$ - базис векторного A/$\pi$A-пространства М/$\pi$М, то \\
$(x_i)_{i\in I}$- минимальная система образующих в М, \\
(iii) для $x\in M$ имеет место следование $ \alpha x=0,x\ne 0 \Rightarrow\pi|\alpha $, \\
($iv$) если $ \pi M = \bigoplus_{i\in I} ,\ ;\ ;\ ;\text{где} \pi х_i\ne О, \;\;\;\text{то} (\bar{x_i})_{i\in I}$ является 
мно-\\жеством линейно независимых элементов векторного $A/ \pi A$-про- 
странства М/$\pi$М.

\medskip

{\bf b.} Предположим, что тгМ является прямой суммой 
однопорожден-\\ных модулей. Доказать, что М является прямой суммой однопоро- \\
жденных модулей. (Указание: если $ \pi M = \bigoplus_{i\in I} ,\;\;\;\text{где} \pi х_i\ne О, $\;\;\;, то

\pagebreak

доказать, что можно дополнить  $(\bar{x_i})_{i\in I}$ до базиса векторного A/$\pi$A-\\ 
пространства M/$\pi$M с помощью  $(\bar{u_i})_{i\in I}$, удовлетворяющих равенствам \\
$\pi u_j$ = 0.) 


{\bf c.} Закончить, используя индукцию по к. 



{\noindent\bf32. Абелевы группы данного порядка n}

Обозначим через $\theta(n)$ число конечных абелевых групп порядка n. 


{\bf a.} Предполагается, что n - степень простого числа р: п = $p^r$. 
На-0\\
писать алгоритм, перечисляющий $\theta(n)$ абелевых групп порядка n, и 
до-\\
казать, что $\theta(n)$ зависит только от r: $\theta(p^r)$ =$p^r$. Перечислить абелевы \\
группы порядка 64. Составить таблицу р(r) для $1 \leqslant r \leqslant 20$. 



{\bf b.} Пусть даны абелева группа $\Omega$ и $ q \in N^{bullet}$. Будем полагать, что \\
$\Omega^{(q)}=\{ x\in\Omega|qx=0 \}$. Если $\Omega$ - абелева группа порядка$ nm$, где $n$\\
и $m$ взаимно просты, то доказать, что$\Omega=\Omega^{(n)} \oplus \Omega^{(m)}$ и что$\Omega^{(n)}$ имеет 
порядок n. 


{\bf c.} Если пит взаимно просты, доказать, что $\theta$(nm) = $\theta(n)\theta(m)$. 
Вы-\linebreak разить в в зависимости от р. Сколько абелевых групп порядка 21 600? 


{\noindent\bf33. Вычисление инвариантных множителей}


{\bf a.} Найти базис подмодуля Е = $\{(x_i,..., x_n) | x_1 + ... + x_n = 0\}$\\ 
и дополнение Е в Z. Каков ранг Е? Тот же вопрос и для подмодуля \\
F =  $\{(x_i,..., x_n) | x_1 + ... + x_n \}$.


{\bf b.} Проверить, что $E \cap F$ = {0}, и доказать, что $Z^n/E\oplus F\simeq Z/nZ$. 


{\noindent\bf34. Другое вычисление инвариантных множителей} 


Рассмотрим фактормодуль М =$Z^4$/Е, где E - подмодуль $Z^4$, 
по-\\рожденный тремя векторами (7,-2,5,-5), (10,-2,8,-2), (24,-6,18, \\
- 12), и обозначим через $\{f_1,f_2,f_3,f_4\}$семейство образующих в М, 
образ канонического базиса $\{e_1,e_2,e_3,e_4\}$. 


{\bf a.} Показать, что М изоморфен $Z/6Z \times Z \times Z$. 


{\bf b.} Выразить через /$\{f_1,f_2,f_3$\;и\; $f_4\}$ три элемента $g_2,g_3,g_4$
Удовлетво-\\ряющих условию М = $Zg_2\oplus Zg_3\oplus Zg_4$, элемент $g_2$ имеет порядок 6, а\\ 
$g_3$ и $g_4$ бесконечного порядка.

\pagebreak

{\noindent\bf35. Инвариантные множители и НОД миноров}

\medskip

Доказать истинность предложения 64, касающегося инвариантных\\ 
множителей и НОД миноров для матрицы, приведенной далее, 
инвари-\\антные множители которой вычислены в тексте. 



$$X =\begin{pmatrix}
6 & 8 & 4 & 20\\
12 & 12 & 18 & 30\\
18 & 4 & 4 & 10 
\end{pmatrix}$$



{\noindent\bf36. Условие наличия одинаковых инвариантных множителей 
относительно $A^n$} 

\medskip

Пусть Е и F - два подмодуля $A^n$, где А - кольцо главных 
идеа-\\лов. Доказать, что Е и F имеют одинаковые инвариантные множители 
относительно $A^n$ тогда и только тогда, когда фактормодули $A^n$/Е и 
$A^n$/F изоморфны. 

\medskip

{\noindent\bf37. Определяющие соотношения }

\medskip

Пусть $\Omega$ - нетривиальная абелева группа, порожденная тремя
об-\\разующими $g_1,g_2,g_3$, удовлетворяющими соотношениям: 


$$ 3g_1+g_2=g_3=0,\;\;\; 25g_1+8g_2+10g_3=0,\;\;\;46g_1+20g_2+11g_3=0 $$. 


Доказать, что $\Omega$ является циклической группой с 19 элементами.\\ 
(Указание: абсолютное значение некоторого определителя равно 19.)\\ 
Найти такую систему образующих в Z/19Z. 

\medskip

{\noindent\bf38. Образующие и определяющие соотношения (продолжение) }

\medskip

Пусть $\Omega$ - нетривиальная абелева группа, порожденная тремя
век-\\торами $g_1,g_2,g_3$, удовлетворяющими условиям 


$$ 2g_1+4g_2-6g_3=0,\;\;\; 8g_1+20g_2-16g_3=0,\;\;\;72g_1+72g_2-324g_3=0 $$. 


Доказать, что порядок $\Omega$ является делителем 288. Полагая к тому же, \\
что $\Omega$ - группа с 96 элементами, доказать, что $\Omega$ изоморфна группе\\ 
$Z_2\times Z_4\times Z_12 $. 

\medskip

{\noindent\bf39. Упражнение Чарльза Лутвиджа Доджсона }

\medskip

"Не is in distress," the Governor explained as they left the court. "Her 
Radiancy has commanded him to place twenty-four pigs in thoses four sties, so



\end{document}
