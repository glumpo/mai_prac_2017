$C = \{ a\in U(\mathbb{Z}_p)\ |\ a^{\frac{p-1}{2}}\}, U(\mathbb{Z}_p) - C = \{ a\in U(\mathbb{Z}_p)\ | \ a^{\frac{p-1}{2}} = -1,\}$

и отсюда получем необходимый результат
%\end{myproof}

\begin{sled}
Число $—1$ является квадратичным вычетом тогда и только тогда, когда $р \equiv 1$ (mod 4).

Рассмотрим известный результат, связывающий два свойства: << $р$— квадрат по модулю $q$ >> и << $q$ — квадрат по модулю $p$ >>.
\end{sled}

\begin{thm}[закон квадратичной взаимности Лежандра — Гаусса]
Пусть $р$ и $q$ — два различных нечетных простых числа. Тогда
\end{thm}
\begin{center}
$(\frac pq)(\frac qp) = (-1)^{\frac{(p-1)(q-1)}{4}}$.
\end{center}

Среди различных доказательств закона взаимности есть такие, которые используют конечные поля. Вот одно из них. Рассмотрим надполе $\Omega_p$ поля $\mathbb{Z}_p$, содержащее $\omega$, — корень степени $q$ из единицы (можно, конечно, рассмотреть алгебраически замкнутое надполе $\mathbb{Z}_p$, но достаточно и любого другого надполя, в котором многочлен $X^q - 1$ имеет корень, отличный от 1, например, надполе $\mathbb{Z}_p/(P)$, где $Р$ — неприводимый множитель многочлена $X^q - 1$). Рассмотрим сумму Гаусса:

\begin{center}
$\tau = \underset{x\in \mathbb{Z}_q^*}{\sum} (\frac xq)\omega^x$
\end{center}

Отметим что так как $\omega^q = 1$, то отображение $\mathbb{F}_q \ni x \rightarrow \omega^z \in \Omega_p$ определено и удовлетворяет равенству $\omega^{x+y} = \omega^x\omega^y$. Например, для $q = 7$ имеем: $\tau = \omega + \omega^2+\omega^4-\omega^3-\omega^5-\omega^6$. В поле $\Omega_p$ элемент $\tau$ — квадратный корень из $\pm q$, как это показывает следующая лемма.

\begin{lemma}
В предыдущих обозначениях $\tau^2 = (\frac{-1}{q})q$.
\end{lemma}

\begin{myproof}
Имеем:
\begin{center}
$\tau^2 = \underset{x}{\sum} (\frac xq)\omega^x \times \underset{y}{\sum} (\frac yq)\omega^y = \underset{x,y}{\sum} (\frac{xy}{q})\omega^{x+y}$.
\end{center}