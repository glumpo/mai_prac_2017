	\lhead{328}
	\rhead{III-2 \ \ \ \  Нормальная форма подгруппы группы ${\mathbb Z}^{n}$}
	
	\noindent
	В силу леммы 20, если $I^{b}_{1} \neq A$, то $\rho (bP) = p$. В тоже время $\rho (bP)$ являет-\linebreak
	ся минимальным кардинальным числом. Если $I^{b}_{1} = A$, то $\rho (bP) \leqslant p - 1$.\linebreak
	Следовательно,
	$$\rho (bP) \leqslant p - 1 \Longleftrightarrow I^{b}_{1} = A \Longleftrightarrow b \in I_{1}.$$
	Аналогично можно показать, что $I_{k} = \{b \in A \ | \ \rho (bP) \leqslant p - k \}$ для\linebreak
	$1 \leqslant k \leqslant p$.
	
	Итак, мы определили $p, I_{1}, I_{2}, \ldots, I_{p}$, и это определение {\bf внутрен-}\linebreak
	{\bf нее}, зависит от $A$-модуля $P$: $P$ изоморфно $Q$, тогда $p = q$ и $I_{i} = J_{i}$.
	
	Приведем сейчас два следствия из теоремы единственности. Отме-\linebreak
	тим различие между посылками: в первом случае ищут изиморфизм\linebreak
	между частными, и $a_{i}$ и $b_{i}$ предполагаются необратимыми; во втором\linebreak
	$-$ речь идет об изоморфизме в окружающей среде, $a_{i}$ и $b_{j}$ могут быть\linebreak
	обратимыми.
	
	\noindent
	{\bf (22) Следствие.}
	
	Пусть $A$ $-$ {\it коммутативное унитарное кольцо}. Если $a_{1}, a_{2}, \ldots, a_{p}$ и\linebreak
	$b_{1, b_{2}, \ldots, b_{q}}$ {\bf необратимые} элементы, удовлетворяющие условиям:
	$$a_{1} |a_{2}|\ldots |a_{p - 1}|a_{p} \ \ \text{и} \ \ b_{1} |b_{2}|\ldots |b_{q - 1}|b_{q},$$
	$$A/(a_{1}) \times A/(a_{2}) \times \ldots \times A/(a_{p}) \ \simeq \ A/(b_{1}) \times A/(b_{2}) \times \ldots \times A/(b_{q}),$$
	то $p = q$ и элементы $a_{i}$ и $b_{i}$ ассоциированы. В частности, если коль-\linebreak
	цо $A$ без делителей нуля, то элементы $a_{i}$ и $b_{i}$ равны с точностью до\linebreak
	обратимого сомножителя.
	
	\noindent
	{\bf (23) Следствие.}
	
	$A$ обозначает опять коммутативное унитарное кольцо. Пусть\linebreak
	$a_{1}, a_{2}, \ldots, a_{p}$ и $b_{1}, b_{2}, \ldots, b_{q}$ - элементы, удовлетворяющие условиям:\linebreak
	$$a_{1} |a_{2}|\ldots |a_{p - 1}|a_{p} \ \ \text{и} \ \ b_{1} |b_{2}|\ldots |b_{q - 1}|b_{q},$$
	$$\exists \varphi : a^{p} \xrightarrow{\simeq} A^{q}, \ \ \varphi (a_{1}A \times a_{2}A \times \ldots \times a_{p}A) = b_{1}A \times b_{2}A \times \ldots \times b_{q}A.$$
	Тогда $p = q$ и элементы $a_{i}$ и $b_{i}$ ассоциированы (т.е. $a_{i}A = b_{i}A$), что\linebreak
	влечет равенство:
	$$a_{1}A \times a_{2}A \times \ldots \times a_{p}A = b_{1}A \times b_{2}A \times \ldots \times b_{q}A.$$
	Если кольцо $A$ без делителей нуля, то элементы $a_{i}$ и $b_{i}$ равны с точно-\linebreak
	стью до обратимого.

	\pagebreak
	
	\lhead{III-3 \ \ \ \  Вычисление образа и ядра матрицы}
	\rhead{329}

	\noindent
	{\bf Доказательство.}
	
	\begin{tabular}{|p{12.5cm}}
	\noindent
	Так как $A^{p}$ изоморфно $A^{q}$ (обозначим изоморфизм $\varphi$), то $p = q$.\linebreak
	Кроме того, изоморфизм $\varphi$ индуцирует, будучи перенесен на част-\linebreak
	ное, изоморфизм:
	$$A/(a_{1}) \times A/(a_{2}) \times \ldots \times A/(a_{p}) \ \simeq \ A/(b_{1}) \times A/(b_{2}) \times \ldots \times A/(b_{q}).$$
	Нельзя прямо применить теорему 18, так как некоторые из $a_{i}$ или\linebreak
	$b_{j}$ могут быть обратимыми. Определим $n$ и $m$ следующим образом:
	$$1 \leqslant i \leqslant n \Longleftrightarrow  (a_{i}) = A \ \text{и} \ 1 \leqslant j \leqslant m \Longleftrightarrow  (b_{j}) = A.$$
	Тогда имеет место изоморфизм:
	$$A/a_{n + 1} \times A/a_{n + 2} \times \ldots \times A/a_{p}$$
	$$A/b_{m + 1} \times A/b_{m + 2} \times \ldots \times A/b_{q}.$$
	Откула получаем $p - n = q - m$, следовательно $n = m$ (потому, что\linebreak
	$p = q$) и $(a_{i}) = (b_{i})$ для $i > n$. Так как $a_{i}$ и $b_{i}$ с индексами, меньшими\linebreak
	или равными $n$, обратимы, то для всех $i (a_{i}) = (b_{i})$.
	\end{tabular}
	
	\noindent
	{\Large {\bf 3 Вычисление образа и ядра матрицы}}
	
	\noindent
	В следующем разделе мы изучим некоторые вычислительные методы,\linebreak
	применяемые к матрицам с целыми элементами. Действительно, толь-\linebreak
	ко одно арифметическое свойство кольца чисел исполбзуется для\linebreak
	этих вычислений, поэтому везде можно заменитьь слова {\it с целыми эле-\linebreak
	ментами} на слова {\it с элементами в КГИ}. Однако, эти вычислительные\linebreak
	методы могут быть применимы, только если в рассматриваемом кольце\linebreak
	известны коэффициенты Безу.
	
	\noindent
	{\large {\bf 3.1 Ступенчатые матрицы}}
	
	\noindent
	Пусть $X$ $n\times m$-матрица с целыми элементами. Прежде всего мы постро-\linebreak
	им базис в  Im$X$ - подмодуле из ${\mathbb Z}^{n}$, порожденном $m$ столбцами матри-\linebreak
	цы $X$. Основная идея этого метода состоит в замене $n\times m$-матрицы\linebreak
	$X$ другой матрицей $X'$, более простой, чем $X$, и эквивалентной справа\linebreak
	матрице $X$ в смысле, определенном ниже.\newline
	
	\noindent
	{\bf (24) Определение.}
	
	{\it Пусть $X$ и $X'$ $-$ две матрицы с целыми элементами}. Они называю-\linebreak

	\pagebreak
	
	\lhead{330}
	\rhead{III-3 \ \ \ \  Вычисление образа и ядра матрицы}
	
	\noindent
	тся {\bf эквивалентными справа} {\it тогда и только тогда, когда существу-\linebreak
	ет обратимая матрица $R$, удовлетворяющая условию $X' = XR$. Это\linebreak
	равенсто определяет, естесственно, отношение эквивалентности}.\linebreak

	Две эквивалентные справа матрицы $X$ и $X'$ имеют один и тот же\linebreak
	образ: действительно, отношение между функциями $f' = fg$, в котором\linebreak
	$g$ обратимо, влечет, что $f$ и $f'$ имеют один образ. Следовательно, если\linebreak
	знать образ одной из этих матриц, то тогда изестен и образ другой\linebreak
	матрицы. Следовательно, необходимо рассмотреть класс матриц, образ\linebreak
	которых легко вычисляется. Матрицы, называемые ступенчатыми в\linebreak
	столбцах, образуют такой класс.
	
	\noindent
	{\bf (25) Определения.}
	
	{\it {\bf Высотой} вектора $x$ из $A^{n}$ ($n$ - произвольное кольцо) называется\linebreak
	целое число $n - i$, где $i$ самое большое из чисел, для которых $x_{j} = 0$ при\linebreak
	$1 \leqslant j \leqslant i$, т.е. $(0, 0, \ldots, 0, x_{i + 1}, x_{i + 2}, \ldots, x_{n})$ при $x_{i + 1} \neq 0$. Термин {\it $\ll$ вы-\linebreak
	сота$\gg$} происходит из того, что это понятие применяют для столбцов\linebreak
	матрицы. Нулевой вектор, и только он, имеет высоту 0.
	
	Матрица X размеров $n \times m$ со столбцами $X_{1}, X_{2}, \ldots, X_{m}$ назы-\linebreak
	вается {\bf ступенчатой в столбцах}, если высота ее столбцов является\linebreak
	убывающей, т.е. существует $k$ такое, что:
	$$h(X_{1}) > h(X_{2}) > \ldots > h(X_{k}) > 0 \ \text{и} \ X_{k + 1} = X_{k + 2} = \ldots X_{m} = 0,$$
	где $h(X_{i})$ обозначает высоту вектора $X_{i}$.}
	
	\noindent
	{\large {\bf 3.2 Вычисление образа матрицы}}
	
	\noindent
	Ступенчатые матрицы имеют простую структуру и обладают несколь-\linebreak
	кими элементарными свойствами, которые широко используются.
	
	\noindent
	{\bf (26) Свойства} (ступенчатых матриц).
	
	{\it Пусть $X$ $-$ ступенчатая матрица над {\bf кольцом без делителей\linebreak
	нуля.}}

	$(i)$ {\it Ненулевые столбцы матрицы $X$ образуют базис пространства,\linebreak
	порожденного столбцами.}

	$(ii)$ {\it Векторы $(e_{i})_{i > k}$, отвечающие нулевым столбцам $(X_{i})_{i > k}$, обра-\linebreak
	зуют базис ядра матрицы $X$.}
	
	Следующая теорема имеет важное значение: она показывает, что\linebreak
	любая матрица с целыми коэффициентами (в общем случае - с коэф-\linebreak
	фициентами в КГИ) может быть преобразована в ступенчатую мат-\linebreak
	рицу.
	
	\pagebreak
	
	\lhead{III-3.2 \ \ \ \  Вычисление образа матрицы}
	\rhead{331}
	
	\noindent
	{\bf (27) Теорема.}
	
	$(i)$ {\it Существует алгоритм, который, будучи применен к матрице $X$\linebreak
	размеров $n\times m$ с целыми коэффициентами, дает обратимое справа пре-\linebreak
	образование, которое превращает матрицу $X$ в ступенчатую матрицу\linebreak
	$X'$. Более точно, этот алгоритм, примененный к $X$, дает пару $(X', R),$\linebreak
	где $X'$ - ступенчатая $n\times m$-матрица, $R$ - $m\times m$-матрица, принад-\linebreak
	лежащая $SL_{m}(\mathbb Z)$ и удовлетворяющая условию $X' = XR$. В этом случае\linebreak
	говорят, что $X'$} {\bf специально эквивалентна справа} {\it матрице $X$, \it что-\linebreak
	бы подчеркнуть существование правой матрицы перехода с} {\bf определи-\linebreak
	телем 1.}
	
	$(ii)$ {\it Результат остается справедливым, если кольцо целых чисел за-\linebreak
	менить на} {\bf КГИ}, {\it в котором имеется алгоритм, позволяющий вычи-\linebreak
	слить коэффициенты Безу. Если основное кольцо предполагается толь-\linebreak
	ко КГИ, то матрицы $R$ и $X'$ существуют независимо от того, можем\linebreak
	мы их представить в явном виде или нет.}
	
	Эта теорема, доказательство которой сопровождается алгоритмом,\linebreak
	имеет следствие <<абстрактной>> природы.
	
	\noindent
	{\bf (28) Следствие.}
	
	$(i)$ {\it Существует алгоритм, который будучи применен к $m$ векторам\linebreak
	из ${\mathbb Z}^{n}$, строит базис подмодуля в ${\mathbb Z}^{n}$, порожденного этими $m$ векторами.
	
	$(ii)$ Пусть $A$ }$-$ {\bf КГИ}; {\it любой подмодуль $A^{n}$ имеет базис, образо-\linebreak
	ванный самое большое из $n$ векторов. В частности, над кольцом глав-\linebreak
	ных идеалов любой подмодуль} {\bf свободного модуля конечного типа}\linebreak
	{\it является свободным конечного типа.}
	
		\fbox{%
		\parbox{10cm}{%
				{\footnotesize\definecolor{light-gray}{rgb}{0.8,0.8,0.8}
					\colorbox{light-gray}{$X'\in M_{nm}, X'_1, X'_2, ...,$ обозначают столбцы матрицы $X'$}\newline
				$X' \leftarrow X; j \leftarrow 1;$\newline
				for ($i = 0; i == n; i++$) \{\newline
				${\ \ \ }$if ($j == m$) \{\newline
				${\ \ \ \ \ \ }$break;\newline
				${\ \ \ }$\}\newline
				${\ \ \ }$for ($k = j + 1; k == m; k += 1$) \{\newline
				${\ \ \ \ \ \ }$\definecolor{light-gray}{rgb}{0.8,0.8,0.8}
				\colorbox{light-gray}{Вычислить ${\scriptsize \begin{pmatrix} \alpha & \gamma \\ \beta & \delta \end{pmatrix}_{jk}}$, удовлетворяющую условиям:}\newline
				${\ \ \ \ \ \ \ \ \ }$\definecolor{light-gray}{rgb}{0.8,0.8,0.8}
				\colorbox{light-gray}{ $\gamma X'_{ij} + \delta X'_{ik} = 0$ и $\alpha \delta - \beta \gamma = 1$}\newline
				${\ \ \ \ \ \ }$$(X'_j, X'_k) \leftarrow (\alpha X'_j + \beta X'_k, \gamma X'_j + \delta X'_k)$;\newline
				${\ \ \ }$\}\newline
				${\ \ \ }$if ($X'(i, j) \neq 0$) \{\newline
				${\ \ \ \ \ \ }$$j \leftarrow j + 1$;\newline
				${\ \ \ }$\}\newline
				\}\newline
				\definecolor{light-gray}{rgb}{0.8,0.8,0.8}
				\colorbox{light-gray}{$X' = XR$, где $R - $матрица с определителем 1. Она не вычислена.}}}}

	{\bf Алгоритм 3.} Приведение матрицы $X$ к ступенчатому\newline
	${\ \ \ \ \ \ \ \ \ \ \ \ \ \ \ \ \ \ \ \ \ \ \ \ \ \ \ \ \ \ \ \ \ \ \ \ \ \ \ }$ в столбцах виду 
	
	\pagebreak
	
	\lhead{332}
	\rhead{III-3 \ \ \ \  Вычисление образа и ядра матрицы}
	
	\noindent
	{\bf Доказательство} (следствия).

	\begin{tabular}{|p{12.5cm}}
	$(i)$ является частным случаем предыдущей теоремы, если рассмо-\linebreak
	треть матрицу, образованную данными $m$ векторами.\linebreak
	Чтобы доказать $(ii)$, рассмотрим подмодуль $M$ кольца $A^n$. Так как\linebreak
	$A^{n}$ - нётерово, то подмодуль $M$ имеет конечное число образую-\linebreak
	щих (обобщение модулей предложения 17 раздела II-2.4, посвящен-\linebreak
	ного свойству конечности идеалов нётерово кольца). Тогда можно\linebreak
	применить пункт $(i)$.
	\end{tabular}
	
	\noindent
	{\bf Неконструктивное доказательство}
	
	Докажем по индукции, что любой подмодуль $E$ кольца $A^{n}$ свободен\linebreak
	с базисом, состоящим из меньшего или равного $n$ числа элементов.
	
	Если $n = 1$, то $E$ - идеал в $A$, следовательно, $E$ имеет вид $Aa$, и\linebreak
	$\{a\}$ - базис $E$ (исключая, может быть, случай $a = 0$, для которого $\emptyset$\linebreak
	будет базисом $E$).
	
	Возьмем $n\geqslant 2$ и пусть $\pi_{n}:A^{n} \ \textrightarrow \ A$ - координатная функция\linebreak
	на $e_n$ - последнем векторе канонического базиса. Если $\pi_{n}(E) = 0$, то\linebreak
	$E\subset A^{n - 1}$ и  имеет (предположение индукции) базис $\{f_1, f_2, \ldots, f_m\}$ для\linebreak
	$m\leqslant n - 1$. Если $\pi_{n}(E) \neq 0$, то $E$ является идеалом $Aa$ кольца $A$ с $a \neq 0$.\linebreak
	Пусть $x \ in E$ и $\pi_{n}(x) = a$. Докажем тогда, что
	
	$$(E \cap A^{n - 1}) \oplus Ax = E \ \ \ \ \ \ \ \ \ \ \ \ \ \ \ (2)$$
	Сумма прямая, потому что, если $A \in E \cap A^{n - 1} \cap Ax$, то $y = \lambda x$ и\linebreak
	$\pi_{n}(y) = 0$. Отсюда $\lambda = 0$, затем $y = 0$. Сумма равна $E$, так как, если для\linebreak
	$y \in E$ взять разложение $y = y' + \lambda x$, то получим $\pi_{n}(y) = \lambda a$, что опре-\linebreak
	деляет $\lambda$ (по предположению $\pi_{n}(y) \in Aa$). Если положить $y' = y - \lambda x$,\linebreak
	то получим $y' \ in E \cap A^{n - 1}$.
	
	По индукции подмодуль $E \cap A^{n - 1}$ кольца $A^{n - 1}$ имеет базис\linebreak
	$\{f_1, f_2, \ldots, f_m\}$ при $m \leqslant n - 1$. Равенство (2) доказывает, что\linebreak
	$\{f_1, f_2, \ldots, f_m, x\}$ - базис $E$.
	
	В последнем пункте предыдущего следствия можно убрать условие\linebreak
	конечности, что приводит к следующему предложению, доказанному в\linebreak
	упражнении 16.
	
	\noindent
	{\bf (29) Предложение.}
	
	{\it Над кольцом главных идеалов любой подмодуль свободного модуля\linebreak
	свободен.}
	
	\pagebreak
	
	\lhead{III-3.2 \ \ \ \  Вычисление образа матрицы}
	\rhead{333}
	
	\noindent
	{\bf Доказательство теоремы 27}
	
	Покажем для начала как вычисляют $X'$; вычисление $R$ может быть\linebreak
	получено, как частный случай этого метода.
	
	Пусть $X'$ - $n\times n$-матрица, определенная по $X$. Ее столбцы обо-\linebreak
	значают через $X'_{1}, X'_{2}, \ldots, X'_{m}$. Благодаря лемме 6 об исключениях, на\linebreak
	столбцах $X'$ можно реализовать операцию:
	$$(X'_1, X'_2) \textleftarrow (\alpha X'_1 + \beta X'_2, \gamma X'_1 + \delta X'_2)$$
	$$\ \ \ \ \ \ \ \ \ \ \ \ \ \ \ \ \ \ \ \ \ \ \ \ \ \ \ \ \ \ \ \ \ \text{при} \ \alpha\delta-\beta\gamma = 1 \ \text{и} \ \gamma X'_{11} + \delta X'_{12} = 0.$$
	С обозначениями из раздела 1 эту операцию можно еще записать в виде:
	$$(X'_1, X'_2) \textleftarrow (X'_1, X'_2) \begin{pmatrix} \alpha & \gamma \\ \beta & \delta \end{pmatrix}_{12}.$$
	Эта операция производит к умножению справа матрицы $X'$ на матрицу\linebreak
	из $SL_{m}(\mathbb Z)$ и обнуляет $X'_{12}$. Таким же образом обнуляются элементы в\linebreak
	$X'_{12}, X'_{13}, \ldots, X'_{1m}$.
	
	При итераци метода представляются два случая:
	
	\begin{itemize}
	\item $X'_{11} \neq 0$. Тогда приводят к ступенчатому виду (по индукции) под-\linebreak
	матрицу $X'(2..n, 2..m)$, образованную $n - 1$ последними строками\linebreak
	и $m - 1$ последними столбцами.
	
	\item $X'_{11} = 0$. Тогда приводят к ступенчатому виду (по индукции) под-\linebreak
	матрицу $X'(2..n, 1..m)$, состоящую из $n - 1$ послеедних строк и\linebreak
	$m$ столбцов.
	\end{itemize}
	
	В обоих случаях результирующая матрица $X'$ будет ступенчатой.\linebreak
	Существование правой матрицы перехода $R$ приводит к тому, что ка-\linebreak
	ждая операция в действительности является умножением справа на эле-\linebreak
	ментарную матрицу.
	\begin{flushright}$\blacksquare$\end{flushright}
	{\ \ \ }Итак, можно дать более простой алгоритм, реализующий преобра-\linebreak
	зование матрицы $X$ в ступенчатую матрицу $X'$, специально эквивалент-\linebreak
	ную справа исходной матрице. Получение правой матрицы перехода $R$,\linebreak
	{\bf использующее только этот алгоритм}, будет объяснено ниже.\linebreak
	Чтобы лучше понять алгоритм 3, нужно отметить, что на данном\linebreak
	{\small
	\begin{minipage}[t]{80mm}\parindent=2em
		\noindent
		этапе выбирают индекс строки $i$ и ин-\linebreak
		декс столбца $j$ и пытаются обнулить все\linebreak
		элементы строки $i$ с индексом $k > j$.\linebreak
		Рассмотрим этот процесс на примере.\linebreak
		В этой матрице $a_{11}$ и $a_{32}$ {\bf не нули}, знач-\linebreak
		ки . обозначают произвольные элемен-\linebreak
		ты, а $\bullet$ - расположение начального мо-
	\end{minipage}
	\hfill
	\begin{minipage}[t]{50mm}
	\[ \bordermatrix{
		& & & \stackrel{j}{\textdownarrow} & & \stackrel{k}{\textdownarrow} & \cr
		& a_{11} & 0 & 0 & 0 & 0 & 0 \cr
		& . & 0 & 0 & 0 & 0 & 0 \cr
		& . & a_{32} & 0 & 0 & 0 & 0 \cr
		& . & . & 0 & 0 & 0 & 0 \cr
		i\textrightarrow  & . & . & \bullet & 0 & \ast & . \cr
	    & . & . & . & . & . & . \cr}
	\]
	\end{minipage}}

	\pagebreak
	
	\lhead{334}
	\rhead{III-3 \ \ \ \  Вычисление образа и ядра матрицы}
	
	\noindent
	мента (осевой столбец). Пытаются обнулить элемент $\ast$ (т.е. элемент в\linebreak
	строке $i$ и столбце $j$).
	
	\noindent
	{\bf Пример приведения к ступенчатому виду}
	
	Вот конкретный пример применения программы, иллюстрирующий\linebreak
	этапы приведения к ступенчатому виду матрицы размеров $4\times 3$. Приво-\linebreak
	дится каждый этап, в котором обнуляемый коэффициент еще не нуль.\linebreak
	Запись \underline{0} обозначает, что данный коэффициент {\it стал нулем на этом\linebreak
	этапе.}
	$$\begin{pmatrix} 6 & 15 & 10 \\ 12 & 0 & 10 \\ -12 & 30 & 0 \\ 18 & 15 & 20\end{pmatrix}, \begin{pmatrix} 3 & \underline{0} & 10 \\ -24 & -60 & 10 \\ 54 & 120 & 0 \\ -21 & -60 & 20\end{pmatrix},$$
	$$\begin{pmatrix} 1 & 0 & \underline{0} \\ 82 & -60 & 270 \\ -162 & 120 & -540 \\ 83 & -60 & 270\end{pmatrix}, \begin{pmatrix} 1 & 0 & 0 \\ 83 & 30 & \underline{0} \\ -162 & -60 & 0 \\ 83 & 30 & 0\end{pmatrix}$$
	
	Пусть $F$ - подпространство из ${\mathbb Z}^4$, порожденное 3 столбцами матрицы\linebreak
	$X$. Процесс приведения к ступенчатому виду из $X$ в $X'$ дает базис $F$.\linebreak
	Действительно, два первых столбца ступенчатой матрицы $X'$ образу-\linebreak
	ет базис $ImX'$, а значит, $F = ImX = ImX'$. Читателю предлагается\linebreak
	решить упражнение 12, в котором сообщается, что хотя $F$ имеет раз-\linebreak
	мерность 2, свободная система $\{X_1, X_2\}$ из $F$ не образует базис $F$ и не\linebreak
	может быть дополнена до базиса $F$.
	
	\noindent
	{\bf (30) Предупреждение.}
	
	{\it Приведем несколько основных классических результатов линейной\linebreak
	алгебры для векторных пространств.
	
	$(i)$ В ${\mathbb Z}^n$ в общем случае нельзя найти базис системы образующих.
	
	$(ii)$ Свободная система из $n$ векторов в ${\mathbb Z}^n$ не имеет никакого отно-\linebreak
	шения к построению базиса в ${\mathbb Z}^n$.
	
	$(iii)$ Дополнить свободную систему ${\mathbb Z}^n$ до базиса ${\mathbb Z}^n$, вообще говоря\linebreak
	невозможно.}

	Посмотрим сейчас, насколько это возможно в процессе приведения\linebreak
	к ступенчатому виду $X \rightarrow X' = XR$, как вычислить матрицу пере-\linebreak
	хода $R$. Достаточно взять переменную $R$ матричного типа размеров\linebreak
	$n\times m$, пока равную единичной, и произвести над ней те же элементар-\linebreak
	ные умножения справа, что и над $X'$. Практически процесс состоит\linebreak
	
	\pagebreak
	
	\lhead{III-3.2 \ \ \ \  Вычисление образа матрицы}
	\rhead{335}
	
	\noindent
	в построении новой матрицы. Приписывая к матрице $X$ {\it снизу} единич-\linebreak
	ную $n\times m$-матрицу, применим к этой $(n + m) \times m$-матрице алгоритм\linebreak
	приведения к ступенчатому виду по столбцам, помня, что достаточ-\linebreak
	но преобразовать к ступенчатому виду только верхнюю часть. Если\linebreak
	обозначить через $X'$ переменную, представляющую $n\times m$-подматрицу,\linebreak
	образованную из $n$ первых строк (т.е. первых строк) и через $R$ - пе-\linebreak
	ременную, представляющую подматрицу размеров $m\times m$, состоящую\linebreak
	из $m$ последних строк (т.е. нижних строк), то соотношение\linebreak
	$$XR = X', \ \ \ \ \ R \in SL_{m}(A),$$
	остается истинным на каждом шаге (обе части этого равенства умно-\linebreak
	жены справа на одну и ту же элементарнуюматрицу из $SL_{m}(A)$). Итак,\linebreak
	это соотношение остается истинным до конца алгоритма приведения к\linebreak
	ступенчатому виду и позволяет вычислить, таким образом, в явном\linebreak
	виде матрицу перехода $R$.
	
	\begin{flushleft}
	\hangindent=1cm \hangafter=0 \noindent
	{\small {\bf Замечание.}${\ \ \ \ \ \ \ }$Отметим, что в силу соотношения $X' = XR$ воз-\linebreak
		можно получить выражение векторов $X'_j$, векторов-столбцов ма-\linebreak
		трицы \  $X'$ \ в зависимости от векторов \ $X_j$, \ столбцов матрицы \ $X$.\linebreak
		Чтобы выразить \ $X_j$ через \ $X'_j$, достаточно умножить на \ $R^{-1}$ обе\linebreak
		части \ равенства, \ что возможно сделать с помощью \ алгоритма.\linebreak
		Можно также построить базис, исходя и з системы образующих,\linebreak
		и выразить в явном виде элемент базиса в этой \ системе образу-\linebreak
		ющих и наоборот.}
	\end{flushleft}
	
	\noindent
	{\bf Пример приведения к ступенчатому виду (продолжение)}
	
	Приведем пример, показывающий получение правой матрицы пере-\linebreak
	хода. Как обычно, показан каждый этап обнуления элемента; это обну-\linebreak
	ление будет обозначаться через подчеркнутый ноль \underline{0}:
	$$\begin{pmatrix} 6 & 15 & 10 \\ 12 & 0 & 10 \\ -12 & 30 & 0 \\ 18 & 15 & 20 \\ 1 & 0 & 0 \\ 0 & 1 & 0 \\ 0 & 0 & 1\end{pmatrix}\begin{pmatrix} 3 & \underline{0} & 10 \\ -24 & -60 & 10 \\ 54 & 120 & 0 \\ -21 & -60 & 20 \\ -2 & -5 & 0 \\ 1 & 2 & 0 \\ 0 & 0 & 1\end{pmatrix}$$
	
	\pagebreak
	
	\lhead{336}
	\rhead{III-3 \ \ \ \  Вычисление образа и ядра матрицы}
	
	$$\begin{pmatrix} 1 & 0 & \underline{0} \\ 82 & -60 & 270 \\ -162 & 120 & -540 \\ 83 & -60 & 270 \\ 6 & -5 & 20 \\ -3 & 2 & -10 \\ 1 & 0 & 3\end{pmatrix}\begin{pmatrix} 1 & 0 & 0 \\ 82 & 30 & \underline{0} \\ -162 & -60 & 0 \\ 83 & 30 & 0 \\ 6 & 0 & 5 \\ -3 & -2 & 2 \\ 1 & 3 & -6\end{pmatrix}.$$
	Сохраняя предыдущие обозначения, покажем на примере, как знание\linebreak
	матрицы $R$ помогает выразить векторы $X'_1, X'_2$ и  базиса $F$ в зависимо-\linebreak
	сти от начальной системы образующих $X_1, X_2, X_3$. Равенство $X' = XR$\linebreak
	влечет, в частности, равенства $X'_1 = XR_1$ и $X'_2 = XR_2$; они могут быть\linebreak
	записны в виде:
	$$X'_1 = 6X_1 - 3X_2 + X_3, X'_2 = -2X_2 + 3X_3.$$
	Отметим, что нулевое значение последней колонки влечет отношение\linebreak
	зависимости между $X_1, X_2$ и $X_3$:
	$5X_1 + 2X_2 - 6X_3 = 0.$
	Чтобы выразить $X_j$ через $X'_j$, достаточно знать матрицу $R^{-1}$. Эту\linebreak
	матрицу лгко получить, применив алгоритм приведения к ступенча-\linebreak
	тому виду. Действительно, $R$ получена, начиная с единичной матрицы,\linebreak
	умножением {\it справа} на элементарные матрицы {\scriptsize $\begin{pmatrix} \alpha & \gamma \\ \beta & \delta \end{pmatrix}_{jk}$}. Как следствие,\linebreak
	достаточно иметь квадратную матрицу $S$ того же размера, что и $R$, и\linebreak
	определенную как единичная. Затем подвергнуть ее умножению {\it слева},\linebreak
	на элементарные обратные к предшествующим, матрицы столько раз,\linebreak
	сколько подвергалась и матрца $R$. Отношение $RS = Id_m$ тогда оста-\linebreak
	нется инвариантным с начала и до конца алгоритма. Вэтом примере\linebreak
	мы получили матрицу $R^{-1}$ и выражение системы образующих в базисе\linebreak
	$$\begin{pmatrix} 6 & 15 & 10 \\ -16 & -41 & -27 \\ -7 & -18 & -12\end{pmatrix}, X_1 = 6X'_1 - 16X'_2$$
	$${\ \ \ \ \ \ \ \ \ \ \ \ \ \ \ \ \ \ \ \ \ \ \ \ \ \ \ \ \ \ \ \ \ \ \ \ \ \ \ \ \ \ }X_2 = 15X'_1 - 41X'_2, {\ \ \ }X_3 = 10X'_1 - 27X'_2.$$
	
	{\bf{\large3.3 Существование решения системы линейных\newline ${\ \ \ \ \ \ \ \ \ }$ уравнений}}
	
	Пусть $A$ - матрица размеров $n\times m$ с целыми элементами и $b$ - век-\linebreak
	тор с $n$ целыми компонентами, ассоциированные с системой линейных\linebreak
	уравнений $Ax = b$. Решить явно эту систему, значит:
	
	\pagebreak
	
	\lhead{III-3.3 \ \ \ \  Существование решения системы линейный уравнений}
	\rhead{337}
	
	{\bf1)} быть способным показать, имеет ли эта система одно решение и\linebreak
	при необходимости предъявить его,
	
	{\bf 2)} если система имеет решения, то найти их все, т.е. найти базис\linebreak
	ядра $KerA$.
	
	Общее решение системы линейных уравнений рассмотренно в разде-\linebreak
	ле 3.5. Здесь мы будем рассматривать только поиск решения. Сказать,\linebreak
	что система $Ax = b$ имеет решение, то же самое, что сказать, что век-\linebreak
	тор $b$ принадлежит образу $ImA$; найти решение - значит представить\linebreak
	вектор $b$ как линейную комбинацию столбцов $A$. Рассмотрим матри-\linebreak
	цу, полученную приписыванием справа к матрице $A$ столбца $b$. Итак,\linebreak
	достаточно изучить следующую задачу.
	
	{\bf 3)} Пусть дана матрица $X$ с $n$ строками и $m$ столбцами. Определить,\linebreak
	является ли последний столбец $X_m$ линейной комбинацией пред-\linebreak
	шествующих столбцов $X_1, X_2, \ldots, X_{m-1}$ и если да, то найти в\linebreak
	явном виде эту линейную комбинацию.
	
	\noindent
	{\bf (31) Теорема.}
	
	{\it Пусть $X$ - $n\times m$-матрица с целыми элементами. Обозначим через\linebreak
	$X'$ и $R$ две матрицы, полученные с помощью алгоритма приведения\linebreak
	к ступенчатому виду по столбцам ($R$ - квадратная матрица порядка\linebreak
	$m$ и обратима, $X' = XR$ \texttwelveudash ${\ }$ матрица, ступенчатая по столбцам). Если\linebreak
	$R = (r_{ij})$, следующие два свойства эквивалентны:
	
	a) последний столбец $X_m$ является линейной комбинацией столбцов\linebreak
	$X_1, X_2, \ldots, X_{m-1}$,
	
	b) столбец $X'_m$ нулевой и $r_{mm}$ обратимо.
	
	Тогда последняя строка матрицы $R$ будет $(0, 0, \ldots, 0, 0, 1)$, и имеет\linebreak
	место равенство:
	$$X_m = -(r_{1m}X_1 + r_{2m}X_2) + \ldots + r_{m - 1m}X_{m - 1},$$
	которое доказывает, что последний столбец $X_m$ матрицы $X$ является\linebreak
	линейной комбинацией предшествующих столбцов.\linebreak
	Эти результаты обобщаются, очевидно, на любое кольцо гланых\linebreak
	идеалов; сложностб вычисления зависит от сложности вычисления ко-\linebreak
	эффициентов Безу.}
	
	\noindent
	{\bf Доказательство.}
	
	\begin{tabular}{|p{12.5cm}}
	Докажем сначала импликацию $b \Rightarrow a$. Равенство $X' = XR$, интер-\linebreak
	претированное в терминах столбцов, описывается так:
	$$\textperiodcentered X'_j = r_{1j}X_1 + r_{2j}X_2 \ldots + r_{mj}X_m \ \  \text{для} \ \  1 \leqslant j \leqslant m.$$
	\end{tabular}
	
	\pagebreak
	
	\lhead{338}
	\rhead{III-3 \ \ \ \  Вычисление образа и ядра матрицы}
	
	\begin{tabular}{|p{12.5cm}}
	\noindent
	Если столбец $X'_m$ нулевой и если $e_{mm}$ обратимо, то из равенства,\linebreak
	приведенного выше, следует при $j = m$ выражение столбца $X_m$ в\linebreak
	виде линейной комбинации $X_1, X_2, \ldots, X_{m - 1}$.\linebreak
	Чтобы доказать обратную тмпликацию ${\bf a \Rightarrow b}$, нужно доказать,\linebreak
	что в любой момент в алгоритме приведения к ступенчатому виду\linebreak
	последняя строка матрицы $R$ всегда будет равна $(0, 0, \ldots, 0, 0, 1)$,\linebreak
	что можно еще записать в виде:
	$$R({\mathbb Z}^{n - 1}\times 0 = {\mathbb Z}^{n - 1}\times 0) \ \  \text{и} \ \ R(e_m) - e_m \in {\mathbb Z}^{n - 1}\times 0 \ \ \ \ \ \ \ \ \ \ \ \ \ \ \ \ \ \ \ \ \ (3)$$
	${\mathbb Z}^{n - 1}\times 0$ обозначает подпространство ${\mathbb Z}^{n - 1}$, образованное $n - 1$ пер-\linebreak
	выми векторами $(e_i)_{1 \leqslant i \leqslant n - 1}$ канонического базиса.\newline
	Это доказывается по индукции: сначала возбмем квадратную ма-\linebreak
	трицу $R$ единичной. На каждом этапе $R$ изменяется - умножается\linebreak
	на элементарные матрицы:
	$$R \ \longleftarrow R{\small \begin{pmatrix} \alpha & \gamma\\ \beta & \delta \end{pmatrix}_{jk}}$$
	Так как множество матриц, удовлетворяющих условию (3), образу-\linebreak
	ют подгруппу $GL_m(\mathbb Z)$, достаточно доказать, что матрица ${\scriptsize \begin{pmatrix} \alpha & \gamma\\ \beta & \delta \end{pmatrix}_{jk}}$\linebreak
	принадлежит этой подгруппе, что очевидно, если $k < m$.\linebreak
	Итак, интересен этап, для которого индексы осевых столбцов будут\linebreak
	$j$ и $m$. Достаточно доказать, что в этом случае $\beta = 0$ и $\delta = 1$.\linebreak
	Если $X'$ обозначает матричную переменную $XR$, то столбец $X'_m$\linebreak
	является линейной комбинацией предыдущих столбцов $X'_1, X'_2, \ldots$,\linebreak
	$X'_{m - 1}$. Действительно:
	$$X'_m \stackrel{\text{def}}{=} ZR(e_m)\in XR({\mathbb Z}^n) = X({\mathbb Z}^n) = X({\mathbb Z}^n \times 0)$$
	$${\ \ \ \ \ \ \ \ \ \ \ \ \ \ \ \ \ \ \ \ \ \ \ }= XR({\mathbb Z}^n \times 0) \stackrel{\text{def}}{=} X'({\mathbb Z}^n \times 0).$$
	Первое равенство верно, так как $R$ обратима. Второе - так как\linebreak
	$X_m = X(e_m)$ является по предположению линейной комбинацией\linebreak
	$X'_1, X'_2, \ldots, X'_{m - 1}$ и так как $R$ удовлетворяет условию (3).\linebreak
	Покажем, что $x'_{ij}$ делит $x'_{im}, i$ - индекс текущей строки. Рассмо-\linebreak
	трим отношение между столбцами:
	$$X'_m = \lambda_1 X'_1 + \lambda_2 X'_2 + \ldots + \lambda_j X'_j + \ldots + \lambda_{m - 1} X'_{m - 1} {\ \ \ \ \ \ \ \ \ \ \ \ \ \ \ \ \ \ \ \ (4)}$$
	Так как матрица $X'$ уже частично ступенчатая, то подматрица,\linebreak
	составленная из $i - 1$ первых строк, будет ступенчатой. Ненулевые\linebreak
	\end{tabular}
	
	\pagebreak
	
	\lhead{III-3.3 \ \ \ \  Существование решения системы линейный уравнений}
	\rhead{339}
	
	\begin{tabular}{|p{12.5cm}}
	\noindent
	столбцы этой подматрицы, т.е. столбцы с индексом $< j$, образу-\linebreak
	ют множество линейно независимых элементов и отношение (4),\linebreak
	ограниченное до $i - 1$ первых строк, образует нулевую линейную\linebreak
	комбинацию этой линейно независимой системы. Следовательно,\linebreak
	$\lambda_1 = \lambda_2 = \ldots = \lambda_{j - 1} = 0$. Элемент отношения (4), индекс стро-\linebreak
	ки которого $i$, описывается как: $x'_{im} = \lambda_i x'_{ij}$, что доказывает ука-\linebreak
	занное отношение делимости. Однако алгоритм оговаривает, что\linebreak
	в случае, когда $x'_{ij}$ делит $x'_{im}$, нужно выбрать элемент $\beta$ матрицы\linebreak
	${\scriptsize \begin{pmatrix} \alpha & \gamma\\ \beta & \delta \end{pmatrix}_{jm}}$ равным 0, а $\delta$ - равным 1.\linebreak
	То, что последний столбец $X'_m$ {\bf финальной} матрицы $X'$ будет нуле-\linebreak
	вым, доказывает, что этот столбец является линейной комбинацией\linebreak
	предыдущих столбцов и что $X'$ {\bf полностью ступенчата}.
	\end{tabular}
	
	\noindent
	{\bf Пример решения}
	
	Вот пример решения линейной системы, иллюстрирующий доказа-\linebreak
	тельство теоремы 31. Рассмотрим линейную систему $Ax = b$ с:\linebreak
	$$\begin{pmatrix} 3 & 2 & 3 & 4 \\ 1 & -2 & 1 & -1 \end{pmatrix} \ \ \text{и} \ \ \begin{pmatrix} -8 \\ -3 \end{pmatrix}.$$
	Приведем этапы процесса ступенчатости:
	$$\begin{pmatrix} 3 & 2 & 3 & 4 & -8 \\ 1 & -2 & 1 & -1 & -3 \end{pmatrix}, \begin{pmatrix} -1 & \underline{0} & 3 & 4 & -8 \\ -3 & 8 & 1 & -1 & -3 \end{pmatrix},$$
	$$\begin{pmatrix} -1 & 0 & \underline{0} & 4 & -8 \\ 3 & 8 & -8 & -1 & -3 \end{pmatrix}, \begin{pmatrix} -1 & 0 & 0 & \underline{0} & -8 \\ 3 & 8 & -8 & -13 & -3 \end{pmatrix},$$
	$$\begin{pmatrix} -1 & 0 & 0 & 0 & \underline{0} \\ -3 & 8 & -8 & -13 & 21 \end{pmatrix}, \begin{pmatrix} -1 & 0 & 0 & 0 & 0 \\ -3 & 8 & \underline{0} & -13 & 21 \end{pmatrix},$$
	$$\begin{pmatrix} -1 & 0 & 0 & 0 & 0 \\ -3 & 1 & 0 & \underline{0} & 21 \end{pmatrix}, \begin{pmatrix} -1 & 0 & 0 & 0 & 0 \\ -3 & 1 & 0 & 0 & \underline{0} \end{pmatrix}.$$
	Получена правая матрица перехода:
	$$\begin{pmatrix} -1 & -2 & -1 & -6 & 50 \\ 1 & -3 & 0 & -7 & 55 \\ 0 & 0 & 1 & 0 & 0 \\ 0 & 3 & 0 & 8 & -63 \\ 0 & 0 & 0 & 0 & 1\end{pmatrix}.$$
	Тогда заключаем, что вектор $x = (-50, -55, 0, 63)$ является частным\linebreak
	решением предложенной системы. 
	
	\pagebreak
	
	\lhead{340}
	\rhead{III-3 \ \ \ \  Вычисление образа и ядра матрицы}
	
	\noindent
	{\large {\bf 3.4 Вычисление ядра матрицы}}
	
	\noindent
	Ступенчатый вид матрицы позволяет также выразить ядро матрицы.
	
	\noindent
	{\bf (32) Предложение.}
	
	{\it $A$ обозначает здесь кольцо главных идеалов. Пусть $X \ \texttwelveudash \  n\times m$-\linebreak
	матрица с элементами в $A$ и $R$ - обратимая матрица порядка $m$ такая,\linebreak
	что $X' = XR$ будет ступенчатой по столбцам. Если $X'_{k + 1}, X'_{k + 2}, \ldots$,\linebreak
	$X'_{m}$ $\ \texttwelveudash \ $ нулевые столбцы матрицы $X'$, то:
	
	$(i)$ последние столбцы $R_{k + 1}, \ldots, R_{m}$ матрицы $R$ образуют базис\linebreak
	подпространства Ker$X$ кольца $A^m$,
	
	$(ii)$ $k$ первых столбцов матрицы $R$ образуют базис дополнения $S$ до\linebreak
	Ker$X$ в $A^m$:
	$$S = \bigoplus\limits_{i = 1}^k AR_i, KerX = \bigoplus\limits_{i = k + 1}^m AR_i \ \text{и} \ A^m = S \oplus KerX,$$
	$(iii)$ для $x \in A^m$ разложение $R^{-1}(x)$ в каноническом базисе будет\linebreak
	иметь вид: $R^{-1}(x) = \sum_{i = 1}^m \lambda_i (x)e_i$. Тогда $\sum_{i = 1}^k \lambda_i (x)R_i$ и $\sum_{i > k}^m \lambda_i (x)R_i$\linebreak
	будут компонентами $X$ в прямой сумме, описанной выше. Следователь-\linebreak
	но, $KerX$ {\bf выражает} прямое слагаемое в $A^m$.}
	
	\noindent
	{\bf Доказательство.}
	
	\begin{tabular}{|p{12.5cm}}
	Равенство $X' = XR$ доказывает для $x\in A^m$ эквивалентность $Rx\in$\linebreak
	$KerX \Longleftrightarrow x\in KerX'$ и, следовательно, $R(KerX') = KerX$. Ясно,\linebreak
	что $e_{k + 1}, e_{k + 2}, \ldots, e_{m}$ образует базис ядра $X'$. Остальную часть\linebreak
	предложения докажем ниже.
	\end{tabular}

	\noindent
	{\bf Пример}
	
	Проиллюстрируем это доказательство на примере матрицы разме-\linebreak
	ров $3\times 4$. Возьмем матрицу $X$, ее ступенчатую форму $X'$, матрицу $R$\linebreak
	перехода от $X$ к $X'$, удовлетворяющую условию $X' = XR$.
	$$X = \begin{pmatrix} 6 & 3 & 6 & 1 \\ 2 & 5 & 6 & -1 \\ 8 & 11 & 15 & -1 \end{pmatrix}, \ \ \ X' = \begin{pmatrix} 1 & 0 & 0 & 0 \\ -1 & -4 & 0 & 0 \\ -1 & -7 & 0 & 0 \end{pmatrix},$$
	$$R = \begin{pmatrix} 0 & 0 & 1 & 0 \\ 0 & -2 & 2 & 3 \\ 0 & 1 & -2 & -2 \\ 1 & 0 & 0 & 3 \end{pmatrix}.$$
	
	\pagebreak
	
	\lhead{III-3.3 \ \ \ \  Общее решение системы линейный уравнений}
	\rhead{341}
	
	Усли обозначить через $R_i$ столбцы $R$, то векторы $R_3$ и $R_4$, ко-\linebreak
	торые соответствуют нулевым столбцам матрицы $X'$, образуют базис\linebreak
	ядра матрицы $X$. Кроме того, имеется разложение ${\mathbb Z}^4$ в прямую сумму:\linebreak
	$${\mathbb Z}^4 = (\mathbb Z R_1\oplus \mathbb Z R_2) \oplus (\mathbb Z R_3\oplus \mathbb Z R_4) = (\mathbb Z R_1\oplus \mathbb Z R_2) \oplus \text{Ker}X,$$
	а знание матрицы $R^{-1}$ влечет разложение: $e_1 = (6R_1 - 2R_2) \oplus (R_3 - 2R_4),$\linebreak
	$e_2 = (6R_3 - 2R_2) \oplus (-R_4), e_3 = (6R_1 - 3R_2) \oplus (-2R_4), e_4 = (R_1) \oplus (0).$
	
	\noindent
	{\large {\bf 3.5 Общее решение систеы линейных уравнений}}
	
	\noindent
	В разделе 3.3. уже было определено, что значит решить линейную систе-\linebreak
	му с целыми коэффициентами $Ax = b$, где $x \in \mathbb Z^m$ и $b \in \mathbb Z^n$. Существо-\linebreak
	вание и поиск частного решения уже были рассмотрены в разделе 3.3.\linebreak
	В то же время, хорошо известно, что общее решение такой системы -\linebreak
	это сумма частного решения и общего решения однородной системы.\linebreak
	Однако ядро матрицы $A$ (множество однородных решений) было вы-\linebreak
	численно в предыдущем разделе. Итак, линейное уравнение, описанное\linebreak
	выше, полностью решено. Но нужно сделать замечание: ядро матрицы\linebreak
	$A$ может быть вычислено во время поиска частного решения, т.е. одно-\linebreak
	временно с процессом приведения к ступенчатому виду матрицы $A$, к\linebreak
	которой присоединен вектор-столбец $b$.
	
	\noindent
	{\bf (33) Предложение.}
	
	{\it Обозначим через $(A | b)$ матрицу, состоящую из матрицы $A$ и присо-\linebreak
	единенного к ней вектора $b$. Тогда $(A | b)$ будет линейным отображени-\linebreak
	ем из $\mathbb Z^m \times \mathbb Z \ \text{в} \ \mathbb Z^n$, которое $x, x_{m + 1}$ ставит в соответствие $Ax + bx_{m + 1}$.
	
	Пусть $(A' | b')$ - ступенчатая матрица, полученная с помощью ал-\linebreak
	горитма приведения к ступенчатому виду из раздела 3.2, примененного\linebreak
	к матрице $(A | b)$. $R$ - матрица перехода, удовлетворяющая условию\linebreak
	$(A | b)\times R = (A' | b')$ (эта матрица $R$ обратима и порядка $m + 1$).
	
	$(i)$ Система $Ax = b$ имеет решение тогда и только тогда, когда\linebreak
	последняя строка матрицы $R$ равна $(0, 0, \ldots, 0, 0, 1)$ и когда вектор $b'$\linebreak
	нулевой. Эти два условия гарантируют частное решение $-\overline{R}_{m + 1}$, где\linebreak
	$\overline{R}_{m + 1}$ - вектор, образованный $m$ первыми компонентами из $R_{m + 1}$.
	
	$(ii)$ Вэтом случае $A'$ - ступенчатая матрица, эквивалентная слева\linebreak
	матрице $A$. Пусть $k \in [1, m]$ такое, что нулевые столбцы матрицы $A$\linebreak
	имеют индексы $> k$. Тогда векторы $\overline{R}_{k + 1}, \overline{R}_{k + 2}, \ldots, \overline{R}_{m} \in \mathbb Z^m$ состоят из\linebreak
	$m$ первых компонент столбцов $R_{k + 1}, R_{k + 2}, \ldots, R_{m}$ образующих базис\linebreak
	$KerA$.}
	
	\pagebreak
	
	\lhead{342}
	\rhead{III-3 \ \ \ \  Вычисление образа и ядра матрицы}
	
	\noindent
	{\bf Доказательство.}
	
	\noindent
	\begin{tabular}{|p{12.5cm}}
	Пункт $(i)$ следует из теоремы 31. Тогда имеет место матричное\linebreak
	равенство:
	$$(A | b)\times R = (A' | b') = (A' | 0) \ \text{для} \ R = \left( \left.\frac{\overline{R}}{0 \ldots 0} \right| \frac{\overline{R}_{m + 1}}{1} \right),$$
	которое ведет к $A' = A\times \overline{R} \ \text{и} \ A\overline{R}_{m + 1} + b = 0$; это доказывает\linebreak
	пункт $(ii)$.
	\end{tabular}
	
	\noindent
	{\bf Пример}
	
	Проиллюстрируем данное доказательство примером, уже рассмо-\linebreak
	тренным в разделе 3.3. Речь идет о системе уравнений $3x_1 + 2x_2 + 3x_3 +$\linebreak
	$+ 4x_4 = -8\ \text{и} \ x_1 - 2x_2 + x_3 - x_4 = -3$. Приведем ступенчатую форму\linebreak
	$(A' | b')$ матрицы $(A | b)$ и матрицу перехода $R$ от $(A | b)$ к $(A' | b')$:
	$$(A' | b') = \begin{pmatrix} -1 & 0 & 0 & 0 & 0 \\ -3 & 1 & 0 & 0 & 0 \end{pmatrix},\ \ \  R = \begin{pmatrix} -1 & -2 & -1 & -6 & 50 \\ 1 & -3 & 0 & -7 & 55 \\ 0 & 0 & 1 & 0 & 0 \\ 0 & 3 & 0 & 8 & -63 \\ 0 & 0 & 0 & 0 & 1 \end{pmatrix}.$$
	\ \ \ Два основных пункта обеспечивают нулевое значение вектора $\b'$ и\linebreak
	то, что последняя строка матрицы $R$ равна $(0, 0, 0, 0, 1)$. Действительно,\linebreak
	если $x \ -$ вектор $(-50, -55, 0, 63)$, то соотношение $(A | b)\times R_5 = 0$ опи-\linebreak
	сывается в виде: $-Ax + r_{55}b = 0$. Следовательно, $x - $ частное решение\linebreak
	системы $Ax = b$. Кроме того, имеет место равенство $A\times\overline{R} = A'$, ма-\linebreak
	трица $\overline{R}$ будет подматрицей матрицы $R$, состоящей из 4 первых строк\linebreak
	и 4 первых столбцов матрицы $R$. Это матричное равенство имеет тот\linebreak
	смысл, что матрица $A'$ является ступенчатой формой матрицы $A$, и\linebreak
	$\overline{R}$ - правая матрица перехода. Векторы $\overline{R}_3$ и $\overline{R}_4$, соответствующие\linebreak
	нулевым столбцам матрицы $A'$, образуют базис $KerA$. Другими слова-\linebreak
	ми, $\overline{R}_3, \overline{R}_4$ образуют базис пространства однородных решений. Итак,\linebreak
	решения записываются в виде:
	$$x_1 = -50 - u - 6v, \ \ \ x_2 = -55 - 7v,$$
	$$ \ \ \ \ \ \ \ \ \ \ \ \ \ \ \ \ \ \ \ \ \ \ \ x_3 = u, x_4 = 63 + 8u \ \ \text{для} \ \ u, v \in \mathbb Z.$$
	
	\pagebreak
	
	\lhead{III-3.6 \ \ \ \  Ранг модулей}
	\rhead{343}
	
	\noindent
	{\large {\bf 3.6 Ранг модулей}}
	
	\noindent
	Вот несколько результатов, относящихся к рангу модулей, которые,\linebreak
	возможно, покажутся смешными для читателя, хорошо знающего те-\linebreak
	орию модулей над кольцом главных идеалов. Предупреждения, фигу-\linebreak
	рирующие в этих формулировках, оправдывают их присутствие здесь.\linebreak
	Первые результаты будут верны не только в кольцах главных идеалов,\linebreak
	а останутся истинным и в коммутативных кольцах: предположение о\linebreak
	том, что кольцо является кольцом главных идеалов, позволяет, однако,\linebreak
	доказать, что приведение к ступенчатому виду предоставляет возмож-\linebreak
	ность эффективного подхода.
	
	\noindent
	{\bf (34) Предложение.}
	
	{\it Два базиса свободного модуля $L$ конечного типа над {\bf коммутатив-\linebreak
	ным} кольцом $A$ имеют одно и тоже число элементов.}
	
	\noindent
	{\bf Доказательство.}
	
	\begin{tabular}{|p{12.5cm}}
	\noindent
	Обозначим через $p$ кардинальное число одного базиса и через $q \ - $\linebreak
	кардинальное число другого базиса. По определению базиса, три\linebreak
	$A$-модуля $L, A^p$ и $A^q$ изоморфны. В силу следствия 19 раздела 2.3\linebreak
	имеем: $p = q$, что и доказывает предложение.
	\end{tabular}
	
	\noindent
	{\bf (35) Определения.}
	
	{\it Если $(i)$ - свободный модуль конечного типа над {\bf коммутатив-\linebreak
	ным} кольцом $A$, то мощность какого-либо его базиса называется ран-\linebreak
	гом модуля $L$ и обозначается через $rang(L)$. В частности, ранг $A^n$ равен\linebreak
	$n$.
	
	$(ii)$ Если $M - A$-модуль конечного типа, то обозначим, как и в\linebreak
	лемме 20 раздела 2.3, через $\rho(M)$ наименьшее число образующих в $M$.}
	
	\noindent
	{\bf (36) Предложение.}
	
	{\it Пусть $A - $ {\bf коммутативное} кольцо. Семейство образующих в $A^n$,\linebreak
	состоящее из $n$ элементов, является базисом в $A^n$. Другими словами,\linebreak
	любая сюръекция модуля $A^n$ является изоморфизмом.}
	
	\noindent
	{\bf Доказательство.}
	
	\begin{tabular}{|p{13cm}}
	\noindent
	Напомним, что квадратная матрица $X$ порядка $n$ является обра-\linebreak
	тимой тогда и только тогда, когда ее определитель обратим; это\linebreak
	равенство в силу равенства $\det(XY) = \det(X)\det(Y)$. Используя это\linebreak
	равенство для матрицы $\widetilde{X}$, присоединенной к матрице $X$, транс-\linebreak
	понируем матрицу алгебраических дополенений матрицы $X: X\widetilde{X} = $\linebreak
	$\widetilde{X}X = \det(X) \text{Id}_n$.
	\end{tabular}
	
	\pagebreak
	
	\lhead{344}
	\rhead{III-3 \ \ \ \  Вычисление образа и ядра матрицы}
	
	\begin{tabular}{|p{12.5cm}}
	\noindent
	Пусть $X - n\times n$матрица, столбцы которой $X_1, X_2, \ldots, X_n$ явля-\linebreak
	ются $n$ векторами порождающими $A^n$. Рассмотрим сначала част-\linebreak
	ный случай $A = \mathbb Z$. Если $X' - $ступенчатая матрица, эквивалентная\linebreak
	справа матрице $X$, то ее столбцы порождают Im$X' = $Im$X = \mathbb Z^n$.\linebreak
	Как следствие, всякий столбец матрицы $X' -$ ненулевой. Матрица\linebreak
	$X' -$ ступенчатая, ее столбцы назависимы и образуют базис $\mathbb Z^n$;\linebreak
	тогда то же верно и для столбцов матрицы $X$. Эти рассуждения\linebreak
	применимы, очевидно, и к любому кольцу главных идеалов.\linebreak
	Рассмотрим Теперь случай произвольного коммутативного кольца.\linebreak
	Так как $X_i$ составляют систему образующих для $A^n$, то существу-\linebreak
	ет семейство $(Y_i) n$ векторов из $A^n$ такое, что $X(Y_i) = e_i$, где\linebreak
	множество $e_i$ обозначает канонический базис. Если $Y$ обозначает\linebreak
	матрицу, столбцами которой являются векторы $Y_1, Y_2, \ldots, Y_n$, то\linebreak
	$XY = \text{Id}_n$ и $\det(X)\det(Y) = 1$. Следовательно, матрица $X$ с обра-\linebreak
	тимым определителем обратима; ее столбцы образуют базис в $A^n$.
	\end{tabular}
	
	Внимание: этот результат не имеет аналогов для множества линейно\linebreak
	независимых элементов (см. пример и предупреждение 30 раздела 3.2).\linebreak
	В упражнении 6 доказан следующий результат.
	
	\noindent
	{\bf (37) Предложение.}
	
	{\it Если $A$ - коммутативное кольцо, то в $A^n n$ векторов свободны\linebreak
	тогда и только тогда, когда их определитель не является делителем\linebreak
	нуля. Другими словами,эндоморфизм $A^n$ инъективен тогда и только\linebreak
	тогда, когда его определитель не является делителем нуля.}
	
	\noindent
	{\bf (38) Предложение.}
	
	{\it Пусть $A$ - произвольное {\bf коммутативное} кольцо.
		
	$(i)$ Множество $n + 1$ векторов из $A^n$ всегда зависимо.
	
	$(ii)$ Множество $n - 1$ векторов из $A^n$ не порождает $A^n$.
	
	$(iii)$ Если модуль $M$ порожден $n$ векторами, то $n + 1$ векторов всегда\linebreak
	линейно зависимы; и еще, если $M$ содержит $m$ линейно независимых\linebreak
	векторов, то $M$ не может быть порожден $m - 1$ векторами.
	
	$(iv)$ Если $L$ - свободный подмодуль модуля конечного типа $M$, то\linebreak
	rang$(L) \leqslant \rho(M)$.
	
	$(v)$ В частности, если $M$ - свободный подмодуль конечного типа,\linebreak
	то rang$(M) \leqslant \rho(M)$.}
	
	\pagebreak
	
	\lhead{III-3.6 \ \ \ \  Ранг модулей}
	\rhead{345}
	
	\noindent
	{\bf Доказательство.}\newline
	\begin{tabular}{|p{13cm}}
	$(i)$ Классическое доказательство состоит в использовании внеш-\linebreak
	него вычисления. Но можно дать более простое доказательство,\linebreak
	имитирующее внешнее вычисление с помощью подопределителей\linebreak
	(упражнение 6). Приведем здесь доказательство в двух частных\linebreak
	случаях. Предположим, например, что $A$ без делителей нуля и по-\linebreak
	грузим $n + 1$ данных векторов в $K^n$, где $K$ - поле частных для $A$.\linebreak
	Элементарный результат линейной алгебры о векторных простран-\linebreak
	ствах дает нам нетривиальное соотношение с коэффициентами в $K$,\linebreak
	которое умножается на общий знаменатель и дает нетривиальное\linebreak
	отношение с коэффициентами в $A$.\newline
	Более простое доказательство над кольцом главных идеалов состо-\linebreak
	ит в использовании ступенчатости. Обозначим через $X$ матрицу с\linebreak
	$n$ строками и $n + 1$ столбцами, образованную $n + 1$ данными век-\linebreak
	торами, через $X' -$ ступенчатую матрицу, эквивалентную справа\linebreak
	матрице $X$, через $R -$ матрицу перехода порядка $n + 1$, обрати-\linebreak
	мую и такую, что $XR = X'$. Матрица $X'$ обладает строго большим\linebreak
	числом столбцом, чем строк. Но так как она ступенчатая, то ее\linebreak
	последний столбец нулевой. Равенство $XR_{n + 1} = 0$ ($R_{n + 1} -$ послед-\linebreak
	ний столбец матрицы $R$) дает {\it нетривиальное} сооьношение между\linebreak
	столбцами матрицы $X$.
	
	$(iii)$ Первая часть следует из $(i)$. Пусть $u -$ сюръекция из $A^n$\linebreak
	на $M$. Рассмотрим тогда $n + 1$ векторов $\mu_1, \mu_2, \ldots, \mu_{n + 1}$ моду-\linebreak
	ля $M$, порожденного $n$ векторами. Если $X_i \in A^n - $ прообраз $\mu_i$ для\linebreak
	$i = 1, 2, \ldots, n + 1$, то $X_i$ в количестве $n + 1$ находятся в $A^n$. Нетриви-\linebreak
	альное линейное соотношение сежду $X_i$ проектируется с помощью\linebreak
	$u$ в нтривиальное линейное соотношение между $\mu_i$.\linebreak
	
	Вторая часть пункта $(iii)$ - переформулировка первой, которая\linebreak
	влечет, в частности, пункт $(ii)$. Пункты $(iv)$ и $(v)$ являются непо-\linebreak
	средственными следствиями из предыдущих пунктов.\linebreak
	Дополнительное замечание: метод приведения к ступенчатому виду\linebreak
	дает явное доказательство пункта $(ii)$, предъявляя вектор канони-\linebreak
	ческого базиса, не принадлежащего пространству, порожденному\linebreak
	$n - 1$ данными векторами из $A^n$. Действительно, рассмотрим ма-\linebreak
	трицу $n$, образованную этими $n - 1$ векторами, и ступенчатую ма-\linebreak
	трицу $X'$, эквивалентную справа матрице $X$. Матрица $X'$ имеет\linebreak
	$n$ строк, $n - 1$ столбцов и тот же образ, что и матрица $X$. Если\linebreak
	ни один из диагональных элементов матрицы $X'$ не нуль, то $e_n$ не\linebreak
	принадлежит образу $X'$, и, следовательно, $e_n \notin$Im$X$. Напротив,\linebreak
	если $X'$ имеет нулевой диагональный элемент, то выберем из таких
	\end{tabular}
	
	\pagebreak
	
	\lhead{346}
	\rhead{III-3 \ \ \ \  Вычисление образа и ядра матрицы}
	
	\begin{tabular}{|p{12.5cm}}
	\noindent
	элементов элемент с самым маленьким индексом $i$; тогда $e_i$ не будет\linebreak
	принадлежать образу $X'$, и, следовательно, $e_i\notin$Im$X$.
	\end{tabular}
	
	\noindent
	{\bf Примеры}
	
	{\bf1.} Первый пример показывает, что 5 вектороы в $\mathbb Z^4$ обязательно\linebreak
	линейно зависимы. Рассмотрим векторы:
	$$\begin{pmatrix} 1 \\ 0 \\ 1 \\ 2 \end{pmatrix}, \begin{pmatrix} 2 \\ -1 \\ 0 \\ 1 \end{pmatrix}, \begin{pmatrix} 0 \\ 2 \\ 2 \\ 1 \end{pmatrix}, \begin{pmatrix} 1 \\ 0 \\ 1 \\ 1 \end{pmatrix}, \begin{pmatrix} 2 \\ 1 \\ 1 \\ -1 \end{pmatrix},$$
	которые записаны в виде столбцов матрицы, внутренняя часть которой\linebreak
	ограничена единичной матрицей порядка 5. Применим к этой матрице\linebreak
	процесс приведения к ступенчатому виду, получим ступенчатую ма-\linebreak
	трицу $X'$ и матрицу перехода $R$:
	$$X' = \begin{pmatrix} 1 & 0 & 0 & 0 & 0\\ 0 & -1 & 0 & 0 & 0 \\ 1 & -2 & -1 & 0 & 0 \\ 2 & -3 & -3 & -1 & 0 \end{pmatrix} \ \ \text{и} \ \ R = \begin{pmatrix} 1 & -2 & 0 & -1 & 5\\ 0 & 1 & -1 & 0 & -4 \\ 0 & 0 & -1 & 0 & -3 \\ 0 & 0 & 0 & 1 & -1 \\ 0 & 0 & 1 & 0 & 2 \end{pmatrix}.$$
	Последний столбец $R_5$ матрицы $R$ в силу равенства $0 = X'_5 = XR_5$ дает\linebreak
	соотношение между данными 5 векторами $\mathbb Z^4$:
	$$5X_1 - 4X_2 - 3_X3 - X_4 + 2X_5 = 0.$$
	${\ \ \ \ }${\bf 2.} Второй пример показывает, что 4 вектора из $\mathbb Z^5$ не могут поро-\linebreak
	ждать $\mathbb Z^5$. Рассмотрим следующие векторы и результирующую ступен-\linebreak
	чатую матрицу:
	$$\begin{pmatrix} 1 \\ 0 \\ 0 \\ 1 \\ 1 \end{pmatrix}, \begin{pmatrix} 1 \\ 3 \\ 2 \\ 0 \\ 2 \end{pmatrix}, \begin{pmatrix} 2 \\ 3 \\ 2 \\ 2 \\ 3 \end{pmatrix}, \begin{pmatrix} 0 \\ 3 \\ 2 \\ 1 \\ 1 \end{pmatrix} \ \ \text{и}\ \ \begin{pmatrix} 1 & 0 & 0 & 0 \\ 0 & 3 & 0 & 0 \\ 0 & 2 & 0 & 0 \\ 1 & -1 & 1 & 0 \\ 1 & 1 & 0 & 1 \end{pmatrix}.$$
	В этом примере вектор $e_3$ канонического базиса $\mathbb Z^5$ не принадлежит\linebreak
	пространству, порожденному векторами-столбцами конечной ступен-\linebreak
	чатой матрицы. Следовательно, вектор $e_3$ не принадлежит простран-\linebreak
	ству, порожденному 4 данными векторами из $\mathbb Z^5$. Отметим, что про-\linebreak
	странство содержит векторы $e_1, e_4$ и $e_5$.
	
	\pagebreak
	
	\lhead{III-4 \ \ \ \  Приведение матрицы}
	\rhead{347}
	
	
	\noindent
	{\bf (39) Предложение.}
	
	{\it Пусть $M$ — модуль конечного типа над кольцом $A$ главных идеалов.\linebreak
	Если $N$ — подмодуль $M$, то $N$ конечного типа и $\rho(N) \leqslant \rho(M)$.}
	
	\noindent
	{\bf Доказательство.}
	
	\begin{tabular}{|p{12.5cm}}
	\noindent
	Пусть $(\mu_1, \mu_2, \ldots, \mu_m)$ — система образующих $M$ с $m = \rho(M)$.\linebreak
	Существует одно и только одно линейное отображение $u$ из $A^m$ в\linebreak
	$M$, удовлетворяющее условию $u(e_i) = \mu_i$ для $i = 1,2, \ldots ,m$.Оно\linebreak
	сюръективно, поскольку $\mu_i$ порождают $M$.\newline
	Для $N \subset M$ верно равенство $N = u(u^{-1}(N))$. Подмодуль $u^{-1}(N) A^m$\linebreak
	имеет базис из $n$ элементов $n \leqslant m$ (предложение 28 раздела 3.2).\linebreak
	Отсюда следует, что $N$ порождено $n$ элементами, и, как следствие,\linebreak
	$\rho(N) \leqslant n\leqslant m$.
	\end{tabular}
	
	\begin{flushleft}
		\hangindent=1cm \hangafter=0 \noindent
		{\small {\bf Замечания.} \ Неравенство для \ $\rho$ \ не будет \ справедливым для мо-\linebreak
			дулей \ конечного \ типа над произвольным \ коммутативным \ коль-\linebreak
			цом. Достаточно рассмотреть идеалы коммутативного кольца (не\linebreak
			кольца главных идеалов), чтобы в этом убедиться.
			
			${\ \ \ \ \ \ }$Можно также \ убедиться в том, что подмодуль \ модуля конеч-\linebreak
			ного типа не обязан быть конечного \ типа. Для этого \ достаточно\linebreak
			рассмотреть идеалы \ коммутативного не нётерова кольца, напри-\linebreak
			мер, \ кольца \ полиномов \ \ $\mathbb Q[X_1, X_2, \ldots, X_i, \ldots]$ \ \ с \ бесконечным \ чи-\linebreak
			слом неизвестных.}
	\end{flushleft}
	
	\noindent
	{\Large {\bf 4 Приведение матрицы}}
	
	\noindent
	В определении 15 была введена эквивалентность двух последовательных\linebreak
	целых чисел $(a_i)_{1 \leqslant i \leqslant n}$ и $(b_j)_{1 \leqslant j \leqslant n}$ через существование автоморфизма\linebreak
	в $\mathbb Z^n$, переводящего подгруппу $\Pi a_i\mathbb Z$ в подгруппу $\Pi b_j\mathbb Z$. Аналогичная\linebreak
	эквивалентность существует для матриц.
	
	\noindent
	{\bf (40) Определение.}
	
	{\it Две матрицы $А$ и $В$, не обязательно квадратные, но одинаковых раз-\linebreak
	меров, называются эквивалентными, если существуют две обратимые\linebreak
	матрицы $L$ и $R$ такие, что $LAR = В$. Они (матрицы $А$ и $В$) называются\linebreak
	специально эквивалентными, если матрицы $L$ и $R$ имеют определители,\linebreak
	равные 1.}
	
	Прочитав это определение, можно подумать, что две квадратные\linebreak
	диагональные матрицы эквивалентны тогда и только тогда, когда по-\linebreak
	следовательности целых чисел, фигурирующие на диагоналях, эквива-\linebreak
	лентны. Это конечно так, но мы не можем легко доказать необходимое
	
	\pagebreak
	

