\documentclass{mai_book}

\defaultfontfeatures{Mapping=tex-text}
\setmainfont{DejaVuSans}
\setdefaultlanguage{russian}

\clearpage
\setcounter{page}{483}

\begin{document}
%на предыдущей странице должно быть \begin{myproof}

Проведем замену переменных~$y = xz$ и, используя то, что ~$(\frac{xy}{q}) = (\frac xq)^2(\frac zq) \\ = (\frac zq) $, получим:
\begin{center}
$\tau^2 = \underset{z}{\sum} (\frac zq) \underset{x}{\sum} \ \omega^{x(1+z)} $
\end{center}
Так как $\sum_{t \in \mathbb{Z}_q} \omega^t = 1 + \omega + \omega^2 + \dots + \omega^{p-1} = 0 $, то $\sum_{t \in \mathbb{Z}_q} \omega^t = ~-1$. Если $z \not=~-1$
, то отображение $x \rightarrow x(1+z) $
 ~- перестановка из $\mathbb{Z}_q^*$
 и, следовательно, $\sum_{x} \omega^{x(1+z) = ~-1}$, а потому:

\begin{center}
$\tau^2 = ~- \underset{z \not= ~-1}{\sum} (\frac zq) + (\frac{-1}{q})(q - 1) $.
\end{center}


Используя то, что число квадратов в $\mathbb{Z}_q^*$
 равно числу не-квадратов, получаем  $\sum_{z \in \mathbb{Z}_q^*} (\frac zq) = 0$, откуда

\begin{center}
$\tau^2 = (\frac{-1}{q}) + (\frac{-1}{q})(q - 1) = (\frac{-1}{q})q $.
\end{center}

%\end{myproof}

\begin{thm}[доказательство квадратичного закона взаимности]
Сумма Гаусса $\tau \in \Omega_p$ - это квадратный корень из $(\frac{-1}{q})q$. Отсюда следует, что $(\frac{-1}{q})q$
 есть квадрат в $\mathbb{Z}_p$
 тогда и только тогда, когда $\tau$
 принадлежит $\mathbb{Z}_p$. Напомним в связи с этим, что элемент $\theta \in \Omega_p$
 принадлежит $\mathbb{Z}_p$
 тогда и только тогда, когда $\theta^p = \theta$: действительно, многочлен $X^p ~- X$
 имеет в $\Omega_p$
 не более $p$ корней и все элементы $\mathbb{Z}_p$
 являются корнями этого многочлена. Вычисление $\tau^p$
 упрощается, потому что мы работаем в характеристике $p$:

\begin{center}
$\tau^p = \underset{x \in \mathbb{Z}_q^*}{\sum} (\frac xq)\omega^{px} = (\frac pq)^{-1} \underset{y \in \mathbb{Z}_q^*}{\sum} (\frac yq)\omega^y = (\frac pq)\tau$.
\end{center}


Отсюда получаем, что $(\frac{-1}{q})q$
 ~- квадрат в $\mathbb{Z}_p$
 тогда и только тогда, когда $p$ ~- квадрат по модулю $q$. Следовательно, 

\begin{center}
$(\frac pq) = (\frac{(\frac{-1}{q})q}{p}) = (\frac{(-1)^{\frac{q-1}{2}}q}{p}) = (\frac{-1}{p})^{\frac{q-1}{2}}(\frac qp) = (-1)^{\frac{p-1}{2}\frac{q-1}{2}} (\frac qp)$.
\end{center}


что и требовалось доказать.
\end{thm}

\newpage

Существует дополнение к закону взаимности, указывающее, когда число 2 является квадратом по модулю простого нечетного числа.

\begin{thm}[дополнение к квадратичному закону взаимности]
Пусть $p$ - нечетное простое число. 2 является квадратом по модулю $p$ тогда и только тогда, когда $p \equiv \pm 1$ (mod \ 8). Это можно записать в виде:

\begin{center}
$(\frac 2p) = (-1)^{\frac{p^2 -1}{8}}$.
\end{center}
\end{thm}
\begin{myproof}

Легко убедиться в том, что для нечетного $х$ число $x^2 -1$
 делится на 8 и $\frac{x^2-1}{8}$
 четно тогда и только тогда, когда $x \equiv \pm 1$ (mod \ 8). По аналогии с доказательством закона взаимности найдем квадратный корень из 2 в расширении поля $\mathbb{Z}_p$.

В поле комплексных чисел равенство $e^{\frac{2i\pi}{e}} = (\sqrt2 + i\sqrt2)/2$
 позволяет записать $\sqrt2 = z + \bar z$
 , где $z = e^{\frac{2i\pi}{8}}$; кроме того, используя равенство $X^4+1 = (X^2 + \sqrt2 X+1)(X^2 - \sqrt2 X+1)$
 , получаем, что $z$ - примитивный корень из 1 степени 8 (что эквивалентно $z^4=~-1$
    ).
Это отчасти объясняет, почему следует рассмотреть подполе $\Omega_p$
 поля $\mathbb{Z}_p$
 , имеющее корень $\alpha$
 8-й степени из единицы (достаточно взять поле, в котором многочлен $X^4+1$
 имеет корень $\alpha$
 ). Тогда:

\begin{center}
$\alpha^2 = -\alpha^{-2} \Longrightarrow (\alpha + \alpha^{-1})^2 = 2$.
\end{center}


Значит, 2 является квадратом по модулю p тогда и только тогда, когда элемент $\omega = \alpha + \alpha^{-1}$
 принадлежит  $\mathbb{Z}_p$
 , т.е. когда $\omega^p = \omega$. Так как мы работаем в поле характеристики $p$, то $\omega^p = \alpha^p + \alpha^{-p}$
 ~- выражение, в котором показатели можно привести по модулю 8 (так как $\alpha^8=1$
    ). Получаем:

\begin{center}
$\alpha^1 + \alpha^{-1} = \omega, \alpha^3 + \alpha^{-3} = -\omega, \alpha^5 + \alpha^{-5} = -\omega, \alpha^7 + \alpha^{-7} = \omega$,
\end{center}


и следовательно, 2 является квадратом по модулю $p$ тогда и только тогда, когда $p$ сравнимо с 1 или 7 по модулю 8.
\end{myproof}

\begin{beznomera}
{\bf Пример: является ли 323 квадратом по модулю 479?}
\end{beznomera}

Это пример применения закона квадратичной взаимности. Введенный в упражнениях 31, 32, символ Якоби позволяет проводить те же самые вычисления, не разлагая используемые числа в произведение множителей. Число 479 - простое, в то время как $323 = 17\times 19$, а потому

\newpage


$(\frac{323}{479}) = (\frac{17}{479})(\frac{19}{479})$. Первый символ Лержандра вычисляем, используя то, что $479 \equiv 3$ (mod \ 17):

\begin{center}
$(\frac{17}{479}) = (-1)^{\frac{479-1}{2}\frac{17-1}{2}}(\frac{479}{17}) = (\frac{3}{17}) = (-1)^{\frac{479-1}{2}\frac{19-1}{2}}(\frac{479}{19}) = -(\frac{4}{19}) = -(\frac{2}{19})^2 = ~-1$
\end{center}

Аналогично, используя сравнение $479 \equiv 4$ (mod \ 19), получаем:

\begin{center}
$(\frac{19}{479}) = (-1)^{\frac{479-1}{2}\frac{19-1}{2}}(\frac{479}{19}) = -(\frac{4}{19}) = -(\frac{2}{19})^2 = -1$,
\end{center}

откуда $(\frac{323}{479}) = 1$
 и значит, число 323 является квадратом по модулю 479. Вопрос теперь состоит в том, как эффективно вычислить квадратный корень из 323 в $\mathbb{Z}_{479}$?


\subsection{Квадратные корни: метод\newline Цассенхауза - Кантора}


Зафиксируем конечное поле K (например, K = $\mathbb{Z}_p$
 для простого $p$) и поставим себе цель: вычислить квадратный корень из элемента $u \in K^*$. Конечно, необходимо, чтобы $u$
 был квадратом, - условие, которое всегда можно реализовать в поле K характеристики 2 (почему? как в этом случае эффективно получить квадратные корни?). Предположим поэтому, что в дальнейшем K обозначает поле характеристики, отличной от 2. В этом случае элемент $u$ является квадратом тогда и только тогда, когда $u^{\frac{|K|-1}{2}} = 1$. Результат сохраняется также во всякой циклической группе G четного порядка: элемент $u \in G$
 ~- квадрат тогда и только тогда, когда $u^{\frac{|G|}{2}} = 1$
 (записать $u$
 в виде степени порождающего элемента).

{\bf Простой случай:} $|K| \equiv 3 (mod \ 4)$

Соотношение $u^{\frac{|K|-1}{2}} = 1$
 можно записать как $u^{\frac{|K|+1}{2}} = u$
 или как $(u^{\frac{|K|+1}{4}})^2 = u$
 . Таким образом, найден квадратный корень из $u$
, а именно $u^{\frac{|K|+1}{4}}$
, который может быть быстро вычислен при помощи метода экспоненциальной дихотомии.


Мы изучим два метода извлечения корней. Первый фигурирует как упражнение в книге Кнута [99] (раздел 4.6.2) и опирается на метод разложения многочленов в конечных полях, принадлежащий Цассенхаузу и Кантору. Второй, изложенный в следующем разделе, принадлежит Шенксу.

\newpage

Принцип метода Цассенхауза~---Кантора состоит в сопоставлении квадрату $u \in K^*$
 кольца $A = K[X]/(X^2 - u)$. В этом кольце тождество $z^{|K|} = z$
 может быть записано как

\begin{center}
$0 = z \times (z^{\frac{|K|-1}{2}} - 1) \times (z^{\frac{|K|-1}{2}} +1)$,
\end{center}


а потому (прочтите в конце раздела отступление о факторизациях) взятое наугад $z$ приводит к делителю многочлена $X^2 - u$
 (а значит, и к квадратному корню из $u$
 в K). Продолжение этого раздела состоит в основном в перечислении благоприятных случаев и доказательстве того, что все вычисления могут быть реализованы без кольца многочленов.

Легко убедиться, что $A = K\oplus K \bar X$
 и умножение в кольце А задается формулой:

\begin{center}
$(a + b \bar X)(a\prime b\prime \bar X) = aa^{\prime} + bb^{\prime} u + (ab^{\prime} + a^{\prime} b) \bar X$,
\end{center}


и А имеет инволютивный автоморфизм $\sigma$
, который переводит $a + b \bar X$
 в $a - b \bar X$, а также мультипликативную норму $N : A \rightarrow K$, определенную как $N(z) = z \sigma (z)$
 или, по другому, $N(a+b \bar X) = a^2 - b^2u$.

Напомним, что найти квадратный корень из $u$
 в K это все равно, что найти делитель нуля в А. В следующей лемме дано несколько уточнений.

\begin{lemma}
Для $y = a+b \bar X \in A - {0}$
 эквивалентны следующие условия:

a)$N(y) = 0$,

b)$b$ не равно нулю и $a/b$
 ~- квадратный корень из $u$,

c)$y$ делитель нуля в А,

d)$y$ не обратим в А.
\end{lemma}

\begin{myproof}
Импликации $a \Rightarrow b$
 и $c \Rightarrow d$
 очевидны. Для доказательства $b \Rightarrow c$
 рассмотрим $y = a + b \bar X$
 , удовлетворяющий равенству $(a/b)^2 = u$
. Тогда $y\sigma(y) = N(y) = 0$
 и $y$
 ~- делитель нуля в А. Наконец, $d \Rightarrow a$:
 если $N(y) \not= 0$,
 , то $y\sigma(y)/N(y) = 1$, что приводит к обратимости $y$
 в А.
\end{myproof}
Пусть $|K| - 1 = 2^kq$
, где $q$ - нечетное, а $k \geqslant 1$
. Каждому элементу $y \in A - {0}$
 поставим в соответствие последовательность элементов из А:

\begin{center}
$y_0 = y^q, y_{i+1} = y_i^2,$ (а значит, $y_i = y^{2^iq}$),
\end{center}


и назовем элемент $y$
 благоприятным, если $N(y) = 0$
 или существует такой индекс $i\geqslant 0$
 , что $N(y_i-1) = 0$
 и $y_i \not= 1$. В противном случае назовем $y$
 неблагоприятным. Если удастся найти благоприятный элемент,

\newpage

то сразу найдется и квадратный корень из $u$
, так как в этом случае найдется {\bf ненулевой} элемент $z \in A$, удовлетворяющий условию предыдущей леммы.

{\bf Частный случай} $|K| \equiv 3 (mod \ 4)$

Этот случай эквивалентен $k = 1$. Если $y = \bar X$
, то

\begin{center}
$y_0 = y^q = (\bar X)^{\frac{|K|-1}{2}} = u^{\frac{|K|-1}{2}}\bar X$,

а значит, $N(y_0 -1) = 1 - y^2_0 = 1 - u^{\frac{|K|-1}{2}} = 0$.
\end{center}

Следовательно, элемент $\bar X$
 благоприятен ($y_0$
 отличен от 1 в K), а это позволяет вычислить квадратный корень из $u$
 в K и снова легко находим $\sqrt u = u^{\frac{|K|+1}{4}}$.

\begin{predl}

$(i)$ Вероятность того, что элемент $y \in A - {0}$
 неблагоприятный, меньше или равна $\frac 13 + \frac{2}{3\cdot 4^k}$.

$(ii)$ Если $k\geqslant 2$
 , т.е. $|K| \equiv 1 (mod \ 4)$
, то эта вероятность $\leqslant \frac 38$.

$(iii)$ Если $k = 1$
 , то элемент $y = \bar X$
 благоприятный.

Для доказательства построим новое кольцо B, изоморфное $A = K[X]/(X^2 - u)$
, в котором множество неблагоприятных элементов легко пересчитать. Пусть $\omega$
 ~- квадратный корень из $u$
 в K. Тогда $X^2 - u = (X-\omega)(X+\omega)$
 и многочлены $X - \omega$
 и $X + \omega$
 взаимно просты. На основании китайской теоремы об остатках
\end{predl}
\begin{center}
$K[X]/(X^2 - u) \simeq K[X]/(X-\omega) \times K[X]/(X+\omega) \simeq K \times K$.
\end{center}


Более того, справедливо утверждение:

\begin{lemma}

$(i)$
 Отображение $K[X]\rightarrow K\times K$
 , которое каждому многочлену $P(X)$
 ставит в соответствие пару $(P(\omega),P(-\omega))$,
 ~- сюръективный гомоморфизм кольца $K[X]$
 в произведение колец $K\times K$
 . Идеал, порожденный $X^2 - u$, ~- ядро этого гомоморфизма. Переходя к факторкольцу, получаем изоморфизм $\psi : K[X]/(X^2 -u)\rightarrow K\times K$, который элементу $a+b\bar X$
 ставит в соответствие пару $(a+b\omega, a - b\omega)$.

$(ii)$ Норма $K[X]/(X^2 - u)$, перенесенная в $K\times K$
, каждой паре $(\alpha,\beta) \in K\times K$
 ставит в соответствие элемент $\alpha\beta$.
\end{lemma}
\newpage

\begin{lemma}

Пусть $G_{-1}=\theta, G_i = \{x \in K \ | \ x^{2^iq} = 1\}$
 для $i\geqslant 0$
 . Если $D$
 ~- множество неблагоприятных элементов из $A - {0}$
 , то 
\end{lemma}
\begin{center}
$\psi(D) = \underset{i=0}{\overset{\bigcup}{k}}(G_i - G_{i-1})\times (G_i - G_{i-1})$.
\end{center}



\begin{myproof}
Заметим сначала, что $G_{-1}\subset G_0 \subset G_1 \subset \dots\subset G_k$ и $G_k = K^*$. Кроме того, если для $y \in A \psi(y) = (\alpha,\beta)$, то $\psi(y_i)=(\alpha_i,\beta_i)$
, где $\alpha_i = \alpha^{2^iq}, \beta_i = \beta^{2^iq}$.
Если $y$
 ~- неблагоприятный, то $N(y)\not= 0$
 , а потому $\alpha\beta\not= 0$
. Пусть $i\leqslant k$
 ~- наименьший индекс, такой, что $\alpha_i = 1$
 (такой индекс существует, ибо $\alpha\not= 0$
 и, следовательно, $\alpha_k = 1$
 ). Тогда $N(y_i-1) = (\alpha_i-1)(\beta_i - 1)=0$
 , откуда $y_i = 1$
 (в противном случае $y$
 будет благоприятным). Следовательно, $\beta_i = 1, \alpha \in G_i - G_{i-1}, \beta\in G_i$. Аналогично можно доказать, что существует индекс $j\leqslant k$
 такой, что $\beta\in G_j-G_{j-1}$
 и $\alpha\in G_j$
 , а значит, $i=j$
, и $(\alpha,\beta)\in (G_i-G_{i-1})\times(G_i-G_{i-1})$
, и включение $\psi(D)\subset\bigcup(G_i-G_{i-1})^2$
 доказано.
Обратное включение $(G_i-G_{i-1})^2\subset\psi(D)$
 доказывается следующим образом: пусть $(\alpha,\beta)=\psi(y)$
 и $\alpha\in G_i-G_{i-1}, \beta\in G_i-G_{i-1}$
 . Тогда $\alpha\not= 0, \beta\not= 0$
 и $N(y) = \alpha\beta\not= 0$
 . Пусть индекс $j$
 такой, что $N(y_i-1)=0$
 . Тогда $(\alpha_j-1)(\beta_j-1)=0$
, и без ограничения общности можно считать, что $\alpha_j = 1$
. Но $\alpha_j =1$
 приводит к тому, что $j\geqslant i$
 (так как $\alpha\in G_i - G_{i-1}$
    ) и $\beta_j = 1$
 (так как $\beta\in G_i$
    ), откуда $y_i = 1$
. Это доказывает, что $y$
 ~- неблагоприятный элемент. Лемма доказана.
\end{myproof}


{\bf Доказательство предложения 65}


Имеем $|D|=|\psi(D)|=\sum_{i=0}^k(|G_i|-|G_{i-1}|)^2$
 и для $i\geqslant 0$
 порядок $G_i$
 равен $2^iq$
 (так как $K^*$
 ~- циклическая группа порядка $2^kq$
 ); для $i=-1$
 положим $|G_i|=0$
 . Получаем:

\begin{center}
$|D|= q^2 + \underset{i=1}{\overset{k}{\sum}}(2^iq-2^{i-1}q)^2 = q^2(4^{k-1}+4^{k-2}+\dots+4+1+1) = q^2(4^k+2)/3$.
\end{center}


Откуда:

\begin{center}
$\frac{|D|}{|A-{0}|} = \frac{q^2(4^k+2)}{3(|K|^2-1)} \leqslant
\frac{q^2(4^k+2)}{3(|K|-1)^2} = \frac{q^2(4^k+2)/3}{3\cdot
4^kq^2} = \frac 13 + \frac{2}{3\cdot
4^k}$.
\end{center}


\newpage

Сделаем несколько замечаний, касающихся реализации алгоритма 5. Пусть $y\in A - {0}$
 такой, что $N(y)\not= 0$
 и $i$
 ~- наименьший индекс, для которого $N(y_i-1)=0$
 (такой индекс существует, так как $y_k=1$
    ). Если $i\geqslant 1$
 , то $y_i-1=y^2_{i-1}=(y_{i-1}-1)(y_{i-1}+1)$
, откуда $N(y_{i-1}+1)=0$
 и можно проверять $N(y_{i-1}+1)$
 вместо $N(y_i-1)$
 .

%
\begin{lstlisting}[mathescape=true]
data_type loop()
    $\text{Выбрать <<наугад>>}$ $t\in K; y\longleftarrow t +\bar X$;
    if($t*t-u == 0$)  $N(y) = t^2 - u\Rightarrow t = \sqrt u$
        return $t$;
    
    while($z\not\in K$) 
        if($N(z+q)=0$) return $(a+1)b^{-1}$;
        else if($N(z-1)=0$) return $(a-1)b^{-1}$;
        else
        $z\leftarrow z^2$;
        
    $y = t+\bar X$ $\;$ $\text{неблагоприятен, рассмотреть другое}$ $t$

\end{lstlisting}


{\bf Алгоритм 5.} Вычисление квадратных корней в конечном поле


С другой стороны, если $y_j\in K$
 для некоторого $j$
 , то для $l\geqslant j$
 равенство $N(y_l -1)=0$
 приводит к тому, что $y_l-1=0$
 (действительно, $y_l-1\in K$
 влечет $N(y_l-1)=(y_l -1)^2$
 ). Поэтому достаточно найти $y_j\in K$
 (вместо того, чтобы продолжать вычислять $y_{j+1},y_{j+2}\ldots$
    ).

Наконец, для элемента $\lambda\in K^*$
 элемент $y\in A - {0}$
 неблагоприятный тогда и только тогда, когда неблагоприятен элемент $ \lambda y$
 . Следовательно, вместо проверки произвольных элементов $y\in A - {0}$
 достаточно проверить элементы вида $y = t + \bar X (t\in K)$
 .

Вероятность того, что выбранные наугад $m$
 элементов из $A - {0}$
 являются неблагоприятными, меньше, чем $(\frac 38)^m$
 , при условии $k\geqslant 2$
. Для $k = 1$
, т.е. при $|K|\equiv 3 (mod \ 4)$
, используя формулу $\sqrt u = u^{\frac{|K|+1}{4}}$
, сразу получаем квадратный корень из $u$
 (разумеется, если $u$
 ~- это квадрат).

Отметим, что этот алгоритм предполагает, что $u$
 является квадратом в $K^*$
 , т.е. удовлетворяет равенству $u^{\frac{|K|-1}{2}} = 1$
 (конечно, все возведения в степень осуществляются при помощи дихотомии).


{\bf Пример}

Приведем пример, иллюстрирующий два результата работы алгоритма 5: $N(z-1)=0$ и $N(z+1)=0$
. В обоих случаях имеются по 3

\newpage

неудачных попытки (данные абсолютно не выдуманные, хотя, как правило, число попыток меньше). Пусть $p = 401$
. Тогда $p = 25\cdot 2^4 +1$
 и найдем квадратные корни из -1 и 29 (проверить, что эти числа действительн о являются квадратами по модулю $p$). Вычисления $\sqrt{-1}$
 дают:

\begin{center}
$z = 1 +\bar X$, $z^{25}=86+86\bar X$, $z^{2\cdot 25}=356\bar X$,
$z^{4\cdot 25}=381$, неудача,

$z = 2+\bar X$, $z^{25}=400\bar X$, $z^{2\cdot 25}=400$, неудача,

$z = 3+\bar X$, $z^{25}=315+86\bar X$, $z^{2\cdot 25}=45\bar X$, $z^{4\cdot 25}=381$, неудача,

$z=4+\bar X$, $z^{25}=302+396\bar X$, $N(z-1)=0$,
\end{center}


откуда $\sqrt{-1} = 301\cdot 396^{-1} mod \ 401 = 20 $
. Вычисления $\sqrt{29}$
 следующие:

\begin{center}

$z = 1 +1\bar X$, $z^{25}=315+255\bar X$, $z^{2\cdot 25}=250\bar X$,
$z^{4\cdot 25}=381$, неудача,

$z = 2+1\bar X$, $z^{25}=272+182\bar X$, $z^{2\cdot 25}=362\bar X$, $z^{4\cdot 25}=400$, неудача,

$z = 3+1\bar X$, $z^{25}=282+331\bar X$, $z^{2\cdot 25}=272+219\bar X$,\\ $z^{4\cdot 25}=39\bar X$, $z^{8\cdot 25}=400$, неудача,

$z=4+1\bar X$, $z^{25}=305+320\bar X$, $z^{2\cdot 25}=188+314\bar X$,\\ $z^{4\cdot 25}=210+170\bar X$, $N(z+1)=0$,

\end{center}


откуда $\sqrt{29}=211\cdot 170^{-1}$ $mod \ 401=164$
.


{\bf Делители 0 и факторизация.}

Начнем с элементарных вещей: элемент $a$
 кольца главных идеалов $А$ называется разложимым, если кольцо $A/a$
 не является областью целостности, т.е. существует ненулевые элементы $\bar x,\bar y\in A/ a$
 такие, что $\bar x,\bar y = 0$
 . Значение делителей нуля, в частности, позволяет найти нетривиальные факторизации $a$: НОД$(x,a)$
 и НОД$(y,a)$
 являются истинными делителями $a$
 . Действительно, они необратимы и, тем более, не эквивалентны $a$
. Аналогичные рассуждения применяются, если в факторкольце $A/a$
 имеется равенство $\bar x_1\bar x_2\ldots \bar x_r = 0$
 , в котором ни один из $\bar x_i$
 не равен нулю. Внимание: в этом случае можно лишь утверждать, что существует такой $x_i$
 , что НОД$(x_i,a)$
 ~- нетривиальный делитель $a$
 (фактически, имеется по крайней мере два необратимых элемента $\bar x_i$
    , которые дают два нетривиальных делителя $a$
).

Разумеется, все это простая переформулировка определения непростого элемента (а значит, разложимого). Чтобы все это выявить, необходимо использовать машину для нахождения равенств $\alpha\beta\ldots\nu=0$
 в факторкольце $A/a$
 . Эти равенства отражают структуру алгебры $A/a$
 и зависят от природы элемента $a$
 .

Проиллюстрируем эти ужасы при помощи простого примера. Предположим, что мы хотим найти квадратный корень из 7 по

\newpage

модулю простого числа $p$. Разумеется, поиски бесполезны, если 7 не яаляется квадратом по модулю $p$ (вычисление $7^{\frac{p-1}{2}}$ $mod \ p$
 позволяет выяснить это). Найти $\sqrt 7$
 в $\mathbb{F}_p$
 все равно, что разложить на множители многочлен $X^2 - 7$
 в $\mathbb{F}_p[X]$
 . Можно ли, используя машину, породить равенство $\alpha\beta=0$
 в кольце $\mathbb{F}_p[X]/(X^2 - 7)$
 ? Да! И ответ лежит в китайской теореме об остатках: так как элемент 7 имеет квадратный корень, то в $\mathbb{F}_p[X]$
 можно записать $X^2 - 7 = (X-\sqrt 7)(X+\sqrt 7)$
 и, применяя китайскую теорему об остатках к взаимно простых многочленам $X - \sqrt 7$, $X+\sqrt 7$
 , получим:

\begin{center}
$\mathbb{F}_p[X]/(X^2 - 7)\simeq \mathbb{F}_p[X]/(X-\sqrt 7)\times\mathbb{F}_p[X]/(X+\sqrt 7)$.
\end{center}


Оба множителя, находящиеся в правой части равенства, изоморфы $\mathbb{F}_p[X]$
, в котором имеется тождество $z^p=z$
. Следовательно, это тождество выполняется и в $\mathbb{F}_p[X]/(X^2-7)$
, что позволяет, как мы увидим позже, находить делители 0.

Небольшое отступление: небесполезно выделить в явом виде китайский изоморфизм жля этого частного случая, чтобы заменить одну очень простую вещь, - многочлен степени 1 может быть найдет, если известны его значения в точках $\pm\sqrt 7$
.

Но какие делители нуля дают равенство $z^p=z$
? Ответ записан в тождестве Цассенхауза ~- Кантора:

\begin{center}
$z^p-z=z\times(z^{\frac{p-1}{2}}-1)\times(z^{\frac{p-1}{2}}+1)=0$.
\end{center}


Если $(p-1)/2$
 четно, то можно разложить и $z^{\frac{p-1}{2}}-1$
 . Например, если $p\equiv 1$ $(mod \ 8)$
, то

\begin{center}
$z\times(z^{\frac{p-1}{8}}-1)\times(z^{\frac{p-1}{8}}+1)\times(z^{\frac{p-1}{4}}+1)\times(z^{\frac{p-1}{2}}+1)=0$.
\end{center}


Но необходимо избежать таких $z$
, что предыдущее равенство было бы вида $a\times 0\times\dots\times b =0$
. Это происходит, например, при $z=0,1$
 или $z\in\mathbb{F}_p$
 . Существуют ли <<хорошие>> $z$
 и насколько их много? Только что изученный материал говорит о том, что удача нам улыбается.

Привести пример? Пусть $p=337$
, $p-1=16q$ и $q=21$
. Тогда:

\begin{center}
$(z^{21}-1)(z^{21}+1)=z^{42}-1$, $(z^{42}-1)(z^{42}+1)=z^{84}-1$,

$(z^{84}-1)(z^{84}+1)=z^{168}-1$, $(z^{168}-1)(z^{168}+1)=z^{336}-1$.
\end{center}


Как оперировать с $z\in\mathbb{F}_p[X]/(X^2-7)$
? Элементы $z\in\mathbb{F}_p[X]/(X^2-7)$
 представляют собой формы вида $a+b\bar X$
 , которые складывают и перемножают, используя равенство $\bar X^2 = 7$
. Если взять $z=$


\newpage

$1+\bar X$
, то мы потерпим неудачу (придем к равенству $0\times a = 0$
    ). Возьмем $z=2+\bar X$
 и получим:

\begin{center}
$z^{21}=150+11\bar X$, $z^{42}=94+267\bar X$, $z^{84}=320\bar X$, $z^{168}=1$.
\end{center}


Получилось! $z^{84}-1\not= 0$, $z^{84}+1\not= 0$, $a(z^{84}-1)(z^{84}+1)=z^{168}-1=0$
. Из-за простоты примера можно без вычисления НОД заметить, что $z^{84}+1=1+320\bar X$
 делит $X^2-7$
 в $\mathbb{F}_{337}[X]$, откуда $\sqrt 7 = 320^{-1} mod \ 337 = 218$
. Мы нашли квадратный корень.

Существуют и другие примеры применения этого принципа. Например, метод Барлекампа [21] разложения многочлена степени $n$
 с коэффициентами из $\mathbb{F}_p$
 испульзует алгебру $B=\mathbb{F}_p[X]/P$
 размерности $n$
 над $\mathbb{F}_p$
 . Вычисляется матрица эндоморфизма $\tau: z\mapsto z^p$
 в каноническом базисе $\{1,\bar X,\bar X^2,\ldots,\bar X^{n-1}\}$
 алгебры $B$, что не вызывает особых проблем. Заием определяется базис ядра $Ker(\tau-Id_B)$ 
 ~- это классическая задача линейной алгебры. Если элемент $z$
 принадлежит $Ker(\tau-Id_B)$
 и не является константой, то произведение

\begin{center}
$z(z-1)(z-2)\ldots(z-(p-1))=z^p-z=0$,
\end{center}


состоит из ненулевых множителей, а потому позволяет разложить $P(X)$:
если элемент $z$
 из  $\mathbb{F}_p[X]/P$
 представляется в виде многочлена $Q\in\mathbb{F}_p[X]$
 степени $<n$
 , то по крайней мере один из НОД$(Q-i,P)$
 является нетривиальным множителем $P(X)$
 . Читатель, заинтересовавшийся этим методом, может получить более точную информацию в [21],[22] и [99].

Последний пример: записать простое число $p$  в виде суммы двух квадратов. Фактически запись $p=x^2+y^2$
 в  $\mathbb{Z}$
 означает, что $p=(x+iy)(x-iy)$
 в  $\mathbb{Z}[i]$
 , а значит, $p$ разложимо в  $\mathbb{Z}[i]$
. Что дает в этом контексте рассмотрение факторкольца $\mathbb{Z}[i]/p$
? Ничего. Если не рассмотреть  $\mathbb{Z}[i]$
 как факторкольцо  $\mathbb{Z}[X]/(X^2+1)$
 и не поменять местами факторкольца:

\begin{center}
$\mathbb{Z}[i]/P\simeq(\mathbb{Z}[X]/(X^2+1))/p\simeq(\mathbb{Z}[X]/p)/(X^2+1)\simeq$

$\simeq \mathbb{F}_p[X]/(X^2+1)$.
\end{center}


Ситуация изменилась: теперь речь идет о разложении на простые множители многочлена $X^2+1$
 по модулю $p$ или, по-другому, об отыскании квадратных корней из -1 по модулю $p$. А это мы уже умеем делать! Если $x^2\equiv -1$ $(mod \ p)$
 , то $z =$НОД$(x+i,p)$
 является собственным делителем $p$
 в $\mathbb{Z}[i]$
 . Это приводит к тому, что $N(z)=p$
 ~- сумма двух найденных квадратов.

\newpage


\subsection{Квадратные корни: метод Шенкса}


Как обычно, запишем $|K|-1=2^kq$
, где $q$
 ~- нечетно. Из китайской теоремы об остатках, примененной к циклической группе $K^*$
 порядка $2^kq$
 , следует, что эта группа является прямым произведением $G_0\times G_1$
 группы порядка $2^k$
 (единственная 2-подгруппа Силова) и группы порядка $q$. Метод Шенкса использует 2-подгруппу, а не всю группу $K^*$
 . В отличие от метода Цассенхауза ~- Кантора, метод Шенкса использует вычисления в K (а не в его квадратичном расширении).

При помощи коэффициентов Безу мы можем вычислить разложение $u=u_0u_1$
 элемента $u\in K^*$
 . Элемент $u_1$
 всегда квадрат в $G_1$
 (так как порядок $G_1$
 нечетен) и мы всегда можем вычислить квадратный корень из $u_1=(u_1^{\frac{q+1}{2}})^2$
 . Поэтому вычисление квадратного корня из $u$
 сводится к вычислению квадратного корня из $u_0$
 . Это приводит нас к изучению группы, порядок которой - степень 2.

Вместо того, чтобы использовать приведенное выше разложение $K^*$
, Шенкс опирается на равенства:

\begin{center}
$(u^{\frac{q+1}{2}})^2 = u\cdot u^q$ или $u=(u^{\frac{q+1}{2}})^2(u^q)^{-1}$,
\end{center}


которые, с одной стороны, доказывают, что $u$
 является квадратом тогда и только тогда, когда $u^2$
 ~- квадрат (это следует также из того, что отображение $u\mapsto u^q$
 есть автоморфизм поля), а с другой стороны, позволяют вычислить квадратный корень из $u$
 , зная квадратный корень из $u^q$
. Конечно, элемент $u^q$
 принадлежит подгруппе $G=\{z\in K^*, z^{2^k}=1\}$
 порядка $2^k$
 . Отметим, что если $|K|\equiv 3$ $(mod \ 4)$
 (что эквивалентно $k = 1$), то из (3) немедленно следует искомый результат: элемент $u$
 является квадратом тогда и только тогда, когда $u^{\frac{|K|-1}{2}}=u^q=1$
 , а значит, в этом случае $u^{\frac{q+1}{4}}$
 ~- один из квадратных корней из $u$
 (вторым корнем является - $u^{\frac{q+1}{4}}$
    ). Остается описать алгоритм вычисления квадратного корня из элемента, принадлежащего циклической (под)группе, порядок которой - степень 2.

\begin{lemma}
Пусть $x$
 и $y$
 ~- два элемента порядка $2n$
 в циклической группе. Тогда произведение $xy$
 имеет порядок, строго меньший $2n$
 (в действительности, этот порядок делит $n$
    ).
\end{lemma}

\begin{myproof}
Так как $(x^n)^2=1$
 и $x^n\not= 1$
 , то $x^n=e$
, где $e$
 обозначает {\it единственный} элемент порядка 2 в нашей группе. Следовательно, $(xy)^n=e^2=1$
 .
\end{myproof}

\newpage

Пусть порядок циклической группы $G$ - степень 2. Предыдущая лемма позволяет найти квадратный корень из элемента $x\in G$
 при помощи последовательного понижения порядка $x$
 , предполагая, тем не менее, что выполняется следующее свойство: для всякого элемента $x\in G$
, порядок которого строго меньше $|G|$
, существует элемент $y$
того же порядка, что и $x$
, квадратный корень из которого мы знаем. Действительно, рассмотрим произведение $xy$
 ($y$
 имеет тот же порядок, что и $x$
 , и нам известен квадратный корень из $y$
) и из равенства $x=xy\cdot y^{-1}$
 получим $\sqrt x = \sqrt{xy}(\sqrt y)^{-1}$
 . Это дает рекуррентную процедуру, так как $ord(xy)<ord x$
.


{\bf Пример}

Рассмотрим простое число $p=641=5\cdot 2^7+1$
. Группа $G = \{x\in U(\mathbb{Z}_p)\ | \ x^{2^7}=1\}$
 имеет порядок 128. Позже будет показано, как определить с помощью вероятностного метода образующий элемент $G$
 ; сейчас же воспользуемся тем, что образующим $G$
 будет элемент $\omega=21$
 ( из равенства $\omega^{2^7}=1$
 следует, что $\omega$
 порождает группу $G$
 ). Следующее фундаментальное равенство непосредственно проверяется для любого образующего элемента $G$
 :

\begin{center}
$\omega^{2^6}=-1$.
\end{center}


Выберем какой-нибудь элемент из $G$
, скажем $x_0 = 160$
, и попытаемся найти квадратный корень из $x_0^{-1}$
. Начальная проверка $x_0^{2^6} = 1$
 доказывает, с одной стороны, что $x_0$
 принадлежит $G$
 , а с другой стороны, что $x_o$
 ~- квадрат. Действительно, порядок $x_0$
 равен 32, а значит:

\begin{center}
$x_0^{2^4}=-1\Rightarrow (x_0\omega^{2^2})^{2^4} = x_0^{2^4}\times\omega^{2^6}=(-1)\times(-1)=1$.
\end{center}


Положим $x_1=x_0\omega^{2^2}=256$
 и заметим, что $x_1^{2^4}$
 . Нетрудно проверить, что порядок $x_1$
 равен 8, откуда:

\begin{center}
$x_1^{2^2}=-1\Rightarrow (x_1\omega^{2^4})^{2^2} = x_1^{2^2}\times\omega^{2^6}=(-1)\times(-1)=1$.
\end{center}


Полагая $x_2=x_1\omega^{2^4}=487$
, получаем $x_2^{2^2}=1$
. Проверяем, что порядок $x_2$
 равен 4, откуда следует:

\begin{center}
$x_2^2 = -1\Rightarrow (x_2\omega^{2^5})^2 = x_2^2\times\omega^{2^6}=(-1)\times(-1)=1$.
\end{center}


В итоге процесс останавливается на $x_3=x_2\omega^{2^5}$
, так как $x_3=1$
. Перепишем цепочку в обратном порядке: $1=x_3=x_2\omega^{2^5}=x_1\omega^{2^4}\omega^{2^5}=$


\newpage

$x_0\omega^{2^2}\omega^{2^4}\omega^{2^5}=x_0\omega^52$
. Следовательно, $x_0^{-1}=(\omega^26)^2=308^2$
, что и требовалось найти.

Ниже приведена итеративная версия этого метода для группы, в которой известен образующий ее 2-подгруппы. Согласно формуле (3) этот метод позволяет вычислять квадратный корень из $x^{-1}$
 быстрее, чем из $x$
 (конечно, с математической точки зрения эти процедуры эквивалентны, однако, с точки зрения программистской необходимо еще и вычисление обратных). Для прямого вычисления квадратного корня из $x$
 было бы предпочтительнее заменить в лемме 68 $xy$
 на $xy^{-1}\dots$

\begin{predl}
Пусть $G^`$
 ~- циклическая группа порядка $2^kq$
 с нечетным $q$
 и $G$
 ~- подгруппа $G^`$
 порядка $2^k$
 . Предположим, что известен образующий $\omega$
 группы $G$
 . Свяжем с ним последовательность $\omega_i=\omega^{2^{k-1}}$
 для $0\leqslant i\leqslant k$
 .

$(i)$ Элемент $\omega_i$
 является элементом $G^`$
 порядка $2^i$
 и, кроме того, $\omega_i = \omega^2_{i+1}$
 для $i<k$
 .

$(ii)$ Элемент $x\in G$
 является квадратом в $G^`$
 тогда и только тогда, когда его порядок (делящий $2^k$
    ) меньше $2^k$
 . В этом случае определим последовательность целых чисел $h_i$
 и элементов $x_i$
 группы $G$
 :
\end{predl}
\begin{center}
$
    \begin{cases}
    \text{$x_1=x$,} \\
    \text{$2^{h_1}=ordx_1$,}
    \end{cases}
    \begin{cases}
    \text{$x_2=x_1\omega_{h_1}$,} \\
    \text{$2^{h_2}=ordx_2$,}
    \end{cases}
    \dots, \ \ \ \
    \begin{cases}
    \text{$x_{m+1}=x_m\omega_{h_m}$,} \\
    \text{$2^{h_{m+1}}=ordx_{m+1}$.}
    \end{cases}
$
\end{center}



Последовательность чисел $h_i$
 строго убывает до 0 (т.е. $h_{m+1}=0$
    , что эквивалентно $x_{m+1}=1$
) и $\omega_{1+h_1}\omega_{1+h_2}\ldots\omega_{1+h_m}$
 ~- квадратный корень из $x^{-1}$
 .


\begin{myproof}
Заметим, что $G$
 ~- единственная подгруппа порядка $2^k$
 в $G^`$
 ($G^`$
 циклическая) и элемент, порядок которого есть степень 2, принадлежит подгруппе $G$
 . В частности, $x\in G$
 является квадратом в $G^`$
 тогда и только тогда, когда он является квадратом в $G$
 .
Согласно предыдущей лемме последовательность $h_i$
 - строго убывающая, а элементарные вычисления показывают, что

\begin{center}
$x_{j+1}= x_1\omega_{h_1}\omega_{h_2}\ldots\omega_{h_j} = x_1(\omega_{1+h_1}\omega_{1+h_2}\ldots\omega_{1+h_j})^2$;
\end{center}


откуда, используя $x_{m+1}=1$
 и $x_1=x$
 получаем искомый результат.
\end{myproof}

Для вычисления произведения $\omega_{1+h_1}\omega_{1+h_2}\ldots\omega_{1+h_m}$
 при программировании введем последовательность $z_j$
 , стоящую из последовательных произведений, т.е. определенную как $z_j = \omega_{1+h_1}\omega_{1+h_2}\ldots\omega_{1+h_j}$
, и

\newpage

удовлетворяющую рекуррентным соотношениям:

\begin{center}
$z_1 = \omega_{1+h_1}$, $z_2=z_1\omega_{1+h_2}$, \ldots, $z_j=z_{j-1}\omega_{1+h_j}$,\ldots
\end{center}


Если $m$
 такой индекс, что $h_{m+1}=0$(т.е. $x_{m+1}=1$)
 , то $x^{-1}=z_m^2$
.

Для рассматриваемой нами проблемы было бы предпочтительнее напрямую вычислять квадратный корень из $u$
 (а не из $u^q$
    ), изменив для этого значения $z_1$
 (см. формулу (3)). Впрочем, можно немного сэкономить при вычислении $\omega_{h_i}$
 и $\omega_{1+h_i}$
 . Это сделано в следующем следствии.

\begin{sled}
Обозначения заимствованы из предыдущего предложения.

$(i)$ Для $z_o\in G^`$
 последовательность $z_i$
 , построенная по правилу $z_i=z_{i-1}\omega_{1+h_i}$
, удовлетворяет равенству $z_m^2=z_0^2x^{-1}$
.

$(ii)$ $\omega_{h_i}=\omega_{h_{i-1}}^{2^{h_{i-1}-1h_i}}$, $\omega_{1+h_i}=\omega_{h_{i-1}}^{2^{h_{i-1}-h_i-1}}$ 
\end{sled}

%
\begin{lstlisting}[mathescape=true]
$(x,z,h,\alpha)\leftarrow (u^q, u^{\frac{q+1}{2}},k,\omega)$;
    
    $\mathrm{Вычислить}\;$ наименьшее$\;$ $j$ $\;$такое,$\;$ что $\;$ $x^{2^j}=1$;
    if($j=k$)
        printf("$u$ -$\;$ не$\;$ квадрат");
        return $0$;
    
    while($j!=0$)
        $ux=z^2$, $ord\alpha = 2^h$, $ord x=2^j$, $j < h$

        $(x,z,h,\alpha)\leftarrow (x\alpha^{2^{h-j}}, z\alpha^{2^{h-j-1}},j,\alpha^{2^{h-j}})$;
    $\mathrm{Вычислить}\;$ минимальное $\;$ $j$ $\;$такое,$\;$ что$/;$ $x^{2^j} = 1$;
   
    return $z$; $x=1$,$\;$и $z$ $\;$-$\;$ квадратный$\;$ корень$\;$ из $\;$ $u$
\end{lstlisting}


\begin{center}
{\bf Алгоритм 6.} Квадратные корни: метод Шенкса
\end{center}

Алгоритм 6 с одной стороны позволяет проверить, является ли элемент $u$
 квадратом, а с другой стороны, в случае положительного ответа, вычисляет квадратный корень из $u$
 в $K^*$
 .
При этом используется предыдущее следствие с $z_0=u^{\frac{q+1}{2}}$
. Кроме того, предполагается известным образующий $\omega$
 группы $G = \{v\in K^* \ | \ v^{2^k} = 1\}$
 (ниже описано, как можно определить вероятностным методом).

Во время реализации можно осуществить еще несколько небольших улучшений. Для нахождения пары $(x,y)$
 вычисляем $x\leftarrow u^{\frac{q-1}{2}}$
 , затем $z=u^{\frac{q-1}{2}}\cdot u = u^{\frac{q+1}{2}}$
 и заканчиваем вычисление так: $x\leftarrow z\cdot x=u^{\frac{q+1}{2}}u^{\frac{q-1}{2}}=u^q$
 .

Чтобы реализовать

\begin{center}
$(x,z,h,\alpha)\leftarrow (x\alpha^{2^{h-j}},z\alpha^{2^{h-j-1}},j,\alpha^{2^{h-j}})$,
\end{center}


\newpage

можно выполнить следующие действия:

\begin{center}
$\alpha\leftarrow\alpha^{2^{h-j-1}}$; $z\leftarrow z\alpha$; $\alpha\leftarrow\alpha^2$; $x\leftarrow x\alpha$; $h\leftarrow j$.
\end{center}


Можно показать, что основной цикл требует порядка $k^2/4$
 умножений в $K$
 . Чтобы найти образующий группы $G=\{v\in K^* \ | \ v^{2^k}=1\}$
, выбираем наугад $t\in K^*$
 до тех пор, пока не выполнится равенство $t^{\frac{|K|-1}{2}}$
 : некоторые $t$
 из $K^*$
 подойдут (не-квадраты в $K^*$
    ). Затем положим $\omega = t^q$
 , это порождающий элемент $G$
. Очевидно, что поиск порождающего элемента 2-силовской подгруппы $G$
 из $K^*$
 эквивалентен поиску не-квадрата в $K^*$
 . Действительно, если $\omega$
 ~- порождающий элемент 2-силовской подгруппы, то $\omega$
 ~- не-квадрат, а если $t$
 ~- не-квадрат, то $t^q$
 порождает 2-силовскую подгруппу.

\section{Факторизация и простота}

Изучим теперь два критерия простоты и метод факторизации. Рассматриваемые нами алгоритмы вычисления модулярных квадратных 
корней выроятностны в том смысле, что их сложность статистически 
ограниченная, однако, когда они выдают ответ, то он корректен.  
Алгоритм факторизации Полларда, который мы изучим в конце раздела, 
является вероятностным в различных смыслах: он может подвести. 
Аналогично этим алгоритмам, тест на простоту Рабина ~—  
Миллера также является вероятностным, впрочем, как и множество других 
алгоритмов теории чисел. Примененный к некоторому числу этот  
алгоритм очень быстро дает один из следующих ответов: число составное 
и этот ответ однозначен, или число, вероятно, простое, т.е. имеются 
лишь достаточно сильные предположения простоты. Это то, что  
называется вероятностным тестом на непростоту. Однако существуют 
числа, для которых имеются детерминированные критерии простоты. 
Это, например, числа Ферма, с которыми мы уже встречались (тест 
Пепина представлен в упражнении 23), и числа Мерсенна, к изучению 
которых мы и переходим. 

\subsection{Числа Мерсенна} 

Огромное преимущество детерминированных критериев простоты  
заключается в том, что они дают доказательство простоты. Без этих 
критериев исследование простоты числа довольно сложная операция, 
которая иногда состоит в доказательстве того, что группа обратимых 

\newpage

элементов циклическая. Числа Мерсенна (которыми являются и самые 
большие известные сейчас простые числа) и Ферма (относительно 5 из 
которых известно, что они простые) позволяют освободиться от этой 
сложной работы. 
В 1644 г. Марен Мерсенн предположил, что единственные простые 
числа вида $2^q-1$
, соответствующие значениям параметра $q$
,  
заключенным в интервале от 2 до 257, это в точности числа, соответствующие 
следующим значениям $q$
:

\begin{center}
$2,3,5,7,13,17,19,31,67,127,257$
\end{center}

Прошло два века, прежде чем появилась идея доказательства  
предположения Мерсенна. В 1772 г. Эйлер доказал, что $x^{31} - 1$
 ~— простое; в 
1870-х годах Лукас доказывает, что $2^{127}-1$
 ~— простое, а $2^{67}-1$
 - 
нет, что приводит к предположению об опечатке $61\leftrightarrow 67$
 в сообщении 
Мерсенна. С 1911 по 1914 г. Пауэрc доказывает, что $2^{89}-1$
 и $2^{107}-1$
 
простые, а в 1922 г. Крайчик показывает, что $2^{257}-1$
 непростое, что 
и заканчивает экспертизу предположения Мерсенна. 

{\bf Числами Мерсенна} называются простые числа $M_q$
 вида $2^q-1$
 , 
где $q$
 ~— простое (если $q$
 непростое, то $M_q$
 легко факторизуется). Из 
того, что $q$
 ~— простое, еще не следует, что $M_q$
 также простое, как это 
доказывают исправления в предположении Мерсенна. Так, например, 
11 и 23 ~— простые числа, а $M_{11}=2047$
 и $M_{23}=8\ 388\ 607$
 простыми не 
являются $(2047=23\times 89,\ 8\ 388\ 607=47\times 178\ 481)$
. 
В таблице 5, возможно и неполной, указаны известные к 1990 году 
числа Мерсенна; причем явно в этой таблице выписаны лишь те числа 
Мерсенна, число цифр в которых не слишком велико. Для оставшихся 
мы указали лишь число десятичных знаков. Забавно, но например,  
запись числа $M_{132049}$
 в эту книгу потребовала бы порядка 20 страниц, 
и если бы какой-нибудь оратор захотел продекламировать это число 
на светском вечере, то ему понадобилось бы примерно 10 часов (без 
остановок и перерывов). Вот два небольших числа Мерсенна:

\begin{center}
$2^{61}-1 = 2\ 305\ 843\ 009\ 213\ 693\ 951$,

$2^{107}-1= 162\ 259\ 276\ 829\ 213\ 363\ 391\ 578\ 010\ 288\ 127$.
\end{center}


Конечно ли множество.чисел Мерсенна? А множество непростых чисел 
Мерсенна?

\newpage

\begin{tabular}[t]{|r|r|l|}

\hline
q & $M_q=2^q-1$ & Открыватель, дата, ЭВМ \\
\hline
2 & 3 & \\
3 & 7 & \\
5 & 31 & \\
7 & 127 & \\
13 & 8\ 191 \\
17 & 131\ 071 & Cataldi, 1588 \\
19 & 524\ 287 & Cataldi, 1588 \\
31 & 2\ 147\ 483\ 647 & Euler, 1772 \\
61 & 19 цифр & Perchouvine, 1883 \\
89 & 27 цифр & Powers, 1911 \\
107 & 33 цифр & Powers et Fauquemberge, 1914 \\
127 & 39 цифр & Lucas et Fauquemberge, 1876-1914 \\
521 & 157 цифр & Robinson, 1952, sur SWAC \\
607 & 183 цифр & Robinson, 1952, sur SWAC \\
1\ 279 & 386 цифр & Robinson, 1952, sur SWAC \\
2\ 203 & 664 цифр & Robinson, 1952, sur SWAC \\
2\ 281 & 687 цифр & Robinson, 1952, sur SWAC \\
3\ 217 & 969 цифр & Riesel, 1957, sur BESK \\ 
4\ 253 & 1\ 281 цифр & Hurwitz, 1961, sur IBM-7090 \\
4\ 423 & 1\ 332 цифр & Hurwitz, 1961, sur IBM-7090 \\
9\ 689 & 2\ 917 цифр & Gillies, 1963, sur ILLIAC-II \\
9\ 941 & 2\ 993 цифр & Gillies, 1963, sur ILLIAC-II \\
11\ 213 & 3\ 376 цифр & Gillies, 1963, sur ILLIAC-II \\
19\ 937 & 6\ 002 цифр & Tuckerman, 1971, sur IBM-360/91 \\
21\ 701 & 6\ 533 цифр & Nickel et Noll, 1978,sur CDC-CYBER-174 \\
23\ 209 & 6\ 987 цифр & Noll, 1979,sur CDC-CYBER-174 \\
44\ 497 & 13\ 395 цифр & Nelson et Slowinski, 1979,sur CRAY-I \\
86\ 243 & 25\ 962 цифр & Slowinski, 1982 \\
110\ 503 & 33\ 264 цифр & Colquitt et Welsh, 1990, sur NEC SX-2 \\
132\ 049 & 39\ 751 цифр & Slowinski, 1983 \\
216\ 091 & 65\ 050 цифр & Slowinski, 1985 \\
\hline
\end{tabular}

\begin{center}
{\bf Таблица 5.} Известные числа Мерсенна 
\end{center}

Мы упоминали, что доказательство простоты числа $р$ часто  
связано с факторизацией, полной или частичной, числа $p-1$
. Лукас  
показал, что в некоторых случаях можно заменить факторизацию $p-1$
 на 
факторизацию $p+1$
; Лемер обобщил эту идею и не так давно показал, 
что существуют и другие способы доказательства простоты [67]. Числа

\newpage

Мерсенна входят в категорию чисел, для которых возможно  
разложение на множители числа $p+1$
. 
Тест на простоту чисел Мерсенна, называемый {\bf тестом Лукаса — 
Лемера}, основывается на изучении порядка элемента $2+\sqrt 3$
 по модулю 
$M_q$
, т.е. в группе обратимых факторкольца $\mathbb{Z}[\sqrt 3]/(M_q)$
. 

\begin{thm}[Лукас — Лемер]

Число Мерсенна $M_q = 2^q -1$
 с нечетным показателем $q$
 большим 
или равным 3
 простое тогда и только тогда, когда $(2+\sqrt 3)^{2^{q-1}}\equiv -1$ $(mod \ M_q)$
 . 
\end{thm}

Вопреки формулировке предыдущей теоремы в тесте Лукаса —  
Лемера используются вычисления только с целыми числами, и позже мы 
дадим более {\sl эффективную} формулировку этой теоремы, в которой уже 
не будет фигурировать $\sqrt 3$
. В любом случае мы видим, что  
доказательство простоты числа Мерсенна эквивалентно доказательству  
некоторого сравнения, что легко реализовать (благодаря алгоритму  
экспоненциальной дихотомии), и имеет сложность порядка $2q$
 ( $q-1$ возведений 
в квадрат и $q-1$
 делений на $M_q$
 ). Докажем теорему Лукаса — Лемера, 
рассматривая по отдельности каждую из импликаций. 

{\bf Из сравнения следует простота} 

При доказательстве этой импликации явно ипользуется арифметика 
кольца $\mathbb{Z}[\sqrt 3] = \mathbb{Z}\oplus\mathbb{Z}\sqrt 3$
. Напомним, что данное кольцо имеет инволютивный автоморфизм, который переводит $\sqrt 3$ в $-\sqrt 3$
 а также  
снабжено мультипликативной нормой $N : \mathbb{Z}[\sqrt 3]\rightarrow\mathbb{Z}$
, определенной как $N(x+y\sqrt 3)=x^2 - 3y^2$
. 

\begin{lemma}

$(i)$ Кольцо $\mathbb{Z}[\sqrt 3]$
 евклидово по отношению к отображению $x+y\sqrt 3\mapsto |x^2 - 3y^2|$
 
(абсолютное значение нормы). В частности, это кольцо — 
кольцо главных идеалов. 

$(ii)$ Элемент $\alpha$
 из $\mathbb{Z}[\sqrt 3]$
 обратим тогда и только тогда, когда его 
норма равна $\pm 1$
. 

$(iii)$ Элемент $2+\sqrt 3$
 обратим в $\mathbb{Z}[\sqrt 3]$
 его обратный $2-\sqrt 3$
 . 
\end{lemma}
\begin{myproof}
 Для определения евклидова деления элемента $\alpha$
 на $\beta$
 введем элемент 
$\alpha/\beta$
 из $\mathbb{Q}[\sqrt 3]$
 , который запишем в виде $r +s\sqrt 3$ с $r$ и $s$ из $\mathbb{Q}$ 
. Рассмотрим такие целые числа $x$
 и $y$
 , что $|r-x|\leqslant 1/2$ и $|s-y|\leqslant 1/2$
.

\newpage

Тогда 

\begin{center}
$-3/4 \leqslant (r-x)^2 - 3(s-y)^2 \leqslant 1/4\Rightarrow |N(\alpha/\beta - (x+y\sqrt 3))| \leqslant$

$\leqslant 3/4\Rightarrow |N(\alpha - \beta(x+y\sqrt 3))| < |N(\beta)|$.
\end{center}


Положив $\gamma = x +y\sqrt 3$ и $\rho = \alpha - \beta\gamma$
 получим, что $\alpha = \beta\gamma + \rho$
 и $|N(\rho)| < |N(\beta)|$.
\end{myproof}

\begin{sled}
Пусть число Мерсенна $M_q = 2^q - 1$
 с нечетным показателем $q$
 , 
большим или равным 3, удовлетворяет сравнению $(2+\sqrt 3)^{2^{q-1}}\equiv -1$ $mod \ M_q$
 . Тогда $M_q$
 простое. 
\end{sled}
\begin{myproof}

 Индукцией по нечетному числу $q$
 , большему или равному 3, легко 
показать, что $M_q\equiv 7$ $(mod \ 12)$
. Заметим, что всякий простой  
делитель $p$
 числа $M_q$
 отличен от 2 и 3, а значит, сравним с $\pm 1$
 или $\pm 5$
 по модулю 12 (иначе с $\pm 1$
 было бы сравнимо и их  
произведение). Отсюда следует, что $M_q$
 имеет по крайней мере один простой 
делитель $p$ с $p\not\equiv \pm 1$ $(mod \ 12)$
. Теперь покажем, что этот простой 
делитель $p$
 равен $M_q$
 , что доказывает простоту $M_q$
. 
В $\mathbb{Z}$
 существуют простые числа, которые уже не являются  
простыми в $\mathbb{Z}[\sqrt 3]$
 (например, 13). Но это не относится к $p$
 . Действительно, 
если $p$
 разложимо в $\mathbb{Z}[\sqrt 3]$
 , то 

\begin{center}
$p = \alpha\beta \Rightarrow p^2 = N(\alpha)N(\beta) \Rightarrow p=\pm N(\alpha) \Rightarrow p=\pm(x^2 - 3y^2)$,
\end{center}


и, учитывая то, что $x^2-3y^2$
 ~— простое число, отличное от $\pm 2$ и $\pm 3$
 , 
легко проверить, что $x^2 - 3y^2 \equiv 1 \ (mod \ 3)$ и $x^2 -3y^2\equiv 1 \ (mod \ 4)$
, 
откуда $p\equiv\pm 1 \ (mod \ 12)$
 ~— противоречие с выбором $p$
 . Значит, $p$

остается простым и в $\mathbb{Z}[\sqrt 3]$
. 

Сравнения 

\begin{center}
$(2+\sqrt 3)^{2^{q-1}}\equiv -1 \ (mod \ p) \ \Rightarrow (2+\sqrt 3)^{2^q}\equiv 1 \ (mod \ p)$
\end{center}

показывают, что $2+\sqrt 3$
 имеет порядок $2^q$ в $U(\mathbb{Z}[\sqrt 3]/(p))$
 а  
поэтому, кольцо $\mathbb{Z}[\sqrt 3]/(p)$
 ~— поле (поскольку $\mathbb{Z}[\sqrt 3]$
 ~— кольцо главных 
идеалов), состоящее из $p^2$
 элементов. Следовательно, $2^q$
 делит $p^2-1$
 
~— порядок группы обратимых элементов этого поля. Так как $p$
 делитель $M_q = 2^q -1$
 , то получаем сравнения 
\end{myproof}

\end{document}