\documentclass{mai_book}

\defaultfontfeatures{Mapping=tex-text}
\setmainfont{DejaVuSans}
\setdefaultlanguage{russian}

%\clearpage
%\setcounter{page}{7} % ВОТ ТУТ ЗАДАТЬ СТРАНИЦУ
\setcounter{thesection}{5} % ТАК ЗАДАВАТЬ ГЛАВЫ, ПАРАГРАФЫ И ПРОЧЕЕ.
% Эти счетчики достаточно задать один раз, обновляются дальше сами


% \newtop{ЗАГОЛОВОК}  юзать чтобы вручную поменть заголовок вверху страници

\begin{document}
\newtop{test}
\noindent\textbf{1. Десятичные цифры простых чисел}\\
\\
\hspace*{15pt}Пусть $l$ -- длина $B$, т.е. $10^{l-1}\leqslant B <10^l$\: и \:$k\geqslant 0$. Тогда числа вида\linebreak
$n\cdot10^{k+1} + m\cdot10^k + c$ являются числами десятичная запись которых\linebreak
содержит последовательность $В$, если $ 0 \leqslant c < 10^k.$ Если $k > 0$, то\linebreak
можно взять $с$ взаимно простым с 10 и тогда числа $10^{k+1}$ и $B\cdot10^k + c$\linebreak
будут взаимно просты. Поэтому можно применить теорему Дирихле. В\linebreak действительности оказывается, что существует бесконечно много про­-\linebreak
стых чисел, содержащих фиксированную последовательность цифр $B$ в\linebreak
заранее заданных позициях.\\
\\
\noindent\textbf{2. Вычисление НОД}\\
\\
Допусти, что $m>n$. Тогда $a^m - b^m = (a^n-b^n)a^{m-n}+($
Это тождество и условия взаимной простоты $a$ и $b$ дает


\end{document} 