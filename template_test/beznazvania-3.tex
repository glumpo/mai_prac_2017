\documentclass{mai_book}

\defaultfontfeatures{Mapping=tex-text}
\setmainfont{DejaVuSans}
\setdefaultlanguage{russian}

\clearpage
\setcounter{page}{127} % ВОТ ТУТ ЗАДАТЬ СТРАНИЦУ
\setcounter{section}{6} % ТАК ЗАДАВАТЬ ГЛАВЫ, ПАРАГРАФЫ И ПРОЧЕЕ.
% Эти счетчики достаточно задать один раз, обновляются дальше сами

% \newtop{ЗАГОЛОВОК}  юзать чтобы вручную поменть заголовок вверху страници

\begin{document}

\subsection{Правильный пятиугольник для всех (без Ферма и Гаусса)}
\hspace*{15pt}Зная, что
\begin{center}
$-cos\dfrac{\pi}{5}=cos\dfrac{4\pi}{5}=2cos^2\dfrac{2\pi}{5}-1=8cos^4\dfrac{\pi}{5}-8cos^2\dfrac{\pi}{5}+1$,
\end{center}
достаточно разложить на множители многочлен $8X^4-8X^2+X+1$
(очевидными корнями которого являются —1 и 1/2). Единственное допустимое решение для $cos\dfrac{\pi}{5}$ - это $\dfrac{1+\sqrt{5}}{4}$. Тогда построение проводится
следующим способом (см. рис. 2):\newline
Картинка
с центром $O$ точку $F$ такую, что $AF=\phi$, и получаем прямоугольный
треугольник, в котором угол $(\widehat{FAB})$ равен $\pi/5$.
Соответствующий центральный угол$(\widehat{FOB})$ равен $2\pi/5$ и позволяет
построить правильный пятиугольник (Слейе-Мишо[48]).
Существуют и другие решения, и вот одно из них особенно про
стое: если $z$ есть корень 5-й степени из единицы, соотношение $z^4+z^3+z^2+z+1=0$ приводит к тригонометрическому уравнению
$2cos\dfrac{2\pi}{5}+2cos\dfrac{4\pi}{5}+1=0$, которое просто решается, давая $cos\dfrac{2\pi}{5}=\dfrac{\sqrt{5}-1}{4}$.
Сразу осуществимо построение. Если $U$ — середина $OB$, строим точ-
ку $W$ , расположенную на радиусе $OA$ и удовлетворяющую равенству
$UW=UV=\sqrt{5}/2$. Точка $X$ , середина $OW$, есть ортогональная проек-
ция точки $Y$ такой, что угол $(\widehat{AOY})$ равен $2\pi/5$.
\subsection{<<Двоичное>> деление нацело}
\hspace*{15pt}\textbf{a.} Вот евклидово деление $a$ на $2b$: $a=2b\times q+r$, с $0\leq r<2b$. Если
$r<b$, то евклидово деление $a$ на $b$ получается сразу: $a=b\times 2q+r$. Зато,
если $b\leq r<2b$, деление таково: $a=b\times(2q+1)+(r-b)$.\newpage
\textbf{b.} Предельный случай евклидова деления появляется, когда дели-\newline
тель больше делимого. Вот рекурсивный алгоритм вычисления частно-
го и остатка, представленный в форме функции \textit{Divide}:\newline
Прога\newline
\hspace*{15pt}\textbf{d.} Недостаток рекурсивного алгоритма следующий: в ходе счета
второй параметр функции \textit{Divide}, который удваивается при каждом
вызове, может превзойти первоначальные величины переменных $a$ и $b$.
Это означает, что хотя данные и результат деления поддаются коди-
рованию, может случиться, что в какой-либо частной реализации вели-
чины приведут к переполнению.\newline
\hspace*{15pt}Например, когда применяют рекурсивный алгоритм к целым чи-
слам 31001 и 15, наблюдается ряд рекурсивных вызовов, последний из\
которых — \textit{Divide}(31001,61440); если целый тип, который используют,
записан и закодирован 16 битами, возникает переполнение, тогда как
можно очень хорошо вычислить результат (2066,11), если остановить
удвоение Ь перед последней итерацией.\newline
Таблица\newline
\begin{center}
\textbf{Алгоритм 8.} Бинарные деления
\end{center}
\newpage

\hspace{15pt}Теперь уже можно — это пригодится в дальнейшем — написать\newline
итерационный алгоритм (8-А), который есть не что иное, как разви-\newline
тие предыдущего рекурсивного алгоритма ($q$ и $r$ образуют результат
алгоритма: частное и остаток от деления). Этот алгоритм яснее вы-
являет поставленную задачу.\newline
\hspace*{15pt}\textbf{e.} Чтобы показать эквивалентность двух алгоритмов, можно на-
чать с доказательства очень простого свойства: $[a/2]<b\leftrightarrow a<2b$.\newline
\hspace*{15pt}На выходе из цикла первого алгоритма имеем свойство
$b/2\leq a<b=b_02^k$ - не забудем, что тело цикла использу-
ется, по крайней мере, один раз.\newline
\hspace*{15pt}Что касается второго алгоритма, если цикл использовался, по край-
ней мере, один раз, имеем свойство $b/2\leq a/2<b=b_02^k$ для $k>1$ и $b$ - четного. После выполнения последних присваиваний обнаруживаем
то же свойство, что и для первого алгоритма. Если цикл не выполнен,
это означает, что $a/2<b_0\leq a$, и после исполнения последних команд
получаем $b_0=b/2\leq a<b=2b_0$.\newline
\hspace*{15pt}Последний алгоритм (8-В) получен, исходя из первой итерационной
версии (8-А) извлечением из первого цикла итерации и ее слиянием с
итерацией, взятой из второго цикла (для этого алгоритма, как и для
предыдущего, результат - частное и остаток - представлен послед-
ними значениями переменных $q$ и $r$).\newline
\subsection{Построение прямых линий методами DDA}
\hspace*{15pt}\textbf{a.} Если $u$ и $v$ не взаимно простые числа, отрезок $[0, (u,v)]$ - не что
иное, как повторение $d=$НОД$(u,v)$ сегментов, идентичных сегменту
$[0,(u/d,v/d)]$.\newline
Картинка
\newpage

\hspace*{15pt}\textbf{b.} Выбрать для $y$ наилучшую целую аппроксимацию величины $vx/u$
- это значит убедиться, что $|vx/u|-y\leq 1/2$, или еще, что
$-u\leq2(vx-uy)<u$. Эта формула является в действительности ин-
вариантом алгоритма. Поскольку нужная прямая имеет тангенс угла
наклона меньше 1, есть только один зажигающийся пиксел на данной
вертикали (т.е. $х$ возрастает при каждой итерации). Этот алгоритм
(рис. 3) избегает, следовательно, пересечений маленьких горизонталь-
ных сегментов, что дает красивый результат, когда нужны тонкие ли-
нии; было бы совсем по-другому, если нужно <<жирное>> построение.\newline
\hspace*{15pt}Кроме того, если желаем построить этим способом прямую с накло-
ном, большим 1, нужен второй алгоритм, идентичный первому с точ-
ностью до перемены ролями $x$ и $y$. Наконец, построение вертикальных,
горизонтальных или диагональных линий может проводиться наивным
способом, который будет всегда более быстрым, чем указанный.
Картинка\newline
\hspace*{15pt}\textbf{c.} Парадоксально, но этот алгоритм (рис. 4) - который может
дать менее элегантные результаты, чем предыдущий, - оказывается
немного более сложным для реализации. Действительно, нельзя больше
предполагать, что координата $x$ возрастает после включения точки; на
самом деле, каждая координата является \textit{доминантой} в полуквадран-
те, о котором рассуждаем. Если рассмотреть четыре точки квадрата,
сразу замечаем, что невозможно требовать - если не разрешены диаго-
нальные перемещения - чтобы вертикальное расстояние правой точки
было меньше $1/2$. Если поддерживать включенную точку в полосе фи-
гуры 1, это влечет, между прочим, что одно из расстояний справа, го-
\newpage

ризонтальное или вертикальное, может быть сохранено меньшим 1, это
означает, что величина $|vx-uy|$ меньше, чем $u$ или $v$. Сказать, что точ-
ка расположена в полосе, очень точно означает, что $|2(vx-uy)|\leq u+v$;
теперь покажем, что можно подтвердить это положение.\newline
\hspace*{15pt}Движение, осуществляемое на экране, будет определяться величи-\newline
ной $\delta(x,y)=2(vx-uy)$, получаемой после этого движения. Возможны
два (не исключительных) случая: увеличивают $x$, величина становится
равной $\delta(x+1,y)=\delta(x,y)+2v$, и нужно убедиться, что она остается
меньше $u+v$, т.е. что $\delta(x,y)\leq u-v$. Аналогичным образом можно
увеличить $y$, только если $u-v\leq\delta(x,y)$. Видим, что положение $\delta(x,y)$
по отношению к $u-v$ определяет осуществляемые перемещения.\newline
\hspace*{15pt}Доказательство сходимости состоит в выявлении, что во время ите-
раций $x\leq u$ и $y\leq v$. Чтобы проверить это, зная, что $x$ или $y$ растут
при каждой итерации, можно предположить, что $x=u$ и $y<v$. В
этих условиях оценим величину $d$: $d\geq\delta(u,v-1)+v-u=u+v>0$.
Как следствие, очередная итерация увеличит $y$, и так будет до тех пор,
пока $y<v$.\newline
\hspace*{15pt}Заметим, что по этому алгоритму построенные две вещественные
прямые, симметричные относительно первой диагонали, не симметрич-
ны относительно этой же диагонали (тест, применяемый к $d$, не экви-
валентен для $x$ и $y$).\newline
\subsection{Построение окружности методами DDA}
\hspace*{15pt}\textbf{a.} Функция $f$ убывает на итервале $x\in[0,R\sqrt{2}/2]$, и ее производная
здесь меньше 1. Как следствие, $0\leq f(x)-f(x+1)\leq1$, значит, учитывая
это, можно ограничиться уменьшением $y$ только на шаг, равный 1.\newline
\hspace*{15pt}\textbf{b.} Исходим из сделанного предположения, что $y-1$ есть луч-
шее приближение f(x+1) тогда и только тогда, когда $y-3/2\leq$
$f(x+1)<y-1/2$, т.е. $f(x+1)<y-1/2$; второе неравенство выте-
кает из предположений по поводу f(x) и результата вопроса $a$.\newline
\hspace*{15pt}Следовательно, оцениваем на каждой итерации величину
$e(x,y)=4y^2-4R^2-4y+4x^2+8x+5$, и нетрудно доказать, что
$y-1/2>\sqrt{R^2-(x+1)^2}$ тогда и только тогда, когда $e(x,y)>0$, и
еще, поскольку оперируем целыми величинами, тогда и только тогда,
когда $d(x,y)=e(x,y)-1\geq 0$. Кроме того, $e(x+1,y)=e(x,y)+8x+12$\newline
и $e(x+1,y-1)=e(x,y)+8(x-y)+20$ и, конечно, $e(0,R)=4-4R$.\newpage
Рисунок\newline
\hspace*{15pt} Рассмотрим теперь более пристально последовательность величин\newline
$e_i=d(x_i,y_i)-1$. Имеем следующие рекуррентные определения:
\begin{center}
\hspace*{15pt}$x_0=0$,\hspace{15pt}$y_0=R$,\hspace{15pt}$e_0=4-4R$,\newline
$x_{i+1}=x_i$,\hspace{15pt}
$y_{i+1}=\left\{ 
      \begin{gathered}
        \hspace{-53pt}y_i-1, \text{ если $e_i\geq0,$} \\
        y_i\text{\hspace{25pt}в противном случае,} \\ 
      \end{gathered} 
\right.$\newline
$e_{i+1}=\left\{ 
      \begin{gathered}
        \hspace{-53pt}e_i+8(x_{i+1}-y_{i+1})+20, \text{ если $e_i\geq0,$} \\
        e_i+8x_{i+1}+12\text{\hspace{45pt}в противном случае.} \\ 
      \end{gathered} 
\right.$
\end{center}
Значит, совершенно ясно, что эти соотношения могут упроститься (по­-
скольку только последовательности $(х_i)$  и $(у_i)$ нас интересуют, и их
определение зависит от знака $(e_i)$) следующим образом:
\begin{center}
\hspace*{15pt}$x_0=0$,\hspace{15pt}$y_0=R$,\hspace{15pt}$e_0=1-R$,\newline
$x_{i+1}=x_i$,\hspace{15pt}
$y_{i+1}=\left\{ 
      \begin{gathered}
        \hspace{-53pt}y_i-1, \text{ если $s_i\geq0$,} \\
        y_i\text{\hspace{25pt}в противном случае,} \\ 
      \end{gathered} 
\right.$\newline
$s_{i+1}=\left\{ 
      \begin{gathered}
        \hspace{-53pt}s_i+2(x_{i+1}-y_{i+1})+5, \text{ если $e_i\geq0,$} \\
        s_i+2x_{i+1}+3\text{\hspace{46pt}в противном случае.} \\ 
      \end{gathered} 
\right.$
\end{center}
Это соотношения, используемые в алгоритме 5.\newline
\hspace*{15pt}Когда работа алгоритма заканчивается, $x\geq y$, и не обязательно за­-
жигать точку $(x,y)$. Действительно, если $x>y$, правила, которые
в верхнем октанте, больше недействительны (точка $(x,y)$ -
в нижнем октанте). Зато, если $x=y$, зажигать точку нужно.\newpage


\hspace*{15pt}\textbf{c.} Аргумент Бреэенхама просто выражает тот факт, что если да­-
на точка $(x,y)$ дискретной окружности и $|(x+1)^2+(y-1)^2-R^2\leq|$
$\leq |(x+1)^2+y^2-R^2|$, то $|\sqrt{(x+1)^2+(y-1)^2}-R|\leq|\sqrt{(x+1)^2+y^2}-R|$
(эта импликация верна также, если обратить знаки неравенств), и в
этом случае будем брать $(x+1,y-1)$ как ближайшую точку дискретной
окружности; в противном случае ближайшей точкой будет $(x+1,y)$.
\hspace*{15pt}Обозначим $d(x,y)$ величину (положительную или отрицательную)
$x^2+y^2-R^2$. Ясно, что $d(x+1,y)>d(x+1,y-1)$. Как следствие, знак
суммы $s(x,y)=d(x+1,y)+d(x+1,y-1)$ указывает, какова лучшая
(в смысле, определенном условием задачи) точка, аппроксимирующая
окружность в этой окрестности: если $s(x,y)\geq$, то $P_2$ является этой
точкой, в противном случае - это $P_1$. Из этого непосредственно выте­-
кает алгоритм, и он соответствует эквивалентной версии алгоритма,
построенного в вопросе \textbf{b}: достаточно положить
\begin{center}
$s_{\text{Брезенхам}}=2s_{\text{вопрос \bf{b}}}$.
\end{center}

\subsection{Сложность возведения в степень}
\hspace*{15pt}Можно сразу заметить, что $P_i$ есть степень $x$:$ P_i=x^{\alpha_i}$, кроме того,
очевидно, что последовательность $(\alpha_i)_{\alpha_i\geq 0}$ является \textit{почти} цепочкой
сложений для $n$: действительно, нет строгого возрастания $\alpha_i$, и цепь
не обязательно начинается с 1; если исключить 1 из возможных зна­-
чений $y_i$ и $z_i$, получили бы настоящую цепочку сложений. Как бы то
ни было, легко видеть, что $\alpha_1\leq2$, $\alpha_2\leq 4$, $\alpha_3\leq8$ и т.д. Как след­-
ствие, $\alpha_t=n\leq2^t$, а это доказывает, что $t\geq log_2n$. Это доказывает
также, что невозможно сделать гораздо лучше, чем это делает дихото­-
мический алгоритм возведения в степень, если разрешаются только пе-
ремножения; точнее, можно было бы получить улучшенные результаты
с оптимизированными цепочками сложений, но от этого не изменится
порядок величины сложности.

\subsection{Последовательность Фибоначчи и золотое сечение}
\hspace*{15pt}\textbf{a.} Запишем равенства:\newline
\hspace*{80pt}$F=F_0+F_1X+F_2X^2+F_3X^3+F_4X^4+...$ ,\newline
\hspace*{71pt}$XF=$\hspace{26pt}$F_0X+F_1X^2+F_2X^3+F_3X^4+...$ ,\newline
\hspace*{67pt}$X^2F=$\hspace{58pt}$F_0X^2+F_1X^3+F_2X^4+...$ ,\newline
которые непосредственно (используя определение последовательности
Фибоначчи) подтверждают, что $F-XF-X^2F=X$. Формальный ряд\newpage



$1-X-X^2$, будучи обратимым (так как его постоянный член - нену-
левой), позволяет отсюда вывести, что $F=\dfrac{X}{(1-X-X^2)}$. Остается только
разложить эту последнюю рациональную дробь на простые слагаемые.
Нулями многочлена $X^2+X-1$ являются $-\phi$ и $-\hat{\phi}$, и быстро получается
разложение:
\begin{center}
$F=\dfrac{-\phi}{\sqrt{5}}\times\dfrac{1}{X+\phi}+\dfrac{\hat{\phi}}{\sqrt{5}}\times\dfrac{1}{X+\hat{\phi}} = \dfrac{1}{\sqrt{5}}\left(\dfrac{-1}{1-\hat{\phi}X}+\dfrac{1}{1-\phi X}\right)$,
\end{center}
с учетом того факта, что $\phi\hat{\phi} = -1$. Еще раз беря разложения ря-
дов, которые появляются в этом последнем выражении, выводим, что
$F_{n}=\dfrac{1}{\sqrt{5}}(\phi^n-\hat{\phi}^n)$.\newline
\hspace*{15pt}\textbf{b.} $\phi^n+\hat{\phi}^n$ является общим членом ряда
\begin{center}
$\dfrac{1}{1-\phi X}+\dfrac{1}{1-\hat{\phi}X}=\dfrac{2-X}{1-X-X^2}=\dfrac{2F}{X}-F$.
\end{center}
Значит, $\dfrac{1}{1-X-X^2}=\dfrac{F}{X}$ является порождающим рядом последовательно-
сти $(F_{n+1})$, следовательно, $\phi^n+\hat{\phi}^n=2F_{n+1}-F_{n}$.\newline
\hspace*{15pt}\textbf{c.} $\phi^n$ есть общий член ряда
\begin{center}
$\dfrac{1}{1-\phi X}=\dfrac{1-\hat{\phi} X}{1-X-X^2}=\dfrac{F}{X}-\hat{\phi}F$.
\end{center}

Но $\phi\hat{\phi}=-1$, и тогда, умножая на $\phi$ равенство $\phi^n=F_{n+1}-\hat{\phi}F_{n}$, получаем
$\phi^{n+1}=\phi F_{n+1}+F_{n}$.
\subsection{Соотношения в последовательности Фибоначчи}
\hspace*{15pt}\textbf{a.} Нетрудно установить, что характеристическим многочленом для
$A$ является $X^2-X-1=(X-\phi)(x-\hat{\phi})$. Матрица $A$ тогда приводится к
диагональному виду, и легко находим базис, образуемый собственными
векторами$(\phi,1)$ и $(\hat{\phi},1)$. Следовательно,\newline
\begin{center}
$A^n=\dfrac{1}{\sqrt{5}}\begin{pmatrix} \phi & \hat{\phi} \\ 1 & 1 \end{pmatrix}\begin{pmatrix} \phi^n & 0 \\ 0 & \hat{\phi}^n \end{pmatrix} \begin{pmatrix} 1 & -\hat{\phi} \\ -1 & \phi \end{pmatrix} = \begin{pmatrix} F_{n+1} & F_{n} \\ F_{n} & F_{n-1} \end{pmatrix}$.
\end{center}
Конечно, простое рассмотрение первых величин $A^n$, вытекающих из
рекуррентности, привело бы к тому же результату, но не будем отка-
зывать себе в маленьком удовольствии (обратиться к упражнению 14
для более общей точки зрения).
\newpage


\hspace*{15pt}Требуемые соотношения получаются соответственно вычислением
определителя матрицы $A^n$ и использованием равенства $A^{n+m}=A^n A^m$.
Эти соотношения обнаруживаются вновь в еще более общей форме в
упражнениях главы II, относящимся к непрерывным дробям.\newline
\hspace*{15pt}\textbf{b.} Общий член произведения двух формальных рядов является
сверткой $n$ первых членов каждого ряда: $\sum_{i \leq n}a_ib_{n-i}$. Для задачи, ко-
торой мы занимаемся, имеем: порождающий ряд последовательности
$f_n$ есть $F^2$ ($F$ - порождающий ряд последовательности Фибоначчи).\newline
$F^2=\dfrac{1}{5}\left(\dfrac{1}{(1-\phi X)^2}-\dfrac{2}{(1-\phi X)(1-\hat{\phi} X)}+\dfrac{1}{(1-\hat{\phi} X)^2}\right)$\newline
\hspace*{15pt}$=\dfrac{1}{5}\left(\sum(n+1)\phi^nX^n-2(\sum\phi^nX^n)(\sum\hat{\phi}^nX^n)+\sum(n+1)\hat{\phi}^nX^n\right)$\newline
\hspace*{15pt}$=\dfrac{1}{5}\left(\sum(n+1)(\phi^n+\hat{\phi}^n)X^n-2\sum(\sum\phi^i\hat{\phi}^{n-i})X^n\right)$.\newline
Но мы знаем $\phi^n+\hat{\phi}^n$ (упражнение 12), и общий член второго ряда есть
$\dfrac{\phi^{n+1}-\hat{\phi}^{n+1}}{\phi-\hat{\phi}}=F_{n+1}$. Значит,
\begin{center}
$F^2=\dfrac{1}{5}\left(\sum(n+1)(2F_{n+1}-F_n)X^n-2\sum F_{n+1}X^n\right)$
\end{center}
\begin{center}
или\hspace{15pt}$\displaystyle\sum_{k=0}^nF_kF_{n-k} = \dfrac{2nF_{n+1}-(n+1)F_n}{5}$.
\end{center}
\subsection{Линейные рекуррентные последовательности $k$-го порядка}
\hspace*{15pt}\textbf{a.} Действуем, выполняя преобразования над порождающими ряда-
ми. Пусть $\mathcal{F}$ - порождающий ряд последовательности. Как для по-
следовательности Фибоначчи, можно легко установить, что $(1-X-$
$X^2)\mathcal{F}=x_0+X(x_1-x_0)$. Следовательно,
\begin{center}
$\mathcal{F}=\dfrac{x_0+X(x_1-x_0}{1-X-X^2}=x_0\dfrac{F}{X}+(x_1-x_0)F$.
\end{center}
Значит, $x_n=x_0F_{n+1}+(x_1-x_0)F_n=x_1F_n+x_0F_{n-1}$.\newline
\hspace*{15pt}\textbf{b.} Ясно, что множество рассмотренных последовательностей явля-
ется векторным пространством размерности $k$, и существует изомор-
физм между этим пространством и $K^k$: он ставит в соответствие ка-\newpage


ждой последовательности ее $k$ первых членов. Полным прообразом ка-
нонического базиса $K^k$ является то, что называется фундаменталь-
ным базисом пространства последовательностей. Обозначим этот базис
$(x^{k-1},...,x^0)$ ($x^0$ есть полный прообраз вектора $(0,...,0,1))$. Матрица
$A$, которая действует в $K^k$, соответствует при этом изоморфизме опе-
ратору сдвига последовательностей. Тогда столбцы матрицы $A$ суть
выборки $(x_k^i,...,x_1^i)$ из последовательностей $x^i$. Это означает еще, что
\begin{center}
$A^n=\begin{pmatrix}
x_{n+k-1}^0 & x_{n+k-1}^1 & \cdots & x_{n+k-1}^{k-1} \\
x_{n+k-2}^0 & x_{n+k-2}^1 & \cdots & x_{n+k-2}^{k-1} \\         
\vdots & \vdots & \ddots & \vdots \\
x_n^0 & x_n^1 & \cdots & x_n^{k-1}
\end{pmatrix}$.
\end{center}
В случае, когда $k=2$, фундаментальный базис пространства последова-
тельностей образован последовательностью Фибоначчи и ее \textit{сдвигами}.
Тут же обнаруживается результат предыдущего вопроса и матричное
равенство упражнения 13.\newline
\subsection{Дихотомия по Горнеру}
\hspace*{15pt}\textbf{a.} Конечно, $P_0=P$ и $P_k=a_k$. Соотношение между $P_i$ и $P_{i-1}$,
которое определяет метод Горнера, есть $P_{i-1}=X P_i+a_{i-1}$.\newline
\hspace*{15pt}\textbf{b.} Если предположить, что $n=\sum b_i2^i$, то можно рассмотреть мно-
гочлен $P=\sum b_iX^i$ и вычислить элементы $x^{P_i(2)}$, используя предыдущее
рекуррентное соотношение. Это дает алгоритм:\newline
Прога\newline
Этот алгоритм может быть оптимизирован исключением ненужных
перемножений следующим образом:\newline
Прога
\newpage



\textbf{c.} Если неизвестно двоичное разложение $n$, то ясно, что нужно бу-
дет тем или иным способом вычислить его более или менее неявно по
ходу выполнения алгоритма. Если известно, что $2^k \leq n < 2^{k+1}$, то это
влечет, что двоичный разряд порядка $k$ числа $n$ равен 1, что позво-
ляет запустить счет. Затем достаточно определить, принадлежит ли
$n-2^k$ интервалу $[a^{k-1},2^k]$, чтобы узнать величину двоичного разряда
порядка $k-1$. Все это приводит к следующему алгоритму (который не
касается случая $n=0$):\newline
Прога\newline
В этом алгоритме все начинается с определения $\alpha=2^k$ такого, что
$a^k \leq n < 2^{k+1}$ (только без того, чтобы результаты промежуточных
вычислений превосходили $n$), затем продолжается вычисление $x^n$.\newline
\hspace*{15pt}\textbf{d.} В этом случае, когда сложность одного перемножения постоянна
(независимо от размеров сомножителей), устанавливаем, что имеется
самое большее $k$ итераций, каждая из которых содержит 1 или 2 умно-
жения; следовательно, сложность оценивается сверху числом $2log_2n$.\newline
\hspace*{15pt}В случае, когда сложность элементарного умножения не постоянна, результат совершенно другой.

\subsection{Вычисление чисел Фибоначчи}
\hspace*{15pt}\textbf{a.} Несложно написать рекурсивный аглоритм, подобный следующему:\newline
Прога\newline
\hspace*{15pt}Число рекурсивных вызовов в действительности равно $2\sum_{i=1}^{n-1}F_i=$
$2(F_{n+1}-1)$ (доказать этот результат), что дает экспоненциальную
сложность.\newpage



Вдохновившись, напротив, методом, использованным для алгорит-
ма Евклида, состоящим в преобразовании пар последовательных эле-
ментов в последовательность: $(F_{n+2},F_{n+1})=(F_{n+1}+F_n,F_{n+1})$ (фор-
мула, которая имеет связь с матрицей упражнения 14), приходим к ал-
горитму, который для $n$ вычисляет пару чисел Фибоначчи $(F_{n+1},F_n)$:\newline
Прога\newline
Этот алгоритм, который легко может быть сделан нерекурсивным,
имеет сложность, пропорциональную $n$ (линейную, а не экспоненциаль-
ную).\newline
\hspace*{15pt}\textbf{Ь.} Любое из соотношений:\newline
$F_{n-1}+\phi F_n=\phi^n$,\hspace{20pt}
$\begin{pmatrix}
F_n \\ F_{n-1}
\end{pmatrix}=
\begin{pmatrix}
1 & 1 \\ 1 & 0
\end{pmatrix}^n
\begin{pmatrix}
1 \\ 0
\end{pmatrix}$,\hspace{20pt}$F_n=\dfrac{\phi^n-\hat{\phi}^n}{\sqrt{5}}$\newline
приводит к мысли, что $F_n$ может быть вычислено методом дихотоми-
ческого возведения в степень. Это действительно так! Обоснуем наше
решение с помощью, например, первого соотношения (другие приво-
дят к тому же методу). В несколько научной манере можно сказать,
что $\phi$ принадлежит кольцу $\mathbb{Z}[\phi]$, состоящему из $a+\phi b$ с целыми $a$, $b$.
Устойчивость относительно произведения в $\mathbb{Z}[\phi]$ вытекает из уравне-
ния $\phi^2=\phi+1$:
\begin{center}
$(a+\phi b)(c+\phi d)=ac+\phi(ad+bc)+\phi^2bd=(ac+bd)+\phi(ad+b(c+d))$.
\end{center}
Эта формула показывает, к тому же, как считать в $\mathbb{Z}[\phi]$ с помощью це
лых операций. Следовательно, вычисление $\phi^n$ может быть осуществлено
через дихотомию; как это видим по вышеупомянутой формуле, слож-
ность одного перемножения в $\mathbb{Z}[\phi]$ есть 4 целых перемножения и 3 целых
сложения, тогда как сложность возведения в квадрат — 3 целых перем-
ножения и 2 сложения: $(a+\phi b)^2=(a^2+b^2)+\phi b(2a+b)$, что приводит
к алгоритму 9-А.\newline
\hspace*{15pt}Но можно сделать лучше; соотношение
\begin{center}
$(a+bX)(c+dX)=ac+\left((a+b)(c+d)-ac-bd\right)X+bdX^2$
\end{center}
доказывает, что можно умножать два многочлена первой степени с по-
мощью только 3 перемножений, но 4 сложений (об этом снова пойдет
\newpage



речь в главе V). Применительно к кольцу $\mathbb{Z}[\phi]$ оно позволяет вычислить
произведение с помощью 3 целых перемножений и 4 целых сложений:
\begin{center}
$(a+\phi b)(c+\phi d)=ac+bd+\phi((a+b)(c+d)-ac)$,
\end{center}
и, конечно, использовались при реализации.\newline
\hspace*{15pt}\textbf{c.} Можно вычислить $\phi^n$ через дихотомию, но применяя метод
упражнения 15 (использование метода Горнера). Заметим, что одна
из компонент аддитивных перемножений в этом способе постоянна
(равна элементу, степень которого хотим вычислить): здесь это умно-
жение на $\phi$, которое не является настоящим умножением, так как
$(a+b\phi)\phi=b+\phi(a+b)$. В алгоритме 9-В последовательные степени
$\phi$ вычислены в паре $(g,g')$, представляющей $g+\phi g'$.\newline
Таблица с программой
\newline
\subsection{$\epsilon$-лексикографический порядок и знакопеременный лексикографический порядок}
\hspace*{15pt}\textbf{a.} $\epsilon$-лексикографический порядок является линейным, если исход-
ные упорядочения линейны. Если функция $\epsilon$ тож дественно равна 1, по-
лучаем тогда обычный лексикографический порядок.
\newpage



\hspace*{15pt}\textbf{b.} Эта история с планкой на



\end{document} 